%% This is file `mlq-tpl.tex',
%% generated with the docstrip utility.
%%
%% The original source files were:
%%
%% template.dtx  (with options: `mlq')
%% 
%% $Id: template.dtx,v 1.80 2005/10/04 16:26:14 uwe Exp $
%% 2007-04-11 -MWL- aktualisiert
%% ====================================================================
\documentclass[mlq,a4paper,fleqn%
% ,finallayout%
]{w-art}

%\documentclass[a4paper,10pt,twoside]{article}


\usepackage{textcase}
\usepackage{ifthen}
\usepackage[latin1]{inputenc}
\usepackage{graphicx, color}
\usepackage{amsmath,amsthm,amssymb,amscd}

\usepackage[T1]{fontenc}
\usepackage{mathpazo,helvet,courier}

\usepackage{latexsym} 
\usepackage{url} 

\usepackage{relsize} 

\usepackage{calc}

\hyphenation{in-ter-pre-ta-tion}
\hyphenation{in-tui-tion-istic-ally}




%%%%%%%%%%%%%%%%%%%%%%%%%%%%%%%        %%%%%%%%%%%%%%%%%%%%%%%%%%%%%%%%%
%%%%%%%%%%%%%%%%%%%%%%%%%%%%%%% Quote  %%%%%%%%%%%%%%%%%%%%%%%%%%%%%%%%%
%%%%%%%%%%%%%%%%%%%%%%%%%%%%%%%        %%%%%%%%%%%%%%%%%%%%%%%%%%%%%%%%%

\newcommand{\theQuote}[1]{
\begin{center}
\begin{minipage}{0.7\textwidth} 
{\em #1} 
\end{minipage}
\end{center}
}


\newcommand{\args}[1]{
                        \text{\begin{smaller}$#1$\end{smaller}}
}



\newcommand{\usftext}[1]{\textsf{\upshape #1}}

\newenvironment{rmk}{\paragraph{Remark.}\it}{\\}


\newcommand{\be}[1][{e:\arabic{equation}}] { \begin{equation}\label{#1} }
\newcommand{\ee} { \end{equation} }




%%stuff - mainly KL
\DeclareMathOperator{\lh}{lh}  %length of encoding of a finite sequence

\newcommand{\mon}{\ensuremath{\usftext{m}}} 
\newcommand{\Mon}{\ensuremath{\usftext{M}}} 

%%commands
\renewcommand{\emptyset}{\varnothing}

\newcommand{\ORi}[1]{\ensuremath{\bigwedge^{#1}_{i=1}}}


\newcommand{\RR}{\ensuremath{\mathbb{R}}}
\newcommand{\NN}{\ensuremath{\mathbb{N}}}
\newcommand{\QQ}{\ensuremath{\mathbb{Q}}}
\newcommand{\II}{\ensuremath{\mathbb{I}}}
\newcommand{\PP}{\ensuremath{\mathbb{P}}}

\newcommand{\zero}{\ensuremath{\mathbf0}}
\newcommand{\one}{\ensuremath{\mathbf1}}
\newcommand{\two}{\ensuremath{\mathbf2}}

\newcommand{\xor}{\ensuremath{\dot\vee}}

%%%%%%%%%%%%%%%%%%%%%%%%%   Proof Theory   %%%%%%%%%%%%%%%%%%%%%%%%%%%%%%%
%%%%%%%%%%%%%%%%%%%%%%%%%   Proof Theory   %%%%%%%%%%%%%%%%%%%%%%%%%%%%%%%
%%%%%%%%%%%%%%%%%%%%%%%%%   Proof Theory   %%%%%%%%%%%%%%%%%%%%%%%%%%%%%%%

%% stuff
\input prooftree

\DeclareMathOperator{\maj}{s-maj} %``majorizes
\DeclareMathOperator{\smaj}{s-maj} %``strongly majorizes
\DeclareMathOperator{\K}{K} %K_A from Howard's WKL ND-int
\DeclareMathOperator{\I}{I} %the ``in Interval predicate from BW



%% Types
\newcommand{\Tp}{\ensuremath{\emph{\protect\textbf{T}}}} %set of finite types T
\newcommand{\PT}{\ensuremath{\emph{\protect\textbf{P}}}} %set of pure types P
\newcommand{\tp}[1]{\ensuremath{^\mathbf{#1}}}

%% Models
\newcommand {\SetO}  { \ensuremath{\mathcal{S} } }
\newcommand {\Som}  { \ensuremath{\SetO^\omega} }
\newcommand {\Set}  { \Som }
\newcommand {\ContO}  { \ensuremath{\mathcal{C} } }
\newcommand {\Cont}  { \ensuremath{\ContO^\omega} }
\newcommand {\MajO}  { \ensuremath{\mathcal{M} } }
\newcommand {\Maj}  { \ensuremath{\MajO^\omega} }

%% Special sets
\newcommand{\universal}{\ensuremath{\emph{\protect\textbf{U}}}} %set of universal axioms

%% Systems
\newcommand{\GA}{\ensuremath{\usftext{G}_n\usftext{A}^\omega}} %GnA iaft
\newcommand{\weha}{\ensuremath{{\usftext{WE-HA}}^{\omega}}} % WE - HA iaft
\newcommand{\wepa}{\ensuremath{{\usftext{WE-PA}}^{\omega}}} % WE - PA iaft
\newcommand{\HA}{\ensuremath{{\usftext{HA}}}} % HA 
\newcommand{\PA}{\ensuremath{\usftext{PA}}} % PA 
\newcommand{\ha}{\ensuremath{{\usftext{HA}}^\omega}} % HA iaft
\newcommand{\pa}{\ensuremath{{\usftext{PA}}^\omega}} % PA iaft
\newcommand{\epa}{\ensuremath{{\usftext{E-PA}}^\omega}} % E - PA iaft
\newcommand{\eha}{\ensuremath{{\usftext{E-HA}}^\omega}} % E - HA iaft
\newcommand{\hrrepa}{\ensuremath{\widehat{\usftext{E-PA}}^\omega\kleene}} %hrr E - PA iaft
\newcommand{\hrreha}{\ensuremath{\widehat{\usftext{E-HA}}^\omega\kleene}} %hrr E - HA iaft
\newcommand{\rreha}{\ensuremath{\usftext{E-HA}^\omega\kleene}} %rr E - HA iaft
\newcommand{\rrweha}{\ensuremath{\usftext{WE-HA}^\omega\kleene}} %rr WE - HA iaft
\newcommand{\kleene}{\ensuremath{\!\!\!\restriction}}   % upper arrow
\newcommand{\hrrwepa}{\ensuremath{\widehat{\usftext{WE-PA}}^\omega\kleene}} %hrr WE - PA iaft
\newcommand{\hrrweha}{\ensuremath{\widehat{\usftext{WE-HA}}^\omega\kleene}} %hrr WE - HA iaft

\newcommand{\HAS}{\ensuremath{\usftext{HAS}}} %second order logic
\newcommand{\HAH}{\ensuremath{\usftext{HAH}}} %higher -/-
\newcommand{\ACA}{\ensuremath{\usftext{ACA}}} %
\newcommand{\RCA}{\ensuremath{\usftext{RCA}}} %
\newcommand{\PRA}{\ensuremath{\usftext{PRA}}} %

%% Principles
\newcommand{\IA}{\ensuremath{\usftext{IA}}} %induction schema
\newcommand{\IP}{\ensuremath{\usftext{IP}}} %induction principle
\newcommand{\IR}{\ensuremath{\usftext{IR}}} %induction rule
\newcommand{\BR}{\ensuremath{\usftext{BR}}} %induction rule
\newcommand{\BI}{\ensuremath{\usftext{BI}}} %induction rule


\newcommand{\IPP}{\ensuremath{\usftext{IPP}}}
\newcommand{\PCM}{\ensuremath{\usftext{PCM}}}
\newcommand{\LEM}{\ensuremath{\Sigma^0_1\usftext{-LEM}}}
\newcommand{\lLEM}{\ensuremath{\usftext{LEM}}}
\newcommand{\BW}{\ensuremath{\usftext{BW}}}
\renewcommand{\AA}{\ensuremath{\usftext{AA}}}
\newcommand{\Limsup}{\ensuremath{\usftext{Limsup}}}
\newcommand{\DNS}{\ensuremath{\usftext{DNS}}}
\newcommand{\CA}{\ensuremath{\usftext{CA}}}
\newcommand{\QF}{\ensuremath{\usftext{QF}}}
\newcommand{\QFm}{\ensuremath{\usftext{QF-}}}
\newcommand{\CAhut}{\ensuremath{\widehat{\CA}}}

\newcommand{\AC}{\ensuremath{\usftext{AC}}} 
\newcommand{\ER}{\ensuremath{\usftext{ER}}} 

\newcommand{\WKL}{\ensuremath{\usftext{WKL}}}
\newcommand{\FAN}{\ensuremath{\usftext{FAN}}}

%% General abreviations
\newcommand{\PiL}{\ensuremath{\Pi^0_1}} 
\newcommand{\PiLm}{\ensuremath{\Pi^0_1\usftext{-}}} 
\newcommand{\SiL}{\ensuremath{\Sigma^0_1}} 
\newcommand{\SiLm}{\ensuremath{\Sigma^0_1\usftext{-}}} 
\newcommand{\m}{\ensuremath{\usftext{-}}}


%% for WKL

\newcommand{\BTree}{\ensuremath{\usftext{BinTree}}}
\newcommand{\BFunc}{\ensuremath{\usftext{BinFunc}}}
\newcommand{\UnBounded}{\ensuremath{\usftext{Unbounded}}}
\newcommand{\Sec}{\ensuremath{\usftext{Sec}}} % Boundedly secured
\newcommand{\BSec}{\ensuremath{\usftext{BarSec}}} % Boundedly secured at bar k
\newcommand{\BSecA}{\ensuremath{\usftext{BarSec}_A}} % Boundedly secured at bar K_A

\newcommand{\B}{\ensuremath{\usftext{B}}} %bar recursor
\newcommand{\rB}{\ensuremath{\usftext{B'}}} %restricted bar recursor
\newcommand{\R}{\ensuremath{\usftext{R}}} %recursor
\newcommand{\bPhi}{                       %special bar recursor
 \raisebox{-1.0pt} {
   \ensuremath{\usftext{\Large {\!$\Phi$\!}}}
 }
}

\newcommand{\T}{\ensuremath{\mathcal{T}}} %G�els T
\newcommand{\M}{\ensuremath{\usftext{M}^\omega}} %Markov principle iaft
\renewcommand{\H}{\ensuremath{\usftext{H}}} %Funny Howards argument

\newcommand{\proves}{\vdash}  %proves |-
\newcommand{\forces}{\Vdash}  %||-
\renewcommand{\models}{\vDash}  %|=


\newcommand{\widefr}[1] {
    \text{%
     \rotatebox[x=0.5\width+0.8ex, y=0.5ex]{-90}{%
      \ensuremath{%
         \left(\text{\rotatebox[origin=c]{90}{\ensuremath{#1}}}\right.%
      }%
     }%
    }%
}

\newcommand{\tup}{\underline} %tuple
\newcommand{\atup}{\ensuremath{\,\underline}} %tuple as a parameter

\newcommand{\Tif}{\text{if}}
\newcommand{\Telse}{\text{else}}

%%%%%%%%%%%%%%%%  shortcuts %%%%%%%%%%%%%%%%%%%%%%

\renewcommand{\phi}{\varphi}
\newcommand{\lb}{\linebreak[0]}
\newcommand{\nbd}{\nobreakdash-}

\newcommand{\lOrd}[1]{\text{$\!\!$
\begin{smaller}<\end{smaller}\nolinebreak[4] $\!#1$\hspace{-3pt}
}}

\newcommand{\lOrdm}[1]{\text{
\lOrd{#1}\nbd 
}}

%\makeatletter
\def\Ddots{\mathinner{\mkern1mu\raise\p@
\vbox{\kern7\p@\hbox{.}}\mkern2mu
\raise4\p@\hbox{.}\mkern2mu\raise7\p@\hbox{.}\mkern1mu}}
%\makeatother

\newcommand{\embeded}{\hookrightarrow}

%%%%%%%%%%%%%%%%%%%%%% repetitive definitions (macros) %%%%%%%%%%
%%%%%%%%%%%%%%%%%%%%%% repetitive definitions (macros) %%%%%%%%%%
%%%%%%%%%%%%%%%%%%%%%% repetitive definitions (macros) %%%%%%%%%%
\newcommand{\xTWf}[3]
{\ensuremath{
\bPhi_0{#1}u_{#2,#3}^{\!0(1(0))(0)}0^0\zero^1
}}

\newcommand{\uWf}[2]
{\ensuremath{
    \begin{cases}
      1 &\Tif\ #2(n,#1(v1^0))\neq_00\\
      #1(v1^0) &\Telse
    \end{cases}
}}

\newcommand{\xTWfM}[2]
{\ensuremath{
\bPhi^*_0{#1}u_{#2}^{\!*0(1(0))(0)}0^0\zero^1
}}



\newcommand{\uWfM}[1]
{\ensuremath{
    {\max}_0\big(1,#1(v1)\big)
}}


\newcommand{\KA}[2]
{\ensuremath{
\begin{cases}
  0&\text{if}\ #1[#2]<\lh(#2) \\ 
  1+\max\big\{K_{#1}( #2 *0),K_A( #2 *1)\big\}&\text{otherwise,}
\end{cases}
}}


\newcommand{\Fn}[1]
{\ensuremath{
    \begin{cases}
      {\min}{}_0\big\{b^0\ :\ b^0=\langle b_1,\ldots,b_n\rangle,\tup b\in\{0,1\}^n\wedge #1(b)=_00\big\}
                     &\text{if such a $b$ exists}\\
      \emptyset&\text{else}
    \end{cases}
}}

\newcommand{\fs}[2]
{\ensuremath{
\begin{cases}
  0&\Tif\ \             \forall n^0\!\leq_0\!\lh(#1)\ #1_{n}\!\leq_0\!1 
              \ \wedge\ #2\geq_0 \lh(#1) 
              \ \wedge\ I(#1,\lh(#1),\widetilde{\ s#2\ })\\
  1&\Telse
\end{cases}
}}

\newcommand{\InIntM}[2]
{\ensuremath{
    #2\leq_0 \lh(#1) \ \wedge\ 
    \left\langle \sum_{i=1}^{#2}\frac{{#1}_i}{2^i}\right\rangle\ \leq_\QQ\ 
    m\ \leq_\QQ\left\langle\sum_{i=1}^{#2}\frac{{#1}_i}{2^i}+\frac{1}{2^{#2}} \right\rangle
}}


\usepackage{times,cite,w-thm}

% our adaptations
%%%%%%%%%%%%%%%%%%%%%%%%%   Theorems   %%%%%%%%%%%%%%%%%%%%%%%%%%%%%%%%%%%
%%%%%%%%%%%%%%%%%%%%%%%%%   Theorems   %%%%%%%%%%%%%%%%%%%%%%%%%%%%%%%%%%%
%%%%%%%%%%%%%%%%%%%%%%%%%   Theorems   %%%%%%%%%%%%%%%%%%%%%%%%%%%%%%%%%%%

\theoremstyle{plain}

\theoremstyle{definition}
\newtheorem{dfn}[theorem]{Definition}

\theoremstyle{remark}
%\newtheorem*{remark}{Remark}
\newtheorem*{eg}{Example}


% \usepackage{w-sidecapt}
%% By default the equations are consecutively numbered. This may be changed by
%% the following command.
%% \numberwithin{equation}{section}
%%
%% The definition of new theorem like environments.
%% Criterion
\theoremstyle{plain}
\newtheorem{criterion}{Criterion}
%% Condition
\theoremstyle{definition}
\newtheorem{condition}[theorem]{Condition}
%%
%% The usage of multiple languages is possible.
%% \usepackage{ngerman}% or
%% \usepackage[english,ngerman]{babel}
%% \usepackage[english,french]{babel}
%% \usepackage[]{graphicx}
%% 
\begin{document}
%%    The information for the title page will be placed between
%%    \begin{document} and \maketitle. The order of most entries
%%    is determined by the class file and can not be changed by
%%    rearranging them. The maketitle command follows after the
%%    abstract.
%%
%%    Most of the following commands will be completed by the publisher.
%%
%%    The copyrightyear is defined in the .clo file as the first argument
%%    of the copyrightinfo command. If the copyrightyear differs from that
%%    value it might be adjusted by the following definition:
%%
%% \renewcommand{\copyrightyear}{2007}% uncomment to change the copyrightyear.
%%
\DOIsuffix{theDOIsuffix}
%%
%% issueinfo for the header line
\Volume{53}
\Month{01}
\Year{2007}
%%
%%    First and last pagenumber of the article. If the option
%%    'autolastpage' is set (default) the second argument may be left empty.
\pagespan{1}{}
%%
%%    Dates will be filled in by the publisher. The 'reviseddate' and
%%    'dateposted' (Published online) entry may be left empty.
\Receiveddate{XXXX}
\Reviseddate{XXXX}
\Accepteddate{XXXX}
\Dateposted{XXXX}
%%
\keywords{Bolzano-Weierstra\ss{}, sequential compactness, hard analysis, Dialectica interpretation, G\"odel functional interpretation,  monotone functional interpretation, computational content, program extraction, proof mining }
\subjclass[msc2000]{03F10, 03F50, 03F60, 03D65}

%% \pretitle{Editor's Choice}

%% We have a short and a long form for the title. The short form
%% (optional argument) goes into the running head.

\title[Interpreting Bolzano-Weierstra\ss{}]{On the Computational Content of the Bolzano-Weierstra\ss{}\\
Principle}

%% Please do not enter footnotes or \inst{}-notes into the optional
%% argument of the author command. The optional argument will go into
%% the header.  If there is only one address the marker \inst{x} may be
%% omitted.

%% Information for the first author.
\author[P. Safarik]{Pavol Safarik\inst{1,}%
  \footnote{Corresponding author\quad E-mail:~\textsf{safarik@mathematik-tu-darmstadt.de} }}
\address[\inst{1}]{TU Darmstadt, FB Mathematik, AG Logik, Schlossgartenstra{\ss}e 7, D-64289 Darmstadt}
%%
%%    Information for the second author
\author[U. Kohlenbach]{Ulrich Kohlenbach\inst{1,}\footnote{The main results of this paper are from the Diploma Thesis 
\cite{Safarik08} 
of the first author written under the supervision of the second author. 
The authors gratefully acknowledge the support by the German Science
Foundation (DFG Project KO 1737/5-1).}}
%\address[\inst{2}]{Second address}
%%
%%    Information for the third author
% \author[T. Author]{Third Author\inst{2,}\footnote{Third author footnote.}}
%%
%%    \dedicatory{This is a dedicatory.}
\begin{abstract}
We will apply the methods developed in the field of `proof mining' to the
Bolzano-Weierstra\ss{} theorem $\BW$ and calibrate the 
computational contribution of using this theorem in proofs
of combinatorial statements. We provide an 
explicit solution of the G\"odel functional interpretation (combined with 
negative translation) as 
well as the monotone functional interpretation of $\BW$ for 
the product space $\prod_{i\in\NN}[-k_i,k_i]$ (with the standard product 
metric). 
This results in 
optimal program and bound extraction theorems for proofs based on 
fixed instances of $\BW$, i.e. for $\BW$ applied to 
fixed sequences in $\prod_{i\in\NN}[-k_i,k_i]$.
\end{abstract}
%% maketitle must follow the abstract.
\maketitle                   % Produces the title.

\section*{Introduction}
\label{s:intro}
\markboth{\small\sffamily\fontseries{c}\selectfont {Introduction}}
         {\small\sffamily\fontseries{c}\selectfont {Introduction}}
During the last 15 years, an applied form of proof theory, also called 
`proof mining', became more and more prominent 
that  is concerned primarily with the extraction 
of additional (often computational) 
information from prima facie ineffective proofs which could not be read of 
directly (see e.g. \cite{Kohlenbach06}). 
Proof interpretations, most importantly forms of functional 
interpretation, are the key proof theoretic methods used in this type of 
proof theory. These methods enjoy a strong modularity in the sense that 
once the solution of the interpretation of a certain key lemma is found, 
that solution can be used without change in any other `unwinding' of proofs 
using this lemma. Most systematically, proof interpretations have been applied 
to proofs in (nonlinear) functional analysis. Here sequential 
compactness principles play an important role in numerous proofs. So far, 
such uses of sequential compactness could be dealt with using an elimination 
procedure due to the second author 
(\cite{Kohlenbach96,Kohlenbach_gp,Kohlenbach98,Kohlenbach00}) 
which replaces (if the underlying context is elementary enough) applications 
of fixed instances of these principles by arithmetical counterparts. However, 
more substantial uses of sequential compactness (e.g. in the context of 
weak compactness arguments) require to deal with these 
principles directly by explicitly solving their functional interpretation 
(see \cite{KohlenbachMints,Kohlenbach(weakcompactness)}. 
For the simplest form of sequential compactness, the principle that 
monotone bounded sequences of real numbers are convergent, this has been 
done already in \cite{KO02}.
\\[1mm]
This paper provides an explicit solution of the (negative translation of the) 
G\"odel functional 
interpretation as 
well as the monotone functional interpretation of the Bolzano-Weierstra\ss{} 
theorem $\BW$ for $[0,1]$ and other compact metric spaces. Moreover, we argue  
that our solution is of optimal complexity. In fact, we will use it to 
get optimal program and bound extraction theorems for proofs based on 
fixed instances $\BW(t)$ of $\BW$, i.e. for $\BW$ applied to 
fixed bounded sequences in $[0,1]$ 
given by a term $t$ whose only free variables are the parameters 
of the theorem to be proved. 
\\[1mm]   
As is known from reverse mathematics (\cite{Simpson99}) the 
Bolzano-Weierstra\ss{} theorem $\BW$ for compact metric spaces can be 
proved using arithmetical comprehension $\CA_{ar}$ (also denoted by 
$\Pi^0_{\infty}$-$\CA$) and -- already for $[0,1]$ -- 
also implies $\CA_{ar}$. All this holds 
irrespectively of whether 
$\BW$ is stated to assert the existence of a cluster point or a convergent 
subsequence and relative to a weak base system $\RCA_0.$ In fact, already 
the special case of $\BW$ stating that every monotone sequence in $[0,1]$ is 
convergent implies $\CA_{ar}$ (much refined results in this direction can be 
found in \cite{Kohlenbach00}). 
\\[1mm] From an inspection of Spector's solution of the functional 
interpretation of classical analysis by his bar recursive functionals 
$\T+(\BR)$ 
(\cite{Spector62}) it follows that the functional interpretation 
of $\CA_{ar}$ and hence of $\BW$ can be solved in the fragment $\T_0+\BR_{0,1}$, 
where only the primitive recursor $\R_0$ for type $0$ and the bar recursor 
$\B_{0,1}$ for the 
types $0,1$ are used (see \cite{Kohlenbach99}). 
Here $0$ denotes the type of natural numbers and $1$ the type of 
number theoretic functions $f:\NN\to\NN.$ In general, for types $\rho,\tau$ 
we denote the type of objects that map objects of type $\rho$ to objects of 
type $\tau$ by $\tau(\rho).$ Pure types are of the form $0$ or $0(\rho)$ 
(where $\rho$ already is a pure type) and can be represented by natural 
numbers via $n+1:=0(n).$ The level or degree $deg(\rho)$ of a type $\rho$
is defined as $deg(\tau(\rho)):=\max (deg(\tau),\deg(\rho)+1)$ with 
$deg(0):=0.$ 
Up to the type 
level $2,$ the functionals definable by closed terms of $\T_0+\BR_{0,1}$ 
(and also of $\T_1+\BR_{0,1}$)  
coincide with those definable in G\"odel's system $\T$ of primitive recursive 
functionals of finite type (\cite{Goedel58,Hilbert(26)}).    
For type $3$ this no longer holds as $\B_{0,1}$ is not definable in $\T$ and 
already for the type $1$ this fails for functionals definable in 
$\T_2+\BR_{0,1},$ $\T_0+\BR_{0,2}$ or $\T_0+\BR_{1,1}$ (see 
\cite{Kohlenbach99} for all this). Here $T_n$ is the fragment of $T$ that only 
contains recursors for primitive recursion of type level $\le n.$  \\ 
However, for a faithful calibration of the contribution of 
(single instances of) 
$\BW$ and the extraction 
of realizers of optimal complexity level from proofs of 
$\forall n\,\exists m$-sentences that are based on uses of $\BW$ even the fact 
that one has a functional interpretation in $\T_0+\BR_{0,1}$ is too crude. 
Indeed, it is crucial that 
the solution of the functional interpretation of $\BW$ uses only minimal 
number 
of nested $\B_{0,1}$-applications. In fact, we will show that a single use 
of $\B_{0,1}$ plus a use of a weak `binary' form of bar recursion (due 
to Howard) suffices. Together with results of Howard and Parson 
this can be used to show that over systems such as 
\[ \mbox{$\hrrwepa\,+\QFm\AC+\Sigma^0_{n+1}$-$\IA$} \] (which has 
a functional interpretation in $\T_n,$ see \cite{Parsons72}) 
the contribution of a use of $\BW$ in the form 
\[ \forall n\, \big(\,\mbox{$\BW$}(\xi(n))\to\exists m\,
\phi_{_\QF}(n,m)\big)\] 
in a proof of a sentence 
\[ \forall n\exists m\,\phi_{_\QF}(n,m) \ \   \mbox{(with quantifier-free 
$\phi_{_\QF}$)} \] 
at most increases the complexity of the 
extractable algorithm $f$ s.t. 
\[ \forall n\, \phi_{_\QF}(n,f(n)) \] 
from $f\in T_n$ to $f\in T_{n+1}.$ We will also show that this increase 
in general is unavoidable, thereby establishing the 
optimality of our result. \\   
Here $\hrrwepa$ is the (weakly extensional) extension of Primitive 
Recursive Arithmetic $\PRA$ to all finite types (i.e. -- in contrast to 
$\wepa$ -- we only include quantifier-free 
induction and primitive recursion of type $\NN$). \\ Moreover, {\sf QF-AC} 
is the schema of quantifier-free choice, i.e.   
\[ {\text{$\QFm\AC$}}\ :\quad
 \forall x^\rho\exists y^\tau   
     \phi_{_\QF}(x,y)\ \ \rightarrow\ \ \exists g^{\tau(\rho)}\forall 
x^\rho \phi_{_\QF}(x,g(x)) \]
with quantifier-free $\phi_{_\QF}$ (also for tuples of variables 
$\tup{x},\tup{y}$ of arbitrary types). Roughly speaking, 
$\hrrwepa\,+\QFm\AC$ is a (conservative) finite type extension of 
(an appropriate version of) $\RCA_0.$ \\     
{\sf $\Sigma^0_n$-IA} is the schema of induction for 
$\Sigma^0_n$-formulas, i.e.
\[ 
   {\Sigma^0_n\m\IA}\quad:\quad \phi(0)\wedge\forall x^0(\phi(x)\rightarrow\phi(x+1))\ \rightarrow\ \forall x^0\phi(x)
\text{,}
\]
where $\phi\in\Sigma^0_n$ may contain arbitrary parameters. 
\\[1mm] These results complement the ones 
obtained in \cite{Kohlenbach98,Kohlenbach00} using the aforementioned 
elimination method. In those papers it is shown that fixed (sequences of) 
instances of $\BW$, when used over systems containing only Kalmar elementary 
functionals (but not the recursor $\R_0$), at most contribute by ordinary 
primitive recursive complexity (i.e. elevate the complexity from being 
Kalmar elementary to $\T_0$). Again this result is optimal. 
So far the elimination method has not been 
developed for systems based on $\T_0$-functionals and stronger ones as above. 
At the same time, the approach in the present 
paper does not seem to be fine enough 
to re-obtain the results based on the elimination method. 
\\[1mm] 
Independently of the motivation given so far, 
our explicit $\BW$-functional seems to be 
of interest in its own as it exhibits the computational content of $\BW$. 
The functional dramatically simplifies if we switch to a majorizing functional 
in the sense of W.A. Howard (and hence to a solution of the monotone 
functional interpretation of $\BW$). In particular, the use of 
Howard's `binary' bar recursion then disappears altogether. 
\\[1mm] In \cite{Kohlenbach06}, the second author has argued 
that the solutions provided by monotone functional interpretation 
of principles $P$ directly 
correspond to the `finitary' versions of $P$ as discussed in Tao's 
program of `finitary analysis' (see \cite{Tao07}). Following \cite{Tao07},  
the discussion in \cite{Kohlenbach06} focuses on the monotone 
convergence principle PCM and the infinitary pigeonhole principle $\IPP$:
\[ \IPP\quad:\quad\forall n\in\NN \,\forall f:\NN\mapsto\{0, \ldots, n\}
\,\exists i\leq n \forall k\in\NN\,\exists m\geq k\ \ (f(m)=i). \] 
Already for $\IPP$, it is nontrivial to arrive at a `correct' 
finitization (see \cite{GasparKohlenbach} for a thorough discussion). 
However, $\IPP$ is nothing else as the special case of $\BW$ for the discrete 
spaces $\{ 0,1,2,\ldots,n\}.$ Hence to carry out the explicit solution 
of the (monotone) functional interpretation is a step further towards  
investigating the role of functional interpretations in connection with 
the program of finitizing analytical principles. 
\\[1mm] In this paper we also treat the Bolzano-Weierstra\ss{} principle for 
the compact (w.r.t. the product metric) metric space $\Pi_{i\in\NN} 
[-k_i,k_i]$ (for sequences $(k_i)$ in $\RR_+$) whose functional 
interpretation has the same complexity as the one for the case $[0,1].$ 
This is of relevance for the logical analysis of proofs that use the weak 
compactness of closed, bounded convex sets in Hilbert spaces which can 
be reduced to the sequential compactness of $\Pi_{i\in\NN} 
[-k_i,k_i]$ (see \cite{KohlenbachMints,Kohlenbach(weakcompactness)}). 
Though certain uses of weak compactness in strong convergence results 
can be eliminated in the course of the logical analysis one in general 
will have to deal with quantitative forms of weak compactness as provided 
by functional interpretation (see \cite{Kohlenbach(Browder)} as well as 
\cite{Kohlenbach(weakcompactness)} for discussions of both points).  
\\[1mm] For simplicity, let us come back to the case of $[0,1]$ and the 
Bolzano-Weierstra\ss{} theorem in the form stating that every sequence 
$(x_n)$ of rational numbers 
in $[0,1]$ has a cluster point. In order to obtain a solution of the 
functional interpretation of optimal complexity one has to start with an 
appropriate proof of this statement: one standard way is to select one of 
the subintervals $[0,\frac{1}{2}],[\frac{1}{2},1]$ that contains infinitely 
many elements of the sequence $(x_n)$ (by $\IPP$ at least one of the two 
intervals has this property) and then to continue with that interval. In 
this way one gets a nested sequence $I_0\supset I_1\supset I_2,\ldots$ 
of intervals $I_k$ of length $2^{-k}$ that converges to a cluster point. 
In order to decide whether an interval $I_k$ contains infinitely 
many elements of $(x_n)$ one needs $\Pi^0_2$-comprehension since 
\[ \forall m\exists n\ge m\, (x_n\in I_k) \in\Pi^0_2. \]
However, in order to get the existence of just {\bf some} 
sequence $I_0\supset I_1\supset I_2,\ldots$ 
of intervals $I_n$ with the above property (rather than deciding 
this property which would be necessary only for finding -- say -- 
a left-most sequence, i.e. for constructing the limit inferior of $(x_n)$ 
which indeed is of strictly greater complexity, see \cite{Kohlenbach00}) 
one can use K\"onig's lemma for 
$0/1$-trees. Note though, that 
this is not a use of what is called weak K\"onig's 
lemma ($\WKL$) in reverse mathematics (see \cite{Simpson99}) since the tree 
predicate is not quantifier-free but is $\Pi^0_2$ (and so, in fact, is 
an instance of what we call $\Pi^0_2$-$\WKL$). Nevertheless, using 
a single instance of $\Sigma^0_1$-comprehension (short: $\Sigma^0_1$-$\CA$) 
one can reduce such a 
$\Pi^0_2$-formula to a $\Pi^0_1$-tree predicate (by absorbing the inner 
existential quantifier). Now $\WKL$ for $\Pi^0_1$-trees (i.e. 
$\Pi^0_1$-$\WKL$) 
can easily be reduced to 
the usual $\WKL$. In this way the use of $\Pi^0_2$-comprehension is replaced 
by a use of $\Sigma^0_1$-$\CA$ plus the use of $\WKL$, where the latter is 
known not to contribute to the complexity of extractable bounds. In fact, 
rather than first reducing $\Pi^0_2$-$\WKL$ using $\Sigma^0_1$-$\CA$ to 
$\Pi^0_1$-$\WKL$ and subsequently to $\WKL$, we work directly with 
$\Sigma^0_1$-$\WKL$ 
and reduce this via $\Sigma^0_1$-$\CA$ in one step to $\WKL$. The functional 
interpretation of $\BW$, therefore, essentially boils down to solving the 
functional interpretation of $\Sigma^0_1$-$\WKL$.  

\subsection*{Notation and Common Expressions}
%
By "$\equiv$" we refer to syntactic identity.
We will write {$\PiL$} and {$\SiL$} for the
purely universal arithmetic formulas, i.e. $\forall n^0\ \phi_{_\QF}(n)$, and
the purely existential arithmetic formulas, i.e. 
$\exists n^0\ \phi_{_\QF}(n)$, where
in general $\phi_{_\QF}$ denotes a quantifier-free formula, 
which may contain parameters of arbitrary type. \\
For the encoding of a given finite sequence $s$ of natural numbers we 
write {$\lh(s)$} for the length of $s$ and
denote by {$[s]$}\footnote{In general we try to adopt the notation of \cite{Kohlenbach06} if possible. Therefor
the usual notation $\hat s$ is used for other purposes -- see Definition \ref{d:hatReal}. } the type one function defined by
\[
  [s](i^0)\ :=_0\ \begin{cases}s(i)&\text{if}\ \ 
i<_0\lh(s),\\0&\text{else.}\end{cases} \]
For a type one function $f$ and a natural number $n$ we define the 
corresponding encoding of
the finite sequence {$\overline{f}n$} of length $n$
as follows:
\[
  \overline{f}n\ :=\ \big\langle f(0), f(1), \cdots, f(n-1) \big\rangle
\text{.} \]
Given two finite sequences $s$ and $t$ we write $s*t$ for the concatenation 
of $s$ and $t$. 
We write shortly $s*\langle 0\rangle$ and $s*\langle 1\rangle$ as
$s*0$ and $s*1$. Following the notation of Avigad and Feferman \cite{AF98}, we 
use 
`$s\subseteq t$' to denote $t$ is an extension of $s$ (i.e. the sequence 
$t$ has the sequence $s$ as an initial segment). 
We denote the empty sequence by {$\emptyset$}.\\
For finite tuples of variables (not necessarily of the same type) 
$x_1,x_2,\ldots,x_k$ we
write $\tup x$. By $\tup x^{\tup \rho}$ we 
mean $x_1^{\rho_1},x_2^{\rho_2},\ldots,x_k^{\rho_k}$.\\
In most cases we will use the Greek letters $\phi$,$\psi$, $\chi$ to 
denote formulas, the lower case Latin letters $f$,$g$,$h$
for functions, the letters $a$,$b$,$i$,$j$,$\ldots$ for 
natural numbers and encodings, and the capitals $A$, $B$,~$\ldots$ for 
functionals.\\
We denote $\lambda n^0.1^0$, $\lambda f^1.1^0$, $\ldots$ by 
$\one\equiv\one^1$, $\one^2$,~$\ldots$ and we use bold numbers to denote the 
type level of a term, e.g. we write $t\tp 1$ for $t^{1(0)}$. We use this 
superscript as a shortcut for a specific type having the given type level. 
So, by $\forall X\tp2$ we mean for all $X$ of an appropriate type,
e.g. $2(1(0))$, not for all $X$ which are of any type with level $2$.\\ 
We distinguish between full schemas and principles and their restrictions
to a concrete instance. For a given principle $\phi$ we write  $\phi(f)$ 
for the concrete instance of  $\phi $ applied to~$f$. We will define the 
principles in this form in most cases. It is also the preferable form to be 
analyzed, since most proofs in mathematics use only concrete instances of 
the common principles. We will write shortly $\phi$ for the full 
second-order closure $\forall f \phi(f)$ of $\varphi(f).$\\



\section{ Representations and Interpretations } \label{s:si}
\subsection{Fragments and Extensions of $\wepa$}

In the following, $\wepa$ denotes weakly extensional (see below) Peano 
arithmetic extended to the language of functionals for all finite types 
(e.g. see \cite{Troelstra73} or \cite{Kohlenbach06}). The intuitionistic 
variant of this system (weakly extensional Heyting arithmetic in all 
finite types) is denoted by $\weha.$
%
Furthermore, we define the extensional systems:
\begin{dfn} 
By replacing the weak extensionality rule
\[ \QFm\ER\quad:\quad
           \begin{prooftree}
             \phi_{{}_{\QF}}\rightarrow s=_\rho t
             \justifies
             \phi_{{}_{\QF}}\rightarrow r[s]=_\tau r[t]
           \end{prooftree}\text{,}\] 
 by the {axioms of higher type extensionality}:
\[ 
{\usftext{E}}_\rho\quad:\quad 
 \forall z^{\rho},x_1^{\rho_1},y_1^{\rho_1},\ldots,x_k^{\rho_k},y_k^{\rho_k}
   \ \Big( 
\bigwedge_{i=1}^{k}
( x_i=_{\rho_i} y_i ) \rightarrow z\tup x=_0 z \tup y \Big)
\text{,} 
\]
in the systems $\weha$ and $\wepa$, where $\rho=0\rho_k\ldots\rho_1$, 
we obtain the systems $\eha$ and $\epa$. Here $s=_{\rho} t$ is 
defined extensionally, 
i.e. as 
\[ \forall x^{\rho_1}_1,\ldots,x^{\rho_k}_k \, (s\underline{x}=_0 
t\underline{x}). \]
\end{dfn}

On the other hand, sometimes it is sufficient to have weaker systems. 
This leads us to the definition of the following fragments:
\begin{dfn}
For all four systems $\weha$, $\wepa$, $\eha$ and $\epa$ let
correspondingly $\hrrweha$, $\hrrwepa$, $\hrreha$, $\hrrepa$ denote
the fragments where we only have the recursor $R_0$ for type-0-recursion and
the induction schema is restricted to the schema of
quantifier-free induction:
\[
\QF\m\IA\quad:\quad
             \left(\phi_{{}_{\QF}}(0)\wedge\forall n^0\big( \phi_{{}_{\QF}}(n)\rightarrow  \phi_{{}_{\QF}}(n+1)\big)\right)
             \rightarrow\forall n^0 \phi_{{}_{\QF}}(n)
\text{,}
\]
where $\phi_{{}_{\QF}}$ is a quantifier-free formula and may 
contain parameters of arbitrary types.
\end{dfn}


The ordered field of {\em rational numbers} within $\weha$ is represented by
codes $j(n,m)$ of pairs $(n,m)$ of natural numbers (in this paper let $j(n,m)$ define
the rational number $\frac{\frac{n}{2}}{m+1}$ if $n$ is even and 
$- \frac{\frac{n+1}{2}}{m+1}$ otherwise,
for $j$ we use the Cantor pairing function). 
The standard relations and operations
(like e.g. $+_\QQ$, $|\cdot|_\QQ$, $<_\QQ$) are defined in a natural way (see e.g. ~\cite{Kohlenbach06}).
By $\langle r\rangle$ we mean the smallest code of the rational number $r$.\\
The Archimedean ordered Field of {\em real numbers} within $\weha$
is represented by sequences of rational numbers with
a fixed rate of convergence $2^{-n}$. 
\begin{dfn}{\em Representation of a real number\\}\label{d:real}
A function $f : \NN \mapsto \NN$ such that
\[
  \forall n\ \big(|fn -_\QQ f(n+1)|_\QQ <_\QQ \langle 2^{-(n+1)}  \rangle\big)
\text{,}
\]
represents a real number.
\end{dfn}
We arrange for each function $f^1$ to code a unique real number in this way.
\begin{dfn} \label{d:hatReal}
Let $f\ :\ \NN\mapsto\NN$ be a function. Define $\widehat{\!f}$ by   
\[
\widehat{\!f}n := 
  \begin{cases}
    fn & \Tif\ \ \forall k<n\ \big(|fk -_\QQ f(k+1)|_\QQ <_\QQ \langle 
2^{-k-1}\rangle\big) \\
    fk & \Telse
  \end{cases}\text{,}
\]
where $k$ is the least number such that $k<n$ 
and  \[|fk-_\QQ f(k+1)|_\QQ \geq_\QQ \langle 2^{-k-1} \rangle\text{.}\]
The function $\widehat{\!f}$ defines a uniquely determined real number, so
we say also that $f$ defines a uniquely determined real number, namely the 
one represented 
by $\widehat{\!f}$.
\end{dfn}
The functional which maps $f$ to $\widehat{\!f}$ can be defined primitive 
recursively in $\weha$. I.e. we can reduce quantifiers ranging over
$\mathbb{R}$ to quantifiers ranging over type $1$ objects. 
The usual operations and relations on $\RR$ can be defined in $\weha$ in
a an intuitive way (see e.g. ~\cite{Kohlenbach06}).\\ 
For the embedding $\QQ\embeded\RR$, we define for the coding $n=\langle 
r\rangle$ of a rational number $r$
its coding $n_{\mathbb{R}} : = \lambda k. n_\QQ$ for the real
number corresponding to $r$. However, usually we omit
the intermediate encoding to a rational number and write shortly $r_\RR$ 
instead of $(\langle r\rangle)_\RR$.
We do not introduce $\mathbb{R}$ as the set of equivalence classes of
representatives, but consider only the representatives themselves. The 
structure
\[              
 (\NN^\NN, +_\RR,\cdot_\RR, 0_{\mathbb{R}}, 1_{\mathbb{R}}, <_{\mathbb{R}})
\]
represents the Archimedean ordered field of real numbers $(\mathbb{R}, +, 
\cdot, 0, 1, <)$ in $\weha$. 

\subsection{G\"odel's Functional Interpretation}
In \cite{Goedel58}, K. G\"odel introduced his famous functional `Dialectica' 
interpretation for Heyting arithmetic HA and -- via some negative translation 
as a pre-processing step -- also Peano arithmetic PA in his 
quantifier-free system $\T$ of primitive recursive functionals 
of all finite types. This interpretation 
was extended to classical analysis (obtained from a finite type extension 
of PA by the addition of the schema of countable choice) by C.~Spector 
in \cite{Spector62} using his bar recursive functionals $\T+\BR$, where for 
the treatment of countable choice for numbers and arithmetical formulas 
(sufficient to derive arithmetical comprehension) 
bar recursion of lowest types is enough (see e.g. 
\cite{Kohlenbach06}). In this paper, it is always the combination of negative 
translation and functional interpretation that is used. This combination becomes 
particularly convenient to formulate if one uses a negative translation due 
to Krivine, as it then coincides with the so-called Shoenfield variant 
\cite{Shoenfield67} (for the fragment $\{ \forall, \vee,\neg\}$) as 
was shown in \cite{Streicher/Kohlenbach}, where this interpretation is 
given for the full language (i.e. $\{ \forall, \exists, \rightarrow,\vee,
\wedge,\neg \}$ with $A\leftrightarrow B:\equiv 
(A\to B) \wedge (B\to A)$). 
\\[1mm]
We now give the definition of this Shoenfield interpretation:   
\begin{dfn}[\cite{Shoenfield67,Streicher/Kohlenbach}] \label{d:FI}
To each formula $\phi(\tup{a})$ in ${\mathcal L}(\wepa)$ with the tuple of
free variables $\tup{a}$ we associate its {Sh-interpretation}
$\phi^{Sh}(\tup{a})$, which is a formula of the form
\[
\phi^{Sh}(\tup{a}):\equiv\forall \tup{u}\,\exists \tup{x}\, 
\phi_{Sh}(\tup{u},\tup{x}, \tup{a})
\text{,}
\]
where $\phi_{Sh}(\tup{a})$ is a quantifier-free formula.
Each of $\tup{x}$ and $\tup{y}$ is a tuple of variables whose
types, as well
as the length of each tuple, depend only on the logical structure of $\phi$. 
We also write $\phi_{Sh}(\tup{x},\tup{y},\tup{a})$ for $\phi_{Sh}(\tup{a})$. 
If some variables
$\tup z$ of $\phi$ are exhibited, as $\phi(\tup z,\tup a)$, then we write
$\phi_{Sh}(\tup x, \tup y, \tup z,\tup{a})$ for $\phi_{Sh}(\tup{a})$.\\
We define the construction of $\phi^{Sh}$ inductively as follows 
(with $\tup{x}\,\tup{y}$ denoting $x_1\tup{y},\ldots,x_n\tup{y}$ for 
$\tup{x}=x_1,\ldots,x_n$). In the inductive steps we assume that 
\[
\phi^{Sh}(\tup{a}):\equiv\forall \tup{u}\,\exists \tup{x}\, 
\phi_{Sh}(\tup{u},\tup{x}, \tup{a}) \ 
\text{and} \ \psi^{Sh}(\tup{b}):\equiv\forall \tup{v}\,\exists \tup{y}\, 
\psi_{Sh}(\tup{v},\tup{y}, \tup{b})
\] are already defined. 
\begin{enumerate}
\item[(S1)] $\phi^{Sh}(\tup{a}) 
\equiv:\phi_{Sh}(\tup{a})$ for atomic 
$\phi(\tup{a}),$
\item[(S2)] $(\neg \phi)^{Sh} \equiv \forall \tup{f} \exists \tup{u} \, 
\neg \phi_{Sh}(\tup{u},\tup{f}\,\tup{u}),$
\item[(S3)] $(\phi \vee \psi)^{Sh} \equiv 
             \forall \tup{u},\tup{v} \exists \tup{x},\tup{y} \, 
\big(\phi_{Sh}(\tup{u},\tup{x}) \vee \psi_{Sh}(\tup{v},\tup{y})\big),$
\item[(S4)] $(\forall z \, \phi)^{Sh} \equiv \forall  z,\tup{u} 
\exists \tup{x} \, \phi_{Sh}(z,\tup{u},\tup{x}),$
\item[(S5)] 
$(\phi{\to} \psi)^{Sh} 
\equiv \forall \tup{f}, \tup{v} \exists \tup{u}, \tup{y} \, \big(\phi_{Sh}
(\tup{u},\tup{f}\, \tup{u}) \to \psi_{Sh}(\tup{v},\tup{y})\big),$
\item[(S6)] 
$(\exists z \,\phi)^{Sh} \equiv \forall \tup{U} \exists z, \tup{f} \, 
\phi_{Sh}(z,\tup{U} \,z\,\tup{f},\tup{f}(\tup{U}\,z\,\tup{f})),$
\item[(S7)] $(\phi \wedge \psi)^{Sh} \leftrightarrow 
\forall \tup{u},\tup{v} \exists \tup{x}, \tup{y} \,
\big( \phi_{Sh}(\tup{u},\tup{x}) \wedge \psi_{Sh}(\tup{v},\tup{y})\big).$ 
\end{enumerate}
\end{dfn}
\begin{remark} The official definition  of (S7) in \cite{Streicher/Kohlenbach} 
is slightly different from the one given above (where our version is called 
(S7$^*$)) 
but is intuitionistically equivalent to that.
\end{remark}
Sometimes, a partial (weaker) interpretation of the implication is sufficient
when the witnessing data from the premise are not needed
for further use of the Sh\nbd interpretation of a given formula. E.g. if the
premise can be proved directly. Typically, we would analyze in such cases 
an implication as:


\[ \forall \tup{f} \exists \tup{y} \, \big( \forall \tup{u}\,
\phi_{Sh}
(\tup{u},\tup{f}\,\tup{u}) \to \psi_{Sh}(\tup{v},\tup{y})\big).\]
\begin{remark} 
The Shoenfield version of the functional interpretation is often -- for 
obvious reasons -- called $\forall\exists$-form, whereas the Dialectica 
interpretation (and hence also the combination ND of some 
negative translation N with 
the Dialectica interpretation D) always is of the form $\exists\forall.$ 
If the Krivine negative translation is used (see 
\cite{Streicher/Kohlenbach}), 
the latter is nothing else but 
the result of a final application of the axiom schema of quantifier-free 
choice $\QFm\AC$ 
to the Shoenfield interpretation. One should stress though that this passage 
from the $\forall\exists$-form to the $\exists\forall$-version which  
also is implicitly present in the soundness theorem of the Shoenfield 
interpretation (stating the extractability of suitable terms realizing 
the $\forall\exists$-form) is necessary for the interpretation to be sound 
for the modus ponens rule.  
\end{remark}


\newcommand{\typeOfXZ}{0(10)(1)}

\subsection{{Bar Recursion}} \label{ss:BR}
We give the definition in the form presented e.g. in 
\cite{Luckhardt73} or \cite{Kohlenbach06}. Implicitly we assume that tuples 
of variables are contracted to single variables. 
Alternatively, one could use a 
simultaneous form of bar recursion (see \cite{Kohlenbach06}).
\begin{dfn}
\label{d:BR}
The {bar recursor} $\B_{\rho,\tau}$ is defined by:
\[ (\BR_{\rho,\tau}) \ :
\B_{\rho,\tau}yzunx:=_\tau\begin{cases}
   zn(\overline{x,n})&\text{if}\ y(\overline{x,n})<_0n\\
   u\big(\lambda D^\rho.\B_{\rho,\tau}yzu(n+1)(\overline{x,n}*D)\big)n
(\overline{x,n})&\text{otherwise,}
   \end{cases}\]
where 
$
(\overline{x,n})(k^0)=_\rho\begin{cases}
  x(k)&\text{if}\ k<_0 n\\
  0^\rho&\text{otherwise}
  \end{cases}
$
 and 
$
(\overline{x,n}*D)(k^0)=_\rho\begin{cases}
  x(k)&\text{if}\ k<_0 n\\
  D&\text{if}\ k=_0 n\\
  0^\rho&\text{otherwise.}
  \end{cases}
$
\end{dfn}
%
\begin{remark}
  Note that for $\rho=0,$ $\overline{f,n}$ is an object of type $1$ 
and so is not the same as $\bar fn$ which has 
type $0.$ 
\end{remark}
%
The form of bar recursion we actually need is the following 
special case of ($\BR$) that  Spector presented in \cite{Spector62}
as an operator $\bPhi$ such that
\[  \bPhi_\rho yunxm:=_{\rho}\left\{ \begin{array}{l} 
        xm,\: \mathrm{if}\ m<_0n\\
        0^\rho,\: \mathrm{if}\ m\geq_0n \wedge y(\overline{x,n})<n\\
        \bPhi_\rho yun' (\overline{x,n}*D_0)m,\: \mathrm{otherwise,}
\end{array}\right. \] 
where \[ D_0=_\rho un(\lambda D^\rho.\bPhi_\rho 
yun'(\overline{x,n}*D)). \]
As in this paper we deal only with arithmetical comprehension
over numbers we don't need bar recursion for all types.
Indeed, $\Phi_0$ (i.e. $\Phi_\rho$ with $\rho=0$) and hence $\B_{0,1}$ (from which $\Phi_0$ can be 
defined via type-0 primitive recursion, see \cite{Safarik08} or 
\cite{Kohlenbach06}) is sufficient.
Furthermore, to be able to properly analyze the complexity of the witnessing
functionals in later sections, we introduce Howard's schemas of restricted 
bar recursion as given in \cite{Howard81}.
\begin{dfn}\label{d:rBR}
The {restricted bar recursor} for {Scheme 1}, $\rB_\one$, is 
defined by:
\begin{align*}
\rB_\one&y^2z^{(2)0}un^0x^1:=_0\\
   &\begin{cases}
   zn(\overline{x,n})&\text{if}\ y(\overline{x,n})<_0n\\
   u\big(\rB_\one yzu(n\!+\!1)(\overline{x,n}\!*\!0)\big)
    \big(\rB_\one yzu(n\!+\!1)(\overline{x,n}\!*\!1)\big)
    n(\overline{x,n})&\text{otherwise.}
   \end{cases}
\end{align*}
\end{dfn}
\begin{remark}Note that, $\rB_\one$ is just a special
 form of $\B_{0,0}$.
\end{remark}

\subsection{Majorizability}

The following important structural property of the closed terms of 
all systems
used in this papers is due to W.A. Howard \cite{Howard73} and (with a 
modification incorporated below) 
M. Bezem \cite{Bezem85}:
\begin{dfn}
\label{d:maj}
The relation {$x^*\ \maj_\rho x$} ($x^*$ 
{strongly majorizes} $x$) between functionals of
type $\rho$ is defined by induction on $\rho$:
\setcounter{equation}{0}
\[ \begin{array}{l} 
 x^*\maj_0 x :\equiv x^*\geq_0x\text{,} \\  
 x^*\maj_{\tau\rho} x :\equiv \forall y^*,y(y^*\maj_\rho y\rightarrow 
        x^*y^*\maj_\tau x^*y,xy)\text{.} \end{array} \]
Naturally, this definition extends to tuples in the expected way.
\end{dfn}

Moreover, Howard and Bezem showed in \cite{Howard73,Bezem85} the following:
\begin{thm}[\cite{Howard73,Bezem85}, see also \cite{Kohlenbach06}] 
\rm For each closed term $t^\rho$ of $\weha+\BR$ one can 
construct a closed term
$t^*$ in $\weha+\BR$ of the same type, such that:
\[ \weha+\ \BR + \BI \,\proves t^* \maj_\rho t, \] 
where $\BI$ is a suitable principle of bar induction (classically: dependent 
choice). \\ $t^*$ only contains recursors or bar recursors of a certain 
type if already $t$ contains these constants. In particular, $\B_{0,1}$ 
(and hence $\bPhi_0$) can be majorized by a term using in addition to 
simple primitive recursive constructions only $\B_{0,1}.$
\end{thm}
Using monotone functional interpretation 
(introduced by the 2nd author in \cite{Kohlenbach(A)}, see also 
\cite{Kohlenbach06}),
one can extract terms which majorize some functionals realizing
the usual functional interpretation directly. In this paper we give
the $Sh$-interpretation first and then majorize the realizers, still 
calling this `monotone $Sh$-interpretation�:
\begin{dfn}\label{d:MSI}
Suppose we have the following $Sh$-interpretation of a formula $\phi$:
\[
\forall \tup{u}\,\exists \tup{x}\, 
\phi_{Sh}(\tup{u},\tup{x}, \tup{a})
\text{,}
\] 
where $\tup{a}$ comprises all the free variables of $\phi.$ \\ 
Then we say that the terms $\tup t^*$ satisfy the
monotone $Sh$-interpretation if
\[
\exists \tup{X}\,\big( 
\tup t^* \maj \tup X\ \wedge\ 
\forall \tup{u}\,\forall \tup{a}\,\phi_{Sh}(\tup{u},\tup{X}\;\tup{a}\;
\tup{u}, \tup{a})
\big)
\text{.}
\]
\end{dfn}
\begin{remark} While the type structures of all strongly majorizable 
functionals $\Maj$ (\cite{Bezem85}) and of all continuous 
functionals $\Cont$ (\cite{Scarpellini71}) are models of $\epa$ with bar 
recursion (see Definition~\ref{d:BR}) the full set theoretic 
type structure $\Set$ is not. However, the first two models start to differ 
only from type $2$ on from the third model, where we still have: 
$\ContO_2\subset\MajO_2\subset\SetO_2$. So, if we use bar recursion to define 
a functional $F$ of type level $2$, we know that it is a well defined 
functional in $\Cont$ and in $\Maj$ and defines a 
total (computable) functional: $\NN^\NN\mapsto\NN$ (see 
\cite{Kohlenbach06} for details on this).
\end{remark}




\subsection {Arithmetical Comprehension}\label{ss:CA}
{\samepage
The {Schema of Comprehension} is known in several forms 
(see e.g. \cite{Troelstra73}).
For us, the following very restricted form is sufficient:
\begin{dfn}[\em Arithmetical Comprehension over numbers for purely 
existential formulas]
\label{d:CA}
\[ \SiLm\CA\quad:\quad \forall f^{1(0)}\,\underbrace{\exists g^1\forall x^0
  \big( \ \big(\exists y^0\ f(x,y)=_00\big) \leftrightarrow g x=_00\ \big)}_{
\equiv : \SiLm\CA(f) } \text{.}\]   
We define $\PiLm\CA$ and $\PiLm\CA(f)$ for purely universal 
formulas analogously.
\end{dfn}
} 
\begin{remark} In $\hrrwepa+\SiLm\CA,$ we can derive any instance of arithmetical 
comprehension 
\[ \CA^0_{ar}: \ \exists f\forall x^0 \, (f(x)=_00 \leftrightarrow A(x)), \] 
where $A$ only contains quantifiers over variables of type $0,$ 
by iterated application of $\SiLm\CA.$ However, this is only the case 
for the full second-order closure 
$\SiLm\CA$ of $\SiLm\CA(f)$ and not for individual instances $\SiLm\CA(f)$.
\end{remark}

As mentioned already above, the Shoenfield interpretation
of $\hrrwepa+\QFm\AC+\SiLm\CA$ can be carried out in $\T_0+\BR_{0,1}$ 
(see \cite{Kohlenbach99} or \cite{Kohlenbach06}).
\\[1mm] 
Below we will need the explicit solution of the Shoenfield interpretation of 
$\SiLm\CA$ which we compute now: \\[1mm] Using 
$\QF\m\AC^{0,0}$ (already in $\hrrwepa$) 
$\SiLm\CA(f)$ is equivalent to the following modified form 
\[
\SiLm\CAhut(f)\quad:\quad\exists g^1 \forall x^0,z^0 \big(fx(gx)=_00 \vee 
fxz\neq_00\big)\text{.}
\]
\begin{lemma}[$Sh$-interpretation of $\forall 
f\SiLm\CAhut(f),$ \cite{Safarik08}] 
\label{p:NDPiLCAhut} 
\[
\forall f\exists g^1 \forall x^0,z^0\ \big(fx(gx)=_00 \vee fxz\neq_00\big)
\text{,}
\]
is $Sh$-interpreted as follows:
\[
  \forall f,X^2,Z^2\ \ \Big(f\big(Xt_g,t_g(Xt_g)\big)=_00 \vee f(Xt_g,Zt_g)
\neq_00\Big)\text{,}
\]
where $t_g:=_1\xTWf{X}{Z}{f}\text{ and }u_{Z,f}n^0v^{1(0)}:=_0\uWf{Z}{f}$.
\end{lemma}
\begin{proof}
Formally we have to prove
\setcounter{equation}{0}
\begin{align}
\forall f, X, Z, \big( f(X(t_g\args{fXZ}), t_g\args{fXZ}(X(t_g\args{fXZ}))) = 0 \vee 
											 f(X(t_g\args{fXZ}), Z(t_g\args{fXZ})) \neq 0  \big).\label{e:ca-ca}
\end{align}
Define a functional $G$ as follows:
\[
G_f:=\lambda x,y,z . \begin{cases}0&\Tif f(x,z)\neq 0, \\ f(x,y) & \Telse,\end{cases}
\]
to obtain
\begin{align}
G_f(x,y,z)=0\quad \leftrightarrow \quad f(x,y)=0 \vee f(x,z)\neq0. \label{e:ca-G}
\end{align}
By the functional interpretation of $\DNS$ as given by Spector in \cite{Spector62} 
(here we use the notation of \cite{Kohlenbach06}, moreover we need only a simplified version for $\Sigma^0_2$ formulas)
applied to the function $G_f$ (instead of $f$) we obtain for all $f^{1(0)}$, $A^{2(0)}$, $Y^2$, $V^2$:
\begin{align}  %% A, T=Y, W=V -- in dipl
\big( G_f(t_x\args{AYV}, &A(t_x\args{AYV})(t_B\args{AYV}),
																		t_B(A\args{(t_x\args{AYV})(t_B\args{AYV})}))=0\rightarrow \label{e:ca-dns1}\\
																 &G_f(Y(t_U\args{AYV}),t_U\args{AYV}(Y(t_U\args{AYV})),V(t_U\args{AYV}))=0 \big), \label{e:ca-dns2}
\end{align}
for suitable terms $t_x$, $t_B$ and $t_U$ as defined in~\cite{Kohlenbach06} in Theorem~11.6 
(note that these terms do not dependent on $f$).
Next we set \[ A:=t_Af:=\lambda x,B . \begin{cases} 1&\Tif\ f(x,B1)\neq0\\ B1&\Telse \end{cases} \]
to ensure that \eqref{e:ca-dns1} holds. Then we have that \eqref{e:ca-dns2} 
holds for this particular $A$. By \eqref{e:ca-G} this concludes the proof of \eqref{e:ca-ca}, 
since for $A=t_Af$, $Y=X$ and $V=Z$ we have also that $t_UAYV=t_gfXZ$. 
\end{proof}
We use this solution to give the 
$Sh$-interpretation
of $\SiLm\CA$.

\begin{thm}[$Sh$-interpretation of $\SiLm\CA$]\label{c:NDSiLCA}
  The schema of arithmetical comprehension over numbers for purely existential
  formulas (for a given function $f^{1(0)}$)
  \[
  \exists h^1\forall x^0 (hx=_00 \leftrightarrow \exists z\ fxz=_00)\text{,}
  \]
is $Sh$-interpreted as follows (using the clause (S7$^*$) for the 
conjunction hidden in `$\leftrightarrow$'):
\begin{align*}
  \forall X^{\typeOfXZ},Z^{\typeOfXZ}\ \ [(\ 
t_h&\args{XZ(X\args{(t_h\args{XZ})(t_z\args{XZ})})}=_00
        \leftarrow \\
        & \quad f(X \args{(t_h\args{XZ})(t_z\args{XZ})}, 
                  Z \args{(t_h\args{XZ})(t_z\args{XZ})} ) =_00)\ \ \wedge\\
 (t_h&\args{XZ(X\args{(t_h\args{XZ})(t_z\args{XZ})})}=_00 
         \rightarrow \\
     &\quad f (
           X\args{(t_h\args{XZ})(t_z\args{XZ})}, 
           t_{z}\args{XZ (X\args{(t_h\args{XZ})(t_z\args{XZ})})
                      (Z\args{(t_h\args{XZ})(t_z\args{XZ})}) } 
      ) =_00\ )]
\text{.}
\end{align*}
The witnessing terms are:
\begin{align*}
t_z&:=\lambda X^{\typeOfXZ},Z^{\typeOfXZ},a^0,b^0\ .\ t_g\args{(t_f\args{X})
(t_f\args{Z})a}\text{,}\\
t_h&:=\lambda X^{\typeOfXZ},Z^{\typeOfXZ},n^0\ .\ 
 \overleftrightarrow{f(n,t_g\args{(t_f\args X)(t_f\args Z)n})}
\text{,}
\end{align*}
where
$
t_f:=\lambda X^{\typeOfXZ},g^1\ .\ 
  X \args{ (\lambda n^0.\overleftrightarrow{f(n,g\args{n})})
(\lambda a^0,b^0 . g\args{a}) }
\text{ and }
\overleftrightarrow{n^0}:=_0\begin{cases}0&\Tif\quad n=_00\\1&\Telse\end{cases}
\text{.}
$\\
The term $t_g$ corresponds to the term
defined in proposition \ref{p:NDPiLCAhut}. The only difference is that 
we give the two type $2$ arguments of $t_g$ explicitly, i.e.,
$t_g$ stands only for the term $t_g$ of type level $3$ and not for 
the type $1$ term $t_gXZ$ as above.
\end{thm}
\begin{proof}
By lemma \ref{p:NDPiLCAhut} we have that:
\[
  \forall X^2,Z^2\ \ (\ 
      f(X\args{(t_g\args{XZ})}, t_g\args{XZ (X \args{(t_gXZ)})})=_00 \vee 
      f(X\args{(t_g\args{XZ})}, Z\args{(t_g\args{XZ})} )\neq_00\ )
\text{.}\tag{+}
\]
Given any $X^{\typeOfXZ}_0$, $Z^{\typeOfXZ}_0$ set $X^2:=t_f\args{X_0}$ and 
$Z^2:=t_f\args{Z_0}$ to
obtain:
\[ (*) \ X(t_g\args{XZ})=t_f\args{X_0}(t_g\args{XZ})=
t_f\args{X_0}(t_g(t_f\args{X_0},t_f\args{Z_0}))=X_0(t_h\args{X_0Z_0})
(t_z\args{X_0Z_0}) \] 
and - analogously - 
\[ (**)\ Z(t_g\args{XZ})=Z_0(t_h\args{X_0Z_0})
(t_z\args{X_0Z_0}). \] 
\begin{itemize}
\item Suppose we have $t_h\args{X_0Z_0}(X_0(t_h\args{X_0Z_0})
(t_z\args{X_0Z_0}))=_00$. It 
follows by $(*)$ that \[t_h\args{X_0Z_0}(X(t_g\args{XZ}))=_00\] and by 
definition of $t_h$ that
\[f(X(t_g\args{XZ}), t_g(t_f\args{X_0})(t_f\args{Z_0})(X(t_g\args{XZ})))
= 0\text{.}\]
By $(*)$ and definition of $X$ and $Z$ we get 
\begin{align*}
&f(X_0(t_h\args{X_0Z_0})(t_z\args{X_0Z_0}),t_g(t_f\args{X_0})
(t_f\args{Z_0})(X_0(t_h\args{X_0Z_0})(t_z\args{X_0Z_0})))=\\
&f( X_0(t_h\args{X_0Z_0})(t_z\args{X_0Z_0}) , 
t_z\args{X_0Z_0}(X_0(t_h\args{X_0Z_0})(t_z\args{X_0Z_0}))
(Z_0(t_h\args{X_0Z_0})(t_z\args{X_0Z_0}))) ) = 0
\text{.}
\end{align*}
\item On the other hand, let 
$f( X_0(t_h\args{X_0Z_0})(t_z\args{X_0Z_0}) , 
Z_0(t_h\args{X_0Z_0})(t_z\args{X_0Z_0}) ) = 0$.
By $(*),(**)$ this yields \\
$f( X(t_g\args{X_0Z_0}),  Z(t_g\args{X_0Y_0})) = 0,$ which 
implies
$f(X(t_g\args{XZ}),t_g\args{XZ}(X(t_g\args{XZ}))) = 0$ by (+). Using $(*)$ and
the definition of $X$, $Z$, and $t_h$ we obtain
\begin{multline*}
 f(X_0(t_h\args{X_0Z_0})(t_z\args{X_0Z_0})  , 
       t_g(t_f\args{X_0})(t_f\args{Z_0})(X_0(t_h\args{X_0Z_0})
(t_z\args{X_0Z_0}))) =  \\
   t_h\args{X_0Z_0}(X_0(t_h\args{X_0Z_0})(t_z\args{X_0Z_0}))) = 0
\text{.}
\end{multline*}
\end{itemize}
\end{proof}
%\comm{Add the MD-int. of $\SiLm\CA$?}

\section {Weak K\"onig's Lemma}\label{s:wkl}
K\"onig's Lemma as well as its weakening called 
Weak K\"onig's Lemma ($\WKL$) are well known
principles. For general context and definitions we refer e.g. 
to \cite{Troelstra74}, \cite{Simpson99}, or \cite{Kohlenbach06}. 
Whereas \cite{Simpson99} uses a language with set variables, 
both \cite{Troelstra74} and \cite{Kohlenbach06} use a formulation 
with function variables that is more convenient in the context of 
functional interpretation.
\begin{dfn}[{$\WKL(\phi)$}] \label{l:WKL-Feferman}
For a given $\phi,$ $\WKL(\phi)$ is the following statement: 
Every infinite $0/1$-tree given by the decision criteria $\phi$
 has an infinite path,
\[
\WKL(\phi)\quad:\quad  \BTree(\phi) \wedge \forall k \UnBounded(\phi,k) 
     \rightarrow \exists b\Big(\BFunc(b)\wedge\forall k\ 
\phi\big(\bar{b}(k)\big)\Big)\text{,}
\]where
$\BFunc(b):\equiv\forall n^0\big(b(n)=_00 \vee b(n)=_01\big)$, \\
$\BTree(\phi):\equiv\forall s \big(\ 
   \phi(s) 
      \rightarrow
 \ s\in\{0,1\}^{<\omega}
   \wedge 
   \forall t\subseteq s\ \phi(t)\ \big)$, 
$\UnBounded(\phi,k^0):\equiv\quad 
  \exists s\in\{0,1\}^{k}\ \phi(s)$.\footnote {
We encode finite binary sequences as natural numbers. A natural number $n$ 
encodes the binary sequence given by all but the first digit of the
binary representation of $n+1$ (i.e. $0=\langle\rangle$, $1=\langle0\rangle$, 
$2=\langle1\rangle$, $3=\langle0,0\rangle$, $\ldots$).
Note that for any $k\in\NN$, any $s\in\{0,1\}^{k}$ is encoded as a natural 
number $n$ for which the following inequalities hold: $2^k\leq n+1<2^{k+1}$.}\\
Furthermore, we define the schema $\Pi_n^0\m\WKL$, as the union of 
$\WKL(\phi)$, where $\phi$ is a $\Pi_n^0$ formula. Also, we write 
$\Pi_n^0\m\WKL(\phi)$ to indicate that we mean the concrete instance 
$\WKL(\phi)$ and that $\phi$ is a $\Pi_n^0$ formula.
(Analogously for $\Sigma^0_n$.)
\end{dfn}
Note that, for every fixed $n \in \NN$, we can always reformulate
the schema $\Pi^0_n\m\WKL$ as a single $2^{nd}$-order axiom. 
We will use this fact implicitly.
However, in the special case for quantifier-free $\phi$ 
we define explicitly:
\begin{dfn}[{$\WKL\equiv\forall f\WKL(f)$} see also~\cite{Troelstra74}] 
\label{d:WKLdelta}
Every infinite binary tree, given by the characteristic function $f$,
 has an infinite path:
\begin{align*}
\WKL(f)\quad:\quad\BTree_K(f) \wedge \forall k\exists x \big(\lh(x)=_0k 
\wedge f(x)=_0&0\big) 
     \rightarrow
 \exists b\leq_1\one^1\ \Big(\forall k\ f\big(\bar{b}(k)\big)=_00\Big)\text{,}
\end{align*}
$
\BTree_K(f^1):\equiv 
  \forall x,y \big(f(x*y)=_00\rightarrow fx=_00\big)\ \wedge\ 
  \forall x,n\big(f(x*\langle n\rangle)=_00\rightarrow n\leq_0 1\big)\text{.}
$
\end{dfn}
We mentioned earlier that the schema $\PiLm\WKL$ is equivalent
to $\WKL$. More precisely, \\ \noindent we have: 
$\hrrweha\proves\PiLm\WKL\leftrightarrow\WKL $.
A proof can be found e.g. in \cite{Safarik08} (see also \cite{Simpson99}).\\
Using a construction from  \cite{Kohlenbach(92)} (see also Proposition 9.18 
in \cite{Kohlenbach06}), we can rewrite $\WKL$ in a logically somewhat 
simpler form:
\begin{dfn}{{$\WKL_\Delta$}$\equiv\forall f\WKL_\Delta(f)$},
\label{d:frown_g} where
\[
\WKL_\Delta(f)\quad :\equiv\quad\exists b^1\forall k^0\ 
  \widefr{\Big(\!\text{\widefr{f}}\Big)_{\!g}} (\bar bk)=_00\text{, with}
\]
\begin{align*}
\widefr{f}n&:=\begin{cases}
  fn &\Tif \ fn\neq0\ \vee\ 
       \big(\forall k,l(k*l=n\rightarrow fk=0)\wedge \forall i<\lh(n)\ 
(n_i\leq 1)\big)\\
  1^0 &\Telse, \end{cases} \\
f_gn&:=\begin{cases}
  fn &\Tif\ f\big(g(\lh(n))\big)=0\wedge \lh\big(g(\lh(n))\big)=\lh(n)\\
  0^0 &\Telse, \end{cases} \\
gk:=g f k&:=\begin{cases}
\min n\leq\overline{\one^1}k\ (\lh(n)=k \wedge fn=0)&\text{if such an $n$ 
exists}\\
0^0&\Telse,
\end{cases}
\end{align*}
where for any given number theoretical function $f^1$, 
$\widefr{f}$ assigns a unique characteristic function of a $0/1$-tree\footnote{
If $f$ was such a characteristic function already, then it is not modified at all (i.e. we would have $\widefr{f}=f$).}
and $f_gn$ adds the full subtree if there is no path of length $n$\footnote{
Again, if $f$ defined an infinite tree already, then it is not modified at all (i.e. we would have $f_g=f$).}
(this may destroy the tree property of $f$ if present). 
The function $g$ simply looks for a path of length $n$ and retruns $0$ if none exists (otherwise the code of the path itself is returned).
\end{dfn}
We have: $\hrrweha\proves\WKL_\Delta\leftrightarrow\WKL$ 
(see \cite{Kohlenbach(92),Kohlenbach06} for the proof).\\
Howard proves in \cite{Howard81} that one can give the realizing 
functionals for the Sh-interpretation of $\WKL$ using only restricted bar 
recursion and $\T_0$. This proof is
discussed in great detail in \cite{Safarik08} and we use
it to obtain the Sh-interpretation of $\WKL(f)$:
\begin{thm}[{The Sh-interpretation of $\WKL_\Delta$}]\label{t:FIwkl}
The Weak K\"onig's lemma for binary trees
\[
  \forall f\exists b^1\forall k^0\ 
  \widefr{\Big(\!\text{\widefr{f}}\Big)_{\!g}} (\bar bk)=_00
\]
is, provably in $\weha+\rB_\one$, Sh-interpreted as follows:
\[ \forall f,A\ \exists b^1\  \widefr{\Big(\!\text{\widefr{f}}\Big)_{\!g}} 
\big(\overline{b}(Ab)\big)=0
\text{,}\]
where $b$ is realized by $b:=_1B^{^\WKL}Af$:
\begin{align*}
B^{^\WKL}(A,f)&:=\big[g\Big(\widefr{\Big(\!\text{\widefr{f}}\Big)_{\!gf}}
\Big) \Big(K^{^\WKL}
(A,\emptyset)\Big)\big] \text{,}  \\
K^{^\WKL}(A,x)&:=\KA{A}{x}  
\end{align*}
where $g$ is the same term as we used in the 
Definition \ref{d:WKLdelta}\footnote{We define $[\cdot]$ analogously 
for codes of sequences
as we did for sequences themselves.}.
Note that $K^{^\WKL}$ is definable by $\rB_\one$ (see also~\cite{Safarik08}).
\end{thm}
\subsection{Majorants for Howard's Solution} \label{ss:majFI}
For the monotone Shoenfield interpretation we need majorants of the terms 
realizing the Shoenfield interpretation. For the solution of $\WKL$ given above 
these majorants are rather trivial and no longer involve any restricted 
bar recursors. Note that the monotone interpretation suffices to calibrate 
the provably total functionals of level $\le 2$ (see the final section 
of this paper).
\\ The following proposition is easily verified:
\begin{prop}
The solution of the Sh-interpretation 
of $\WKL(f)$ is, provably in $\hrrweha+\rB_\one$, majorized as follows:
\setcounter{equation}{0}
\[
K^*:=_{1(0)(2)}\lambda A^2,x^0.A^2\one\ \maj_{1(0)(2)}\ K^{^\WKL}, \quad
B^*:=_{1(2)}\lambda A^2. \one^1 \maj_{1(2)}\ B^{^\WKL}
\text{.}
\]
\end{prop}


\section{Interpreting Bolzano-Weierstra{\ss}} \label{s:bw}
In this section we will use bar recursion
to interpret the Bolzano-Weierstra{\ss} theorem.
\begin{dfn}\label{d:BWfinal}
{The Bolzano-Weierstra{\ss} Principle}\\
Let $x$ be a sequence in $\PP:=\prod_{i\in\NN}[-k_i,k_i]$ for a known 
sequence $(k_i)_{i\in\NN}$ with $k_i$ in $\QQ^+$.
Let $d^\omega$ denote the standard product metric (as defined e.g. in 
\cite{Simpson99}):
\[
d^\omega (a,b):=_\RR\sum_{i=0}^\infty 
\frac{1}{2^i}
\frac{|a_i-b_i|}{1+|a_i-b_i|},\text{ for } a,b\in\PP.
\]
In the following we tacitly 
rely on our representation of real numbers by which 
sequences of real numbers are represented by objects $a^{1(0)}$ and sequences
of sequences of real numbers by objects $x^{1(0)(0)}$ (and 
each such object is a representative of a unique such sequence).
We define
\[
 \BW^\omega_\RR\quad:\quad
  \forall x^{1(0)(0)}\in\PP^\NN
      \underbrace{ 
         \exists a^{10}\in\PP\ \ \forall k^0\exists l^0\geq_0 k
           \ \ d^\omega(xl,a)\leq_\RR 2^{-k} 
      }_{\equiv:\BW^\omega_\RR(x^{1(0)(0)})}\text{,}
\]
where by $x\in\PP^\NN$ we mean a sequence of elements of $\PP$ -- i.e. a 
sequence ($x$) of sequences ($xn\in\PP,\ n\in\NN$) of real numbers in the
corresponding intervals ($(xn)i\in[-k_i,k_i], i\in\NN$).\footnote{
One can easily construct effective transformations which assign to any $x$ 
of the type $1(0)(0)$ (or $1(0)$) a unique $\tilde x$ in $\PP^\NN$ 
(or $\PP$), see theorem \ref{c:ND-bw}.}
\end{dfn}

\subsection{A Simple Proof of $\BW$ based on $\SiLm\WKL$} \label{ss:spuWKL}
To demonstrate the main idea of the proof we only treat $\BW^\omega_\RR$ for 
sequences of rational numbers in the unit interval $[0,1]$ -- denoted by 
$\BW_\QQ$ -- which obviously is implied by (and in fact equivalent to) 
\[
 {\BW_\QQ'}\quad:\quad
  \forall s^1  
   \underbrace{\exists a^1\forall k^0\exists l^0\geq_0 k
     \ |\widetilde{\hat a(k+1)}-_\QQ \widetilde {\ sl\ }|\leq_\QQ\langle2^{-(k+1)}
\rangle }_{ \equiv:\BW_\QQ'(s^1)}
\text{,}
\]
where $\tilde n := \min{}_\QQ\{\langle 1\rangle,\max{}_\QQ\{\langle 0\rangle,n\}\}$.
 Consider a tree representation of the unit interval $[0,1]$ which 
splits the unit interval at level $n$ into $2^n$ intervals of 
length $2^{-n}$.
Note that we can define each node via the path from the 
root to this interval. This path can be represented 
by a binary sequence $b$, where the $n$-th element defines which branch 
to take.\\
We define a predicate $I(b^0,n^0,m^0)$, which tells us, whether the 
rational number $r$ encoded by $m=\langle r \rangle$ belongs to an 
interval defined by such a finite binary sequence $b$
of length $\geq n$, i.e. in an interval of length $2^{-n}$ given by $b$: 
\begin{align*}
I(b^0,n^0,m^0)  :&\equiv\ \ \  
  \InIntM{b}{n}\\
&\Leftrightarrow\ \ \ n\leq \lh(b) \ \wedge\  
             \sum_{i=1}^{n}\frac{b_i}{2^i}\ \leq\ 
             r\ \leq\sum_{i=1}^{n}\frac{b_i}{2^i}+\frac{1}{2^n} 
\text{.}
\end{align*}
We know that for a given finite binary sequence $b$ and 
an infinite sequence of encodings of rational
numbers $s$, there is a function $f_{s}^{1(0)}$, primitive recursive in $b$ 
and $s$, such that:
\[
 f_{s}(b,k)=_00\ \ \leftrightarrow\ \ \Big(\ k > \lh(b)\wedge 
I\big(b,\lh(b),\widetilde{ sk }\big)\ \Big)
\text{.}\]
%
Now, by $\SiLm\CA(f_s)$ we obtain a function $g_s$, s.t. 
$
  \forall b^0\ \Big(\ g_sb=_00 \leftrightarrow \exists k^0\ 
\big(f_{s}(b,k)=_00\big)\ \Big)
\text{.}$
%
In other words we have for all $b^0$:
\[
  g_{s}(b)=_00\ \leftrightarrow \exists k^0>_0 \lh(b)\ \  
I\big(b,\lh(b),\widetilde{ sk }\big)
\text{.}
\tag{+}
\]

To show $\BTree(g_s)$, consider any finite binary sequence $b$:
\begin{align*}
  g_s(b)=_00\ \wedge\ x\subseteq b 
           &{\ \rightarrow\ }  \exists k^0>_0\lh(b)\ \ 
I\big( b,\lh(b),\widetilde{ sk }\big)\ \wedge\ 
                              \lh(x)\leq_0\lh(b)\ \wedge\ 
                              x\subseteq b \\
               &{\ \rightarrow\ } \exists k^0>_0\lh(x)\ \ 
I\big( b,\lh(x),\widetilde{ sk }\big)\ \wedge\ 
                              x\subseteq b\  \\
               &{\ \rightarrow\ } \exists k^0>_0\lh(x)\ \ 
I\big( x,\lh(x),\widetilde{ sk }\big)
               {\ \rightarrow\ }g_s(x)=_00
\text{.}
\end{align*}
To show 
\[
\forall k\exists x \big(\lh(x)=_0k \wedge g_s(x)=_00\big) \tag{++}
\]
just consider any given 
natural number $k$. By the definition of our tree, it splits the $[0,1]$ 
interval at any level, in particular on level $k$, completely. 
Therefore, we have: $I( b, \lh(b), \widetilde{ sk })$ and $ \lh(b)=_0k$ for 
a suitable $b.$  
As we started with arbitrary $k$, this implies (++).\\
Now, we can apply $\WKL(g_s)$ to get:
\[
\exists b^1 \big(\BFunc(b) \wedge \forall k\ g_s(\overline{b}k)=_00\big) 
\tag{$*$}
\text{.}
\]
Note that in $(*)$ (and from now on) $b^1$ is 
a binary function and $g_s$ takes
the encoding of the initial segment, $\langle b(0),\ldots,b(k-1)\rangle$, 
of this infinite sequence as its type $0$ argument.
Using (+) we can conclude that $(*)$ is equivalent to:
$
\exists b^1\leq\one\ \forall n\ \exists k> n\ \ I(\overline{b}n,n,
\widetilde{\ sk\ })
$. This means that $\BW_\QQ'(s)$ is satisfied by $\hat a$ where 
$a$ is defined as:\\
$
a(n^0):=_\QQ \left\langle \sum_{i=1}^{n+1} \frac{b(i-1)}{2^i}\ +
               \frac{1}{2^{n+2}}\right\rangle
$ provided that $\widehat{a}=_1 a.$ \\%
It, therefore, remains to show that $a$ represents a real 
number in the sense of Definition~\ref{d:real}.
W.l.o.g, at this point, we use $r$, $=$, $|\cdot|, \ldots$ directly
instead of the proper syntactic form $\langle r\rangle$, 
$=_\QQ$, $|\cdot|_\QQ,\ldots$ to achieve better readability.
To prove $a=_1\hat a\in\RR$, 
take any natural number $n$. We have:
\begin{align*}
|a(n)-a(n+1)|&=
\left|
\sum_{i=1}^{n+2} \frac{b(i-1)}{2^i}\ +
 \frac{1}{2^{n+3}}
-
\left(\sum_{i=1}^{n+1} \frac{b(i-1)}{2^i}\ +
 \frac{1}{2^{n+2}}\right)
\right|
\\
&=
\left|
\frac{b(n+1)}{2^{n+2}}+\frac{1}{2^{n+3}}
-
\frac{1}{2^{n+2}}
\right|
=
2^{-(n+3)} < 2^{-(n+1)}
\text{,}
\end{align*}
which concludes the proof.\\
%
%

%
The only relevant difference for the general case (i.e. for sequences in 
$\PP$) is the definition
of $f_s$. If we wanted each node $i$ at level $n$ to define a subspace 
$\PP^n_i\subseteq\PP$ such that
$\exists a^n_i\in\PP^n_i\forall b\in\PP^n_i d^\omega(a,b)\leq2^{-n}$, then 
the number of children couldn't be bounded by a constant. \\
It turns out that it is simpler to define a representation of $\PP$ by a 
binary tree, where any infinite path defines a single
element of $\PP$ and provide a function which returns the sufficient level 
to satisfy the condition above. \\
We define such a tree as follows. We start by splitting the first dimension 
into two halves, i.e. the two children represent the spaces 
$\PP^1_0=[-k_0,0]\times\prod_{j=1}^\infty[-k_j,k_j]$ and 
$\PP^1_1=[0,k_0]\times\prod_{j=1}^\infty[-k_j,k_j]$. Next two levels arise 
by first splitting the new intervals in the first dimension and then 
splitting the second dimension into two halves. At level $\frac{l(l+1)}{2}$ 
we create the next $l+1$ levels by splitting the new intervals for the first 
$l$ dimensions and by splitting the original interval for the 
$(l+1)^\text{th}$ dimension. We define formally:
\renewcommand{\l}{\ensuremath{\mathrm {lv}}}
\begin{dfn}\label{d:PPs}
Let $w$ be the primitive recursive function representing the number of times
we split dimension $d$ up to level $n$:
\[w(n^0,d^0):= {\max}{} _\NN\{l: l>0 \wedge 1+\frac{(d+l)(d+l-1)}{2}+d\leq n\ 
\vee\ l=0\}.\]
For an encoding of a finite binary sequence $b=\langle b_0,b_1,\ldots,b_{n-1} 
\rangle$ we define
\begin{itemize}
\item $D(b^0,d^0):=\langle b_{d(d+1)/2+d}, b_{(d+1)(d+2)/2+d}, \ldots, 
b_{(d+w(n,d)-1)(d+w(n,d))/2+d} \rangle$\\ (the code for the 
splittings of dimension $d$ corresponding to the node defined by~$b$\\ 
-- for $w(n,d)=0$ we define $D$ to be the empty sequence\\ 
-- using the Cantor pairing function we could also write $D(b^0,d^0):=\langle 
b_{\langle i,d\rangle}:i\in\{0,\ldots,w(n,d)-1\} \rangle$),
\item $\PP^b_d:=\begin{cases}
[-k_d+\sum_{i=0}^{w(n,d)}\frac{D(b,d)(i)}{2^i}k_d, k_d-\sum_{i=0}^{w(n,d)}
\frac{1-D(b,d)(i)}{2^i}k_d]&\Tif\ d\leq w(n,0),\\
[-k_d,k_d]&\Telse.
\end{cases}$\\ (the partition
of the dimension $d$ relevant at the node defined by $b$),
\item $\PP^b:=\prod_{i=0}^{\infty}\PP^b_i$\\ (the subspace
corresponding to the node defined by $b$).
\end{itemize}
For a sequence $s\subseteq\PP$ we define $f_s\tp1$
as follows:
\[
f_s(b,k):=\begin{cases}
0&\Tif\quad\ k>\lh(b)\ \wedge\ \forall i\!<\!\lh(b)\ b(i)\leq1\ \wedge \\
&\quad\ \ \ \bigwedge_{i=0}^{w(\lh(b),0)} (\widetilde{sk})ik
\in_\QQ \PP^b_i , \\
1&\Telse.
\end{cases}
\]
\end{dfn}
From now on our notation refers to this definition.
\begin{lemma}\label{l:ldn}
\newcommand{\x}{\lceil\log_2(k_d(n+2))\rceil+n} %x-d
Define the functions
\begin{align*}
\l_d^1(n)&:=_\NN1+\frac{(\x+1)(\x)}{2}+d,\\
\l(n)&:=_\NN{\max}{} _\NN\{ \l_d(n) : d\leq n+1\}.
\end{align*}
Then the following holds for all finite binary sequences $b$ and $n\in\NN$:
\[
\lh(b)\geq \l(n)\rightarrow \forall x,y\in\PP^b\ \left( 
d^\omega(x,y)\leq2^{-n} \right).
\]
For the specific $x$, s.t. $x_d$ is the center of $\PP^b_d$ for all 
dimensions $d$ we have even:
\[
A^{^\BW}([b]) :=_{1(0)}\ \lambda d^0,n^0.\ \bigg\langle   
-k_d+\sum_{i=0}^{w(\l(n),d)-1}\frac{D(\overline{[b]}
(\l(n)),d)(i)}{2^i}k_d  + \frac{1}{2^{w(\l(n),d)}}k_d\bigg\rangle,\]\[
\lh(b)\geq \l(n)\rightarrow \forall y\in\PP^b\ \left( 
d^\omega(A^{^\BW}([b]),y)\leq2^{-n-1} \right).
\]
Moreover, we have 
\[
f_s(b,k)=0 \wedge \lh(b)\geq \l(n)\quad \rightarrow\quad d^\omega((\widetilde{sk}),A^{^\BW}([b]))<2^{-n}.
\]
\end{lemma}
\begin{proof}
\newcommand{\x}{\lceil\log_2(k_d(n+2))\rceil} 
W.l.o.g let $l:=\lh(b)=\l(n)$.
By definition we have $w(l,d)\geq\x+n+1-d$. This means
$
\x-w(l,d)\leq d-n-1.
$\\
So $(n+2)k_d2^{-w(l,d)}\leq 2^{-n-1+d}$ and
$
2k_d-\sum_{i=0}^{w(l,d)}2^{-i}k_d\leq \frac{2^{-n-1+d}}{n+2}.
$\\
By definition of $\PP^b_d$ we obtain for all $d\leq n+1 (\leq w(l,0))$:
$
|\PP^b_d|\leq \frac{2^{-n-1+d}}{n+2},
$\footnote{
By $|[a,b]|,\ a,b\in\QQ$ we mean the length of the rational interval $[a,b]$.
}
which implies\\
$
\sum^{n+1}_{d=0}2^{-d}|\PP^b_d|\leq 2^{-n-1}
$
and 
$
\sum^{\infty}_{d=0}2^{-d}\frac{|\PP^b_d|}{1+|\PP^b_d|} \leq 2^{-n}.
$\\
To show $d^\omega((\widetilde{sk}),A^{^\BW}([b]))<2^{-n}$ suppose 
$\bigwedge_{i=0}^{w(\lh(b),0)} (\widetilde{sk})ik\in \PP^b_i$. This implies
there is an $y\in\PP^b$ s.t. $\forall d\ |y_d-(\widetilde{sk})_d|\leq 2^{-k}$. Therefore
$d^\omega((\widetilde{sk}),A^{^\BW}([b])) \leq 2^{-n-1}+2^{-k}$ and since
$k>\lh(b)\geq \l(n) \geq n+2$ also $d^\omega((\widetilde{sk}),A^{^\BW}([b]))<2^{-n}$.\\
\end{proof}
Furthermore, we need to show the following property of our tree representation 
of $\PP$.
\begin{lemma}\label{l:split}
At any level $n\in\NN$, the union of all spaces corresponding to the 
paths of length $n$ is the whole space $\PP$:
\[ \bigcup_{b\in\{b^0:\lh(b)=n\wedge 
\bigwedge_{i=0}^{n}b(i)\leq_0 1\}}\PP^b = \PP.
\]
\end{lemma}
\begin{proof}
Let $I_d$ denote the set of indices within a given, arbitrary long binary 
sequences $b$ used by $D$ to generate the subsequence $D(b,d)$ ( $I_d = 
\{ (d+i)(d+i+1)/2+d = \langle i,d\rangle : i\in\NN \}$).
Since the Cantor pairing function is bijective, it follows that
for $d_1 \neq d_2$ the intersection $I_{d_1}\cap I_{d_2}$ is empty and we 
can choose the binary sequences
for each dimension independently. \\
Therefore
it suffices to show the following (we scale by $2k_d$ and shift by 
$\frac 1 2$):
\[
 \forall n\forall x\in [0,1]\exists b\ \big(\lh(b)=n\wedge x\in
 [\sum_{i=1}^n\frac{b(i-1)}{2^i}, 1-\sum_{i=1}^{n}\frac{1-b(i-1)}{2^i}]
 \big).
\]
This holds for any $n$ and $x$ when we choose $b$ as the following binary sequence:
\[
b(i):=\begin{cases}
1&\Tif\ x \geq \sum_{j=1}^i\frac{b(j-1)}{2^{j}}+\frac1{2^{i+1}},\\
0&\Telse.
\end{cases}
\]
\end{proof}

\subsection{Functional Interpretation of $\BW^\omega_\RR$} \label{ss:fafi}

From now on we consider $s$ to be an infinite sequence of points
in $\PP$, and $f_s$ to be the characteristic function of
the corresponding tree (as defined in \ref{d:PPs} above).
From section~\ref{s:wkl}, we know that using an 
appropriate formula $\phi^{\WKL}$
we can write $\WKL$ as
\[
\forall(h^1)\WKL_\Delta(h)
\equiv
\forall (h^1)\exists b^1\forall k\ \ \phi^{\WKL}(h,b,k)
\equiv
\forall(h^1)\exists b^1\forall k\ \ 
\widefr{\Big(\!\text{\widefr{h}}\Big)_{\!g(h)}} (\bar bk)=_00
\text{,}
\]
where $\phi^{\WKL}$ is quantifier-free.\\
We introduce the following notations for $\SiLm\CA$:
\begin{align*}
\exists g^1\forall x^0 &\phi^{\CA(f)}_{\Sigma^0_1}(x,gx)\\
  &\equiv \exists g^1\forall x^0 \big(gx=_00 \leftrightarrow 
            \exists z^0\ f(x,z)=_00\big)\\
\Leftrightarrow\ 
\exists g^1\forall x^0 &\forall z^0_2\exists z^0_1 
\phi^{\CA(f)}(x,gx,z_1,z_2)\\
&\equiv\
\exists g^1\forall x^0\forall z^0_2\exists z^0_1
   \big((gx=_00 \rightarrow f(x,z_1)=_00)\wedge(gx=_00 \leftarrow 
f(x,z_2)=_00) \big)
\text{,}
\end{align*}
where $\phi^{\CA(f)}$ is a quantifier-free formula. \\
The essential step in the proof above is the following implication:
\[ \big(\SiLm\CA(f) \wedge \WKL\big)\quad \rightarrow\quad 
  \Big( 
   \exists g,b\forall x,k\ \big(\ \phi^{\CA(f)}_{\SiL}(x,gx)\wedge 
\phi^{\WKL}(g,b,k)\ \big) 
   \Big)\  
\tag{+}\text{,} \]
since its conclusion is essentially the same as
$\WKL(\psi_f)$, where
$\psi_f(k^0)$  $\leftrightarrow$ $\exists n^0 f(k,n)\!=_0$~$\!0$
and $k$ is the variable which is bound by the last for-all quantifier in 
$\WKL$.
Moreover, for $f_s$ as defined above it actually directly implies $\BW$, 
whereas 
considered as a schema for arbitrary $f\tp1$ it corresponds to $\SiLm\WKL$.\\
The Sh-interpretation of (+) using this representation and applying $\QFm\AC$ 
is as follows:
\begin{align*}
\exists G,Z_1,B,H',&X',Z_2',K'\ \ \forall B',Z_1',G',X,Z_2,K\\
\Big(
 \big(
  \phi^{\CA(f)}(&\args{
    X'\args{(G'\args{X'Z_2'})(Z_1'\args{X'Z_2'})},
    G'\args{X'Z_2'(X'\args{(G'\args{X'Z_2'})(Z_1'\args{X'Z_2'})})},
    Z_1'\args{X'Z_2'(X'\args{(G'\args{X'Z_2'})(Z_1'\args{X'Z_2'})})
   (Z_2'\args{(G'\args{X'Z_2'})(Z_1'\args{X'Z_2'})})},}\\
    &\args{Z_2'\args{(G'\args{X'Z_2'})(Z_1'\args{X'Z_2'})}
  })\wedge
  \phi^{\WKL}(\args{
     H', B'\args{H'K'}, K'\args{(B'\args{H'K'})}
  })
 \big)\rightarrow\\
 \big(
  \phi^{\CA(f)}(&\args{
     X\args{GZ_1B},
     G\args{(X\args{GZ_1B})},
     Z_1\args{(X\args{GZ_1B})(Z_2\args{GZ_1B})},
     Z_2\args{GZ_1B}
  })\wedge
  \phi^{\WKL}(\args{
     G,
     B,
     K\args{GZ_1B}  
  })
 \big)
\Big)
\end{align*}

where, again, each exists-variable (i.e. $G$, $Z_1$, $B$, $H'$, $X'$, $Z_2'$, 
and $K'$) 
may depend on any 
for-all-variable (i.e. $B'$, $Z_1'$, $G'$, $X$, $Z_2$, and $K$). E.g. by $G$ 
we mean in 
fact $(GB'Z_1'G'XZ_2K)$.  This interpretation
yields the following functional equations:
\setcounter{equation}{0}
\begin{align*}
      X'\args{(G'\args{X'Z_2'})(Z_1'\args{X'Z_2'})}&=
                 X\args{GZ_1B} \text{,}&                        %2
     H'&=G \text{,}\tag{1,5}\\                                                %5
     G'\args{X'Z_2'(X'\args{(G'\args{X'Z_2'})(Z_1'\args{X'Z_2'})})}&=
                G\args{(X\args{GZ_1B})}\text{,}&         %1
     B'\args{H'K'}&=B \text{,}\tag{2,6}\\                                               %6  
     Z_1'\args{X'Z_2'(X'\args{(G'\args{X'Z_2'})(Z_1'\args{X'Z_2'})})
   (Z_2'\args{(G'\args{X'Z_2'})(Z_1'\args{X'Z_2'})})}&=
                Z_1\args{(X\args{GZ_1B})(Z_2\args{GZ_1B})} \text{,}&    %3
     K'\args{(B'\args{H'K'})}&=K\args{GZ_1B}\text{,}\tag{3,7}\\      %7
     Z_2'\args{(G'\args{X'Z_2'})(Z_1'\args{X'Z_2'})}&=
                Z_2\args{GZ_1B} \text{.} \tag{4}                    %4
\end{align*}
We use a very similar approach to the one used by Gerhardy in 
\cite{GerhardyX} to
solve such equations for finite $\DNS$. First, we conclude from (5) and (6)
that $B=B'GK'$ and from (1) and (2) that $G=G'X'Z_2'$. Using (6), we can set 
$K'$ to $\lambda b.KGZ_1b$ according to (7). This is not that trivial
for $X'$ and $Z_2'$. However, as pointed out by Gerhardy in \cite{GerhardyX}, 
in the presence of the $\lambda g$ and $\lambda z_1$, which as we know will 
stand for the input of $G$ and $Z_1$, the objects  $X'$ and 
$Z_2'$ become well definable terms:
\[
t_{X'}:=\lambda g,z_1.Xgz_1(B'g(\lambda b.Kgz_1b))\text{,}\quad
t_{Z_2'}:=\lambda g,z_1.Z_2gz_1(B'g(\lambda b.Kgz_1b))\text{.}
\]
This makes the rest of our terms we need  well defined. This is easy
to see since for each term all dependencies are only on
the terms defined above:
\begin{align*} 
t_{Z_1}:&=Z_1't_{X'}t_{Z_2'}\text{,}&
t_{H'}:&=t_{G}\text{,}\\
t_{G}:&=G't_{X'}t_{Z_2'}\text{,}&
t_{K'}:&=\lambda b.Kt_{G}t_{Z_1}b\text{,}\\
t_{B}:&=B't_{G}t_{K'}\text{.}
\end{align*} 
We have found the realizing terms for the Shoenfield 
interpretation $(+)^{Sh}$ of $(+)$ 
for any $G'$, $Z_1'$ and $B'$. To finally obtain
the Shoenfield interpretation of $(\SiLm\WKL(\phi))$ we just need to 
define these three functionals in such a way that the 
assumptions $\phi^{\WKL}$ and $\phi^{\CA(f)}$ are always true.\\
For $\phi^{\CA(f)}$, as we know from the functional interpretation of 
$\SiLm\CA$ 
(see section \ref{ss:CA}), we get:
\[
      {G'}\tp3=t_h \text{,}\quad
      {Z_1'}\tp3=t_z\text{,} 
\]
where $t_h$ and $t_z$ are defined as in the Sh-interpretation of $\SiLm\CA$ 
(see corollary \ref{c:NDSiLCA}).\\
For $B'$, from the interpretation of $\WKL$, we know the following
equality holds:
\[ 
\underbrace{\overline {B'H'K'}\big(K'(B'H'K')\big)}_{\text{as above}}\quad = 
\underbrace{\overline{B}(AB)}_{\text{as in section \ref{s:wkl}}} 
\text{.} 
\]
We use the same notation as we used to define $B$ in section \ref{s:wkl} 
and define:
\[
        B':=\lambda h.\lambda A.\big[F_h\big(K_{A}(\emptyset)\big)\big] 
\text{,}
\]
where $F_h$ and $K_A$ are defined as in the Sh-interpretation of $\WKL$
(see Theorem \ref{t:FIwkl}).\\
The terms defined above, using these definitions for $G'$ and $B'$, then 
satisfy the Shoenfield interpretation of the conclusion of $(+):$
\begin{align*}
\forall X,Z_2,K\ (\ &\phi^{\CA(f)}(
 Xt_{G}t_{Z_1}t_{B},
 t_{G}(Xt_{G}t_{Z_1}t_{B}),
 t_{Z_1}( Xt_{G}t_{Z_1}t_{B} )( Z_2t_Gt_{Z_1}t_{B} ),
 Z_2t_Gt_{Z_1}t_{B})\ \wedge\\
 &\phi^{\WKL}(t_{G},t_{B},Kt_{G}t_{Z_1}t_{B})\ ) \text{.}
\end{align*}
Using that 
$\forall X^{\typeOfXZ},Z^{\typeOfXZ},a^0,b^0\ t_zXZab=_0t_zXZa0 = t_Za0$ 
we conclude:
\begin{lemma} \label{p:ND-PI02WKL}
$ $
\\
The principle (which essentially represents\footnote{
It trivially implies $\SiLm\WKL(\phi)$ (provably in $\hrrwepa$). Let us note, 
however, that the actual computation of the witnesses for $\SiLm\WKL(\phi)$ 
still involves some highly non-trivial technical work. 
}  $\SiLm\WKL(\phi)$):
\[
\exists g^1,b^1\forall x^0,k^0\ \big( \phi^{\CA(f)}_{\SiL}(x,gx)\wedge 
\phi^{\WKL}(g,b,k) \big)
\text{,}
\]
is Sh-interpreted by ($\tau=01(10)1\ $\footnote{Or, in a more illustrative 
notation: 
$\tau = (\NN \rightarrow \NN) \rightarrow (\NN \rightarrow \NN \rightarrow 
\NN) \rightarrow (\NN \rightarrow \NN) \rightarrow \NN$.}):
\begin{align*}
\forall X^\tau,Z^\tau,K^\tau\ \Big(\ &\phi^{\CA(f)}\big(
 Xt_{G}t_{Z}t_{B},
 t_{G}( Xt_{G}t_{Z}t_{B} ),
 t_{Z}( Xt_{G}t_{Z}t_{B} ) 0,
 Zt_Gt_Zt_{B}\big)\ \wedge\\
 &\phi^{\WKL}(t_{G},t_{B},Kt_{G}t_{Z}t_{B})\ \Big) \text{,}
\end{align*}
where
\newcommand{\tpA}{0(10)1}
\newcommand{\tpz}{10}
\begin{align*}
        t_B    &:=_1 B^{^\WKL}(\lambda b.Kt_Gt_Zb,t_G)\text{,}&
        t_{X}' &:=_{\tpA}\lambda g^1,z^{\tpz} . 
Xgz(B^{^\WKL}(\lambda b.Kgzb,g))\text{,}\\
        t_Z    &:=_{\tpz} t_zt_{X}'t_{Z}'\text{,}&
        t_{Z}' &:=_{\tpA}\lambda g^1,z^{\tpz} . 
Zgz(B^{^\WKL}(\lambda b.Kgzb,g))\text{,}\\ 
        t_G    &:=_1 t_ht_{X}'t_{Z}'\text{,}\\
\end{align*}
The remaining terms are defined as in previous sections.\footnote{
 $B^\WKL$ in Theorem~\ref{t:FIwkl}, and $t_z,t_h$ in Theorem~\ref{c:NDSiLCA}.
}
\end{lemma}

\begin{thm}\label{c:ND-bw}
The Bolzano-Weierstra{\ss} principle $\BW^\omega_\RR$ for an infinite 
sequence $s$ of elements in $\PP$ (let $\rho=(10)0$ and $\sigma=0(10)1$, 
$\widetilde{x^{1(0)}}:=\lambda i^0.{\min} _{\RR}\big(-k_i, 
{\max}_\RR(k_i, xi)\big)$, recall $\PP=\prod_{i\in\NN}[-k_i,k_i]$):
\[
  \forall s^{\rho}
         \exists a^{1(0)} \forall m^0\exists l^0>_0 m
           \ \ d^\omega(\widetilde{sl},\tilde a)<_\RR 2^{-m}
\text{.}
\]
is Sh-interpreted\footnote{The quantifier in $<_\RR$ is irrelevant, 
see the remark after the theorem.} by:
\[
\forall s^{\rho}, M^{\sigma} \exists L^{1},a^{1(0)}
\underbrace{
  \big(\ \ L\args{(M\args{La})}>_0 M\args{La}\ \wedge\ 
  d^\omega( \widetilde { s\args{(L\args{(M\args{La})})} },\tilde{a})
<_\RR2^{-M\args{La}}\ \ \big) }_{\BW_{Sh}(a,MLa,L(MLa),s) :=}\] 
where $L$ and $a$ are realized by the terms $t_L\tp3$ and $t_A\tp3$ 
(we use the notation from Definition~\ref{d:PPs} 
and Lemma~\ref{l:ldn}):
\begin{align*} t_L(s^{\rho},M^{\sigma})&:=_1\ 
\lambda n^0.t_Z\big(\overline{t_B}(\l(n))\big)0 \text{,}\\
t_A(s^{\rho},M^{\sigma})&:=_{1(0)} A^{^\BW}(t_B) =\ \lambda d^0,n^0.\ 
\bigg\langle   
-k_d+\sum_{i=0}^{w(\l(n),d)-1}\frac{D(\overline{t_B}
(\l(n)),d)(i)}{2^i}k_d  + \frac{1}{2^{w(\l(n),d)}}k_d\bigg\rangle
\text{,}
\end{align*}
Here $t_B$ is defined as above, i.e. ($B^\WKL$ is defined in Theorem~\ref{t:FIwkl}):
\begin{align*}
t_B&=_1 B^{^\WKL}(M't_Gt_Z,t_G),
\end{align*}
where as before (with $f:=f_s$, $z^-:\equiv_1\lambda n . z^{10} n 0$, 
$g^+:\equiv_{10}\lambda a,b\ . g^1a$ and 
$g\!\restriction\!{_{f_s}}:\equiv \lambda n. \overleftrightarrow{f_s(n,gn) }$):
\begin{align*}
t_Z&=_{10}\! \bigg(\! \xTWf
    	{\big(    \args \lambda g^1 . X \args {
    														 (g\!\restriction\!{_{f_s}})
    			    									 g^+
    					    							 \big(B^{^\WKL}(M'(g\!\restriction\!{_{f_s}})g^+,g\!\restriction\!{_{f_s}})\big)	}			 \big)}
		  {\big(    \args \lambda g^1 . Z \args {
    														 (g\!\restriction\!{_{f_s}})
    			    									 g^+
    					    							 \big(B^{^\WKL}(M'(g\!\restriction\!{_{f_s}})g^+,g\!\restriction\!{_{f_s}})\big)	}			 \big)}{f_s}
      \!\bigg)^+\\
t_G&=_1 ((t_Z)^-)\!\restriction\!{_{f_s}}\text{.}
\end{align*}
Here $t_B$ and $t_Z$ are shortcuts for $t_B\args{XZM'}$ and 
$t_Z\args{XZM'}$ with fixed $X$, $Z$:
\[
X:=_\tau\lambda g^1,z^{10},b^1\ .\ B(M'gzb,g,{z^-},b)\text{,}\quad
Z:=_\tau\lambda g^1,z^{10},b^1\ .\ N(M'gzb,g,{z^-},b)\text{,}\quad
\text{,}
\]
where $(M')^\tau$ (recall $\tau=01(10)1\ $) is defined for any given 
$M^\sigma$ similarly as $X$ and $Z$ 
as follows:
\setcounter{equation}{0}
\begin{equation}\label{e:Mprime}
{M'(g^1,z^{10},b^1)}:=_0 \l\Bigg( M\args{
 \underbrace{  
   \bigg( \lambda n^0.z^-(\overline{b}(\l(n))) \bigg)
 }_{\text{\normalsize $\thicksim t_L$}}
 \underbrace{  
   \bigg( A^{^\BW}(b)
   \bigg)
 }_{\text{\normalsize $\thicksim t_A$}} 
} \Bigg)
\text{.}
\end{equation}
The terms $B$, $X_n$\footnote{The existence of the required $x$ follows from 
Lemma~\ref{l:split}.} and $N$ are primitive recursive, though not trivial, 
case distinctions:
\begin{align*}
B(m^0,g^1,z^1,b^1)&:=_0
\begin{cases}
  {\min}_0 \left\{ x^0\ \left| \ \ 
  \begin{minipage}{4.8cm}
  	$\lh(x)\leq_0 m\ \wedge\ x\in\{0,1\}^{\lh(x)}\ \wedge$ \\ 
  	$\neg\phi^{\CA}(x,gx,zx,Nmgzb) $
  \end{minipage} \right. \right\} &\text{if it exists,} \\
  X_n(m,g,b)&\Telse\text{,}
\end{cases}  \displaybreak[2] \\
X_n(m^0,g^1,b^1)&:=_0
  \begin{cases}
    \bar bm &\Tif\ g(\bar bm)=_00\text{,}\\
    {\min}_0 \{ x^0 | \lh(x)=m \wedge f_s(x,m+1)=_00 \} &\Telse\text{,}
  \end{cases}  \displaybreak[2] \\
N(m^0,g^1,z^1,b^1)&:=_0\begin{cases}
m+1 &\Tif\ f_s(X_n(m,g,b), m+1)=_00\text{,}\\
z(X_nmgb) &\Telse\text{.}
\end{cases}  
\end{align*}
\end{thm}

\begin{proof}
Unwinding $\phi^\CA$ and $\phi^\WKL$ 
we get by lemma \ref{p:ND-PI02WKL}:
\begin{multline}
\forall K^{\tau},X^{\tau},Z^{\tau}\ 
                     ((t_G\args{ XZK (X \args{
                                          (t_G\args{XZK})
                                          (t_Z\args{XZK})
                                          (t_B\args{XZK})
                                        } )
                           } 
                       )=_00  \rightarrow\\
                       f_s( (X \args{
                                  (t_G\args{XZK})
                                  (t_Z\args{XZK})
                                  (t_B\args{XZK})
                                } )  ,
                            t_Z \args{ XZK (X \args{
                                                (t_G\args{XZK})
                                                (t_Z\args{XZK})
                                                (t_B\args{XZK})
                                               } ) 0
                                 }  
                       )=_00
                      )
                      \label{e:CA1}
\end{multline}
and
\begin{multline}
\forall K^{\tau},X^{\tau},Z^{\tau}\ 
                     ((t_G\args{ XZK (X \args{
                                          (t_G\args{XZK})
                                          (t_Z\args{XZK})
                                          (t_B\args{XZK})
                                        } )
                           } 
                       )=_00  \leftarrow\\
                       f_s( X \args{
                                  (t_G\args{XZK})
                                  (t_Z\args{XZK})
                                  (t_B\args{XZK})
                                }   ,
                            Z \args{
                                 (t_G\args{XZK})
                                 (t_Z\args{XZK})
                                 (t_B\args{XZK})
                               } 
                       )=_00
                      )
                      \label{e:CA2}
\end{multline}
and
\be[f:three]
\forall K^{\tau},X^{\tau},Z^{\tau}\ 
  \widefr{\ (\widefr{ t_G \args{XZK} }
             )_{g(t_G\args{XZK}) }\ 
          } (\overline{
                t_B\args{XZK}
             } (K \args{
                    (t_G\args{XZK})
                    (t_Z\args{XZK})
                    (t_B\args{XZK})
                  }
               )
            )=_00
\text{.}
\ee
Fix an arbitrary $M^\sigma$.\\
We set $X$, $Z$ and $K:=_\tau M'$ -- see \eqref{e:Mprime} --
as in the theorem. We will use the following abbreviations:
\begin{align*}
x_0:&\equiv_0X\args{   (t_G\args{XZK}) (t_Z\args{XZK}) (t_B\args{XZK})  }, & 
\gamma:&\equiv_1t_G\args{XZK},\\
z_0:&\equiv_0Z\args{   (t_G\args{XZK}) (t_Z\args{XZK}) (t_B\args{XZK})  }, & 
z:&\equiv_1\lambda n^0. t_Z\args{XZKn0},\\
k_0:&\equiv_0K\args{   (t_G\args{XZK}) (t_Z\args{XZK}) (t_B\args{XZK})  }, & 
b:&\equiv_1t_B\args{XZK}.
\end{align*}
Note that by \eqref{e:CA1}
the equality $\gamma(x_0)=_00$ implies $f_s(x_0,zx_0)=_00$ and thereby 
$\BFunc([x_0])$.\\
We will not be able to show $\BTree(\gamma)$ but fortunately we need only
to show:
\begin{align}
\forall x^0 
\ \ \Big(\ \big( x\subseteq x_0 \ \wedge \ \gamma (x_0)=0 \big) \rightarrow  
\gamma(x)=_00 \ \Big) \label{e:treeX}
\end{align}
and
\begin{align}
    \lh(x_0)=_0 k_0 \ \wedge \ \gamma(x_0)=_00.
    \label{e:infX}
\end{align}
Note that if $x_0$ was not equal to $X_n(k_0,\gamma,b)$ then 
\[\neg \phi^{^\CA} (x_0, \gamma x_0, z x_0, 
\underbrace { N(k_0,\gamma, z, b) }_{ =z_0} ) \] would hold,
which is a contradiction to \eqref{e:CA1} or a contradiction to \eqref{e:CA2}.
So we can assume that $ x_0 = X_n(k_0,\gamma,b)$. Similarly, we have that
\begin{align}
      (\gamma (q) =_00 \rightarrow f_s( q, z(q) )=_00 )\ \ \wedge\ \ 
      (\gamma (q) =_00 \leftarrow  f_s( q, z_0  )=_00 )
\label{e:caFin}
\text{.}
\end{align}
Suppose \eqref{e:caFin} would not hold for some $q'$ with $\lh(q') 
\leq k_0$, then $x_0$ is equal to such 
a $q'$ by the definition of $X$ and we get a contradiction to \eqref{e:CA1} 
$\wedge$ \eqref{e:CA2} again.\\
\begin{itemize}
\item To prove \eqref{e:treeX} suppose:
\addtocounter{equation}{1}
\begin{align*}
&x\subseteq 
       x_0
      \quad \wedge \tag{\arabic{equation}a}\label{e:trAssX-A}\\
&\gamma
       (x_0)=0
                   \tag{\arabic{equation}b}\label{e:trAssX-B}
\end{align*}
holds for some $x$. Together with \eqref{e:trAssX-B} we obtain from 
\eqref{e:caFin}:
\[
f_s\big( x_0, z(x_0 )\big)=_00
\text{,}
\]
since $\lh(x_0)\leq k_0$ (by the definition of $X$). 
We follow the definition of $Z$. We see that 
either $z_0$ directly equals $z(X_n(k_0,\gamma,b))$ or 
we have $f_s(X_n(k_0,\gamma,b),k_0+1)=0$ and it equals $k_0+1$.
This means that in both cases we obtain (using that $x_0 = X_n(k_0,\gamma,b)$)
\[
f_s( x_0, z_0 )=_00
\text{.}
\]
By the definition of $f_s$, see also section \ref{ss:spuWKL}, and 
\eqref{e:trAssX-A} this implies
\[
f_s( x, z_0 )=_00
\text{.}
\]
From \eqref{e:trAssX-A} we get $\lh(x)\leq k_0$ and by \eqref{e:caFin} we 
obtain
\[
\gamma(x)=0
\text{,}
\]
which concludes the proof of \eqref{e:treeX}.\\

\item Recall that $x_0 = X_n(k_0,\gamma,b) $.
This proves the first part of \eqref{e:infX}:
\[\lh(x_0)=_0 k_0.\]
Now we follow the definition of $X_n$. Either $\gamma (\overline{b}(k_0)) = 0$ 
and therefore
$x_0= \overline{b}(k_0)$ and we  
obtain \eqref{e:infX} immediately, or we have that:
\[
f_s\big( X_n(k_0,\gamma,b), k_0+1 \big) = 0.
\]
In that case, we can infer that 
\[z_0 =
   N(k_0,\gamma, z, b) = k_0+1,\]
and we get 
\[
f_s( x_0, z_0 ) = 0.
\]  
Finally, applying \eqref{e:CA2} concludes the proof of \eqref{e:infX}.
\end{itemize}
This concludes the proofs of \eqref{e:treeX} and \eqref{e:infX}.\\
To show $\gamma\big(\overline{b}(k_0)\big)=0$ assume towards contradiction that
\[\gamma\big(\overline{b}(k_0)\big)\neq 0.\]
If so, then by definition \ref{d:frown_g} we have also that
\[\widefr{\gamma}\big(\overline{b}(k_0)\big)\neq 0.\]
By \eqref{e:infX} we know that $\gamma(x_0)=0$. Using \eqref{e:treeX} we know 
that also $\widefr{\gamma}(x_0)=0$ and since $\lh(x_0)=k_0
=\lh(\overline{b}(k_0))$ we have that
\[
\widefr{\gamma}_{g\gamma}\big(\overline{b}(k_0)\big)\neq0.
\]
By definition \ref{d:frown_g} this is a contradiction to \eqref{f:three} and 
we obtain (recall that we started with an arbitrary $M$):
\[
 \forall M^{\sigma}\ \gamma\big(\overline{b}(k_0)\big)=_00
\text{.}
\]
This implies that $x_0=\overline{b}(k_0)$ (by the definition of $X_n$) and 
therefore it follows by \eqref{e:CA1} that:
\begin{align*}
 \forall M^{\tau} \ \ f_s\big(\ \overline{b}(k_0), z(\overline{b}k_0) \ \big) 
=_00
\text{.}
\end{align*}
Using the terms $t_L$ and $t_A$ in the short notation 
(i.e. $t_L$ instead of $t_L\args{sM}$, $t_B$ instead of $t_B\args{XZM'}$ and 
similarly for $t_A$ and $t_Z$)
this becomes
\begin{align*}
 \forall M^{\sigma} \ \ f_s\bigl(\  
     \overline{t_B}(\l(M\args{t_Lt_A})),\ 
     t_Z(\overline{t_B}
                   (\l(M\args{t_Lt_A})))0
 \ \bigr)=_00
\text{.}
\end{align*}
This implies by Lemma~\ref{l:ldn} (note that 
$t_L(M\args{t_Lt_A})\equiv t_Z(\overline{t_B}(\l(M\args{t_Lt_A})))0$):
\[
\forall M^{\sigma}\ 
\Big(\ t_L(Mt_Lt_A) > Mt_Lt_A\  \wedge\ 
  d^\omega(\widetilde{t_A},\widetilde{s (t_L(M\args{t_Lt_A})) })
<_\RR 2^{ -M\args{t_Lt_A} } \Big)
\text{.}
\] Finally observe that for all $n^0$ we have $(t_A(s,M))n=_{\RR} \widetilde{(t_A(s,M))}n$ 
and that
\begin{align*} 
%============= t _ Z =================
t_Z&=_{10} t_zt_{X}'t_{Z}'
	  =\big(t_g(t_f\args t_X')(t_f\args t_Z')\big)^+ \\
   &=\lambda a, b\ .\ \underbrace { t_g\args{
    	\big( \args \lambda g^1 . X     		
	(g\!\restriction\!{_{f_s}})
    	g^+
    	\big(B^{^\WKL}(M'(g\!\restriction\!{_{f_s}})g^+,g\!\restriction\!{_{f_s}})\big)	 \big)
    	\big( \args \lambda g^1 . Z 
        (g\!\restriction\!{_{f_s}})
    	g^+
    	\big(B^{^\WKL}(M'(g\!\restriction\!{_{f_s}})g^+,g\!\restriction\!{_{f_s}})\big)	 \big)
    	a} }_{=\xTWf
    	{\big( \args \lambda g^1 . X 
        (g\!\restriction\!{_{f_s}}) g^+
    	\big(B^{^\WKL}(M'(g\!\restriction\!{_{f_s}})g^+,g\!\restriction\!{_{f_s}})\big)	 \big)}
		  {\big( \args \lambda g^1 . Z 
        (g\!\restriction\!{_{f_s}})
    	g^+ \big(B^{^\WKL}(M'(g\!\restriction\!{_{f_s}})g^+,g\!\restriction\!{_{f_s}})\big)
        \big)}{f_s}a} ,\\
%============= t _ G =================
t_G &=_1t_ht_{X}'t_{Z}' 
     =\big(t_g(t_f\args t_X')(t_f\args t_Z')\big)\!\restriction\!{_{f_s}} . 
\end{align*} 
\end{proof}


\begin{remark}
From the proof we actually see how to realize the hidden quantifier in 
$<_\RR$. Namely before using Lemma~\ref{l:ldn} we can simply 
apply the definition of $f_s$ and obtain
the following equivalent universal formula (for $\rho=(10)0$ and $\sigma=0(10)1$ 
and writing ${t_L}$ instead of $t_LsM$ and ${t_A}$ instead of $t_AsM$):
\[
\forall s^{\rho}, M^{\sigma}
  \big( {t_L}\args{(M\args{{t_L}{t_A}})}>_0 M\args{{t_L}{t_A}}\wedge\ 
  \!\!\bigwedge_{d=0}^{w(\l(M\args{{t_L}{t_A}}))}
         (\widetilde{s\args{({t_L}\args{(M\args{{t_L}{t_A}})})}})d({t_L}\args{(M\args{{t_L}{t_A}})})
         \in_\QQ \PP^{\overline{t_B}(\l(M\args{t_Lt_A}))}
\text{.}
\]
\end{remark}
\begin{remark} 
One can also extend Theorem \ref{c:ND-bw} to cover the case where the 
sequence $(k_d)$ is a sequence in $\RR_+$ rather than $\QQ_+$ provided that 
one adapts the definition of $f_s$ in such a way that also the boundaries 
of the interval $[-k_d,k_d]$ are replaced by suitable rational 
approximations etc. Since the application of our analysis of the 
Bolzano-Weierstra\ss{} theorem given in \cite{Kohlenbach(Browder)} (referred 
to in the introduction) only uses the monotone functional interpretation 
given below (Theorem \ref{c:NMD-bw}) as will be usually the case, 
we restrict ourselves in this paper to treat the case with 
real $k_d$ only in that context where things are particularly simple 
(and no approximation of the type mentioned above is needed). 
\end{remark}

Usually, the Bolzano-Weierstra\ss{} theorem is formulated to state 
the existence of a converging subsequence rather than the existence of
a cluster point. The next theorem gives the solution for the 
Shoenfield interpretation of this formulation: 

\begin{thm}\label{t:ND-ssbw}
The version of the Bolzano-Weierstra\ss{} theorem stating the existence 
of a convergent subsequence is Sh-interpreted as follows (where 
$t_L,t_A$ are as in theorem \ref{c:ND-bw}):
\begin{align*}
\forall s,M \exists f,p ( f(M\args{fp}+1) > f(M\args{fp})
\wedge
d^\omega ( \tilde{p},  \widetilde{s(f(M\args{fp}+1))} ) < 2^{-M\args{fp}}
)
\end{align*}
Let $M'$ be obtained from $M$ by 
\[
M'\args{la}:=\begin{cases}
l^{M(\lambda n . l^n0)a}0&\Tif\ \big(\forall n<M(\lambda n . 
l^n0)a\big)\ \big(l^{n+1}0>l^n0\big),\\
l^{\min{} _0\{n:\ l^{n+1}0\leq l^n0\}}0&\Telse.
\end{cases}
\] 
Define the functionals
\[
t_F\args{sM}:=\lambda n.(t_L\args{sM'})^n0\quad\text{and}\quad 
t_P\args{sM}:=t_A\args{sM'},
\]
then the following holds:
\begin{align*}
\forall s,M \quad (\ &t_F\args{sM}(M(t_F\args{sM})(t_P\args{sM})+1) > 
t_F\args{sM}(M(t_F\args{sM})(t_P\args{sM}))
\ \wedge\\
&d^\omega ( \widetilde { s(t_F\args{sM}(M(t_F\args{sM})(t_P\args{sM})+1)) } , 
\widetilde{t_P\args{sM}} ) < 2^{-M(t_F\args{sM})(t_P\args{sM})}
\ ).
\end{align*}
\end{thm}
\begin{proof} 
Consider any given $s$ and $M$ and define
$M'$ as in the theorem being proved. We denote $t_L\args{sM'}$, 
$t_A\args{sM'}$ and $M\args{(\lambda n.(t_L\args{sM'})^n0)
(t_A\args{sM'})}$ by $l_0$, $a_0$ and $m_0$. 
Unwinding the terms $t_P$ and $t_F$ in the statement of the theorem leads to:
\begin{align*}
\forall s,M \quad (\ &l_0^{m_0+1}0 > l_0^{m_0}0 \ \wedge d^\omega 
( \tilde{a}_0, \widetilde{s(l_0^{m_0+1} 0 )})\leq 2^{-m_0}\ ).
\end{align*}
\begin{enumerate}
\item If $\big(\forall n<m_0\big)\ \big(l_0^{n+1}0>l_0^n0\big)$ then the 
claim follows directly
from theorem~\ref{c:ND-bw} applied to $s$ and $M'$ (i.e.
$ M'\args{l_0a_0} $ becomes $l_0^{m_0}0$ ) 
and its proof:
\begin{align*}
l_0\args{(l_0^{m_0}0)} > l_0^{m_0}0\ \wedge 
d^\omega( \widetilde { s\args{(l_0\args{(l_0^{m_0}0)})} },\tilde{a}_0)
<_\RR2^{-l_0^{m_0}0}
  \tag{\ref{c:ND-bw} for $M'$}\label{ND-bw4Kprime}
\end{align*}
where the proof allows us to omit the final step from:
\begin{align*}
  l_0\args{(l_0^{m_0}0)}>_0 
  l_0^{m_0}0\ \wedge  \bigwedge_{d=0}^{l_0^{m_0}0}
        |(a_0)\args{(l_0^{m_0}0+1)d}-_\QQ 
\widetilde{s\args{(l_0\args{(l_0^{m_0}0)})}}\args{d(l_0^{m_0}0+1)}|
         \quad<_\QQ\quad 2^{-(l_0^{m_0}0+2)}
\text{.}
\end{align*}
To see that $l_0^{m_0}\geq m_0$ recall that $\forall n<m_0\ l_0^{n+1}0>l_0^n0$.
\item Otherwise we have $l_0^{i+1}0\leq l_0^i0$ for some $i<m_0$ and 
$M'l_0a_0=l_0^i0$. However
this is a contradiction to
 theorem~\ref{c:ND-bw} which implies
\[
l_0^{i+1}0 = l_0(l_0^i0) = l_0 ( M'l_0a_0) > M'l_0a_0 = l_0^i0.
\]
\end{enumerate}
\end{proof}
\begin{remark}
Analogously to the previous remark, the proof of Theorem~\ref{c:ND-bw} 
gives us a more strict\\ $Sh$\nbd interpretation.
\end{remark}

Applying the majorization results from section~\ref{ss:majFI} we obtain a 
(much easier) solution to the {\bf monotone} Shoenfield interpretation 
of $\BW^\omega_{\RR}$ (which is all the information on $\BW^\omega_{\RR}$ 
one needs to extract uniform bounds from proofs that use $\BW^\omega_{\RR}).$ 
\\ Instead of having to majorize explicitly the complicated construction of 
$t_A$ we can rely on the fact that elements in 
compact intervals $[-K_d,K_d]$ with 
$K_d\in\NN$ have a representations $x\le_1 N_{K_d},$ where $N_m(k)$ is a 
fixed (non-decreasing) primitive recursive function in 
$m,k$ which can be taken as 
(see \cite{Kohlenbach06}, p.93) 
\[ N_m(k):=j(m2^{k+3}+1,2^{k+2}-1) \ \ \mbox{for the Cantor 
pairing function $j.$} \] 
From the proof of Theorem~\ref{c:ND-bw} it follows that instead of 
$(A^{\BW}(b))(d)$ we can take any other representative of the same 
real number, in particular the representative $\le N_{K_d}$ from 
the construction in \cite{Kohlenbach06}(p.93) for $K_d \ge k_d.$ As a 
result, we can replace $t_A(s,M)(d)$ by another representative 
that is -- for all $s,M$ --  
majorized by $N_{K_d}.$ In fact, we may even allow $k_d$ to be a real number 
where then in the definition of $f_s$ the clause `$(\widetilde{sk})ik
\in_\QQ \PP^b_i$' has to be replaced by `$(\widetilde{sk})i
\in_\RR \PP^b_i$'. The resulting function $f_s$ then no longer is computable 
but still trivially majorizable by the constant-1 function. \\ 
\begin{remark} The shape of $N_m$ above is due to the particular Cauchy representation 
used in \cite{Kohlenbach06}. If one uses the so-called signed-digit representation, it 
can be improved to the constant-$(2m+3)$ function, see prop. 14 in \cite{Engracia}.
\end{remark}
Before we state the monotone functional interpretation of $\BW^\omega_{\RR}$ 
we first need, however, a simple lemma: 
\begin{lemma} Define (for $\NN\ni K_d\ge k_d$ for all $d\in\NN$)
\[ \l^*(n):=(\max\limits_{d\leq n+1} \{ K_d,n\}+2)^4. \]
Then $\l^*\,\maj\, \l.$
\end{lemma}
\begin{proof}
observe that for $d\leq n+1$ we have that
 \[\l_d(n)\leq n+2+(\lceil k_d (n+2)\rceil+n)^2
 \leq (n+2)+(n+2)^2(\lceil k_d + 1\rceil)^2
 \leq (n+2)^2(\lceil k_d + 2\rceil)^2.\] Moreover $\l^*$ is non-decreasing.
\end{proof}
To give the monotone version of Theorem \ref{t:ND-ssbw}, we need the 
following definition:
\begin{dfn} A majorant for $\bPhi_0$ is given by (see also 
\cite{Kohlenbach06} or \cite{Bezem85}):
\[\bPhi^*_0{y}u^{\!0(1(0))(0)}nx:=_1\max\left( 
(\bPhi_0{y^{\mon}}u_{x}^{\!0(1(0))(0)}){^\Mon} nx,x{^\Mon}\right),\]
where 
\begin{align*}
w{^\Mon}(n):&=_\rho\max\{w(i):i\leq_0 n\}\quad \text{ 
(for $w^{\rho0}$, $\rho\in\Tp$)}, \\
y{^\mon}(x):&=_0y(x{^\Mon}), \\
u_x(n,v):&=_1\max\{x{^\Mon},v(unv)\}.
\end{align*}
\end{dfn}

\begin{thm}\label{c:NMD-bw} Let $(k_d)$ be a sequence of non-negative reals 
and $(K_d)$ be a sequence of natural numbers with $K_d\ge k_d$ for all 
$d\in\NN.$ 
The realizing terms of the
$Sh$-interpretation of the Bolzano-Weierstra{\ss} principle $\BW^\omega_\RR$ 
for an infinite sequence $s$
of elements in $\PP$ (defined by $(k_d)$, see also Theorem~\ref{c:ND-bw} above)
can be majorized by terms $t_L^*$ and $t_A^*$, i.e. $t_L^*$ and 
$t_A^*$ satisfy the monotone $Sh$-interpretation of $\BW^\omega_\RR$.
The term $t_L^*$ depends only
on $M$ (but not on $s$) and $t_A^*$ even is independent from both $s$ and 
$M,$ indeed:
\begin{align*}
t_L^*(M)&:=_1
   \lambda n^0.\xTWfM{(X^*M)}{(Z^*M)}
\big(\bar\one(\l^*(n))\big)\text{,} \\
t_A^*&:=_{1(0)}\lambda d^0,n^0.N_{K_d}
\text{,}
\end{align*}
where 
\[ N_m(k):=j(m2^{k+3}+1,2^{k+2}-1) \]
and $X^*M$ and $Z^*M$ are defined primitive recursively in $M$:
\begin{align*}
{X^*M}&:=\lambda g^1\ .\ \bar\one(M'g)  \text{,} &
{Z^*M}&:=\lambda g^1\ .\ {\max}_0\big(\ M'g,\ g(\bar\one(M'g))\ \big)
\text{.}
\end{align*}
The term $M'$ is a similar primitive recursive modification of $M$ as before: 
\[
{M'}g:= \l^*\!\!\left(M
   \left( \lambda n^0.g(\overline{\one}(\l^*(n))) \right)
   \left( t_A^* \right)\right).
\]
The majorized versions of $u$ and $\l$, are given by:
\[
u^*_{Z}n^0v^{1(0)}  := \uWfM{Z} \text{,}
\quad \l^*(n):=(\max_{d\leq n+1} {} _\NN \{K_d,n\}+2)^4
\text{.}
\]
\end{thm}
Though it is not entirely obvious how we obtain this theorem, the
actual steps are purely elementary. 

Similarly, we can obtain the majorized version of Theorem~\ref{t:ND-ssbw}:
\begin{thm}\label{t:NMD-BW}
The terms $t_F^*$ and $t_P^*$ satisfy the monotone $Sh$-interpretation of 
$\BW^\omega_\RR$ 
stating the existence of  a converging subsequence (see also 
Theorem~\ref{t:ND-ssbw} above):
\[
t_F^*\args{M}:=\lambda n.(t_L^*\args{M''})^n0,\quad t_P^*:=t_A^*,
\]
with $t_L^*$ and $t_A^*$ defined as in the Theorem~\ref{c:NMD-bw} and the 
term and where 
the term $M''$ is a similar primitive recursive modification of $M$ as in 
Theorem~\ref{t:ND-ssbw}: 
\[
{M''}la:= l^{M(\lambda n . l^n(0))a}(0)\text.
\]
\end{thm}
\begin{proof}
For $M^*$ s-maj $M$, we have that 
$(M^*)''$ s-maj $M',M''.$ Hence 
$t_L^*\args{((M^*)'')}(n)\ge t_L\args{sM'}(n)$  
and the function $t_L^*\args{((M^*)'')}$ 
is non-decreasing. Therefore
$t_F^*\args{M^*}$ is non-decreasing as well and majorizes $t_F\args{sM}.$
\end{proof}


\subsection {Analysis of the Complexity of the Realizers} \label{ss:acr}

\newcommand{\tn}{$t\in\T_n$}
\newcommand{\tsn}{$t\in\T_0$}

Theorem \ref{c:NMD-bw} implies that we can use only a single
application of $\B_{0,1}$ on primitive recursive functionals
and primitive recursion to obtain the realizing terms for $\BW^\omega_\RR.$ \\
Now, we can investigate how the principle $\BW^\omega_\RR$ does affect the
complexity of the realizers of a given theorem proved using this principle.
Depending on the way the $\BW^\omega_\RR$-principle is used in such a proof, 
we get the results stated in theorems \ref{t:PEfBW} and \ref{t:PEBW} 
respectively.\\
%
\begin{thm}{\em Program extraction for proofs based on an instance of 
$\BW^\omega_\RR$.\\} 
\label{t:PEfBW}
For $n\geq1$, given a proof using $\BW^\omega_\RR$ on a known sequence $sx$ 
of elements in $\PP$ 
specified by a closed term $s$ of $\hrrwepa,$ 
i.e. $s\in(\PP^\NN)^\NN$ defines a sequence of such sequences  
(where the bounds $(k_d)$ used in forming 
$\PP$ are also given by a term $rx$ that may depend on $x$), and a 
quantifier-free formula $\varphi_{_\QF}(x,y)$ containing only $x,y$ as 
free variables we have: from a proof 
\[
\hrrwepa\ +\ \QFm\AC\ +\ \Sigma^0_n\usftext{-IA}\ \ \vdash\ \ 
    \forall x^0\big(\BW^\omega_\RR(sx)\rightarrow\exists y^0 
\phi_{_\QF}(x,y)\big)\text{,}
\]
we can extract a function $f\in\T_n$ by $Sh$-interpretation s.t.
\[
\Som\ \ \models\ \ \forall x^0 \phi_{_\QF}\big(x,f(x)\big) \text{.}
\]
\end{thm}
\begin{remark}
\begin{enumerate}
\item
In particular, for $n=1$ we obtain a \lOrdm{\omega^{\omega^\omega}} 
recursive realizer.
Moreover, for the special case $n=0$, the 2nd author showed 
in~\cite{Kohlenbach98} that when some weaker systems
than $\hrrwepa\ +\ \QFm\AC$ are used, namely 
$\usftext{G}_{\infty}\usftext{A}^\omega+\ \QFm\AC$,
one obtains even \lOrdm{\omega^\omega}recursive realizers. Note that this 
case is not covered by Theorem \ref{t:PEfBW} above.
\item 
Instead of $\hrrwepa$ we may also have the system resulting from this 
by adding the recursors $R_{\rho}$ with $deg(\rho)\le n-1.$
\end{enumerate} \mbox{ }
\end{remark}
\begin{cor} Theorem \ref{t:PEfBW} also holds with $\hrrepa$ instead of 
$\hrrwepa$ once we restrict $\QFm\AC$ to the types $(\rho,\tau):=(1,0)$ and 
$(\rho,\tau):=(0,1).$
\end{cor} 
\begin{proof} Apply elimination of extensionality to theorem \ref{t:PEfBW} 
(see \cite{Luckhardt73} or \cite{Kohlenbach06}). 
\end{proof}


\begin{proof}[ of Theorem \ref{t:PEfBW}]
By soundness of $\hrrwepa$ (see e.g.~\cite{Kohlenbach06} and 
\cite{Streicher/Kohlenbach}) we know that one can construct 
closed terms $t_M$ and $t_Y$ in $T_{n-1}$ realizing the Sh-interpretation 
of the analyzed proof:
\[
\forall x,L,a\, (\BW_{Sh}(a,t_MxLa,L(t_MxLa),sx)\rightarrow\phi_{Sh}(x,t_YxLa))
\text{.}
\]
Now, using the solution terms $t_A,t_L$ for the Shoenfield interpretation of 
$\BW^{\omega}_{\RR}$ from theorem \ref{c:ND-bw} we define 
\[ a:=t_A(sx,t_Mx) \ \ \mbox{and} \ \ L:=t_L(sx,t_Mx) \] 
and obtain that 
\[ \BW_{Sh}(a,t_MxLa,L(t_MxLa),sx). \] 
Hence for 
\[ tx:=t_YxLa=t_Yx(t_L(sx,t_Mx))(t_A(sx,t_Mx)) \] 
we have that $\varphi_{qf}(x,tx).$
\\ Now let $t^*_M,t^*_Y$ be majorants for $t_M,t_Y$ in $T_{n-1}.$ 
Then --using the majorants $t^*_L,t^*_A$ of $t_L,t_A$ from theorem 
\ref{c:NMD-bw} -- we define 
\[ t^*x:=t^*_Yx(t^*_L(t^*_Mx))(t^*_A(t^*_Mx)). \] 
Since $x \ \maj_0 \, x$ we get that $t^*x\ge tx$ for all $x.$ It, therefore, 
suffices to show 
that the function denoted by $t^*$ can be defined in $T_n:$
Using Parsons' result
from \cite{Parsons71} (p.~361) we know that
$t^*_Mx$ has computational sizes $<\omega_{n}(\omega)$.
This fact allows us to apply Howard's proposition, see~\cite{Howard81} (p.~23),
from which it follows that $t^*_L\args{(t^*_Mx)}y^0$ has 
computational size $<\omega_{n+1}(\omega)$ and so (using Parson's result 
again) can be defined by a term in $T_n$ which finishes the proof as 
$t^*_A$ even is in $T_0.$\\
\end{proof}

%space necessary

The Theorem~\ref{t:PEfBW} is optimal in the following sense:
%
\begin{prop}\label{p:optPEfBW}
Any function $h$ given by a closed term in $\T_n$ can be proven to be total
in $\hrrwepa\ +\ \QFm\AC\ +\ \Sigma^0_n\usftext{-IA}$ using a concrete 
instance of
$\BW(s)$ for a suitable closed term $s^1$ in $\T_0$.
\end{prop}
\begin{proof}

In \cite{Kohlenbach00} (Proposition 5.5), the second author gave a 
construction of a
functional $F\tp2$ such that 
(relative to $\hrrwepa$) $\forall 
f^{1(0)}(\PCM(F(f))\rightarrow\PiLm\CAhut(f)),$ 
where $\PCM(f^{1(0)})$ is the principle of 
monotone convergence defined as follows:
\[
\forall n\big(0\leq_\RR f(n+1)\leq_\RR f(n)\big)\ \rightarrow 
       \exists g\forall k\forall m_1,m_2\geq gk
\left(|fm_1-_\RR fm_2|\leq\frac{1}{k+1}\right)
\text{.}
\]
 Moreover, in the presence of $\QFm\AC$
we also have $\forall f^{1(0)}( \BW_\RR(G(f)) \rightarrow\PCM(f) )$ for a 
suitable functional $G.$ 
Hence there are functionals $F_\exists$, and $F_\forall$ such that 
(relative to $\hrrwepa\ +\ \QFm\AC$) for any
function $f^{1(0)}$ the instance $\BW(F_\exists f)$, resp. $\BW(F_\forall f)$, 
of $\BW$ implies the instance $\SiLm\CA(f)$ of $\SiLm\CA$, 
resp. $\PiLm\CA(f)$ of $\PiLm\CA$.\\ 
By Parsons's results in \cite{Parsons72} we know that
$\hrrwepa\ +\ \QFm\AC$ proves the $\Pi^0_2$ sentence stating
the totality of $h$ using additionally just finitely many, say $m$, instances
of $\Sigma^0_{n+1}\usftext{-IA}$ with number parameters only.\\
%
Assume that $n$ is odd. Each of these $m$ instances
can be reformulated 
as an instance of $\Sigma^0_{n}\usftext{-IA}$ with number and 
function parameters plus the instances 
$\forall \tup a^{\tup0} \PiLm\CA(t_i'(\tup a))$ of $\Pi^0_1$-comprehension, 
where the term $t_i'\in\T_0$ ($i\in\{1\ldots m\}$) corresponds to the 
quantifier-free matrix of the induction formula in the $i^{th}$ instance of 
$\Sigma^0_{n+1}\usftext{-IA}$ (i.e. $t_i'$ essentially is the characteristic 
term of that matrix). 
Furthermore, for suitable closed terms $t_i\in\T_0$
 each formula $\forall \tup a^{\tup0} \PiLm\CA(t_i'(\tup a))$ is
equivalent to an instance $\PiLm\CA(t_i)$ of $\PiLm\CA$.
Moreover, $\bigwedge^{m}_{i=0} \PiLm\CA(t_i)$ is 
equivalent to $\PiLm\CA(t)$ for a suitable \tsn (see \cite{Kohlenbach96} for 
this).\\
Analogously, if $n$ is even we obtain the equivalence to $\SiLm\CA(t)$.\\
Finally, $\PiLm\CA(t)$ and $\SiLm\CA(t)$ are
derivable from the instances $\BW(F_\forall t)$ resp. $\BW(F_\exists t)$ of 
$\BW$.\\
\end{proof} 

Theorem~\ref{t:PEfBW} should be contrasted to the case of 
proofs based on the full (2nd order closure -- also w.r.t. the 
bounding sequence $(k_d)$ from $\PP$ -- of the) 
Bolzano-Weierstra\ss{} principle 
$\BW$ where the following (equally optimal) result follows from 
the literature: 
\begin{thm} 
\label{t:PEBW}
Given a proof 
using $\BW^\omega_\RR$ on any given sequence in $\PP$: Let $\phi_{_\QF}(x,y)$ be 
a quantifier-free formula containing only the free variables $x,y.$ Then 
from a proof 
\[
\hrrwepa\ +\ \QFm\AC\ +\ \Sigma^0_\infty\usftext{-IA}\ +\ \BW^\omega_\RR\ \ 
\vdash\ \ 
    \forall x^1\exists y^0 \phi_{_\QF}(x,y)
\]
we can extract by Sh-interpretation a closed term $t^1\in\T$ s.t.
\[
\Som\ \ \models\ \ \forall x^0 \phi_{_\QF}\big(x,t(x)\big) \text{.}
\]
\end{thm} 
\begin{proof}
Over $\hrrwepa\ + \QFm\AC$ the schema of arithmetical comprehension 
$\CA^0_{ar}$ clearly implies both $\BW^\omega_{\RR}$ as well as 
$\Sigma^0_{\infty}\usftext{-IA}.$ The system 
$\hrrwepa\ + \QFm\AC +\CA^0_{ar}$, however, has a $Sh$-interpretation by terms 
in  $\T_0\ +\ \B_{0,1}$ (see e.g. \cite{Kohlenbach06}, Theorem 11.14).
Hence we get a term $t\in \T_0\ +\ \B_{0,1}$ satisfying
 $\forall x^1\phi_{_\QF}(x,t(x))$. Finally, using Corollary 4.4.1 
from \cite{Kohlenbach99}, 
we can conclude that $t$ can be
rewritten as a functional in $\T$.\\
\end{proof}



\begin{thebibliography}{10}

\bibitem{AF98}
J.~Avigad and S.~Feferman.
\newblock G\"odels's functional ("{D}ialectica") interpretation {\em in} {S}.
  {R}. {B}uss (editor): Handbook of proof theory.
\newblock {\em Studies in Logic and the foundations of mathematics}, 
  137:337--405, Elsevier, Amsterdam, 1998.

\bibitem{Bezem85}
M.~Bezem.
\newblock Strongly majorizable functionals of finite type: A model for bar
  recursion containing discontinuous functionals.
\newblock {\em The Journal of Symbolic Logic}, 50:652--660, 1985.

\bibitem{Engracia} P.~Engr\'acia. Proof-theoretical studies on the bounded functional 
interpretation. \newblock PhD Thesis, University of Lisbon, 2009. 


\bibitem{Feferman77} 
S.~Feferman.
\newblock {\em Theories of finite type related to mathematical practice {\em
  in} {J}. {B}arwise (Editor): Handbook of Mathematical Logic}.
\newblock pages 913--972, North Holland, Amsterdam, 1977.


\bibitem{GasparKohlenbach}
J.~Gaspar and U.~Kohlenbach.
\newblock On Tao's `finitary' infinite pigeonhole principle.
\newblock {\em The Journal of Symbolic Logic}, 75:355--371, 2010.

\bibitem{GerhardyX}
P.~Gerhardy.
\newblock {\em Functional interpretation and modified realizability
  interpretation of the double-negation shift}.
\newblock {P}reprint. BRICS, Department of Computer Science, University of
  Aarhus, Denmark, 2006.



\bibitem{Goedel58}
K.~G\"odel.
\newblock \"{U}ber eine bisher noch nicht ben\"utzte {E}rweiterung des finiten
  {S}tandpunktes.
\newblock {\em Dialectica}, 12:280--287, 1958.


\bibitem{Hilbert(26)} D.~Hilbert.
\newblock \"Uber das
Unendliche. \newblock {\em Math. Ann.}, 95:161-190, 1926. 

\bibitem{Howard73}
W.~A. Howard.
\newblock Hereditarily majorizable functionals of finite type.
\newblock In \cite{Troelstra73}, pages 454--461. 

\bibitem{Howard81}
W.~A. Howard.
\newblock Ordinal analysis of simple cases of bar recursion.
\newblock {\em The Journal of Symbolic Logic}, 46:17--31, 1981.

\bibitem{Kohlenbach(92)} 
U.~Kohlenbach.
\newblock Effective bounds from
ineffective proofs in analysis: an application of functional
interpretation and majorization. 
\newblock {\em The Journal of Symbolic Logic}, 57:1239--1273, 1992.

\bibitem{Kohlenbach(A)} 
U.~Kohlenbach. 
\newblock Analysing proofs in
analysis. 
\newblock In: W. Hodges, M. Hyland, C. Steinhorn, J. Truss,
editors, {\em Logic: from Foundations to Applications. European
Logic Colloquium} (Keele, 1993), pp.225--260, Oxford University
Press, 1996.

\bibitem{Kohlenbach98}
U.~Kohlenbach.
\newblock Arithmetizing proofs in analysis. 
\newblock In: {L}arrazabal, {J.M.},
  {L}ascar, {D}., {M}ints, {G}. (editors): Logic colloquium '96.
 {\em Springer Lecture Notes in Logic}, 12:115--158, 1998.

\bibitem{Kohlenbach96}
U.~Kohlenbach.
\newblock Elimination of Skolem functions for monotone formulas in analysis.
\newblock {\em Arch. Math. Logic}, 37:363--390, 1998.

\bibitem{Kohlenbach_gp}
U.~Kohlenbach.
\newblock On the arithmetical content of restricted forms of comprehension,
  choice and general uniform boundedness.
\newblock {\em Ann. Pure and Applied Logic}, 95:257--285, 1998.

\bibitem{Kohlenbach99}
U.~Kohlenbach.
\newblock On the no-counterexample interpretation.
\newblock {\em The Journal of Symbolic Logic}, 64:1491--1511, 1999.

\bibitem{Kohlenbach00}
U.~Kohlenbach.
\newblock Things that can and things that cannot be done in {PRA}.
\newblock {\em Ann. Pure Applied Logic}, 102:223--245, 2000.



\bibitem{Kohlenbach06}
U.~Kohlenbach.
\newblock {\em Applied Proof Theory: Proof Interpretations and their use in
  Mathematics.}
\newblock Springer Monographs in Mathematics. Springer-Verlag,
  Berlin $\cdot$ Heidelberg $\cdot$ New York, 2008.

 
\bibitem{KohlenbachMints}
U.~Kohlenbach.
\newblock On the logical analysis of proofs based on nonseparable Hilbert space
  theory.
\newblock {\em To appear in: Feferman, S., Sieg, W. (eds.), Festschrift for 
G. Mints}.

\bibitem{Kohlenbach(Browder)}
U.~Kohlenbach.
\newblock A quantitative version of a theorem due to F.E. Browder. 
\newblock {\em Preprint}, submitted for publication, 2009.

\bibitem{Kohlenbach(weakcompactness)} U.~Kohlenbach, On the functional 
interpretation of weak sequential compactness. In preparation. 
  
\bibitem{KO02}
U.~Kohlenbach and P.~Oliva.
\newblock Proof mining: A systematic way of analyzing proofs in mathematics.
\newblock In {\em Proceedings of the Steklov Institute of Mathematics, Vol.
  242}, pages 136--164, 2003.


\bibitem{Luckhardt73}
H.~Luckhardt.
\newblock Extensional {G}\"odel functional interpretation.
\newblock {\em Springer Lecture Notes in Mathematics}, vol. 306, 1973.


\bibitem{Parsons71}
C.~Parsons.
\newblock Proof-theoretic analysis of restricted induction schemata (abstract).
\newblock {\em The Journal of Symbolic Logic}, 36:361, 1971.

\bibitem{Parsons72}
C.~Parsons.
\newblock On n-quantifier induction.
\newblock {\em The Journal of Symbolic Logic}, 37:466--482, 1972.

\bibitem{Safarik08}
P.~Safarik.
\newblock {\em The Interpretation of the {B}olzano-{W}eierstra{\ss}
  Principle Using Bar Recursion}.
\newblock Diploma Thesis, TU Darmstadt, 2008.

\bibitem{Scarpellini71}
B.~Scarpellini.
\newblock A model of bar recursion of higher types.
\newblock {\em Compositio Mathematica}, 23:123--153, 1971.



\bibitem{Shoenfield67} J.S. Shoenfield. Mathematical Logic.
Addison-Wesley Publishing Company (Reading, Massachusetts) 1967.

\bibitem{Simpson99}
S.~G. Simpson.
\newblock {\em Subsystems of Second Order Arithmetic}.
\newblock Perspectives in Mathematical Logic. Springer-Verlag, Berlin $\cdot$
  Heidelberg $\cdot$ New York, 1999.

\bibitem{Spector62}
C.~Spector.
\newblock Provably recursive functionals of analysis: a consistency proof of
  analysis by an extension of principles formulated in current intuitionistic
  mathematics.
\newblock In {\em Recursive function Theory, Proceedings of Symposia in Pure
  Mathematics, vol. 5 (J.C.E. Dekker (ed.)), AMS, Providence, R.I.}, pages
  1--27, 1962.

\bibitem{Streicher/Kohlenbach} T. Streicher and U. Kohlenbach.   
Shoenfield is G\"odel after Krivine. Math. Log. Quart. {\bf 53}, pp. 
176-179, 2007. 


\bibitem{Tao07}
T.~Tao.
\newblock Soft analysis, hard analysis, and the finite convergence
  principle.
\newblock Essay posted May 23, 2007. \newblock {\em Appeared in: 
T. Tao, Structure and Randomness: Pages from Year One of a Mathematical 
Blog}. \newblock AMS, 298pp., 2008.

\bibitem{Troelstra73}
A.~S. Troelstra, editor.
\newblock {\em Mathematical Investigation of Intuitionistic Arithmetic and
  Analysis}.
\newblock Springer Lecture Notes in Mathematics, vol. 344. 1973.

\bibitem{Troelstra74}
A.~S. Troelstra.
\newblock Note on the fan theorem.
\newblock {\em The Journal of Symbolic Logic}, 39:584--596, 1974.

\end{thebibliography}


\end{document}