The paper ... presents a direct computational interpretation of the
Bolzano-Weierstrass principle (BW), via a combination of the negative translation
and G\"odel's dialectica interpretation (in the form of Shoenfield's interpretation).
It is know in reverse mathematics that BW is equivalent to arithmetical
comprehension. The goal of the paper is to find the precise complexity of the
functional needed for the interpretation, in particular, when limited induction
is available and a single instance of the BW principle is used ... \\
By naively looking at the proof of BW, one might think that $\Pi^0_2$ comprehension
is needed even when a single instance of BW is used in a proof. The
clever analysis in the paper shows that in fact all one needs is $\SiL$ comprehension
plus weak K\"onig's lemma (and WKL can be easily dealt with via the monotone
interpretation). In fact, the authors work essentially directly with $\SiL$-WKL ...\\
The authors show that a single instance of BW ... raises the complexity of the realiser by one ... The fact that this jump is optimal is
argued ... This exploits the connection between BW and
arithmetical comprehension, and standard results of Parson's on the complexity
of induction for 
$\Sigma^0_n$ formulas. The monotone functional interpretation of (a
single instance of) BW is also considered, resulting in much simpler realisers ...
