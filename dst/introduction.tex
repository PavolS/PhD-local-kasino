\subsection*{Motivation}

In this section we give functional interpretations for both constructive and classical systems of nonstandard arithmetic. 
Aprat from the two aspects: they show that the nonstandard systems are conservative over ordinary (standard) ones and they show how terms can be extracted from nonstandard proofs; we present the results of a first step towards the interpretation of the countable saturation principle (which will hopefully one day lead to a functional interpretation of proofs based on Loeb measures).\\

Let us have a short look at the work of Nelson on conservation results for non-standard systems, also because it was a major source of inspiration for \cite{BBS12} (large part of this chapter is based on that article). The idea of Nelson was to add a new unary predicate symbol $\st$ to $\ZFC$ for an object ``being standard''. Using this predicate, he added three new axioms to $\ZFC$ governing its use and 
formalizing the basic non-standard principles. He calls these axioms Idealization, Standardization and Transfer resulting in the system he calls $\IST$, which actually stands for Internal Set Theory. His main logical result about $\IST$ is that it is a conservative extension of $\ZFC$, so any theorem provable in $\IST$ (which does not involve the $\st$-predicate, of course) is provable also in $\ZFC$. 

He proved the conservativity twice. In the original paper introducing Internal Set Theory \cite{nelson77} (reprinted in Volume 48, Number 4 of the \emph{Bulletin of the American Mathematical Society} in recognition of its status as a classic), and in a later publication \cite{nelson88}. The latter proof is done syntactically by providing a ``reduction algorithm'' (a rewriting algorithm) for converting proofs performed in ${\IST}$ to ordinary ${\ZFC}$-proofs. There is a remarkable similarity between this reduction algorithm and the Shoenfield interpretation \cite{shoenfield01} (see also Definition~\ref{d:FI}). This observation was the starting point for~\cite{BBS12}.

Let us point out that~\cite{BBS12} shows that if one defines a Dialectica-type functional interpretation using the new application, with implication interpreted \emph{\`a la} Diller--Nahm \cite{dillernahm74}, will interpret and eliminate principles recognizable from nonstandard analysis. By combining that functional interpretation with negative translation, we were able to define a Shoenfield-type functional interpretation for classical nonstandard systems as well. In this way we also obtained conservation and term extraction results for classical systems. The resulting functional interpretations in~\cite{BBS12} (see sections~\ref{s:dst:dialectica} and~\ref{ss:dst:shoenfield}) have some striking similarities with the bounded functional interpretations introduced by Ferreira and Oliva in \cite{ferreiraoliva05} and \cite{ferreira09} (see also \cite{gaspar09}).
