This paper introduces a new realizability and functional interpretation where throughout the
interpretation one works with non-empty finite sequences (sets) of witnesses, rather than single
witnesses. This, surprisingly, gives rise to functional interpretations which are extremely suitable
to deal with the kind of principles used in non-standard analysis such as overspill, underspill
and the idealization principle, amongst others. The proposed set of techniques is much more
involved that the usual modifed realizabilty and dialectica interpretations, in that one has to
deal with two different kinds of quantifiers (internal and external) and a new atomic predicate
$\st(x)$, and their corresponding defining axioms. The benefit of this is that most of the
classical principles of non-standard analysis become either interpretable or are even eliminated by
the interpretation, leading to very strong conservation results which extends previous results of
Moerdijk and Palmgren (1997) and Avigad and Helzner (2002). This is a well-written paper
introducing some novel and powerful ideas in the area of functional interpretation,
with some impressive applications to non-standard analysis...
% I have checked in detail more of the proofs,
% and am quite confident of the correctness of the results.
% Therefore, I would strongly recommend that the paper be accepted for publication.
