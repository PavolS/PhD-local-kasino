

We recall:

\begin{align*}
\I\ &:\quad \forallst x^\sigma \exists y^\tau \forall i \leq |x| \, \varphi((x)_i, y) \to \exists y^\tau \forallst x^\sigma \, \varphi(x, y),\\
\HAC_\intern\ &:\quad  \forallst x \existsst y \, \phi(x,y) \to \existsst F \forallst x \exists i\leq |F(x)|\,  \phi (x,F(x)_i), \quad F(x)_i:\equiv(F(x))_i, \\
\AC_0^{\st}\ &:\quad \forallst n^0 \existsst x \Phi(n,x)\ \rightarrow \existsst f \forallst n^0 \Phi(n,f(n)),\\
\CSAT\ &:\quad \forallst n^0 \, \exists y^\tau \, \Phi(n, y) \to \exists f^{0 \to \tau} \, \forallst n^0 \, \Phi(n, f(n))
\end{align*}

\begin{thm}[Main claim]
%More precisely:
\[
\epast+\I+\HAC_\intern+\AC_0^{\st}\ \vdash\ \CSAT.
\]
\end{thm}

\begin{proof}
Assuming \[\epast+\I+\HAC_\intern\ \vdash\  \Phi(x,y) \leftrightarrow \forallst u\existsst v\ \phi(u,v,x,y),\]
(which follows from Proposition 8.5 in Berg et al.) it suffices to show that
\begin{align}
\forallst x^0\exists y\forallst u\existsst v\ \phi(u,v,x,y) \label{e:s5p1}
\end{align}
implies
\begin{align}
\exists \tilde y^1\forallst x^0,u\existsst v\ \phi(u,v,x,\tilde y(x)) \label{e:s5p2}.
\end{align}
By $\HAC_\intern$ we get that \eqref{e:s5p1} implies
\[
\forallst x^0\exists y\existsst V\forallst u\exists i\leq |V(u)|\ \phi(u,V(u)_i,x,y),
\]
which is by $\IL$ equivalent to
\[
\forallst x^0\existsst V\exists y\forallst u\exists i\leq |V(u)|\ \phi(u,V(u)_i,x,y),
\]
and by $\AC_0^{\st}$ this implies that
\begin{align*}
\existsst V\forallst x^0\exists y\forallst u\exists i\leq |V(x,u)|\ \phi(u,V(x,u)_i,x,y). \tag{\ref{e:s5p1}'}\label{e:s5f1}
\end{align*}
We can strengthen \eqref{e:s5p2} to an under $\HAC_\intern$ (on $u$) and $\AC_0^{\st}$ (on $x$) equivalent statement
\[
\exists \tilde y\existsst V \forallst x^0,u\exists i\leq |V(x,u)|\ \phi(u,V(x,u)_i,x,\tilde y(x)),
\]
which is by $\IL$ equivalent to
\[
\existsst V \exists \tilde y\forallst x^0,u\exists i\leq |V(x,u)|\ \phi(u,V(x,u)_i,x,\tilde y(x)). 
\]
By $\I$, this follows from
\[
\existsst V \forallst x^0,u\exists \tilde y\ \forall j\leq |x|,k\leq |u|\ \exists i\leq |V(x_j,u_k)|\quad \phi(u_k,V(x_j,u_k)_i,x_j,\tilde y(x_j)).
%\existsst V \forallst x^0,u\exists \tilde y\ \forall j\leq |x|\ \forall k\leq |x|\ \exists i\leq |V(x_j,u)|\quad \phi(u,V(x_j,u)_i,x_j,\tilde y(x_j)).
 \tag{\ref{e:s5p2}'}\label{e:s5f2}
\]
Hence it suffices to show that \eqref{e:s5f1} implies \eqref{e:s5f2}.\\
Now to do so, let some standard $V$ satisfy \eqref{e:s5f1}, i.e. we have 
\begin{align}
\forallst x^0\exists y\forallst u\exists i\leq |V(x,u)|\ \phi(u,V(x,u)_i,x,y), \label{e:allX}
\end{align}
and fix an arbitrary but standard $x$ of type $0$. Then we have for some finite (standard) natural number $n$ that
\[
x = \langle x_1,\ldots,x_n \rangle,
\]
with $x_i^0$ standard. So by \eqref{e:allX} we have for each $x_j\in\{x_1,\ldots,x_n\}$ an $y_j$ s.t.
\[
\forallst u\exists i\leq |V(x_j,u)|\ \phi(u,V(x_j,u)_i,x_j,y_j), 
\]
in other words we obtain (simply consider $y=\langle y_1,\ldots,y_n \rangle$)
\[
\exists y \forallst u \forall j\leq |x| \exists i\leq |V(x_j,u)|\ \phi(u,V(x_j,u)_i,x_j,y_j).
\]
Finally, let $\tilde y$ be the function $x_j\mapsto y_j$, (for $1\leq j\leq n$), and constant $0$ otherwise.
%about finite choice
\footnote{Note that such a $\tilde y$ is primitive recursively definable in $x$ and $y$, in Particular it is a term in $\T_0$ with parameters $x$ and $y$. However, it is not necessarily standard, as the parameter $y$ -- though its size is finite -- does not need to code only standard elements and, therefore, does not need to be standard itself.}
Then we have that
\[
\forallst u \forall j\leq |x| \exists i\leq |V(x_j,u)|\ \phi(u,V(x_j,u)_i,x_j,\tilde y(x_j)),
\]
so in particular also
\[
\exists \tilde y\forallst u \forall j\leq |x| \exists i\leq |V(x_j,u)|\ \phi(u,V(x_j,u)_i,x_j,\tilde y(x_j)),
\]
and hence
\[
\forallst u\exists \tilde y\ \forall j\leq |x|,k\leq |u|\  \exists i\leq |V(x_j,u)|\ \phi(u,V(x_j,u)_i,x_j,\tilde y(x_j)),
\]
and therefore also \eqref{e:s5f2}.
\end{proof}