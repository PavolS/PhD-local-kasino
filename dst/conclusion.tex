\section{Conclusion and plans for future work}

We hope this paper lays the groundwork for future uses of functional interpretations to analyse nonstandard arguments and systems. There are many directions, both theoretical and applied, in which one could further develop this research topic. We conclude this paper by mentioning a few possibilities which we would like to take up in future research.

First of all, we would like to see if the interpretations that we have developed in this paper could be used to ``unwind'' or ``proof-mine'' nonstandard arguments. Nonstandard arguments have been used in areas where proof-mining techniques have also been successful, such as metric fixed point theory (for methods of nonstandard analysis applied to metric fixed point theory, see \cite{aksoykhamsi90, kirk03}; for application of proof-mining to metric fixed point theory, see \cite{briseid09, gerhardy06, kohlenbach05, kohlenbachleustean03, kohlenbachleustean10, leustean07}) and ergodic theory (for a nonstandard proof of an ergodic theorem, see \cite{kamae82}; for applications of proof-mining to ergodic theory, see \cite{avigad09, avigadgerhardytowsner10, gerhardy08, gerhardy10, kohlenbach11, kohlenbachleustean09, safarik11}), therefore this looks quite promising. For the former type of applications to work in full generality, one would have to extend our functional interpretation to include types for abstract metric spaces, as in \cite{gerhardykohlenbach08, kohlenbach05b}.

But there are also a number of theoretical questions which still need to be answered. Several have been mentioned already: for example, mapping the precise relationships between the nonstandard principles that we have introduced. Another question was whether $\epaststar + \I + \HAC_\intern + \TPA$ is conservative over $\epastar$. Another question is whether our methods allow one to prove conservativity results over $\weha$ and $\wepa$ as well: this will be important if one wishes to combine the results presented here with the proof-mining techniques from \cite{Kohlenbach08}. 

In addition, we would also like to understand the use of saturation principles in nonstandard arguments. These are of particular interest for two reasons: first, they are used in the construction of Loeb measures, which belong to one of the most successful nonstandard techniques. Secondly, for certain systems it has turned out that extending them with saturation principles has resulted in an increase in proof-theoretic strength (see \cite{hensonkeisler86, keisler07}). 

The general saturation principle is
\[ \SAT: \quad \forallst x^\sigma \, \exists y^\tau \, \Phi(x, y) \to \exists f^{\sigma \to \tau} \, \forallst x^\sigma \, \Phi(x, f(x)). \]
Whether this principle has a $\D$-interpretation within G\"odel's $\Tstar$, we do not know; but
\[ \CSAT: \quad \forallst n^0 \, \exists y^\tau \, \Phi(n, y) \to \exists f^{0 \to \tau} \, \forallst n^0 \, \Phi(n, f(n)) \]
has and that seems to be sufficient for the construction of Loeb measures. Interpreting $\CSAT$ and $\SAT$ in the classical context using the $\Sh$-interpretation is probably quite difficult and it is possible that they require some form of bar recursion. We hope to be able to clarify this in future work.
