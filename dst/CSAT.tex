\section{Functional interpretation for IST}


%\reviewQuote{dst/review}{KS13}{Nov 22, 2012}
\reviewQuote{dst/review}{BBS12}{Apr 26, 2012}

\subsection*{Motivation}

In this section we give functional interpretations for both constructive and classical systems of nonstandard arithmetic. 
Aprat from the two aspects: they show that the nonstandard systems are conservative over ordinary (standard) ones and they show how terms can be extracted from nonstandard proofs; we present the results of a first step towards the interpretation of the countable saturation principle (which will hopefully one day lead to a functional interpretation of proofs based on Loeb measures).\\

Let us have a short look at the work of Nelson on conservation results for non-standard systems, also because it was a major source of inspiration for \cite{BBS12} (large part of this chapter is based on that article). The idea of Nelson was to add a new unary predicate symbol $\st$ to $\ZFC$ for an object ``being standard''. Using this predicate, he added three new axioms to $\ZFC$ governing its use and 
formalizing the basic non-standard principles. He calls these axioms Idealization, Standardization and Transfer resulting in the system he calls $\IST$, which actually stands for Internal Set Theory. His main logical result about $\IST$ is that it is a conservative extension of $\ZFC$, so any theorem provable in $\IST$ (which does not involve the $\st$-predicate, of course) is provable also in $\ZFC$. 

He proved the conservativity twice. In the original paper introducing Internal Set Theory \cite{nelson77} (reprinted in Volume 48, Number 4 of the \emph{Bulletin of the American Mathematical Society} in recognition of its status as a classic), and in a later publication \cite{nelson88}. The latter proof is done syntactically by providing a ``reduction algorithm'' (a rewriting algorithm) for converting proofs performed in ${\IST}$ to ordinary ${\ZFC}$-proofs. There is a remarkable similarity between this reduction algorithm and the Shoenfield interpretation \cite{shoenfield01} (see also Definition~\ref{d:FI}). This observation was the starting point for~\cite{BBS12}.

Let us point out that~\cite{BBS12} shows that if one defines a Dialectica-type functional interpretation using the new application, with implication interpreted \emph{\`a la} Diller--Nahm \cite{dillernahm74}, will interpret and eliminate principles recognizable from nonstandard analysis. By combining that functional interpretation with negative translation, we were able to define a Shoenfield-type functional interpretation for classical nonstandard systems as well. In this way we also obtained conservation and term extraction results for classical systems. The resulting functional interpretations in~\cite{BBS12} (see sections~\ref{s:dst:dialectica} and~\ref{ss:dst:shoenfield}) have some striking similarities with the bounded functional interpretations introduced by Ferreira and Oliva in \cite{ferreiraoliva05} and \cite{ferreira09} (see also \cite{gaspar09}).

\subsection{Formalities}

In this section, we follow very closely~\cite{BBS12} and extend our first section to cover the necessary
technicalities to formalize proofs in non-standard analysis.

\subsubsection*{The system $\ehastar$}

In this chapter, $\ehastar$ will be the extension of the system called $\ehazero$ in \cite{Troelstra73} and $\ehaarrow$ in \cite{troelstravandalen88b} with types for finite sequences, see also Definition~\ref{d:eha_epa} in first chapter (though note that we treat things here a little differently -- see below). More precisely, the collection of types $\Tpstar$ 
(similarly to $\Tp$, see Definition~\ref{d:Tp}) will be smallest set closed under the following rules:
\begin{enumerate}
\item[(i)] $0 \in \Tpstar$;
\item[(ii)] $\sigma, \tau \in \Tpstar \Rightarrow (\sigma \to \tau) \in \Tpstar$;
\item[(iii)] $\sigma \in \Tpstar \Rightarrow \sigma^* \in \Tpstar$.
\end{enumerate}

In dealing with tuples, we will follow the notation and conventions of the first chapter, \cite{Troelstra73} and \cite{Kohlenbach08}. Specifically 
for this context see also Details in~\cite{BBS12}. This includes the enrichment of the term language (G\"odel's $\T$); it now also includes a constant $\et_\sigma$ of type $\sigma^*$ and an operation $c$ of type $\sigma \to (\sigma^* \to \sigma^*)$ (for the empty sequence and the operation of prepending an element to a sequence, respectively), as well as a list recursor $\tup L_{\sigma, \tup \rho}$ satisfying the usual axioms (again see~\cite{BBS12} and also
\cite[p. 456]{troelstravandalen88b} or \cite[p. 48]{Kohlenbach08}). In addition, we have the recursors and combinators for all the new types in G\"odel's $\T$, satisfying the usual equations.  The resulting extension we will denote by $\Tstar$.

Differently to previous chapters, we will have a primitive notion of equality at every type and equality axioms expressing that equality is a congruence (as in \cite[p. 448-9]{troelstravandalen88b}). Since decidability of quantifier-free formulas is not essential for this chapter, this choice will not create any difficulties. In addition, we assume the axiom of extensionality for functions:
\begin{displaymath}
\begin{array}{l}
f =_{\sigma \to \tau} g \leftrightarrow \forall x^\sigma \, fx =_{\tau} gx.
\end{array}
\end{displaymath}

The axiom schemas of the underlying system (i.e. $\eha$) apply to all formulas in the language (i.e., also those containing variables of sequence type and the new terms that belong to $\Tstar$).\\
Finally, we add the following sequence axiom:
\[ \SA: \quad \forall y^{\sigma^*} \, ( \, y = \et_\sigma \lor \exists a^\sigma, x^{\sigma^*} \, y = c(a,x) \, ). \]
In the usual formalization of $\eha$, as in \cite{Kohlenbach08} or \cite{troelstravandalen88b} (or simply our system from Definition~\ref{d:eha_epa}), for example, one can also talk about sequences, but these have to be coded up (see \cite[p. 59]{Kohlenbach08}). As a result, $\ehastar$ is a definitional extension of, and hence conservative over, $\eha$ as defined in \cite{Kohlenbach08} or \cite{troelstravandalen88b}.

\subsubsection*{The system $\ehaststar$}

\begin{dfn}[As given in~\cite{BBS12}] The language of the system $\ehaststar$ is obtained by extending that of $\ehastar$ with unary predicates $\st^\sigma$ as well as two new quantifiers $\forallst x^\sigma$ and $\existsst x^\sigma$ for every type $\sigma \in \Tpstar$. Formulas in the language of $\ehastar$ (i.e., those that do not contain the new predicate $\st_\sigma$ or the two new quantifiers $\forallst x^\sigma$ and $\existsst x^\sigma$) will be called \emph{internal, denoted -- as before -- by small Greek letters e.g. $\phi$, $\psi$}. Formulas which are not internal will be called \emph{external, denoted by capital Greek letters, e.g. $\Phi$, $\Psi$}.
\end{dfn}

\begin{dfn}[$\ehaststar$, in~\cite{BBS12}] The system $\ehaststar$ is obtained by adding to $\ehastar$ the axioms $\EQ, \Tst$ and  $\IA^{\st{}}$, where
\begin{itemize}
\item $\EQ$ stands for the defining axioms of the external quantifiers:
\begin{eqnarray*}
\forallst x \, \Phi(x) & \leftrightarrow &  \forall x \, (\, \st(x)\rightarrow\Phi(x) \, ),\\
\existsst x \, \Phi(x) & \leftrightarrow & \exists x \, (\, \st(x)\wedge\Phi(x) \, ),
\end{eqnarray*}
with $\Phi(x)$ an arbitrary formula, possibly with additional free variables.
\item $\Tst$ consists of:
\begin{enumerate}
\item the axioms $\st(x) \land x = y \to \st(y)$,
\item the axiom $\st(t)$ for each closed term $t$ in $\Tstar$,
\item the axioms $\st(f)\wedge\st(x)\rightarrow\st(fx)$.
\end{enumerate}
\item $ \IA^{\st{}}$ is the external induction axiom:
\[
\IA^{\st{}} \quad : \quad\big(\Phi(0)\wedge\forallst n^0 (\Phi(n)\rightarrow\Phi(n+1) )\big)\rightarrow\forallst n^0 \Phi(n),
\]
where $\Phi(n)$ is an arbitrary formula, possibly with additional free variables.
\end{itemize}
Here it is to be understood that in $\ehaststar$ the laws of intuitionistic logic apply to all formulas, while the induction axiom from $\ehastar$
\[ \quad\big(\varphi(0)\wedge\forall n^0 (\varphi(n)\rightarrow\varphi(n+1) )\big)\rightarrow\forall n^0 \varphi(n) \]
applies to internal formulas $\varphi$ only.
\end{dfn}

\begin{lemma}[\cite{BBS12}] \label{congruence}
$\ehaststar \vdash \Phi(x) \land x= y\to \Phi(y)$ for every formula $\Phi$.
\label{le:extensionality}
\end{lemma}

\begin{lemma}[\cite{BBS12}]
$\ehaststar \vdash \st^0(x) \land y \leq x \to \st^0(y)$.
\end{lemma}

\begin{dfn}[\cite{BBS12}]
For any formula $\Phi$ in the language of $\ehaststar$, we define its \emph{internalization} $\Phi^{\intern}$ to be the formula one obtains from $\Phi$ by replacing $\st(x)$ by $x = x$, and $\forallst x$ and $\existsst x$ by $\forall x$ and $\exists x$, respectively.
\end{dfn}

One of the reasons $\ehaststar$ is such a convenient system for the proof-theoretic investigations in~\cite{BBS12} is because we have the following easy result:

\begin{prop}[\cite{BBS12}] \label{conservativeint}
If a formula $\Phi$ is provable in $\ehaststar$, then its internalization $\Phi^{\intern}$ is provable in $\ehastar$. Hence $\ehaststar$ is a conservative extension of $\ehastar$ and $\eha$.
\end{prop}

\subsubsection*{Operations on finite sequences}

We have all the standard operations on finite sequences, see~\cite{BBS12} for 
details and the following lemma.
\begin{lemma}[\cite{BBS12}] \label{presstandardness}
     \begin{enumerate}
       \item $\ehaststar \vdash \st(x^{\sigma^*}) \to \st (|x|),$
      \item $\ehaststar \vdash \st(x^{\sigma^*}) \to \st ((x)_{i}),$
\item $\ehaststar \vdash \st(x^{\sigma}_0) \land \ldots \land \st(x^{\sigma}_n) \to \st (\langle x^{\sigma}_0,\ldots,x^{\sigma}_n\rangle),$
      \item $\ehaststar \vdash \st(x^{\sigma^*}) \land \st(y^{\sigma^*}) \to \st (x*_{\sigma}y).$
%       \item $\ehast \vdash \st(x^{\sigma\to\tau}) \land \st(y^{\sigma}) \to \st (x[y])$
\item $\ehaststar \vdash \st(F^{0 \to \sigma^*}) \land \st(n^0) \to \st(F(0) * \ldots * F(n-1))$.
    \end{enumerate}
\end{lemma}
\begin{proof} Follows from the $\Tst$-axioms together  with the fact that the list recursor $L$ belongs to $\Tstar$.
\end{proof}

\subsubsection*{Finite sets}

Most of the time, as in~\cite{BBS12}, we will regard finite sequences as stand-ins for finite sets. We also use the notion of an element and that of one sequence being contained in another, as given in~\cite{BBS12}.

\begin{dfn}[\cite{BBS12}]\label{def:element}
For $s^{\sigma},t^{\sigma^*}$ we write $s \in_{\sigma} t$ and say that $s$ \emph{is an element of} $t$ if
\[
         \exists i < |t| (\, s =_{\sigma} (t)_i \, ).
\]
%(We will mostly write simply $\preceq$.)
For $\tup{s}^{\tup{\sigma}}=s_0^{\sigma_0},\ldots,s_{n-1}^{\sigma_{n-1}}$ and $\tup{t}^{\tup{\sigma}^*}=t_0^{\sigma^*_0},\ldots,t_{n-1}^{\sigma^*_{n-1}}$ we write $\tup{s} \in_{\tup{\sigma}} \tup{t}$ and say that $\tup{s}$ \emph{is an element of} $\tup{t}$ if
\[
        \bigwedge_{k=0}^{n-1} \, s_k \in_{\sigma_k} t_k.
\]
In case no confusion can arise, we will drop the subscript and write simply $\in$ instead of $\in_{\sigma}$ or $\in_{\sigma^*}$.
\end{dfn}

\begin{lemma}[\cite{BBS12}] \label{elemstsetset}
$\ehaststar \vdash \st(x^{\sigma^*}) \land y \in_\sigma x \to \st(y^\sigma)$.
\end{lemma}

\begin{dfn}[\cite{BBS12}]\label{def:preorder}
For $s^{\sigma^*},t^{\sigma^*}$ we write $s \preceq_{\sigma} t$ and say that $s$ \emph{is contained in} $t$ if
\[
         \forall x^\sigma \, ( \, x \in s \to x \in t \, ),
\]
or, equivalently,
\[ \forall i < |s| \, \exists j < |t| \, (s)_i =_{\sigma} (t)_j. \]
%(We will mostly write simply $\preceq$.)
For $\tup{s}^{\tup{\sigma}^*}=s_0^{\sigma^*_0},\ldots,s_{n-1}^{\sigma^*_{n-1}}$ and $\tup{t}^{\tup{\sigma}^*}=t_0^{\sigma^*_0},\ldots,t_{n-1}^{\sigma^*_{n-1}}$ we write $\tup{s} \preceq_{\tup{\sigma}} \tup{t}$ and say that $\tup{s}$ \emph{is contained in} $\tup{t}$ if
\[
        \bigwedge_{k=0}^{n-1} \, s_k \preceq_{\sigma_k} t_k.
\]
%\[
%        \forall k \leq n\forall i\leq |s_k| \exists j \leq |t_k| ((s_k)_i=_{\sigma_k} (t_k)_j)
%\]
\end{dfn}

\begin{lemma}[\cite{BBS12}] $\ehastar$ proves that $\preceq_{\sigma}$ determines a preorder on the set of objects of type $\sigma^*$. More precisely, for all $x^{\sigma^*}$ we have $x \preceq_{\sigma} x$, and for all $x^{\sigma^*},y^{\sigma^*},z^{\sigma^*}$ with $x \preceq_{\sigma} y$ and $y \preceq_{\sigma} z$, we have $x \preceq_{\sigma} z$.
\end{lemma}

\begin{dfn}[\cite{BBS12}]
A property $\Phi(\tup{x}^{\tup{\sigma}^*})$ is called \emph{upwards closed in $\tup{x}$} if
$\Phi(\tup{x})\land \tup{x} \preceq \tup{y} \to \Phi(\tup{y})$
and \emph{downwards closed in $\tup{x}$} if
$\Phi(\tup{x})\land \tup{y} \preceq \tup{x} \to \Phi(\tup{y})$.
\end{dfn}

\subsubsection*{Induction and extensionality for sequences}

\begin{prop}[\cite{BBS12}]
$\ehastar$ proves the induction schema for sequences:
\[ \varphi(\et_\sigma) \land \forall a^\sigma, y^{\sigma^*} \, ( \, \varphi(y) \to \varphi(c(a, y) \, ) \to \forall x^{\sigma^*} \, \varphi(x). \]
\end{prop}

A consequence of this is the principle of extensionality for sequences. We follow~\cite{BBS12} and call two elements $x^{\sigma^*},y^{\sigma^*}$ extensionally equal, and write $x =_{e, \sigma^*} y$, iff
\[ |x| =_0 |y| \land \forall i < |x| \, ( \, (x)_i =_\sigma (y)_i \, ). \]
\begin{prop}[\cite{BBS12}] \label{extprincforseq}
$\ehastar$ proves 
\[ \forall x^{\sigma^*},y^{\sigma^*} \, ( \, x=_{e, \sigma^*} y \to x =_{\sigma^*} y \, ). \]
\end{prop}

\begin{cor}[\cite{BBS12}]
$\ehaststar$ proves
\[ \forall x^{\sigma^*} \, \st(|x|) \land \forall i < |x| \, \st((x)_i) \to \st(x). \]
\end{cor}

\begin{cor}[\cite{BBS12}]
$\ehaststar$ proves the external induction axiom for sequences:
\[  \Phi(\et_\sigma) \land \forallst a^\sigma, y^{\sigma^*} \, ( \, \Phi(y) \to \Phi(c(a, y) \, ) \to \forallst x^{\sigma^*} \, \Phi(x). \]
\end{cor}

\subsubsection*{Finite sequence application}

We already said, that we have all the usual operations on sequences. The following, more involved, operations are crucial for this chapter.

\begin{dfn}[Finite sequence application and abstraction,~\cite{BBS12}]
If $s$ is of type $(\sigma \to \tau^*)^*$ and $t$ is of type $\sigma$, then
\[
         s[t] := (s)_0 (t) *\ldots * (s)_{|s|-1}(t):\tau^*.
\]
For every term $s$ of type $\sigma \to \tau^*$ we set 
\[ \Lambda x^\sigma.s(x):=\langle\lambda x^\sigma . s(x)\rangle: (\sigma \to \tau^*)^*. \]
\end{dfn}

Note that we have
\[ (\Lambda x.s(x))[t] =_{\tau^*} (\lambda x.s(x))(t)=_{\tau^*} s(t). \]
Also, the same conventions as for ordinary application and abstraction apply.

Moreover, in~\cite{BBS12} we also define recursors $\tup{\mathcal{R}}_{\tup{\rho}}$ for each tuple of types $\tup{\rho}^* = \rho^*_0, \ldots, \rho^*_k$, such that
\begin{eqnarray*}
          \tup{\mathcal{R}}_{\tup{\rho}} (0,\tup{y},\tup{z}) &=_{\tup{\rho}^*} & \tup{y}, \\
           \tup{\mathcal{R}}_{\tup{\rho}} (n+1,\tup{y},\tup{z}) &=_{\tup{\rho}^*} & \tup{z}[n,\tup{\mathcal{R}}_{\tup{\rho}} (n,\tup{y},\tup{z})],
\end{eqnarray*}
(where $y_i$ is of type $\rho^*_i$ and $z_i$ is of type $(0 \to \rho^*_0 \to \ldots \to \rho^*_k  \to \rho^*_i)^*$). Indeed, by letting
\[
      \tup{\mathcal{R}}_{\tup{\rho}} :=\lambda n^0,\tup{y},\tup{z}. \tup{R}_{\tup{\rho}^*}(n,\tup{y},(\lambda \tup{s}^{\tup{\rho}^*}, t^0 . \tup{z} [t,\tup{s}])),
\]
where $\tup{R}_{\tup{\rho}}$ are constants for simultaneous primitive recursion as in~\cite{Kohlenbach08},
we get
\[
          \tup{\mathcal{R}}_{\tup{\rho}} (0,\tup{y},\tup{z}) =_{\tup{\rho}^*} \tup{R}_{\tup{\rho}^*}(0,\tup{y},(\lambda \tup{s}^{\tup{\rho}^*}, t^0 . \tup{z} [t,\tup{s}]))  =_{\tup{\rho}^*} \tup{y}
\]
and
\begin{eqnarray*}
           \tup{\mathcal{R}}_{\tup{\rho}} (n+1,\tup{y},\tup{z}) &=_{\tup{\rho}^*} & \tup{R}_{\tup{\rho}^*}(n+1,\tup{y},(\lambda \tup{s}^{\tup{\rho}^*}, t^0 . \tup{z} [t,\tup{s}])) \\
           &=_{\tup{\rho}^*} & (\lambda \tup{s}^{\tup{\rho}^*}, t^0 . \tup{z} [t,\tup{s}]) (\tup{R}_{\tup{\rho}^*} (n,\tup{y},  (\lambda \tup{s}^{\tup{\rho}^*}, t^0 . \tup{z} [t,\tup{s}])  ),n) \\
           &=_{\tup{\rho}^*} &  \tup{z}[n,\tup{R}_{\tup{\rho}^*} (n,\tup{y},  (\lambda \tup{s}^{\tup{\rho}^*}, t^0 . \tup{z} [t,\tup{s}])  )] \\
           &=_{\tup{\rho}^*} &  \tup{z}[n,\tup{\mathcal{R}}_{\tup{\rho}} (n,\tup{y},\tup{z})].
\end{eqnarray*}

We have the following concerning the preorder defined above.

\begin{lemma}[\cite{BBS12}]\label{le:herbrand:new_application} $\ehastar$ proves
      \begin{enumerate}
         \item If $s^{(\sigma\to\tau^*)^*} \preceq \tilde{s}^{(\sigma\to\tau^*)^*}$, then
                   $s[t] \preceq \tilde{s}[t]$, for all $t^{\sigma}$.
         \item If $s \preceq \tilde{s}$, then $s[\tup{t}] \preceq \tilde{s}[\tup{t}]$ for all $\tup{t}$ of suitable types.
         \item If $\tup{s} \preceq \tup{\tilde{s}}$, then $\tup{s}[\tup{t}] \preceq \tup{\tilde{s}}[\tup{t}]$ for all $\tup{t}$ of suitable types.
    \end{enumerate}
\end{lemma}

\begin{lemma}[\cite{BBS12}] $\ehaststar$ proves \[         \st^{(\sigma\to\tau^*)^*}(x) \land \st^{\sigma}(y) \to \st^{\tau^*} (x[y]) \] and \[ \st^{\sigma \to \tau^*}(s) \to \st^{(\sigma \to \tau^*)^*}(\Lambda x^\sigma. s(x)). \]
\end{lemma}




\subsection{Nonstandard principles}

Let us motivate this section with a similar introduction as in~\cite{BBS12}.\\
Nonstandard analysis employs the existence of nonstandard models of the first-order theory of the reals (or the natural numbers). One uses the compactness theorem for first-order logic (or, alternatively, the existence of suitable nonprincipal ultrafilters) to show that there are extensions of the natural numbers (or the reals, or other structures) that are \emph{elementary}, i.e. satisfy the same first-order sentences (or formulas with parameters from the original structure). E.g. for the natural numbers, this means that there are structures ${}^*\NN$ and embeddings $i: \NN \to {}^*\NN$ that satisfy
\[ {}^*\NN \models \varphi(i(n_0), \ldots, i(n_k)) \Longleftrightarrow \NN \models \varphi(n_0, \ldots, n_k) \]
for all first-order formulas $\varphi(x_0, \ldots, x_k)$ and natural numbers $n_0, \ldots, n_k$.\\
The image of $i$ is then called the \emph{standard} natural numbers, while those that do not lie in the image of $i$ are the \emph{nonstandard} natural numbers. Also, it is common to add a new predicate $\st$ to the structure ${}^*\NN$, which is true only for the standard natural numbers.\\
The elementarity of the embedding implies that ${}^*\NN$ is still a linear order in which the nonstandard natural numbers must be infinite (i.e., bigger than any standard natural number). The point of nonstandard systems is that one can use these infinite natural numbers to prove theorems in the nonstandard structure ${}^*\NN$, which must then be true in $\NN$ as well, since the embedding $i$ is elementary. The same applies to the reals, with the addition of infinitesimals (nonstandard reals having an absolute value smaller than any positive standard real). Typically, these infinitesimals are then used to prove theorems in analysis in ${}^*\RR$ to show that they must hold in $\RR$ as well.\\
Of course, this makes sense only \emph{first-order, internal} statements. So, using nonstandard models requires some understanding about what can and what can not be expressed in first-order logic as well as whether formulas are internal.

Let us the following, most important principles in nonstandard analysis:
\begin{enumerate}
\item Overspill: if $\varphi(x)$ is internal and holds for all standard $x$, then $\varphi(x)$ also holds for some nonstandard $x$.
\item Underspill: if $\varphi(x)$ is internal and holds for all nonstandard $x$, then $\varphi(x)$ also holds for some standard $x$.
\item Transfer: an internal formula $\varphi$ (possibly with standard parameters) holds in ${}^*\NN$ iff it holds in $\NN$.
\end{enumerate}


These principles will provide us with three criteria with which we will be able to measure the success of the different interpretations. Let us have a closer look. 

\begin{remark}
Unless we state otherwise, the formulas in this chapter may have additional parameters besides those explicitly shown. 
\end{remark}


\subsubsection*{Overspill}

When formalised in $\ehaststar$, overspill (in type 0) is the following statement:
\[ \OS_0: \forallst x^0 \, \varphi(x) \to \exists x^0 \, ( \, \lnot \st(x) \land \varphi(x) \, ). \]

\begin{prop} {\rm \cite{palmgren98}}
In $\ehaststar$, the principle $\OS_0$ implies the existence of nonstandard natural numbers,
\[ \ENS_0: \exists x^0 \, \lnot \st(x), \]
as well as:
\[ \LLPO_0: \forallst x^0, y^0 \, ( \, \varphi(x) \lor \psi(y) \, ) \to \forallst x^0 \, \varphi(x) \lor \forallst y^0 \, \psi(y) . \]
\end{prop}

Formulated for all types, we get:
\[ \OS: \forallst x^\sigma \, \varphi(x) \to \exists x^\sigma \, ( \, \lnot \st(x) \land \varphi(x) \, ). \]

Generalized, this becomes a higher-type version of Nelson's idealization principle \cite{nelson77}:
\[ \I: \forallst x^{\sigma^*} \exists y^\tau \forall x' \in_{\sigma} x \, \varphi(x', y) \to \exists y^\tau \forallst x^\sigma \, \varphi(x, y). \]

\begin{prop} {\rm \cite{palmgren98}} In $\ehaststar$, the idealization principle $\I$ implies overspill, as well as the statement that for every type $\sigma$ there is a nonstandard sequence containing all the standard elements of that type:
\[ \USEQ: \exists y^{\sigma^*} \, \forallst x^\sigma \, x \in_{\sigma} y. \]
\end{prop}

\begin{prop}[\cite{BBS12}]\label{prop:LLPO} In $\ehaststar$, the idealization principle $\I$ implies the existence of nonstandard elements of any type,
\[ \ENS: \exists x^\sigma \, \lnot \st(x), \]
as well as $\LLPO$ for any type:
\[ \LLPO: \forallst x^\sigma, y^\sigma \, ( \, \varphi(x) \lor \psi(y) \, ) \to \forallst x^\sigma \, \varphi(x) \lor \forallst y^\sigma \, \psi(y) . \]
\end{prop}

Of course, classically (intuitionistically, things are not so clear), idealization is equivalent to its dual
\[ \R: \forall y^\tau \existsst x^\sigma \, \varphi(x, y) \to \existsst x^{\sigma^*} \forall y^\tau \exists x' \in x \, \varphi(x', y), \]
which was dubbed the \emph{realization principle} in~\cite{BBS12}.

Our interpretation for constructive nonstandard analysis actually eliminates the stronger \emph{nonclassical realization principle}:
\[ \NCR: \forall y^\tau \existsst x^\sigma \, \Phi(x, y) \to \existsst x^{\sigma^*} \forall y^\tau \exists x' \in x \, \Phi(x', y), \]
where $\Phi(x, y)$ can be any formula. This is quite remarkable, as $\NCR$ is incompatible with classical logic (hence the name) in that one can prove:

\begin{prop}[\cite{BBS12}]
In $\ehaststar$, the nonclassical realization principle $\NCR$ implies the undecidability of the standardness predicate:
\[ \lnot \forall x^\sigma \, ( \, \st(x) \lor \lnot \st(x) \, ). \]
\end{prop}

\subsubsection*{Underspill}

Underspill (in type 0) is the following statement:
\[ \US_0: \forall x^0 \, ( \, \lnot \st(x) \to \varphi(x) \, ) \to \existsst x^0 \, \varphi(x). \]
In a constructive context it has the following nontrivial consequence (compare \cite{avigadhelzner02}):

\begin{prop}[\cite{BBS12}] \label{US_0impliesMP_0}
In $\ehaststar$, the underspill principle $\US_0$ implies
\[ \MP_0: \big( \, \forallst x^0 \, ( \, \varphi(x) \lor \lnot \varphi(x) \, ) \land \lnot \lnot \existsst x^0 \varphi(x) \, \big) \to \existsst x^0 \varphi(x). \]
In particular, $\ehaststar + \US_0 \vdash \lnot \lnot \st^0 (x) \to \st^0(x)$.
\end{prop}

Also underspill has a direct generalization to higher types:
\[ \US: \forall x^\sigma \, ( \, \lnot \st(x) \to \varphi(x) \, ) \to \existsst x^\sigma \, \varphi(x). \]


\subsubsection*{Transfer}

Following Nelson \cite{nelson77}, the transfer principle is usually formulated as follows:
\[ \TPA: \forallst \tup t \, ( \, \forallst x \,  \varphi(x, \tup t) \to \forall x \, \varphi(x, \tup t) \, ), \]
where, this time, $x$ and $\tup t$ include all free variables of the formula $\varphi$. This is classically, but not intuitionistically, equivalent to the following:
\[ \TPE: \forallst \tup t \, ( \, \exists x \,  \varphi(x, \tup t) \to \existsst x \, \varphi(x, \tup t) \, ), \]
where, once again, we do not allow parameters.

Interpreting transfer is very difficult, especially in a constructive context (in fact, Avigad and Helzner have devoted an entire paper \cite{avigadhelzner02} to this issue).
In~\cite{BBS12}, we discuss three problems as follows:
\begin{enumerate}
\item Transfer principles together with overspill imply instances of the law of excluded middle, as was first shown by Moerdijk and Palmgren in \cite{moerdijkpalmgren97}. In our setting we have:
\begin{prop}[\cite{BBS12}]
\begin{enumerate}
\item In $\ehaststar$, the combination of $\ENS_0$ and $\TPA$ implies the law of excluded middle for all internal arithmetical formulas.
\item In $\ehaststar$, the combination of $\USEQ$ and $\TPA$ implies the law of excluded middle for all internal formulas.
\end{enumerate}
\end{prop}
\item As Avigad and Helzner observe in \cite{avigadhelzner02}, also the combination of transfer principles with underspill results in a system which is no longer conservative over Heyting arithmetic. More precisely, adding $\US_0$ and $\TPA$, or $\US_0$ and $\TPE$, to $\ehaststar$ results in a system which is no longer conservative over Heyting arithmetic $\HA$. The reason is that there are quantifier-free formulas $A(x)$ such that
\[ \HA \not\vdash \lnot \lnot \exists x \, A(x) \to \exists x \, A(x). \]
Since one can prove a version of Markov's Principle in $\ehaststar + \US_0$, adding either $\TPA$ or $\TPE$ to it would result in a nonconservative extension of $\HA$ (and hence of $\ehastar$). We refer to \cite{avigadhelzner02} for more details.
\item The last point applies to functional interpretations only. As is well-known, in the context of functional interpretations the axiom of extensionality always presents a serious problem and when developing a functional interpretation of nonstandard arithmetic, the situation is no different. Now, $\ehaststar$ includes an internal axiom of extensionality (as it is part of $\ehastar$), but for the functional interpretation that we will introduce in Section 5 that will be harmless. What will be very problematic for us, however, is the following version of the axiom of extensionality: if for two elements $f, g$ of type $\sigma_1 \to (\sigma_2 \to \ldots \to 0))$, we define
\[ f =^{\st} g \, :\equiv \, \forallst x_1^{\sigma_1}, x^{\sigma_2}_2, \ldots \, ( \, f\tup x =_0 g\tup x \, ), \]
then extensionality formulated as
\[ \forallst f \, \forallst x, y \, ( \, x =^{\st} y \to fx =^{\st} fy \, ) \]
will have no witness definable in $\ZFC$. But that means that also $\TPA$ can have no witness definable in $\ZFC$: for in the presence of $\TPA$ both versions of extensionality are equivalent.
\end{enumerate}

To cope with this, we follow the route taken in most sources (beginning with \cite{moerdijk95}), to have transfer not as a principle, but as a \emph{rule}. As we will see, this turns out to be feasible. In fact, we will have two transfer rules (which are not equivalent, not even classically):
\[ \begin{array}{ccc}
\infer[\TRA]{\forall x \, \varphi(x)}{\forallst x \, \varphi(x)} & & \infer[\TRE]{\existsst x \, \varphi(x)}{\exists x \, \varphi(x)}
\end{array} \]
(this time, no special requirements on the parameters).

%\section{Herbrand realizability}

In this section we will introduce a new realizability interpretation, which will allow us to prove our first consistency and conservation results in the context of $\ehaststar$. Our treatment here will be entirely proof-theoretic; for a semantic approach towards Herbrand realizability, see \cite{berg12}.

\subsection{The interpretation}

The interpretation works by associating to every formula $\Phi(\tup x)$ a formula $\Psi(\tup t,\tup x)$, also denoted by $\tup t \hr \Phi(\tup x)$, where $\tup t$ is a tupe of new variables \emph{all of which are of sequence type}, determined solely by the logical form of $\Phi(\tup x)$. The soundness proof will then involve showing that for every formula $\Phi(\tup x)$ of $\ehaststar$ with $\ehaststar \vdash \Phi(\tup x)$ there is an appropriate tuple $\tup t$ of terms from $\Tstar$ such that $\ehaststar \vdash \tup t \hr \Phi(\tup x)$.

The idea of the interpretation is that we interpret internal quantifers uniformly and that we do not attempt to give them any computational content. In a sense, the only predicate to which we will assign any computational content is the standardness predicate $\st$: to realize $\st^\sigma(x)$, however, it suffices to provide a nonempty finite list of terms of type $\sigma$, one of which will have to be equal to $x$. Therefore to realize a statement of the form $\existsst x^\sigma \, \Phi(x)$ one only needs to provide a finite list $\langle y_0, \ldots, y_n \rangle$ and to make sure that $\Phi(y_i)$ is realized for some $i \leq n$ (which is like giving a Herbrand disjunction; hence the name ``Herbrand realizability'').

The precise definition is as follows:
\begin{dfn}[Herbrand realizability for $\ehaststar$]\label{d:herRel}
\begin{align*}
[]&\hr \phi  & &:\equiv& &\phi\quad \text { for an internal atomic formula $\phi$}, \\
%t&\hr \st(s) & &:\equiv& &t=\langle t_0, \ldots, t_n\rangle\text{ and $s=t_i$ for some $i\leq n$},\\
s&\hr \st(x) & &:\equiv& & x \in s, \\
\tup s, \tup t&\hr (\Phi\vee\Psi) & &:\equiv& &\tup s\hr\Phi\vee \tup t\hr\Psi,\\
 \tup s, \tup t & \hr (\Phi \wedge \Psi) & &:\equiv& &\tup {s}\hr\Phi \, \land \, \tup {t}\hr\Psi,\\
\tup s&\hr (\Phi \rightarrow \Psi) & &:\equiv& & \forallst \tup t\ (\ \tup t\hr\Phi
 \rightarrow\tup s \, [\tup t \, ] \hr\Psi),\\
\tup s&\hr \exists x \, \Phi(x) & &:\equiv& & \exists x\ (\tup s\hr\Phi(x)),\\
\tup s&\hr \forall x \, \Phi(x) & &:\equiv& & \forall x\ (\tup s\hr\Phi(x)),\\
s, \tup t &\hr \existsst x \, \Phi(x) & &:\equiv & & \exists s' \in s \, \big( \, \tup t \hr \Phi(s') \, \big),\\
\tup s &\hr \forallst x \, \Phi(x) & &:\equiv& & \forallst x\ \big(\tup s \, [x] \hr \Phi(x)\big).
\end{align*}
\end{dfn}

Before we show the soundness of the interpretation, we first prove some easy lemmas:

\begin{dfn}[The $\exists^{\st{}}$-free formulas]
We call a formula (in the language of $\ehaststar$) $\exists^{\st{}}$-free, if it is built up from atomic formulas (including $\bot$) using the connectives $\wedge$, $\lor$, $\to$ and the quantifiers $\exists x$, $\forall x$ and $\forallst x$. Alternatively, one could say that these are the formulas in which $\st$ and $\existsst$ do not occur. We denote such formulas by $\Phi_{\not\exists^{\st}}$.
\end{dfn}

\begin{lemma}[Interpretation of $\exists^{\st{}}$-free formulas]\label{l:IntOfEFFormulas} All the interpretations $\tup t \hr \Phi(\tup x)$ of formulas $\Phi(\tup x)$ of $\ehaststar$ are $\exists^{\st{}}$-free. In addition, every $\exists^{\st{}}$-free formula is interpreted by itself. Hence the interpretation is idempotent.
\end{lemma}

The following lemma will be crucial for what follows:

\begin{lemma}[Realizers are provably upwards closed]\label{l:hrealizersext} The formula $\tup t \hr \Phi(\tup x)$ is provably upwards closed in $\tup t$, that is: 
\[ \ehaststar \vdash \tup s\hr\Phi \land \tup s \preceq \tup t \to \tup t\hr\Phi. \]
\end{lemma}
\begin{proof}
By induction on the structure of $\Phi$, using the monotonicity of the new application in the first component in the clauses for $\to$ and $\forallst$.
\end{proof}

\begin{thm}[Soundness of Herbrand realizability]
Let $\Phi$ be an arbitrary formula of $\ehaststar$ and $\Delta_{\not\exists^{\st}}$ be an arbitrary set of $\existsst$-free sentences. Whenever
\[\ehaststar + \Delta_{\not\exists^{\st}} \quad \vdash\quad\Phi(\tup x),\]
then one can extract from the formal proof closed terms $\tup t$ in $\Tstar$, such that
\[\ehaststar + \Delta_{\not\exists^{\st}} \quad\vdash\quad \tup t \hr \Phi(\tup x).\]
\end{thm}
\begin{proof}
As for the logical axioms and rules, the differences with the usual soundness proof of modified realizability for $\eha$ (as in \cite{Troelstra73} or Theorem 5.8 in~\cite{Kohlenbach08}) are
\begin{enumerate}
\item[(a)] that we require the realizing terms to be closed,
\item[(b)] that we have a nonconstructive interpretation of disjunction, and 
\item[(c)] that we interpret the quantifiers in a uniform fashion.
\end{enumerate}
Therefore one has to make the following modifications:
\begin{enumerate}
\item The contraction axiom $A \lor A \to A$ is realized by $\Lambda \tup x, \tup y. \tup x * \tup y$, using that the collection of realizers is provably upwards closed.
\item The weakening axiom $A \to A \lor B$ is realized by $\Lambda \tup x.\tup x, {\cal O}$.
\item The permutation axiom $A \lor B \to B \lor A$ is realized by $\Lambda \tup x, \tup y.\tup y, \tup x$.
\item The axioms of $\forall$-elimination $\forall x\Phi(x)\rightarrow\Phi(t)$ and $\exists$-introduction $\Phi(t)\rightarrow\exists x\Phi(x)$ are realized by the identity tuple $\Lambda \tup x\ .\ \tup x$.
\item The expansion rule $\frac{A \to B}{A \lor C \to A \lor C}$: if $\tup t \hr A \to B$, then $\Lambda \tup x, \tup y. \tup t[\tup x], \tup y$ is a Herbrand realizer of $A \lor C \to A \lor C$.
\item The $\forall$-introduction rule $\frac{\Phi\rightarrow\Psi(x)}{\Phi\rightarrow\forall x\Psi(x)}$ is interpreted, because $\tup s\hr (\Phi\rightarrow\Psi(x))$ implies $\tup s \hr  (\Phi\rightarrow\forall x\Psi(x))$: for if $\tup t \hr \Phi$ and $\st(\tup t)$, then $\tup s[\tup t]  \hr \Psi(x)$ and therefore $\tup s[ \tup t] \hr  \forall x\Psi(x)$.
\item The $\exists$-introduction rule $ \frac{\Phi(x)\rightarrow\Psi}{\exists x\Phi(x)\rightarrow\Psi}$ is interpreted, because $\tup s \hr (\Phi(x)\rightarrow\Psi)$ implies $\tup s \hr (\exists x\Phi(x)\rightarrow\Psi)$: for if $\tup t \hr \exists x\Phi(x)$ and $\st(\tup t)$, then there is an $x$ such that $ \tup t\hr \Phi(x)$, from which it follows that $\tup s[\tup t] \hr \Psi$. 
\end{enumerate}
The sentences from $\Delta_{\not\exists^{\st}}$ and the axioms of $\ehastar$, including $\SA$ and the defining axioms for equality, successor, combinators and recursion, are $\exists^{\st{}}$-free and therefore realized by themselves. Therefore it remains to show the soundness of the following rules and axioms:
\begin{enumerate}
\item The external quantifier axioms $\EQ$: both directions in $\existsst x \Phi(x) \leftrightarrow  \exists x (\st(x)\wedge\Phi(x))$ are interpreted by the the identity. In $\forallst x \Phi(x) \leftrightarrow   \forall x (\st(x)\rightarrow\Phi(x))$ the right-to-left direction is realized by $\Lambda \tup s, x. \tup s \, [\langle x \rangle]$, while the left-to-right direction is realized by $\Lambda \tup s, x. \tup s[x_0] * \ldots * \tup s[x_{|x|-1}]$.
\item The axiom schemes $\Tst$: the principle $\st(x) \land x = y \to \st(y)$ is realized by the identity, while $\st(t)$ is realized by $\langle t \rangle$. In addition, $\st(f) \land \st(x) \to \st(fx)$ is realized by $\Lambda f, x. \langle f_i(x_j) \rangle_{i < |f|, j < |x|}$. 
\item The induction schema $\IA^{\st{}}$:
suppose $\tup s \hr \Phi(0)$ and $\tup t \hr \forallst n (\Phi(n)\rightarrow\Phi(n+1))$, with $\st(\tup s)$ and $\st(\tup t)$. Then $\ehaststar$ proves by external induction that for standard natural numbers $n$, the term $\tup {\mathcal R} (n,\tup s,\atup t)$ is standard and $\tup {\mathcal R} (n,\tup s,\atup t) \hr \Phi(n)$. Therefore $\Lambda \tup x,\tup y, n\ .\ \tup {\mathcal R} (n,\tup x,\atup y) \hr \IA^{\st{}}$.
\end{enumerate}
\end{proof}

\subsection{The characteristic principles of Herbrand realizability}

In this section we will prove that $\HAC$ (the herbrandized axiom of choice), $\HIP_{\not\exists^{\st}}$ (the herbrandized independence of premise principle for $\exists^{\st}$-free formulas) and $\NCR$ axiomatize Herbrand realizability:
\begin{enumerate}
\item $\HAC$ : \[
     \forallst x \existsst y \, \Phi(x,y) \to \existsst F \forallst x \exists y \in F(x) \,  \Phi (x,y),
            \]
            where $\Phi(x,y)$ can be any formula.
           If $\Phi(x,y)$ is upwards closed in $y$, then this is equivalent to
           \[
               \forallst x \existsst y  \, \Phi(x,y) \to \existsst F \forallst x \, \Phi (x,F(x)).
            \]
\item $\HIP_{\not\exists^{\st}}$: \[
    \big( \Phi \to\existsst y \, \Psi(y)\big)\rightarrow \existsst y \, \big( \Phi \to \exists y' \in y \, \Psi(y')\big),
            \]
            where $\Phi$ has to be an $\existsst$-free formula and $\Psi(y)$ can be any formula.
           If $\Psi(y)$ is upwards closed in $y$, then this is equivalent to
           \[
        \big( \Phi \to\existsst y\Psi(y)\big)\rightarrow \existsst y \, \big(\Phi \to \Psi(y)\big).  
            \]
\item $\NCR$: \[
                \forall x \existsst y \,  \Phi (x,y) \to \existsst y \, \forall x \, \exists y' \in y \, \Phi(x,y') ,
            \]
            where $\Phi(x, y)$ can be any formula.
           If $\Phi(x,y)$ is upwards closed in $y$, then this is equivalent to
           \[
               \forall x \existsst y \, \Phi (x,y) \to \existsst y \forall x  \, \Phi(x,y) .
            \]
\end{enumerate}

\begin{thm}[Characterization theorem for Herbrand realizability]
\qquad 
\begin{enumerate}
\item For any instance $\Phi$ of $\HAC$, $\HIP_{\not\exists^{\st}}$ or $\NCR$, there are closed terms $\tup t$ in $\Tstar$ such that \[ \ehaststar \vdash \tup t \hr \Phi. \]
\item For any formula $\Phi$ of $\ehaststar$, we have
\[ \ehaststar + \HAC + \HIP_{\not\exists^{\st}} +\NCR \vdash \Phi \leftrightarrow \existsst \tup x \, ( \tup x \hr \Phi). \]
\end{enumerate}
\end{thm}
\begin{proof}
Soundness of $\HAC$: If $\tup r = s, \tup t$ and $\tup r \hr \forallst x\existsst y \Phi(x,y)$, then for every standard $x$ there is an $s' \in s[x]$ such that $\tup t[x] \hr \Phi(x, s')$. Hence $\langle \lambda x. s[x] \rangle, \tup t \hr \existsst F \forallst x \exists y \in F(x) \,  \Phi (x,y)$. So $\HAC$ is realized by $\Lambda x, \tup y.\langle \lambda z. x[z] \rangle, \tup y$.

Soundness of $\HIP_{\not\exists^{\st}}$: Suppose $\Phi$ is $\existsst$-free, $\tup r = s, \tup t$ and $\tup r \hr \Phi \to \existsst y \Phi(y)$. This means that if $\Phi$ would hold, then there would be an $s' \in s$ such that $\tup t \hr \Psi(s')$. Hence $\langle s \rangle, \tup t \hr \existsst y (\Phi \to \exists y' \in y \Psi(y'))$. So  $\HIP_{\not\exists^{\st}}$ is realized by $\Lambda x, \tup y.\langle x \rangle, \tup y$.

In a similar manner one checks that also $\NCR$ is realized by $\Lambda x, \tup y.\langle x \rangle, \tup y$. This completes the proof of item 1.

Item 2 one proves by induction on the logical structure of $\Phi$. We discuss implication as an illustrative case, as it is by far the hardest, and leave the other cases to the reader. We reason in $\ehaststar + \HAC + \HIP_{\not\exists^{\st}} +\NCR$. By induction hypothesis, we have that $\Phi \leftrightarrow \existsst \tup t (\tup t \hr \Phi)$ and $\Psi \leftrightarrow \existsst \tup s (\tup s \hr \Psi)$ and therefore
\[ \Phi \to \Psi \]
is equivalent to
\[ \existsst \tup t (\tup t \hr \Phi) \to \existsst \tup s (\tup s \hr \Psi), \]
which in turn is equivalent to:
\[ \forallst \tup t \, \big( \, \tup t \hr \Phi \to \existsst \tup s \, ( \, \tup s \hr \Psi \, ) \, \big). \]
Because $\tup t \hr \Phi$ is $\existsst$-free and $\tup s \hr \Psi$ is upwards closed in $\tup s$, we can use $\HIP_{\not\exists^{\st}}$ to rewrite this as:
\[ \forallst \tup t \, \existsst \tup s \, \big( \, \tup t \hr \Phi \to \tup s \hr \Psi \, \big). \]
As $\tup t \hr \Phi \to \tup s \hr \Psi$ is upwards closed in $\tup s$ and finite sequence application and ordinary application are interdefinable, we can use $\HAC$ to see that this is equivalent to:
\[ \existsst \tup s \, \forallst \tup t \, \big( \, \tup t \hr \Phi \to \tup s[\tup t] \hr \Psi \, \big), \]
which is precisely the meaning of $\existsst \tup s \, ( \, \tup s \hr (\Phi \to \Psi) \, )$.
\end{proof}

\begin{thm}[Main theorem on program extraction by $\hr$] Let $\forallst x \existsst y \Phi(x, y)$ be a sentence of $\ehaststar$ and $\Delta_{\not\exists^{\st}}$ be an arbitrary set $\existsst$-free sentences. Then the following rule holds
\begin{align*}
\ehaststar + \HAC + \HIP_{\not\exists^{\st}} + \NCR + \Delta_{\not\exists^{\st}} & \vdash \forallst x \, \existsst y \, \Phi(x, y) \Rightarrow \\
\ehaststar + \HAC + \HIP_{\not\exists^{\st}} + \NCR + \Delta_{\not\exists^{\st}} & \vdash \forallst x \, \exists y \in  t(x) \, \Phi(x, y),
\end{align*}
where $t$ is a closed term from $\Tstar$ which is extracted from the original proof using Herbrand realizability.

In the particular case where both $\Phi(x, y)$ and $\Delta_{\not\exists^{\st}}$ are internal, the conclusion yields
\[ \ehastar + \Delta_{\not\exists^{\st}} \vdash \forall x \,  \exists y \in t(x) \, \Phi(x, y). \]
If we assume that the sentences from $\Delta_{\not\exists^{\st}}$ are not just internal, but also true (in the set-theoretic model), the conclusion implies that $\forall x \,  \exists y \in t(x) \, \Phi(x, y)$ must be true as well.
\end{thm}
\begin{proof}
If
\begin{align*}
\ehaststar + \HAC + \HIP_{\not\exists^{\st}} + \NCR + \Delta_{\not\exists^{\st}} & \vdash \forallst x \, \existsst y \, \Phi(x, y),
\end{align*}
then the soundness proof yields terms $r, \tup s$ such that
\[ \ehaststar + \Delta_{\not\exists^{\st}} \vdash r, \tup s \hr \forallst x \, \existsst y \, \Phi(x, y). \]
Since $r, \tup s \hr \forallst x \, \existsst y \, \Phi(x, y)$ is by definition $\forallst x \, \exists y \in r[x] ( \, \tup s \hr \Phi(x, y) \,)$, the first statement follows by taking $t = \lambda x. r[x]$.

If both $\Phi(x, y)$ and $\Delta_{\not\exists^{\st}}$ are internal, then $\tup s$ is empty and we get
\[ \ehaststar + \Delta_{\not\exists^{\st}} \vdash \forallst x \,  \exists y \in t(x) \, \Phi(x, y). \]
By internalizing the statement, we obtain
\[ \ehastar + \Delta_{\not\exists^{\st}} \vdash \forall x \,  \exists y \in t(x) \, \Phi(x, y). \]
Since all the axioms of $\ehastar$ are true, this implies that $\forall x \,  \exists y \in t(x) \, \Phi(x, y)$ will be true, whenever $\Delta_{\not\exists^{\st}}$ is.
\end{proof}

\subsection{Discussion}

The main virtue of Herbrand realizability may be that it points one's attention to principles like $\HAC$ and $\NCR$ and that it gives one a simple proof of their consistency. However, as a method for eliminating nonstandard principles from proofs, Herbrand realizability has serious limitations. It does eliminate the realization principle $\R$ (it even eliminates the nonclassical principle $\NCR$), but overspill, the idealization principle $\I$ and the transfer rules are $\exists^{\st}$-free and therefore simply passed to the verifying system. Even worse, the underspill principle $\US_0$ does not have a computable Herbrand realizer:
\begin{prop}
$\MP_0$ and $\US_0$ do not have computable Herbrand realizers.
\end{prop}
\begin{proof}
It is well-known that there can be no computable function witnessing the modified realizability interpretation of Markov's principle, because its existence would imply the decidability of the halting problem. A similar argument shows that $\MP_0$ does not have a computable Herbrand realizer: Kleene's $T$-predicate $T(e, x, n)$ is primitive recursive and hence 
\[ \ehaststar \vdash \forallst e, x, n \big( \, T(e,x, n) \lor \lnot T(e, x, n) \, \big). \]
So if $\MP_0$ would have a computable realizer, then so would 
\[ \forallst e \, \big( \, \lnot\lnot \existsst n\,  T(e,e, n) \to \existsst n \, T(e, e, n) \, \big). \]
But if $t$ would be such a realizer, we could decide the halting problem by checking $T(e, e, n)$ for all $n \in t[e]$.

Since $\US_0$ implies $\MP_0$ (see Proposition \ref{US_0impliesMP_0}), it follows that $\US_0$ does not have a computable realizer either.
\end{proof}

In the next section, we will show that these problems can be overcome by moving from realizability to, more complicated, functional interpretations.
%
%
%
%%%%%%%%%%%%%%%%%%%%%%%%%%%%%%%%%%%%
\subsection{A functional interpretation for $\ehaststar$}\label{s:dst:dialectica}

In this section we will define and study a functional interpretation for $\ehaststar$
introduced in~\cite{BBS12}.

\subsubsection*{The interpretation}\label{ss:dst:dialectica}

The basic idea of the $\D$-interpretation (the nonstandard Dialectica interpretation) is to associate to every formula $\Phi (\tup{a})$ a new formula $\Phi(\tup{a})^{\D}\equiv\existsst \tup{x}\forallst \tup{y} \, \phi_{\D} (\tup{x},\tup{y},\tup{a})$
%(with the same free variables $\tup{a}$)
such that 
\begin{enumerate}
\item all variables in $\tup x$ are of sequence type and
\item $\phi_{\D} (\tup{x},\tup{y},\tup{a})$ is upwards closed in $\tup{x}$.
\end{enumerate}
We will interpret the standardness predicate $\st^{\sigma}$ similarly to the case for Herbrand realizability: For a realizer for the interpretation of $\st^{\sigma}(x)$ we will require a standard finite list $\langle y_0,\ldots,y_n \rangle$ of candidates, one of which must be equal to $x$.

\begin{dfn}[The $\D$-interpretation for $\ehaststar$,~\cite{BBS12}]
We associate to every formula $\Phi (\tup{a})$ in the language of $\ehaststar$ (with free variables among $\tup{a}$) a formula $\Phi(\tup{a})^{\D}\equiv\existsst \tup{x}\forallst \tup{y} \, \phi_{\D} (\tup{x},\tup{y},\tup{a})$ in the same language (with the same free variables) by:
%s.t. $\phi_{\D} (x,y)$ is upwards closed in $x$.
    \begin{itemize}
      \item[(i)] $\phi(\tup{a})^{\D}:\equiv \phi_{\D}(\tup{a}):\equiv \phi(\tup{a})$ for internal atomic formulas $\phi(\tup{a})$,
      \item[(ii)] $\st^{\sigma}(u^{\sigma})^{\D}:\equiv \existsst x^{\sigma^*} u \in_{\sigma} x$.
    \end{itemize}
Let $\Phi(\tup{a})^{\D}\equiv\existsst \tup{x}\forallst \tup{y} \, \phi_{\D} (\tup{x},\tup{y},\tup{a})$ and $\Psi(\tup{b})^{\D}\equiv\existsst \tup{u}\forallst \tup{v} \, \psi_{\D} (\tup{u},\tup{v},\tup{b})$. Then
    \begin{itemize}
      \item[(iii)] $(\Phi (\tup{a}) \land \Psi (\tup{b}))^{\D} :\equiv \existsst \tup{x},\tup{u} \forallst \tup{y},\tup{v} \, \big(\phi_{\D} (\tup{x},\tup{y},\tup{a}) \land \psi_{\D} (\tup{u},\tup{v},\tup{b})\big),$
      \item[(iv)] $(\Phi (\tup{a}) \lor \Psi (\tup{b}))^{\D}  :\equiv \existsst \tup{x},\tup{u} \forallst \tup{y},\tup{v} \, \big(\phi_{\D} (\tup{x},\tup{y},\tup{a}) \lor \psi_{\D} (\tup{u},\tup{v},\tup{b})\big),$

      \item[(v)] $(\Phi(\tup{a}) \to \Psi(\tup{b}))^{\D}  :\equiv   \existsst \tup{U},\tup{Y} \forallst \tup{x},\tup{v} \, \big(\forall \tup{y} \in \tup{Y}[\tup{x},\tup{v}]\, \phi_{\D} (\tup{x},\tup{y}, \tup{a})    \to\psi_{\D} (\tup{U}[\tup{x}],\tup{v}, \tup{b})\big).$
      \end{itemize}
Let $\Phi(z,\tup{a})^{\D}\equiv\existsst \tup{x}\forallst \tup{y} \, \phi_{\D} (\tup{x},\tup{y},z,\tup{a})$, with the free variable $z$ not occuring among the $\tup{a}$. Then
    \begin{itemize}
 \item[(vi)] $(\forall z\Phi(z,\tup{a}))^{\D} :\equiv \existsst \tup{x} \forallst \tup{y} \forall z \, \phi_{\D} (\tup{x},\tup{y},z,\tup{a}),$
 \item[(vii)] $(\exists z\Phi(z,\tup{a}))^{\D} :\equiv \existsst \tup{x} \forallst \tup{y} \exists z \forall \tup{y'} \in \tup{y}\, \phi_{\D} (\tup{x},\tup{y'},z,\tup{a}),$
 \item[(viii)] $(\forallst z\Phi(z,\tup{a}))^{\D} :\equiv \existsst \tup{X} \forallst z,\tup{y}  \, \phi_{\D} (\tup{X}[z],\tup{y},z,\tup{a}),$
 \item[(ix)] $(\existsst z\Phi(z,\tup{a}))^{\D} :\equiv \existsst \tup{x},z \, \forallst \tup{y} \, \exists z' \in z \, \forall \tup y' \in \tup{y}\, \phi_{\D} (\tup{x},\tup{y'},z',\tup{a}).$
\end{itemize}
\end{dfn}
\begin{dfn}[\cite{BBS12}]
We say that a formula $\Phi$ is a $\forallst$-formula if $\Phi \equiv \forallst \tup{x}\,  \phi (\tup{x})$, with $\phi (\tup{x})$ internal.
\end{dfn}
\begin{lemma}[\cite{BBS12}]\label{le:forallst-formulas}
Let $\Phi$ be a $\forallst$-formula. Then $\Phi^{\D}\equiv \Phi$.
\end{lemma}

Note that the clause for $\existsst z$ causes the interpretation to be not idempotent and that realizers are upwards closed:


\begin{lemma}[\cite{BBS12}] Let $\Phi (\tup{a})$ be a formula in the language of $\ehaststar$ with interpretation $\existsst \tup{x}\forallst \tup{y} \, \phi_{\D} (\tup{x},\tup{y},\tup{a})$. Then the formula $\phi_{\D} (\tup{x},\tup{y},\tup{a})$ is provably upwards closed in $\tup{x}$, i.e.,
\[
          \ehastar \vdash \phi_{\D} (\tup{x},\tup{y},\tup{a}) \land \tup{x} \preceq \tup{x}' \to  \phi_{\D} (\tup{x}',\tup{y},\tup{a}).
\]
\end{lemma}

%Before proving the soundness of the $\D$-interpretation
The $\D$-interpretation will allow us to interpret the nonclassical realization principle $\NCR$, and also both $\I$ and $\HAC$. Additionally we will be able to interpret a herbrandized independence of premise principle for formulas of the form $\forallst x \, \phi(x)$, and also a herbrandized form of a generalized Markov's principle:
\begin{enumerate}
\item $\HIP_{\forallst}$:
\[
           \big( \forallst x \, \phi(x) \to\existsst y\Psi(y)\big)\rightarrow \existsst y \, \big(\forallst x \,  \phi (x)\to \exists y' \in y \, \Psi(y')\big),
            \]
           where $\Psi(y)$ is a formula in the language of $\ehaststar$ and $\phi(x)$ is an internal formula. If $\Psi(y)$ is upwards closed in $y$, then this is equivalent to
           \[
        \big( \forallst x \, \phi(x) \to\existsst y\Psi(y)\big)\rightarrow \existsst y \, \big(\forallst x \,  \phi (x)\to \Psi(y)\big).
\]
\item $\HGMP$:
\[
  ( \forallst x \, \phi(x) \to\psi)\to \existsst x \, \big(\forall x' \in x\,  \phi (x')\to\psi\big),
            \]
           where $\phi(x)$ and $\psi$ are internal formulas in the language of $\ehaststar$. If $\phi(x)$ is downwards closed in $x$, then this is equivalent to
           \[
       ( \forallst x \, \phi(x) \to\psi)\to \existsst x (  \phi (x)\to\psi).
\]
The latter gives us a form of Markov's principle by taking $\psi \equiv 0=_0 1$ and $\phi (x) \equiv \lnot \phi_0 (x)$ (with $\phi_0(x)$ internal and quantifier-free), whence the name.
\end{enumerate}

\begin{thm}[Soundness of the $\D$-interpretation,~\cite{BBS12}] \label{soundnessDst}
Let $\Phi (\tup{a})$ be a formula of $\ehaststar$ and let $\Delta_{\intern}$ be a set of internal sentences.
If
\[
    \ehaststar + \I + \NCR + \HAC + \HGMP
    %+ \LLPO
    +  \HIP_{\forallst} + \Delta_{\intern} \vdash \Phi (\tup{a})
\]
and $\Phi(\tup{a})^{\D}\equiv\existsst \tup{x}\forallst \tup{y} \, \phi_{\D} (\tup{x},\tup{y},\tup{a})$,
then from the proof we can extract closed terms $\tup{t}$ in $\Tstar$ such that
\[
    \ehastar +\Delta_{\intern} \vdash \forall \tup{y} \, \phi_{\D} (\tup{t},\tup{y},\tup{a}).
\]
\end{thm}

\begin{remark} \label{hanst} We could define a system $\ehanststar$ by adding primitive predicates $\nst^{\sigma}$ (``nonstandard'') to $\ehaststar$ for each finite type $\sigma$, along with axioms
\[
          \forall x^{\sigma} \big(\nst (x) \leftrightarrow \lnot \st (x)\big).
\]
If we then extend the $\D$-interpretation by
\[
      \big(\nst^{\sigma}(x^{\sigma})\big)^{\D} :\equiv \forallst y^{\sigma}  y \neq_{\sigma} x),
\]
we get an analogue of Theorem~\ref{soundnessDst}, since $\big(  \nst (x) \to \lnot \st (x) \big)^{\D}$
is provably equivalent to
\[
     \existsst Y \forallst z \left( \forall y \in Y[z] (y \neq x) \to  x \not\in z \right)
\]
and $\big( \lnot \st (x) \to \nst(x)\big)^{\D}$
to
\[
     \existsst Z \forallst y \left( \forall z' \in Z[y] x \not\in z' \to y\neq x \right),
\]
so that we can take $Y[z] :=z$ and $Z[y] := \langle \langle y \rangle \rangle$  respectively .


\end{remark}

\subsubsection*{The characteristic principles of the nonstandard functional interpretation}

In~\cite{BBS12} we proved that the characteristic principles of the nonstandard functional interpretation are $\I$, $\NCR$, $\HAC$, $\HIP_{\forallst}$, and $\HGMP$. For notational simplicity we will let
\[
     \HH := \ehaststar + \I + \NCR + \HAC + \HIP_{\forallst} + \HGMP.
\]
\begin{thm}[Characterization theorem for the nonstandard functional interpretation,~\cite{BBS12}] $\, $

\begin{enumerate}

\item For any formula $\Phi$ in the language of $\ehaststar$ we have
%\[
%          \ehaststar + \I + \NCR + \HAC + \HIP_{\forallst} + \HGMP \vdash \Phi \leftrightarrow \Phi^{\D}.
%\]
\[
          \HH \vdash \Phi \leftrightarrow \Phi^{\D}.
\]
\item For any formula $\Psi$ in the language of $\ehaststar$ we have: If for all $\Phi$ in
$\mathcal{L}(\ehaststar)$ (with $\Phi^{\D}\equiv \existsst \tup{x}\forallst \tup{y}\, \phi_{\D}(\tup{x},\tup{y})$) the implication
\begin{equation}\label{eq:char_thm_dst}
          \HH +\Psi \vdash \Phi  \quad \Longrightarrow \quad \mbox{there are closed terms $\tup{t}\in\Tstar$ s.t. }
          \eha \vdash \forall \tup{y}\, \phi_{\D}(\tup{t},\tup{y})
\end{equation}
holds, then $\HH \vdash \Psi$.
%implies that there exist closed terms $\tup{T}$ in G\"{o}del's $\T$ such that
%\[
%         \eha \vdash \forall \tup{y}\, \phi(\tup{x},\tup{y}),
%\]
%where $\Phi^{\D}\equiv \existsst \tup{x}\forallst \tup{y}\, \phi(\tup{x},\tup{y})$, we have
%\[
%          \HH \vdash \Psi
%\]


\end{enumerate}
\end{thm}

Theorem~\ref{soundnessDst} allows us to extract a finite sequence of candidates for the existential quantifier in formulas of the form $\forallst x \, \existsst y \, \phi (x,y)$, in the following sense:

\begin{thm}[Main theorem on program extraction by the $\D$-interpretation,~\cite{BBS12}]
Let $\forallst x \existsst y \, \phi (x,y)$ be a sentence of $\ehaststar$ with $\phi(x,y)$ an internal formula,
and let $\Delta_{\intern}$ be a set of internal sentences.
If
\[
    \ehaststar + \I + \NCR + \HAC + \HGMP
    +  \HIP_{\forallst} + \Delta_{\intern} \vdash \forallst x \, \existsst y \, \phi (x,y),
\]
then from the proof we can extract a closed term $t$ in $\Tstar$ such that
\[
    \ehastar +\Delta_{\intern} \vdash \forall x \, \exists y \in t(x)\, \phi (x,y).
\]
\end{thm}

It follows from the soundness of the $\D$-interpretation (Theorem \ref{soundnessDst}) that it can be used to eliminate nonstandard principles, like overspill, realization and idealization, from proofs. It also allows one to eliminate underspill, since we have the following result (recall that $\R$ is the realization principle from Section 4.1):
\begin{prop}[\cite{BBS12}] \label{USfromRHGMP} We have
\[ \ehaststar + \R + \HGMP \vdash \US, \]
and therefore the underspill principle $\US$ is eliminated by the $\D$-interpretation.
\end{prop}

We also have:
\begin{prop}[\cite{BBS12}]
The system $\HH :\equiv \ehaststar + \I + \NCR + \HAC + \HGMP +  \HIP_{\forallst}$ is closed under both transfer rules, $\TRA$ and $\TRE$.
\end{prop}

%Therefore our functional interpretation $\D$ meets all the benchmarks that we discussed in Section 4. 


\subsection{The system $\epaststar$ and negative translation}\label{ss:dst:negative}

By combining the functional interpretation from the previous section with negative translation we can obtain conservation and term extraction results for classical systems as well. We will work out the details in this and the next section.

First, we need to set up a suitable classical system $\epaststar$. It will be an extension of $\epastar$, which is $\ehastar$ with the law of excluded middle added for all formulas. When working with classical systems, we will often take the logical connectives $\lnot, \lor, \forall$ as primitive and regard the others as defined. In a similar spirit, the language of $\epaststar$ will be that of $\epastar$ extended just with unary predicates $\st^\sigma$ for every type $\sigma \in \Tpstar$; the external quantifiers $\forallst, \existsst$ are regarded as abbreviations:
\begin{eqnarray*}
\forallst x \, \Phi(x) & :\equiv &  \forall x ( \, \st(x)\rightarrow\Phi(x) \, ),\\
\existsst x \, \Phi(x) & :\equiv & \exists x (\, \st(x)\wedge\Phi(x) \, ).
\end{eqnarray*}

\begin{dfn}[$\epaststar$] The system $\epaststar$ is
\[\epaststar := \epastar + \Tst + \IA^{\st{}} \]
where
\begin{itemize}
\item $\Tst$ consists of:
\begin{enumerate}
\item the schema $\st(x) \land x = y \to \st(y)$,
\item a schema providing for each closed term $t$ in $\Tstar$ the axiom $\st(t)$,
\item the schema $\st(f)\wedge\st(x)\rightarrow\st(fx)$.
\end{enumerate}
\item $ \IA^{\st{}}$ is the external induction axiom:
\[
\IA^{\st{}} \quad : \quad\big(\Phi(0)\wedge\forallst n^0 (\Phi(n)\rightarrow\Phi(n+1) )\big)\rightarrow\forallst n^0 \Phi(n).
\]
\end{itemize}
Again we warn the reader that the induction axiom from $\epastar$
\[ \quad\big(\varphi(0)\wedge\forall n^0 (\varphi(n)\rightarrow\varphi(n+1) )\big)\rightarrow\forall n^0 \varphi(n) \]
is supposed to apply to internal formulas $\varphi$ only.
\end{dfn}

As for $\ehaststar$, we have:
\begin{prop}
If a formula $\Phi$ is provable in $\epaststar$, then its internalization $\Phi^{\intern}$ is provable in $\epastar$. Hence $\epaststar$ is a conservative extension of $\epastar$ and $\epa$.
\end{prop}

We will now show how negative translation provides an interpretation of $\epaststar$ in $\ehaststar$. Various negative translations exist, with the one due to G\"odel and Gentzen being the most well-known. Here, we work with two variants, the first of which is due to Kuroda~\cite{Kuroda51}.

\begin{dfn}[Kuroda's negative translation for $\epaststar$]
For an arbitrary formula $\Phi$ in the language of $\epast$, we define
its Kuroda negative translation in $\ehaststar$ as
\[
\Phi^\ku\ :\equiv\ \neg\neg\Phi_\ku,
\]
where $\Phi_\ku$ is defined inductively on the structure of $\Phi$ as follows:
\begin{align*}
 \Phi_\ku   &:\equiv \Phi\quad \text { for atomic formulas $\Phi$}, \\
%t&\hr \st(s) & &:\equiv& &t=\langle t_0, \ldots, t_n\rangle\text{ and $s=t_i$ for some $i\leq n$},\\
 \big(\neg\Phi\big)_\ku  &:\equiv \neg\Phi_\ku,\\
 \big(\Phi\vee\Psi\big)_\ku  &:\equiv  \Phi_\ku\vee \Psi_\ku,\\
 \big(\forall x \, \Phi(x)\big)_\ku  &:\equiv  \forall x \, \neg\neg\Phi_\ku(x).
\end{align*}
\end{dfn}

\begin{thm} \label{soundnesskuroda} $\epaststar \vdash \Phi \leftrightarrow \Phi^\ku$ and if $\epaststar + \Delta \vdash\Phi$ then $\ehaststar + \Delta^\ku \vdash\Phi^\ku$.
\end{thm}
\begin{proof}
It is clear that, classically, $\Phi$, $\Phi_\ku$ and $\Phi^\ku$ are all equivalent. The second statement is proved by induction on the proof of $\epaststar + \Delta \vdash\Phi$. For the cases of the axioms and rules of classical logic and $\epastar$, see, for instance, \cite[Proposition 10.3]{Kohlenbach08}. As the Kuroda negative translation of every instance of $\Tst$ or $\IA^{\st{}}$ is provable in $\ehaststar$ using the same instance of $\Tst$ or $\IA^{\st{}}$, the statement is proved.
\end{proof}

It will turn out to be convenient to introduce a second negative translation, extracted from the work of Krivine by Streicher and Reus (see \cite{krivine90, streicherreus98, streicherkohlenbach07}). This translation will interpret $\epaststar$ into $\ehanststar$ (see Remark \ref{hanst}).

%The formal definition of the interpretation is as follows.
\begin{dfn}[Krivine's negative translation for $\epaststar$]
For an arbitrary formula $\Phi$ in the language of $\epaststar$, we define its Krivine negative translation in $\ehanststar$ as
\[
\Phi^\kr\ :\equiv\ \neg\Phi_\kr,
\]
where $\Phi_\kr$ is defined inductively on the structure of $\Phi$ as follows
\begin{align*}
 \phi_\kr   &:\equiv \neg\phi\quad \text { for an internal atomic formula $\phi$}, \\
 \st(x)_\kr  &:\equiv \nst(x), \\
%t&\hr \st(s) & &:\equiv& &t=\langle t_0, \ldots, t_n\rangle\text{ and $s=t_i$ for some $i\leq n$},\\
 \big(\neg\Phi\big)_\kr  &:\equiv \neg\Phi_\kr,\\
 \big(\Phi\vee\Psi\big)_\kr  &:\equiv \Phi_\kr\wedge \Psi_\kr,\\
 \big(\forall x \, \Phi(x)\big)_\kr  &:\equiv  \exists x \, \Phi_\kr(x).\\
\end{align*}
\end{dfn}

\begin{thm}\label{l:kukr}
For every formula $\Phi$ in the language of $\epaststar$, we have:
\begin{enumerate}
\item $\ehanststar\vdash\Phi^\kr\leftrightarrow\Phi^\ku$.
\item If $\epaststar + \Delta \vdash \Phi$, then $\ehanststar + \Delta ^\kr \vdash \Phi^\kr$.
\end{enumerate}
\end{thm}
\begin{proof}
Item 1 is easily proved by induction on the structure of $\Phi$. Item 2 follows from item 1 and Theorem \ref{soundnesskuroda}.
\end{proof}

\subsection{A functional interpretation for $\epaststar$} \label{s:Shoenfield}

We will now combine negative translation and our functional interpretation $\D$ to obtain a functional interpretation of the classical system $\epaststar$.

\subsubsection*{The interpretation}\label{ss:dst:shoenfield}

\begin{dfn} \label{d:stS} ($\Sh$-interpretation for $\epaststar$.) To each formula $\Phi(\tup{a})$ with free variables $\tup{a}$ in the language of $\epaststar$ we associate its {$\Sh$-interpretation}
\[
\Phi^\Sh(\tup{a}):\equiv\forallst \tup{x}\,\existsst \tup{y}\,
\phi_\Shb(\tup{x},\tup{y}, \tup{a})
\text{,}
\]
where $\phi_\Shb$ is an internal formula. Moreover,  $\tup{x}$ and $\tup{y}$ are tuples of variables whose length and types depend only on the logical structure of $\Phi$. The interpretation of the formula is defined inductively on its
structure. If
\[
\Phi^\Sh(\tup{a}):\equiv\forallst \tup{x}\,\existsst \tup{y}\, \
\phi_\Shb(\tup{x},\tup{y}, \tup{a}) \
\text{ and } \ \Psi^\Sh(\tup{b}):\equiv\forallst \tup{u}\,\existsst \tup{v}\, \
\psi_\Shb(\tup{u},\tup{v}, \tup{b}),
\]
then
\begin{enumerate}
\item[(i)] $\phi^\Sh
:\equiv \phi$ for atomic internal
$\phi(\tup{a}),$
\item[(ii)] $\big(\st(z) \big)^\Sh :\equiv \existsst x \, ( z = x)$,
\item[(iii)] $(\neg \Phi)^\Sh :\equiv \forallst \tup{Y} \existsst \tup{x} \,
\forall \tup y \in \tup Y[\tup x] \neg \phi_\Shb(\tup{x}, \tup y, \tup a),$
\item[(iv)] $(\Phi \vee \Psi)^\Sh :\equiv
             \forallst \tup{x},\tup{u} \existsst \tup{y},\tup{v} \,
\big(\phi_\Shb(\tup{x},\tup{y}, \tup a) \vee \psi_\Shb(\tup{u},\tup{v}, \tup b)\big),$
\item[(v)] $(\forall z \, \phi)^\Sh :\equiv \forallst \tup{x}
\existsst \tup{y} \forall z \exists {\tup y}' \in \tup y \, \phi_\Shb(\tup x,\tup{y}', z).$
\end{enumerate}
\end{dfn}

\begin{thm} \label{soundnessshoenfield} {\rm (Soundness of the $\Sh$-interpretation.)} Let $\Phi(\tup a)$ be a formula in the language of $\epaststar$ and suppose $\Phi(\tup a)^\Sh\equiv\forallst \tup x \, \existsst \tup y \, \phi(\tup x, \tup y, \tup a)$. If $\Delta_{\intern}$ is a collection of internal formulas and
\[ \epaststar + \Delta_{\intern} \vdash \Phi(\tup a), \]
then one can extract from the formal proof a sequence of closed terms $\tup t$ in $\Tstar$ such that
\[
\epastar + \Delta_{\intern} \vdash\ \forall \tup x\exists \tup y \in \tup t(\tup x)\ \phi(\tup x,\tup y, \tup a).
\]
\end{thm}

Our proof of this theorem relies on the following lemma:
\begin{lemma}\label{p:ShD}
Let $\Phi(\tup a)$ be a formula in the language of $\epaststar$ and assume
\begin{eqnarray*}
\Phi^\Sh & \equiv & \forallst \tup x\existsst  \tup y \, \phi(\tup x, \tup y, \tup a) \quad \mbox{and} \\
(\Phi_\kr)^\D & \equiv & \existsst \tup u \forallst \tup v \, \theta(\tup u,\tup v, \tup a).
\end{eqnarray*}
Then the tuples $\tup x$ and $\tup u$ have the same length and the variables they contain have the same types. The same applies to $\tup y$ and $\tup v$. In addition, we have
\[
\epastar\ \vdash\ \phi(\tup x, \tup y, \tup a)\leftrightarrow \lnot \theta(\tup x, \tup y, \tup a).
\]
\end{lemma}
\begin{proof} The proof is by induction on the structure of $\Phi$.
\begin{enumerate}
\item[(i)] If $\Phi \equiv \psi$, an internal and atomic formula, then $\varphi \equiv \psi$ and $\theta \equiv \lnot \psi$, so $\epastar \vdash \varphi \leftrightarrow \lnot \theta$.
\item[(ii)] If $\Phi \equiv \st(z)$, then $\varphi \equiv y = z$ and $\theta \equiv y \not= z$, so $\epastar \vdash \varphi \leftrightarrow \lnot \theta$.
\item[(iii)] If $\Phi \equiv \lnot \Phi'$ with $(\Phi')^\Sh \equiv \forallst \tup x\existsst  \tup y \, \phi'(\tup x, \tup y, \tup a)$ and $(\Phi'_\kr)^\D \equiv \existsst \tup u \forallst \tup v \, \theta'(\tup u,\tup v, \tup a)$, then $\varphi \equiv \forall \tup y' \in \tup Y[\tup x]\ \neg\phi'(\tup x,\tup y')$ and $\theta \equiv \neg\forall \tup i\in \tup Y[\tup x]\ \theta'(\tup x,\tup i)$. Since $\epastar \vdash \varphi' \leftrightarrow \lnot \theta'$ by induction hypothesis, also $\epastar \vdash \varphi \leftrightarrow \lnot \theta$.
\item[(iv)] If $\Phi \equiv \Phi_0 \lor \Phi_1$ with \[ \Phi_i^\Sh \equiv \forallst \tup x\existsst  \tup y \, \phi_i(\tup x, \tup y, \tup a) \] and \[ ((\Phi_i)_\kr)^\D \equiv \existsst \tup u \forallst \tup v \, \theta_i(\tup u,\tup v, \tup a),\] then $\varphi \equiv \varphi_0 \lor \varphi_1$ and $\theta \equiv \theta_0 \land \theta_1$. Since $\epastar \vdash \varphi_i \leftrightarrow \lnot \theta_i$ by induction hypothesis, also $\epastar \vdash \varphi \leftrightarrow \lnot \theta$.
\item[(v)] If $\Phi \equiv \forall z \, \Phi'$ with \[ (\Phi')^\Sh \equiv \forallst \tup x\existsst  \tup y \, \phi'(\tup x, \tup y, z, \tup a) \] and \[ (\Phi'_\kr)^\D \equiv \existsst \tup u \forallst \tup v \, \theta'(\tup u,\tup v, z, \tup a),\] then $\varphi \equiv \forall z \exists \tup y' \in \tup y \varphi'(\tup x, \tup y', z, \tup a)$ and $\theta \equiv \exists z\forall  \tup y' \in \tup y\ \theta'(\tup x,\tup y',z, \tup a)$. Since $\epastar \vdash \varphi' \leftrightarrow \lnot \theta'$ by induction hypothesis, also $\epastar \vdash \varphi \leftrightarrow \lnot \theta$.
\end{enumerate}
\end{proof}

\begin{remark}
This lemma is the reason why we introduced the system $\ehanststar$ in Remark \ref{hanst}: it would fail if we would let the Krivine negative translation land directly in $\ehaststar$ with $\st(z)_\kr = \lnot \st(z)$. As it is, this lemma yields a quick proof of the soundness of the $\Sh$-interpretation.
\end{remark}

\begin{proof} (Of the soundness of the $\Sh$-interpretation, Theorem \ref{soundnessshoenfield}.) Let $\Phi(\tup a)$ be a formula in the language of $\epaststar$ and let  $\varphi$ and $\theta$ be such that
\begin{eqnarray*}
\Phi^\Sh & \equiv & \forallst \tup x\existsst  \tup y \, \phi(\tup x, \tup y, \tup a), \\
(\Phi_\kr)^\D & \equiv & \existsst \tup x \forallst \tup y \, \theta(\tup x,\tup y, \tup a)
\end{eqnarray*}
and $\epastar \vdash \varphi \leftrightarrow \lnot \theta$, as in Lemma \ref{p:ShD}.

Now, suppose that $\Delta_\intern$ is a set of internal formulas and $\Phi(\tup a)$ is a formula provable in $\epaststar$ from  $\Delta_\intern$. We first apply soundness of the Krivine negative translation (Theorem~\ref{l:kukr}) to see that
\[ \ehanststar + \Delta_\intern^\kr \vdash \Phi^\kr, \]
where $\Phi^\kr \equiv \lnot \Phi_\kr$. So if $(\Phi_\kr)^\D \equiv \existsst \tup x \forallst \tup y \, \theta(\tup x,\tup y, \tup a)$, then
\[ (\Phi^\kr)^\D \equiv \existsst \tup Y \forallst \tup x \exists \tup y \in \tup Y[\tup x] \lnot \theta(\tup x, \tup y, \tup a). \]
It follows from the soundness theorem for $\D$ (Theorem \ref{soundnessDst}) and Remark \ref{hanst} that there is a sequence of closed terms $\tup s$ from $\Tstar$ such that
\[ \ehastar + \Delta_\intern^\kr \vdash \forall \tup x \exists \tup y  \in \tup s[\tup x] \lnot \theta(\tup x, \tup y, \tup a). \]
Since $\epastar \vdash \Delta_\intern^\kr \leftrightarrow \Delta_\intern$ and $\epastar \vdash \varphi \leftrightarrow \lnot \theta$ we have
\[
\epastar + \Delta_{\intern} \vdash\ \forall \tup x\exists \tup y \in \tup t(\tup x)\ \phi(\tup x,\tup y, \tup a),
\]
with $\tup t \equiv \lambda \tup x. \tup s[\tup x]$.
\end{proof}

\subsubsection*{Characteristic principles}

The characteristic principles of our functional interpretation for classical arithmetic are idealization $\I$ (or, equivalently, $\R$: see Section 4.1) and $\HAC_\intern$
\[     \forallst x \existsst y \, \varphi(x,y) \to \existsst F \forallst x \exists y \in F(x)\,  \varphi (x,y),         \]
which is the choice scheme $\HAC$ restricted to internal formulas. To see this, note first of all that we have:

\begin{prop}
For any formula $\Phi$ in the language of $\epaststar$ one has:
\[ \epaststar + \I + \HAC_\intern \vdash \Phi \leftrightarrow \Phi^\Sh. \]
\end{prop}
\begin{proof}
An easy proof by induction on the structure of $\Phi$, using $\HAC_\intern$ for the case of negation and $\I$ (or rather $\R$) in the case of internal universal quantification.
\end{proof}

For the purpose of showing that $\I$ and $\HAC_\intern$ are interpreted, it will be convenient to consider the ``hybrid'' system $\ehanststar + \LEM_\intern$, where $\LEM_\intern$ is the law of excluded middle for internal formulas. For this hybrid system we have the following easy lemma, whose proof we omit:

\begin{lemma} We have:
\begin{enumerate}
\item[1.] $\ehanststar + \LEM_\intern \, \vdash \, \varphi^\ku \leftrightarrow \varphi$, if $\varphi$ is an internal formula in the the language of $\epaststar$.
\item[2.] $\ehanststar + \LEM_\intern + \I \, \vdash \, \I^\ku$.
\item[3.] $\ehanststar + \LEM_\intern + \HAC_\intern + \HGMP \, \vdash \, \HAC^\ku_\intern$.
\end{enumerate}
\end{lemma}

This means we can strengthen Theorem  \ref{soundnessshoenfield} to:

\begin{thm} {\rm (Soundness of the $\Sh$-interpretation, full version.)}\label{fullsoundnessSst} Let $\Phi(\tup a)$ be a formula in the language of $\epaststar$ and suppose $\Phi(\tup a)^\Sh\equiv\forallst \tup x \, \existsst \tup y \, \phi(\tup x, \tup y, \tup a)$. If $\Delta_{\intern}$ is a collection of internal formulas and
\[ \epaststar + \I + \HAC_\intern + \Delta_{\intern} \vdash \Phi(\tup a), \]
then one can extract from the formal proof a sequence of closed terms $\tup t$ in $\Tstar$ such that
\[
\epastar + \Delta_{\intern} \vdash\ \forall \tup x\exists \tup y\in \tup t(\tup x)\ \phi(\tup x,\tup y, \tup a).
\]
\end{thm}
\begin{proof} The argument is a slight extension of the proof of Theorem \ref{soundnessshoenfield}. So, once again, let $\Phi(\tup a)$ be a formula in the language of $\epaststar$ and $\varphi$ and $\theta$ be such that
\begin{eqnarray*}
\Phi^\Sh & \equiv & \forallst \tup x\existsst  \tup y \, \phi(\tup x, \tup y, \tup a), \\
(\Phi_\kr)^\D & \equiv & \existsst \tup x \forallst \tup y \, \theta(\tup x,\tup y, \tup a)
\end{eqnarray*}
and $\epastar \vdash \varphi \leftrightarrow \lnot\theta$, as in Lemma \ref{p:ShD}.

This time we suppose $\Delta_\intern$ is a set of internal formulas and $\Phi(\tup a)$ is a formula provable in $\epaststar$ from $\I + \HAC_\intern + \Delta_\intern$. We first apply soundness of the Kuroda negative translation (Theorem~\ref{soundnesskuroda}), which yields:
\[ \ehanststar + \I^\ku + \HAC^\ku_\intern + \Delta_\intern^\ku \vdash \Phi^\ku. \]
Then the previous lemma implies that:
\[ \ehanststar + \LEM_\intern + \I + \HAC_\intern + \HGMP + \Delta_\intern^\ku \vdash \Phi^\ku. \]
Note that $\ehanststar \vdash \Phi^\ku \leftrightarrow \Phi^\kr$,  $\Phi^\kr \equiv \lnot \Phi_\kr$ and
\[ (\Phi^\kr)^\D \equiv \existsst \tup Y \forallst \tup x \exists \tup y \in \tup Y[\tup x] \lnot \theta(\tup x, \tup y , \tup a). \]
Therefore the soundness theorem for $\D$ (Theorem \ref{soundnessDst}), in combination with Remark \ref{hanst} and the fact that the axiom scheme $\LEM_\intern$ is internal, implies that there is a sequence of closed terms $\tup s$ from $\Tstar$ such that
\[ \ehastar + \LEM + \Delta_\intern^\ku \vdash \forall \tup x \exists \tup y \in \tup s[\tup x] \lnot \theta(\tup x, \tup y, \tup a). \]
Since $\epastar \vdash \LEM$, $\epastar \vdash \Delta_\intern^\ku \leftrightarrow \Delta_\intern$ and $\epastar \vdash \varphi \leftrightarrow \lnot \theta$, we have
\[
\epastar + \Delta_{\intern} \vdash\ \forall \tup x\exists \tup y \in \tup t(\tup x)\ \phi(\tup x,\tup y, \tup a)
\]
with $\tup t \equiv \lambda \tup x. \tup s[\tup x]$.
\end{proof}

The following picture depicts the relation between the various interpretations we have established:
\begin{figure}[hbt]%
\[
\begin{xy}
  \xymatrix{
      \epaststar + \I + \HAC_\intern \ar[rd]^{(\cdot)\ku} \ar[dd]_{(\cdot)^\Sh} &  \\
      {} & \ehanststar+\LEM_\intern+\I+\NCR + \HAC + \HGMP + \HIP_{\forallst} \ar[ld]^{(\cdot)^\D}  \\
                                   \epastar &   {}
  }
\end{xy}\]
\caption{The Shoenfield and negative Dialectica interpretations.}
\label{f:ShD}
\end{figure}

\subsubsection*{Conservation results and the transfer principle}

Theorem \ref{fullsoundnessSst} immediately gives us the following conservation result:

\begin{cor}
$\epaststar + \I + \HAC_\intern$ is a conservative extension of $\epastar$ and hence of $\epa$.
\end{cor}

\subsubsection*{Interpreting DNS for standard arguments}

We already discussed that solving the negative interpretation of $\DNS$ and thereby that
of choice and comprehension was of major interest to proof mining in the introduction and 
implicitly at several occasions in the first chapter.\\
Therefore it seems natural to have a closer look at the corresponding scenario in
our non-standard setting, namely the $\Sh$-interpretation of the $\DNS{\st}$ principle. Recall:
\[
\DNS{\st}\quad :\quad \forallst n^0 \, \exists y^\tau \, \Phi(n, y) \to \exists f^{0 \to \tau} \, \forallst n^0 \, \Phi(n, f(n)).
\]

It turns out that the interpretation (obtained with the help of an automated 
interpreter due to myself) remains very similar to the standard scenario and reads as follows:

\begin{small}
\begin{align*}
 &\existsst \hat U,\hat Y,\tilde N\forallst \hat V,N,\hat X( \\ 
 &\quad \forall \hat n\leq |\hat Y[\hat X,\hat V,N]|, 
                \hat p\leq |\tilde N[\hat X,\hat V,N]| \\
 &\quad \neg\forall \hat b\leq |\hat X[{\tilde N[\hat X,\hat V,N]}_{\hat p}][{\hat Y[\hat X,\hat V,N]}_{\hat n}]| \\
 &\quad \neg\forall \hat a\leq |{\hat Y[\hat X,\hat V,N]}_{\hat n}[{\hat X[{\tilde N[\hat X,\hat V,N]}_{\hat p}][{\hat Y[\hat X,\hat V,N]}_{\hat n}]}_{\hat b}]| \\ 
 &\quad\quad 
    \phi({\tilde N[\hat X,\hat V,N]}_{\hat p},{\hat X[{\tilde N[\hat X,\hat V,N]}_{\hat p}][{\hat Y[\hat X,\hat V,N]}_{\hat n}]}_{\hat b},{{\hat Y[\hat X,\hat V,N]}_{\hat n}[{\hat X[{\tilde N[\hat X,\hat V,N]}_{\hat p}][{\hat Y[\hat X,\hat V,N]}_{\hat n}]}_{\hat b}]}_{\hat a})\ \rightarrow\\
 &\quad \neg\forall \hat m\leq |\hat U[\hat X][\hat V,N]|\\
 &\quad \neg\forall \hat k\leq |\hat V[{\hat U[\hat X][\hat V,N]}_{\hat m}]|,\hat l\leq |N[{\hat U[\hat X][\hat V,N]}_{\hat m}]| \\ 
 &\quad\quad \phi({N[{\hat U[\hat X][\hat V,N]}_{\hat m}]}_{\hat l},{\hat U[\hat X][\hat V,N]}_{\hat m}[{N[{\hat U[\hat X][\hat V,N]}_{\hat m}]}_{\hat l}],{\hat V[{\hat U[\hat X][\hat V,N]}_{\hat m}]}_{\hat k})).
\end{align*}
\end{small}

In particular, note that the equations which need to be solved correspond one to one to the standard case (see~\ref{ss:DNS} or~\cite{Kohlenbach08}), except that here, of course, we have to use our sequence application $\Lambda$. In fact, to obtain a solution we might need to define a modified version of the bar recursor, similarly as we did for normal recursion $\tup{\mathcal{R}}_{\tup{\rho}}$.
 % <-- make this an example above

%\section{Conclusion and plans for future work}

We hope this paper lays the groundwork for future uses of functional interpretations to analyse nonstandard arguments and systems. There are many directions, both theoretical and applied, in which one could further develop this research topic. We conclude this paper by mentioning a few possibilities which we would like to take up in future research.

First of all, we would like to see if the interpretations that we have developed in this paper could be used to ``unwind'' or ``proof-mine'' nonstandard arguments. Nonstandard arguments have been used in areas where proof-mining techniques have also been successful, such as metric fixed point theory (for methods of nonstandard analysis applied to metric fixed point theory, see \cite{aksoykhamsi90, kirk03}; for application of proof-mining to metric fixed point theory, see \cite{briseid09, gerhardy06, kohlenbach05, kohlenbachleustean03, kohlenbachleustean10, leustean07}) and ergodic theory (for a nonstandard proof of an ergodic theorem, see \cite{kamae82}; for applications of proof-mining to ergodic theory, see \cite{avigad09, avigadgerhardytowsner10, gerhardy08, gerhardy10, kohlenbach11, kohlenbachleustean09, safarik11}), therefore this looks quite promising. For the former type of applications to work in full generality, one would have to extend our functional interpretation to include types for abstract metric spaces, as in \cite{gerhardykohlenbach08, kohlenbach05b}.

But there are also a number of theoretical questions which still need to be answered. Several have been mentioned already: for example, mapping the precise relationships between the nonstandard principles that we have introduced. Another question was whether $\epaststar + \I + \HAC_\intern + \TPA$ is conservative over $\epastar$. Another question is whether our methods allow one to prove conservativity results over $\weha$ and $\wepa$ as well: this will be important if one wishes to combine the results presented here with the proof-mining techniques from \cite{Kohlenbach08}. 

In addition, we would also like to understand the use of saturation principles in nonstandard arguments. These are of particular interest for two reasons: first, they are used in the construction of Loeb measures, which belong to one of the most successful nonstandard techniques. Secondly, for certain systems it has turned out that extending them with saturation principles has resulted in an increase in proof-theoretic strength (see \cite{hensonkeisler86, keisler07}). 

The general saturation principle is
\[ \SAT: \quad \forallst x^\sigma \, \exists y^\tau \, \Phi(x, y) \to \exists f^{\sigma \to \tau} \, \forallst x^\sigma \, \Phi(x, f(x)). \]
Whether this principle has a $\D$-interpretation within G\"odel's $\Tstar$, we do not know; but
\[ \CSAT: \quad \forallst n^0 \, \exists y^\tau \, \Phi(n, y) \to \exists f^{0 \to \tau} \, \forallst n^0 \, \Phi(n, f(n)) \]
has and that seems to be sufficient for the construction of Loeb measures. Interpreting $\CSAT$ and $\SAT$ in the classical context using the $\Sh$-interpretation is probably quite difficult and it is possible that they require some form of bar recursion. We hope to be able to clarify this in future work.


\subsection{Countable saturation}\label{ss:dst:csat}

The main idea behind providing a functional interpretation for nonstandard proofs, is that  we would like to see if the interpretation that we have developed in this chapter could be used to ``unwind'' or ``proof-mine'' nonstandard arguments. Nonstandard arguments have been used in areas where proof-mining techniques have also been successful, such as metric fixed point theory (for methods of nonstandard analysis applied to metric fixed point theory, see \cite{aksoykhamsi90, kirk03}; for application of proof-mining to metric fixed point theory, see \cite{briseid09, Gerhardy06, Kohlenbach05fpt, kohlenbachleustean03, kohlenbachleustean10, leustean07}) and ergodic theory (for a nonstandard proof of an ergodic theorem, see \cite{kamae82}; for applications of proof-mining to ergodic theory, see \cite{avigad09, AGT10, gerhardy08, Gerhardy2010, Kohlenbach2011, kohlenbachleustean09, Safarik(11)}), therefore this looks quite promising. For the former type of applications to work in full generality, one would have to extend our functional interpretation to include types for abstract metric spaces, as in \cite{GK08, Kohlenbach05meta}.

Towards that purpose, one should extend our methods to allow one to prove conservativity results over $\weha$ and $\wepa$ as well: this will be important if one wishes to combine the results presented here with the proof-mining techniques from \cite{Kohlenbach08}. 

Most importantly for now, we would like to understand the use of saturation principles in nonstandard arguments. These are of particular interest since they are used in the construction of Loeb measures, which belong to one of the most successful nonstandard techniques. 
The general saturation principle is
\[ \SAT: \quad \forallst x^\sigma \, \exists y^\tau \, \Phi(x, y) \to \exists f^{\sigma \to \tau} \, \forallst x^\sigma \, \Phi(x, f(x)). \]
Whether this principle has a $\D$-interpretation within G\"odel's $\Tstar$, we do not know; but
\[ \CSAT: \quad \forallst n^0 \, \exists y^\tau \, \Phi(n, y) \to \exists f^{0 \to \tau} \, \forallst n^0 \, \Phi(n, f(n)) \]
has and that seems to be sufficient for the construction of Loeb measures. However, as in the standard scenario, the more interesting
context here is the classical one. Interpreting $\CSAT$ and $\SAT$ in using the $\Sh$-interpretation is actually quite difficult and it is most probable that the analysis will require some form of bar recursion. \\
Actually, as we can see already on the formalization of $\CSAT$ above, it turns out that the situation is very similar as discussed in section~\ref{ss:AC} regarding $\AC$.



We recall:

\begin{align*}
\I\ &:\quad \forallst x^\sigma \exists y^\tau \forall i \leq |x| \, \varphi((x)_i, y) \to \exists y^\tau \forallst x^\sigma \, \varphi(x, y),\\
\HAC_\intern\ &:\quad  \forallst x \existsst y \, \phi(x,y) \to \existsst F \forallst x \exists i\leq |F(x)|\,  \phi (x,F(x)_i), \quad F(x)_i:\equiv(F(x))_i, \\
\AC_0^{\st}\ &:\quad \forallst n^0 \existsst x \Phi(n,x)\ \rightarrow \existsst f \forallst n^0 \Phi(n,f(n)),\\
\CSAT\ &:\quad \forallst n^0 \, \exists y^\tau \, \Phi(n, y) \to \exists f^{0 \to \tau} \, \forallst n^0 \, \Phi(n, f(n))
\end{align*}

\begin{thm}[Main claim]
%More precisely:
\[
\epast+\I+\HAC_\intern+\AC_0^{\st}\ \vdash\ \CSAT.
\]
\end{thm}

\begin{proof}
Assuming \[\epast+\I+\HAC_\intern\ \vdash\  \Phi(x,y) \leftrightarrow \forallst u\existsst v\ \phi(u,v,x,y),\]
(which follows from Proposition 8.5 in Berg et al.) it suffices to show that
\begin{align}
\forallst x^0\exists y\forallst u\existsst v\ \phi(u,v,x,y) \label{e:s5p1}
\end{align}
implies
\begin{align}
\exists \tilde y^1\forallst x^0,u\existsst v\ \phi(u,v,x,\tilde y(x)) \label{e:s5p2}.
\end{align}
By $\HAC_\intern$ we get that \eqref{e:s5p1} implies
\[
\forallst x^0\exists y\existsst V\forallst u\exists i\leq |V(u)|\ \phi(u,V(u)_i,x,y),
\]
which is by $\IL$ equivalent to
\[
\forallst x^0\existsst V\exists y\forallst u\exists i\leq |V(u)|\ \phi(u,V(u)_i,x,y),
\]
and by $\AC_0^{\st}$ this implies that
\begin{align*}
\existsst V\forallst x^0\exists y\forallst u\exists i\leq |V(x,u)|\ \phi(u,V(x,u)_i,x,y). \tag{\ref{e:s5p1}'}\label{e:s5f1}
\end{align*}
We can strengthen \eqref{e:s5p2} to an under $\HAC_\intern$ (on $u$) and $\AC_0^{\st}$ (on $x$) equivalent statement
\[
\exists \tilde y\existsst V \forallst x^0,u\exists i\leq |V(x,u)|\ \phi(u,V(x,u)_i,x,\tilde y(x)),
\]
which is by $\IL$ equivalent to
\[
\existsst V \exists \tilde y\forallst x^0,u\exists i\leq |V(x,u)|\ \phi(u,V(x,u)_i,x,\tilde y(x)). 
\]
By $\I$, this follows from
\[
\existsst V \forallst x^0,u\exists \tilde y\ \forall j\leq |x|,k\leq |u|\ \exists i\leq |V(x_j,u_k)|\quad \phi(u_k,V(x_j,u_k)_i,x_j,\tilde y(x_j)).
%\existsst V \forallst x^0,u\exists \tilde y\ \forall j\leq |x|\ \forall k\leq |x|\ \exists i\leq |V(x_j,u)|\quad \phi(u,V(x_j,u)_i,x_j,\tilde y(x_j)).
 \tag{\ref{e:s5p2}'}\label{e:s5f2}
\]
Hence it suffices to show that \eqref{e:s5f1} implies \eqref{e:s5f2}.\\
Now to do so, let some standard $V$ satisfy \eqref{e:s5f1}, i.e. we have 
\begin{align}
\forallst x^0\exists y\forallst u\exists i\leq |V(x,u)|\ \phi(u,V(x,u)_i,x,y), \label{e:allX}
\end{align}
and fix an arbitrary but standard $x$ of type $0$. Then we have for some finite (standard) natural number $n$ that
\[
x = \langle x_1,\ldots,x_n \rangle,
\]
with $x_i^0$ standard. So by \eqref{e:allX} we have for each $x_j\in\{x_1,\ldots,x_n\}$ an $y_j$ s.t.
\[
\forallst u\exists i\leq |V(x_j,u)|\ \phi(u,V(x_j,u)_i,x_j,y_j), 
\]
in other words we obtain (simply consider $y=\langle y_1,\ldots,y_n \rangle$)
\[
\exists y \forallst u \forall j\leq |x| \exists i\leq |V(x_j,u)|\ \phi(u,V(x_j,u)_i,x_j,y_j).
\]
Finally, let $\tilde y$ be the function $x_j\mapsto y_j$, (for $1\leq j\leq n$), and constant $0$ otherwise.
%about finite choice
\footnote{Note that such a $\tilde y$ is primitive recursively definable in $x$ and $y$, in Particular it is a term in $\T_0$ with parameters $x$ and $y$. However, it is not necessarily standard, as the parameter $y$ -- though its size is finite -- does not need to code only standard elements and, therefore, does not need to be standard itself.}
Then we have that
\[
\forallst u \forall j\leq |x| \exists i\leq |V(x_j,u)|\ \phi(u,V(x_j,u)_i,x_j,\tilde y(x_j)),
\]
so in particular also
\[
\exists \tilde y\forallst u \forall j\leq |x| \exists i\leq |V(x_j,u)|\ \phi(u,V(x_j,u)_i,x_j,\tilde y(x_j)),
\]
and hence
\[
\forallst u\exists \tilde y\ \forall j\leq |x|,k\leq |u|\  \exists i\leq |V(x_j,u)|\ \phi(u,V(x_j,u)_i,x_j,\tilde y(x_j)),
\]
and therefore also \eqref{e:s5f2}.
\end{proof}
