\subsection{Obtaining a bound}\label{s:terms}
%\begin{dfn}
%By a formula $\phi$ containing
%\[
%t_1 \leq_l t_2,
%\]
%we mean that there is a small natural number $c$, s.t. for all $l$ the formula $\phi$ holds
%with\footnote{instead of $t_1 \leq_l t_2$} $t_1 \leq t_2 + 2^{-l}$ (note that we require the existance of
%a fixed $c$ independently on all other parameters and/or variables).
%\end{dfn}
The goal of this section is to roughly describe how to find a bound for $m$ in
\[
\forall l,g \exists m\ \big( \|A_m-A_{m+g(m)}\|\leq 2^{-l} \big).
\]
Similarly as before, for better understanding we leave some technical details
for later sections and disregard the monotonicity of the bounds as well as some
small corrections needed in the exponent of $2^{-l}$.
We handle these two aspects very carefully in the sections~\ref{s:Main} and~\ref{s:Lemmas}, where we make also
all of the following steps more explicit.\\
Let the assumptions in the proof of Theorem~\ref{t:fin21ar} hold and let $K$, 
$B$ and $C$ be as before as well.\\
Furthermore, we assume that $N_0$ and $P_0$ are the witnessing terms for~\eqref{w:1} (note that in
fact this means rather that we have a witness for Kreisel's no-counterexample interpretation -- n.c.i. see~\cite{Kreisel1951, Kreisel1959} -- of the
convergence of $\|X_n\|$ in the first place) and~\eqref{w:2} as given in~\cite{Kohlenbach08}, i.e. we have that:
\begin{align*}
\forall l,h \forall m,n\in[N_0(l,h) ; N_0(l,h)&+g(N_0(l,h))]\ \big( 
\langle X_{m+k},X_{n+k}\rangle \leq \langle X_{m},X_{n}\rangle + 2^{-l} \big)
\end{align*}
and
\[
\forall l,f \ \big( C(l,n_l,P_0(l,f))\leq C(l,n_l,f(P_0(l,f))) + 2^{-l} \big)
\]
for any $n_l$ satisfying~\eqref{w:1}.\\ This structure accounts for the already mentioned nested iteration.  
Eventually, we have to specify a counterfunction $f$ s.t. it is sufficient that this inequality holds for that particular $f$. 
This $f$ has to depend on $n_l$, which will obviously be defined as an iteration, and $f$ itself will be iterated by $P_0$.\\
%Also, from now on, we will use the more formal $\lambda$-abstraction rather than the notation we used above. 
%For example we write $F(\tup x)=_1\lambda n.n^2$ rather than $F(\tup x)(n)=n^2$, to express that $F$ is a functional, 
%which given the arguments $\tup x$ returns the number theoretic square function
%(see e.g. Lemma 3.15 in~\cite{Kohlenbach08} for a precise definition within a subsystem of $\AHilb$). \\
To obtain a bound for $m$, we follow the proof from the last section backwards. We define 
\[
M_0(l,n,p):=2(p + n + K(l))B2^l,
\]
since then we have that $\|Z(l,n_l,p_l,M_0(l,n_l,p_l))-A_{m+1}\|\leq 2^{-l}$, for the 
right values of $l$, $n_l$ and $p_l$ (which we don't know yet).\\
From the proof we can infer that the largest $p$ needed in~\eqref{w:2} is $K(l)+m+g(m)+n_l+p_l$ (for details
see proof of Lemma~\ref{l:Zs}.\eqref{e:Zup}), therefore we need that
\[f(P_0(l,f))=K(l)+m+g(m)+n_l+P_0(l,f),\]
(where $m$ and $m+g(m)$ correspond to the $m$ and $n$ in~\eqref{w:1}). \\
Moreover, we can see that the largest $m$ or $n$ needed in~\eqref{w:1} is $n_l+p_l+m+g(m)+K(l)$ (for details
see proof of Lemma~\ref{l:scpb} and its application in proof of Lemma~\ref{l:Zs}.\eqref{e:Zdown}). So we need that
$h(N_0(h,l))=p_l+m+g(m)+K(l)$. \\
Keeping in mind that $N_0(l,h)$ corresponds to $n_l$ (and to $n$ below) and $P_0(l,f)$ to $p_l$ (and to $p$ below) 
we obtain\footnote{
Here and in the following, when considering a complex term $t[a_1,\ldots,a_n]$ in several variables $a_1,\ldots,a_n$,
the notation $\lambda a_i.t[a_1,\ldots,a_n]$ defines (as usual in logic) the function $a_i\mapsto t[a_1,\ldots,a_n]$.}
\begin{align*}
 h(l,g) = \lambda n\ .\ &\underbrace{P_0(l,\lambda p\ .\ K(l)+M_0(l,n,p)+g(M_0(l,n,p))+p+n)}_{p_l} + \\
&+M_0(p_l,n,l) + g(M_0(p_l,n,l))+K(l),
\end{align*}
and define
\[
N(l,g):=N_0(l, h(l,g)).
\]
Given $N(l,g)$, which corresponds to $n_l$, and again keeping in mind that $P_0(l,f)$ corresponds to $p_l$ (and to $p$ below),
we can define $P(l,g)$ as well:
\[
P(l,g):=P_0(l, f(l,g)),
\]
where
$ f(l,g) = \lambda p\ .\ K(l)+M_0(p,N(l,g),l)+g(M_0(p,N(l,g),l))+p+N(l,g).$
Finally we can define the desired witness for $m$ as follows:
\[
M(l,g):=M_0( l, N(l,g), P(l,g) ).
\]



