\documentclass[1p]{elsarticle}
\usepackage{amssymb,latexsym}
\usepackage{enumerate}


%%%% Put my macros here:
%%%%%%%%%%%%%%%%%%%%%%%%%%%   usepackage   %%%%%%%%%%%%%%%%%%%%%%%%%%%%%%%%%%%%%%%
%%%%%%%%%%%%%%%%%%%%%%%%%%%   usepackage   %%%%%%%%%%%%%%%%%%%%%%%%%%%%%%%%%%%%%%%
%%%%%%%%%%%%%%%%%%%%%%%%%%%   usepackage   %%%%%%%%%%%%%%%%%%%%%%%%%%%%%%%%%%%%%%%
%\usepackage{graphicx, color}
\usepackage{amsmath,amsthm,amssymb,amscd}
\usepackage[all]{xy}

\newcommand{\usftext}[1]{\textsf{\upshape #1}}

%%%%%%%%%%%%%%%%%%%%%%%%%%%   Layout   %%%%%%%%%%%%%%%%%%%%%%%%%%%%%%%%%%%%%%%
%%%%%%%%%%%%%%%%%%%%%%%%%%%   Layout   %%%%%%%%%%%%%%%%%%%%%%%%%%%%%%%%%%%%%%%
%%%%%%%%%%%%%%%%%%%%%%%%%%%   Layout   %%%%%%%%%%%%%%%%%%%%%%%%%%%%%%%%%%%%%%%

% \setlength{\parskip}{1.5ex plus .3ex minus .2ex}        % Distance between paragraphs
%                                                         % +- indicates tolerance for TeX
% 

%%margins      
%\newlength{\topmarginmod}
%\newlength{\bottommarginmod}
%\topmarginmod=0.5in
%\bottommarginmod=0.4in
%\addtolength{\topmargin}{-\topmarginmod}
%\addtolength{\textheight}{\topmarginmod}
%\addtolength{\textheight}{\bottommarginmod}
%% so it gets all on a page
%\textwidth=13.4cm
%\oddsidemargin=0.9cm
%%should be pagewidth-1in-\oddsidemargin-\textwidth-1in~=21cm-2.8cm-1cm-13cm-2.8cm=1.4cm
%\evensidemargin=1.92cm

\newcommand{\todo}[1]{{\it #1}
  \marginpar{\center\texttt{ToDo}}}    % Issues to clarify

%%%%%%%%%%%%%%%%%%%%%%%%%%%%% get a blank page %%%%%%%%%%%%%%%%%%%%%%%%



%%%%%%%%%%%%%%%%%%%%%%%%%%%%%%%        %%%%%%%%%%%%%%%%%%%%%%%%%%%%%%%%%
%%%%%%%%%%%%%%%%%%%%%%%%%%%%%%% Quote  %%%%%%%%%%%%%%%%%%%%%%%%%%%%%%%%%
%%%%%%%%%%%%%%%%%%%%%%%%%%%%%%%        %%%%%%%%%%%%%%%%%%%%%%%%%%%%%%%%%

\newcommand{\theQuote}[1]{
\begin{center}
\begin{minipage}{0.7\textwidth} 
{\em #1} 
\end{minipage}
\end{center}
}




%%%%%%%%%%%%%%%%%%%%%%%%%%%   Remarks   %%%%%%%%%%%%%%%%%%%%%%%%%%%%%%%%%%%%%%
%%%%%%%%%%%%%%%%%%%%%%%%%%%   Remarks   %%%%%%%%%%%%%%%%%%%%%%%%%%%%%%%%%%%%%%
%%%%%%%%%%%%%%%%%%%%%%%%%%%   Remarks   %%%%%%%%%%%%%%%%%%%%%%%%%%%%%%%%%%%%%%

%\newcommand{\rem}{{\tt ?}\marginpar{\Large \tt\centering ?}}
\newcommand{\Rem}[1]{{\tt ?}\marginpar{\raggedright #1}}
\newcommand{\OK}{\marginpar{\begin{center}\unitlength1.5em
        \begin{picture}(1,1) \put(0.5,0.5){\circle{1}}
        \put(0.5,0.4){\makebox(0,0){$\smile$}}
        \put(0.4,0.7){\makebox(0,0){$\cdot$}}
        \put(0.6,0.7){\makebox(0,0){$\cdot$}}
        \end{picture}\end{center}}}
\newcommand{\nOK}{\marginpar{\begin{center}\unitlength1.5em
        \begin{picture}(1,1) \put(0.5,0.5){\circle{1}}
        \put(0.5,0.5){\makebox(0,0){$\ddot{\frown}$}}
        \end{picture}\end{center}}}

\newenvironment{rmk}{\paragraph{Remark.}\it}{\\}

%%%%%%%%%%%%%%%%%%%%%%%%%%%   equations   %%%%%%%%%%%%%%%%%%%%%%%%%%%%%%%%%%%%%%
\newcommand{\be}[1][{e:\arabic{equation}}] { \begin{equation}\label{#1} }
\newcommand{\ee} { \end{equation} }



%%%%%%%%%%%%%%%%%%%%%%%%%%%%   General Maths   %%%%%%%%%%%%%%%%%%%%%%%%%%%%%%%
%%%%%%%%%%%%%%%%%%%%%%%%%%%%   General Maths   %%%%%%%%%%%%%%%%%%%%%%%%%%%%%%%
%%%%%%%%%%%%%%%%%%%%%%%%%%%%   General Maths   %%%%%%%%%%%%%%%%%%%%%%%%%%%%%%%

%%stuff - mainly KL
\DeclareMathOperator{\lh}{lh}  %length of encoding of a finite sequence
\DeclareMathOperator{\TMaj}{Maj}
\DeclareMathOperator{\TAN}{W}
\DeclareMathOperator{\Id}{id}
%\DeclareMathOperator{\P}{P}  %secured

%%commands
\renewcommand{\emptyset}{\varnothing}

\newcommand{\ORi}[1]{\ensuremath{\bigwedge^{#1}_{i=1}}}


\newcommand{\RR}{\ensuremath{\mathbb{R}}}
\newcommand{\NN}{\ensuremath{\mathbb{N}}}
\newcommand{\QQ}{\ensuremath{\mathbb{Q}}}
\newcommand{\II}{\ensuremath{\mathbb{I}}}

\newcommand{\zero}{\ensuremath{\mathbf0}}
\newcommand{\one}{\ensuremath{\mathbf1}}
\newcommand{\two}{\ensuremath{\mathbf2}}

\newcommand{\xor}{\ensuremath{\dot\vee}}

%%%%%%%%%%%%%%%%%%%%%%%%%   Proof Theory   %%%%%%%%%%%%%%%%%%%%%%%%%%%%%%%
%%%%%%%%%%%%%%%%%%%%%%%%%   Proof Theory   %%%%%%%%%%%%%%%%%%%%%%%%%%%%%%%
%%%%%%%%%%%%%%%%%%%%%%%%%   Proof Theory   %%%%%%%%%%%%%%%%%%%%%%%%%%%%%%%

%% stuff
%\input prooftree

\DeclareMathOperator{\maj}{maj} %``majorizes
\DeclareMathOperator{\smaj}{s-maj} %``strongly majorizes
\DeclareMathOperator{\K}{K} %K_A from Howard's WKL ND-int
\DeclareMathOperator{\I}{I} %the ``in Interval predicate from BW



%% Types
\newcommand{\Tp}{\ensuremath{\emph{\protect\textbf{T}}}} %set of finite types T
\newcommand{\PT}{\ensuremath{\emph{\protect\textbf{P}}}} %set of pure types P
\newcommand{\tp}[1]{\ensuremath{^\mathbf{#1}}}

%% Models
\newcommand {\SetO}  { \ensuremath{\mathcal{S} } }
\newcommand {\Som}  { \ensuremath{\SetO^\omega} }
\newcommand {\Set}  { \Som }
\newcommand {\ContO}  { \ensuremath{\mathcal{C} } }
\newcommand {\Cont}  { \ensuremath{\ContO^\omega} }
\newcommand {\MajO}  { \ensuremath{\mathcal{M} } }
\newcommand {\Maj}  { \ensuremath{\MajO^\omega} }

%% Special sets
\newcommand{\universal}{\ensuremath{\emph{\protect\textbf{U}}}} %set of universal axioms

%% Systems
\newcommand{\Ax}{\ensuremath{\mathcal{A}^\omega}} %GnA iaft
\newcommand{\AHilb}{\ensuremath{\mathcal{A}^\omega[X,\langle\cdot,\cdot\rangle]}} %A Hilbert space
\newcommand{\AHilbS}{\ensuremath{\mathcal{A}^\omega[X,\langle\cdot,\cdot\rangle,S]}} %A Hilbert sp. + S
\newcommand{\GA}{\ensuremath{\usftext{G}_n\usftext{A}^\omega}} %GnA iaft
\newcommand{\weha}{\ensuremath{{\usftext{WE-HA}}^{\omega}}} % WE - HA iaft
\newcommand{\wepa}{\ensuremath{{\usftext{WE-PA}}^{\omega}}} % WE - PA iaft
\newcommand{\HA}{\ensuremath{{\usftext{HA}}}} % HA 
\newcommand{\PA}{\ensuremath{\usftext{PA}}} % PA 
\newcommand{\ha}{\ensuremath{{\usftext{HA}}^\omega}} % HA iaft
\newcommand{\pa}{\ensuremath{{\usftext{PA}}^\omega}} % PA iaft
\newcommand{\epa}{\ensuremath{{\usftext{E-PA}}^\omega}} % E - PA iaft
\newcommand{\eha}{\ensuremath{{\usftext{E-HA}}^\omega}} % E - HA iaft
\newcommand{\hrrepa}{\ensuremath{\widehat{\usftext{E-PA}}^\omega\kleene}} %hrr E - PA iaft
\newcommand{\hrreha}{\ensuremath{\widehat{\usftext{E-HA}}^\omega\kleene}} %hrr E - HA iaft
\newcommand{\rreha}{\ensuremath{\usftext{E-HA}^\omega\kleene}} %rr E - HA iaft
\newcommand{\rrweha}{\ensuremath{\usftext{WE-HA}^\omega\kleene}} %rr WE - HA iaft
\newcommand{\kleene}{\ensuremath{\!\!\!\restriction}}   % upper arrow
\newcommand{\hrrwepa}{\ensuremath{\widehat{\usftext{WE-PA}}^\omega\kleene}} %hrr WE - PA iaft
\newcommand{\hrrweha}{\ensuremath{\widehat{\usftext{WE-HA}}^\omega\kleene}} %hrr WE - HA iaft

\newcommand{\HAS}{\ensuremath{\usftext{HAS}}} %second order logic
\newcommand{\HAH}{\ensuremath{\usftext{HAH}}} %higher -/-
\newcommand{\ACA}{\ensuremath{\usftext{ACA}}} %
\newcommand{\RCA}{\ensuremath{\usftext{RCA}}} %

%% Principles
\newcommand{\IA}{\ensuremath{\usftext{IA}}} %induction schema
\newcommand{\IP}{\ensuremath{\usftext{IP}}} %induction principle
\newcommand{\IR}{\ensuremath{\usftext{IR}}} %induction rule
\newcommand{\BR}{\ensuremath{\usftext{BR}}} %induction rule


\newcommand{\IPP}{\ensuremath{\usftext{IPP}}}
\newcommand{\PCM}{\ensuremath{\usftext{PCM}}}
\newcommand{\LEM}{\ensuremath{\Sigma^0_1\usftext{-LEM}}}
\newcommand{\lLEM}{\ensuremath{\usftext{LEM}}}
\newcommand{\BW}{\ensuremath{\usftext{BW}}}
\renewcommand{\AA}{\ensuremath{\usftext{AA}}}
\newcommand{\Limsup}{\ensuremath{\usftext{Limsup}}}
\newcommand{\DNS}{\ensuremath{\usftext{DNS}}}
\newcommand{\CA}{\ensuremath{\usftext{CA}}}
\newcommand{\QF}{\ensuremath{\usftext{QF}}}
\newcommand{\QFm}{\ensuremath{\usftext{QF-}}}
\newcommand{\CAhut}{\ensuremath{\widehat{\CA}}}

\newcommand{\AC}{\ensuremath{\usftext{AC}}} 
\newcommand{\ER}{\ensuremath{\usftext{ER}}} 

\newcommand{\WKL}{\ensuremath{\usftext{WKL}}}
\newcommand{\FAN}{\ensuremath{\usftext{FAN}}}

%% General abreviations
\newcommand{\PiL}{\ensuremath{\Pi^0_1}} 
\newcommand{\PiLm}{\ensuremath{\Pi^0_1\usftext{-}}} 
\newcommand{\SiL}{\ensuremath{\Sigma^0_1}} 
\newcommand{\SiLm}{\ensuremath{\Sigma^0_1\usftext{-}}} 
\newcommand{\m}{\ensuremath{\usftext{-}}}


%% for WKL

\newcommand{\BTree}{\ensuremath{\usftext{BinTree}}}
\newcommand{\BFunc}{\ensuremath{\usftext{BinFunc}}}
\newcommand{\UnBounded}{\ensuremath{\usftext{Unbounded}}}
%\newcommand{\Bounded}{\ensuremath{\usftext{Bounded}}}
\newcommand{\Sec}{\ensuremath{\usftext{Sec}}} % Boundedly secured
\newcommand{\BSec}{\ensuremath{\usftext{BarSec}}} % Boundedly secured at bar k
\newcommand{\BSecA}{\ensuremath{\usftext{BarSec}_A}} % Boundedly secured at bar K_A

\newcommand{\B}{\ensuremath{\usftext{B}}} %bar recursor
\newcommand{\rB}{\ensuremath{\usftext{B'}}} %restricted bar recursor
\newcommand{\R}{\ensuremath{\usftext{R}}} %recursor
\newcommand{\bPhi}{                       %special bar recursor
 \raisebox{-1.0pt} {
   \ensuremath{\usftext{\Large {\!$\Phi$\!}}}
 }
}

\newcommand{\T}{\ensuremath{\mathcal{T}}} %G�els T
\newcommand{\M}{\ensuremath{\usftext{M}^\omega}} %Markov principle iaft
\renewcommand{\H}{\ensuremath{\usftext{H}}} %Funny Howards argument

\newcommand{\proves}{\vdash}  %proves |-
\newcommand{\forces}{\Vdash}  %||-
\renewcommand{\models}{\vDash}  %|=



\newcommand{\tup}{\underline} %tuple
\newcommand{\atup}{\ensuremath{\,\underline}} %tuple as a parameter

\newcommand{\Tif}{\text{if}}
\newcommand{\Telse}{\text{else}}

%%%%%%%%%%%%%%%%%%%%%%%%%   Theorems   %%%%%%%%%%%%%%%%%%%%%%%%%%%%%%%%%%%
%%%%%%%%%%%%%%%%%%%%%%%%%   Theorems   %%%%%%%%%%%%%%%%%%%%%%%%%%%%%%%%%%%
%%%%%%%%%%%%%%%%%%%%%%%%%   Theorems   %%%%%%%%%%%%%%%%%%%%%%%%%%%%%%%%%%%


\theoremstyle{plain}
\newtheorem{thm}{Theorem}[section]
\newtheorem{lemma}[thm]{Lemma}
%\newtheorem*{lemma*}{Lemma}
\newtheorem{prop}[thm]{Proposition}
\newtheorem{cor}[thm]{Corollary}
\newtheorem{con}[thm]{Conjecture}

\theoremstyle{definition}
\newtheorem{dfn}[thm]{Definition}

\theoremstyle{remark}
\newtheorem*{remark}{Remark}
\newtheorem*{eg}{Example}

%\newenvironment{lemma*}[2][]{\noindent{\bf Recall Lemma~\ref{#2}} ({#1}).}{\\}
\newenvironment{lemma*}[2][]{\noindent{\bf Recall Lemma~\ref{#2}} ({#1}). \begin{it}}{\end{it}\\}



%  from Klaus:
 \renewenvironment{proof}[1][]{\noindent{\bf Proof{#1}. }}{\nopagebreak[4]{\hspace*{\fill}
%   \rule{1.2ex}{1.2ex} % Full big box
  $\Box$              % Empty box
 }{\vspace{2ex}}}

%%%%%%%%%%%%%%%% get it stylish - shortcuts %%%%%%%%%%%%%%%%%%%%%%
%%%%%%%%%%%%%%%% get it stylish - shortcuts %%%%%%%%%%%%%%%%%%%%%%
%%%%%%%%%%%%%%%% get it stylish - shortcuts %%%%%%%%%%%%%%%%%%%%%%

\renewcommand{\phi}{\varphi}
\newcommand{\lb}{\linebreak[0]}
\newcommand{\pb}{\pagebreak[0]}
\newcommand{\nbd}{\nobreakdash-}

\newcommand{\lOrd}[1]{\text{$\!\!$
\begin{smaller}<\end{smaller}\nolinebreak[4] $\!#1$\hspace{-3pt}
}}

\newcommand{\lOrdm}[1]{\text{
\lOrd{#1}\nbd 
%\hspace{-4pt}
}}

%\makeatletter
\def\Ddots{\mathinner{\mkern1mu\raise\p@
\vbox{\kern7\p@\hbox{.}}\mkern2mu
\raise4\p@\hbox{.}\mkern2mu\raise7\p@\hbox{.}\mkern1mu}}
%\makeatother

\newcommand{\embeded}{\hookrightarrow}




%% A numbered theorem with a fancy name:

\newtheorem{mainthm}[thm]{Main Theorem}

%% Numbered objects of "non-theorem" style (text roman):

\theoremstyle{definition}
\newtheorem{defin}[thm]{Definition}
\newtheorem{rem}[thm]{Remark}
\newtheorem{exa}[thm]{Example}

\begin{document}


%%%%%%%%%%%


\title{A quantitative nonlinear strong ergodic theorem for Hilbert spaces}
\author{Pavol Safarik}
\ead{pavol.safarik@gmail.com}
\address{Department of Mathematics, Technische Universit\"at Darmstadt, Schlossgartenstra{\ss}e 7, 64289 Darmstadt, Germany}

\date{}

\begin{abstract}
We give a quantitative version of a strong nonlinear ergodic theorem for (a class of possibly even discontinuous) 
selfmappings of an arbitrary subset of a Hilbert space due to R.~Wittmann and 
outline how the existence of uniform bounds in such quantitative formulations of ergodic theorems
can be proved by means of a general logical metatheorem.
In particular these bounds depend neither on the operator nor on the initial point.
Furthermore, we extract such
uniform bounds in our quantitative formulation of Wittmann's theorem, implicitly using the 
proof-theoretic techniques on which the metatheorem is based. However, we present our result
and its proof in analytic terms without any reference to logic as such. Our bounds turn out to involve
nested iterations of relatively low computational complexity. While in theory these kind
of iterations ought to be expected, so far this seems to be the first occurrence
of such a nested use observed in practice. %unprecedented form.
\end{abstract}

\begin{keyword}
Proof mining \sep uniform bounds \sep functionals of finite type \sep nonlinear ergodic theory \sep strong convergence \sep Ces{\`a}ro means \sep hard analysis.\\
\MSC[2010]{03F10, 47H25, 37A30}.
\end{keyword}


\maketitle

\section*{Introduction}
\label{s:intro}
\markboth{\small\sffamily\fontseries{c}\selectfont {Introduction}}
         {\small\sffamily\fontseries{c}\selectfont {Introduction}}
\addcontentsline{toc}{section}{Introduction}
%
%
\mySubsection{Foreword}
In \cite{Tao07}, a recent Fields medallist, Terence Tao, 
published some thoughts on the relation between
 ``hard (quantitative, finitary)'' and ``soft (qualitative, 
infinitary)'' analysis. In this essay, he also emphasized
the importance of the so called ``hard'' analysis:
\theQuote{ ... I therefore feel that it is often profitable for a practitioner 
of one type of analysis to learn about the other, as they both offer their own 
strengths, weaknesses, and intuition, and knowledge of one gives 
more insight into the workings of the other. ... }
His point is illustrated by the proof of Theorem~1.6 in~\cite{tao-2007},
which is carried out in the context of ``finitary ergodic theory'': 
\theQuote{ ... The main advantage of working in a finitary setting ... is that
the underlying dynamical system becomes extremely explicit. ... }
He goes on to connect the finitisation to the methods we will employ
in this thesis:
\theQuote{ ... In proof theory, this finitisation is known as G\"odel functional
interpretation ... }
In the case of convergence theorems Tao calls the finitary formulation
metastability and the corresponding explicit content rate(s) of metastability.
It was recently observed by Avigad and Rute \cite{Avigad/Rute} that in the 
case of the von Neumann mean ergodic theorem a particular such rate of metastability 
(extracted in \cite{kohlenbachleustean09}) can be used to obtain even a simple effective (and 
also highly uniform) bound on the number of fluctuations (for the case of Hilbert spaces this 
was already obtained with an even better bound in \cite{Jones}).\\
This leads us to a very natural question: What kind of effective bounds (explicit information) is extractable
under which general logical conditions on convergence proofs.  \\
In this thesis we discuss four natural kinds of such finitary information and analyze the corresponding
conditions.\\
While the results we obtain in the process explain a very common form of realizers for strong ergodic theorems
in seemingly unrelated logical circumstances, there is a notable exception to this pattern published in~\cite{Safarik(11)}. We will see how the nested realizer from~\cite{Safarik(11)} relates to 
a separating example for two of our kinds of finitary information mentioned above.\\
Once we established this basis for computational content extraction, we also investigate, so to say, the opposite approach. 
Namely, given the fact that one is interested in such extraction, the question arises in which systems this is still possible and how.
We answer this question for a very prominent case in the last chapter, where we introduce an intrinsic formalisation
of non-standard analysis and discuss the extractability of effective bounds from proofs in that system.

%As follows from the soundness of the N(M)D-interpretation (see theorems \ref{t:sFI} and \ref{t:mfi})
%the precise realizing terms of the Bolzano-Weierstra{\ss} theorem, which we derive in this thesis,  
%constitute the computational contribution of this theorem in proofs.\\
%As it turns out, in contrast to Tao's concern indicated by his reference
%to the Paris-Harrington theorem (see~\cite{PH77}), this contribution is comfortably low. 
%In many contexts, which include systems formalizing large parts of Analysis, 
%%this contribution is covered by the $<\nolinebreak[4]\!\!\epsilon_0$\nbd recursion and, in some cases,
%this contribution is covered by \lOrdm{\epsilon_0}re\-cur\-sion or, in some cases,
%%even by $<\!\!\omega^{\omega^{\omega}}$-recursion.
%even by \lOrdm{\omega^{\omega^{\omega}}}re\-cur\-sion (see theorems \ref{t:PEfBW} and \ref{t:PEBW}). 
%Furthermore, it follows
%from Kohlenbach's results in \cite{Kohlenbach98} that for very weak systems 
%the contribution can in fact be covered by \lOrdm{\omega^{\omega}}re\-cur\-sion.
%
\mySubsection{Scientific Context}
At least since the presentation of G\"odel's functional 
interpretation (also called Dialectica or simply ``D-'' interpretation)
by K. G\"odel in 1958, proof theoretic methods can be used not only simply 
to analyze the syntactic structure of proofs, 
but also to deliver new information about the mathematical theorems being proved. The
first to formulate this idea of ``unwinding proofs'' was G.~Kreisel: 
\theQuote{ What more do we know if we have proved a theorem by 
  restricted means than if we merely know that it is true. }
%In our case, as mentioned above, it will be the theorem of Bolzano-Weierstra{\ss}.
%The first work in this new direction is also due to G. Kreisel in early 50's and
%his no-counterexample interpretation of Peano arithmetic (further $\PA$) experiencing
%a rather great interest and development thereafter. He used
%W. Ackermann's $\epsilon$-substitution method to prove
%that for any theorem of the first-order $\PA$ one can find ordinal recursive functionals,
%of order type \lOrd{\epsilon_0}, which realize the theorem's Herbrand
%normal form (see \cite{Kreisel51}, \cite{Kreisel52}).\\
Kreisel's ``Unwinding of Proofs'' 
developed into a new field of Mathematical Logic providing many new 
results in Algebra, Number Theory, and Numerical and Functional Analysis. 
To describe better the character and aim of this field, D. Scott suggested to call
this discipline \defkey{Proof Mining} and the new name was quickly adopted.\\
In this thesis, we make mainly use of the above mentioned functional interpretation (in various forms, 
we will even give a new variant suitable for our formalization of non-standard analysis in the last chapter).
G\"odel's functional interpretation requires an \em{intuitionistic} proof, 
(also called a \em{constructive} proof), which is -- roughly speaking -- a proof which doesn't make any use
of the law of excluded middle:
\[ \usftext{LEM}\quad:\quad \phi \vee \neg \phi\text{.} \]
Of course, one often works with semi-intuitionistic (semi-constructive) systems, 
where $\usftext{LEM}$ is only partially added, e.g. only for negated formulas (i.e. formulas in the form $\neg\phi$) -- denoted by $\usftext{LEM}_\neg$ -- or for formulas with only one, existential quantifier (i.e. formulas in the form $\exists\phi_\QF$) -- denoted by $\LEM$.
An intuitionistic proof can be obtained from a classical one by applying the negative translation
as a pre-processing step. The negative translation also goes back to 
G\"odel who defined a transformation of 
classically provable arithmetical formulas into classically equivalent and 
intuitionistically provable formulas in early 30's. We will use a 
modified version of this interpretation as 
presented by Kuroda in 1951 (see definition \ref{d:NT}).
The application of the combination of these interpretations, 
i.e. first the negative translation and then
the functional interpretation, is often referred to as the 
ND-interpretation (ND for negative D-interpretation). We will discuss all of this
in great detail in the first chapter and partially in the next paragraph.
For even more thorough and self-contained presentation of all these topics,
see U. Kohlenbach's book \cite{Kohlenbach08}. \\
% At this point, it is important to mention a result of C. Spector \cite{Spector62}.
In 1962, Spector \cite{Spector62} defined a particularly simple form of bar recursion (recursion on well-founded trees), which is sufficient
to solve the functional interpretation of (negative translation of) 
the schema of full comprehension over numbers:
\[
\CA^0\ :\quad \exists f:\NN\to\NN\forall x\in\NN\ \big(f(x)=_\NN 0 \leftrightarrow \phi(x)\big)\text{.}
\]
One could call this result a major break-through for proof mining, since it suffices to
prove most of classical analysis within a simple logical system based on Peano arithmetic (again see also \cite{Kohlenbach08}).\\
In particular, with Spector's bar recursion one can extract realizers (using ND-interpretation) for almost all
theorems of classical analysis (given their formal proof).\\
In fact, for many theorems the existence of a uniform bound on the realizers is guaranteed by Kohlenbach's metatheorems
introduced in~\cite{Kohlenbach05meta} and refined in~\cite{GK08}. Additionally, 
proof theoretic methods such as Kohlenbach's monotone functional interpretation (see~\cite{Kohlenbach96mfi}) can be used to
systematically obtain such effective bounds from their proofs, which tend to be a lot simpler than the actual realizers, though they remain
equally useful (again \cite{Kohlenbach08} is a very good reference for a comprehensive discussion on this topic). \\
We use Spector's solution together with Kohlenbach's monotone interpretation of Weak K�nig's Lemma (WKL, see~\ref{l:WKL-Feferman} in first chapter or for more details and background
e.g. \cite{Kohlenbach92, Kohlenbach08, Howard81, Troelstra74}) to give the ND-interpretation
of the Bolzano-Weierstrass theorem:
\[
 \BW_\RR\quad:\quad
  \forall f^{1(0)}
      \underbrace{ \exists a^1\forall k^0\exists l^0\geq_0 k
           \ |\hat a-_\RR\widetilde {fl\ }|\leq_\RR (\lambda n.\langle 2^{-k}\rangle) }_{\equiv:\BW_\RR(f^{1(0)})}
\text{.}
\] \\
The computational contribution of (instances of) BW and other principles 
describing sequential compactness
has been analyzed by Kohlenbach in \cite{Kohlenbach98} for proofs in somewhat 
weaker base systems: the Grzegorczyk arithmetic of level $n$, \defkeyn{$\GA$}. 
Kohlenbach and Oliva gave bounds for the bar-recursive realizers of the 
Principle of monotone convergence, $\PCM$, in the last section of \cite{KO02} and
discuss various general scenarios, where $\PCM$ has provably significantly smaller 
computational contribution.\\ 
Still, it is a bit surprising that the computational contribution of 
this basic theorem was first fully (in the above sense) analyzed in \cite{Safarik08}.
%? cite KohlenbachSafarik10
Thereafter the interest in BW's computational content continued. It was analyzed as part of
two different larger projects. Encouragingly, as far as the results can be compared, in both cases
the computational content corresponds very closely to the one obtained in \cite{Safarik08}. \\
The more recent of these results were obtained by P. Oliva and T. Powell in \cite{OP12}, where the authors
take the same approach as in \cite{Safarik08}. The main difference is, that
Spector's bar recursion is replaced by the use of selection functions. The authors
show the equivalence of unbounded products of selection functions and bar-recursion in \cite{OP12br}
and demonstrate how this can be used to give a different formalization
of the realizers in \cite{Safarik08}. Moreover, they give a very nice game-theoretic
interpretation of the realizers which allows for better understanding and intuition
for he original bar-recursive solutions.\\
On the other hand, V. Brattka analyzed the BW theorem as part of the Weihrauch-Lattice (see e.g. \cite{BG11, HP11})
in \cite{BGM12}. Though using completely different techniques, also he arrives at the conclusion
that while using BW in its full generality in a proof implies very strong forms of recursion, when used
in a more natural way, i.e. applied as a single instance of the principle, the
complexity is rather low.\\
In general, the proof mining experience shows that very often proofs of well known and influential theorems
do not need this kind of strong principles, i.e. principles like $\CA^0$ requiring more than normal recursion 
(though in most cases the proofs themselves are formulated in ways, which would suggest otherwise).\\
In \cite{AGT10}, J. Avigad, P. Gerhardy and H. Towsner analyzed the mean ergodic theorem,
implicitly showing that, when proving that the sequence of averages is Cauchy (rather than its
full convergence), it can be proved with strictly arithmetical
principles\footnote{more precisely, with arithmetic principles and arithmetic versions of analytical and higher principles},
i.e. statements about numbers without the need of a specific true higher type object like an actual 
cluster point for a sequence as in the BW theorem (please refer to section~\ref{s:ArProof} for more details on arithmetization in this context and
to \cite{Kohlenbach98} in general). \\
%Wittmann intro
Shortly after \cite{AGT10}, Kohlenbach and Leu{\c s}tean gave a more efficient analysis of MET in \cite{kohlenbachleustean09},
which was further investigated by 
Avigad and Rute in \cite{Avigad/Rute}, where they derive the above discussed
bound on the number of fluctuations (roughly speaking a bound on how many times does the distance 
between the values of the sequence exceed a given $\epsilon$, see Definition~\ref{d:nfluc}).
Such a bound gives us, of course, much more than simply a rate of metastability. The question under which
circumstances can a fluctuation bound be (systematically) obtained -- hoping for a similarly clear-cut answer
as in the case of metastability -- was answered in~\cite{KS13} and led to a new but very natural and straight-forward
definition of (effective) learnability. As expected, the kind of computational information we get, strongly depends on the ``amount of classicallity'' we use. Or in other words, how much of $\usftext{LEM}$ was truly needed to prove a given theorem.  \\
Interestingly, to some extent, we get a bit of ``classicallity''
in constructive formalization(s) of non-standard analysis as well (see e.g.~\cite{avigadhelzner02}).
Of course, this makes logical formalization of non-standard analysis and possible related conservation results interesting on its own account. Moreover, it would be hugely beneficial, to have an extraction procedure
and some meta-theorems to guarantee specific uniform bounds in the non-standard context as well.
Inspired by the work of Nelson~\cite{nelson77, nelson88}, Berg et al. give a formalization
of non-standard analysis together with an adapted form of Shoenfield interpretation (while the similarity between Shienfield's interpretation and Nelson's translation procedure between classical and non-standard proofs was actually the original point of departure), which are both intended for extraction of computation content from proofs based on non-standard methods, in~\cite{BBS12}. A major
step towards that goal would be solving the interpretation of the saturation principles, which can be used to formalize the concept of Loeb measures -- one of the most prominent non-standard techniques.  
For certain systems, it has turned out that extending them with saturation principles has resulted in an increase in proof-theoretic strength (see \cite{hensonkeisler86, keisler07}). So far, it seems that in the
context of~\cite{BBS12} it is also the case and we need to add a version of bar recursion
which corresponds precisely to Spector's, but adopted to our special form of sequence application as defined in~\cite{BBS12}.




% Finally talk a little about Intiotionistic vs. classic and how this is partially included in NSA
% and that we deal with it.


%%
%
% Here comes the part about complexity
%
%%%%%%%%%%%%%
%Of course, we are interested in the computational complexity of these functionals. 
%While investigating this kind of questions definitely should
%be considered being proof mining, it hardly fits into the proof-theoretic bounds. 
%It is not easy, if not impossible, to assign every
%single problem or rather approach to solve a problem of mathematical logic 
%to a specific branch of symbolic logic, nevertheless one would probably consider this
% sort of analysis more a part of recursion theory than proof theory. 
%
%

\mySubsection{Goals and Organization of the Material}
While in the context description above, we follow more or less the chronological order of
development, this thesis is structured according to the underlying proof-theoretic strength
or, correspondingly, the complexity of the computational content in question. We start
with low complexity realizers for simple convergence and work towards bar-recursive solutions
in the last two chapters.
Large parts of this thesis are based on previous publications. We precede
each section with a short quote of some of the journal's referee comments on the corresponding
article. These quotes are intended to offer a short abstract on the topic at hand from a more detached perspective.
In the following
we give an overview of the contents of the chapters, their relation and which
publications are they based on.\\
%
The {\em Preliminaries chapter} is somewhat different from the others. It is based on parts from
both a joint paper with U. Kohlenbach~\cite{KS10} and~\cite{Safarik08}.
It gives a quick introduction to the basic 
methods and results in the area of proof mining, which might be considered folklore or common knowledge. We do give some new results (and fill tiny gaps in known proofs), but mostly it is a collection of
known or not surprising material and can be found in
standard literature (mainly~\cite{Troelstra73,Howard81,Kohlenbach08}). In fact, it is meant more
as a quick reference for results we need in later chapters. We capture the most relevant results in short summaries at the end of the corresponding sections. For an interested reader, 
it is probably best to refer to \cite{Kohlenbach08} directly instead.\\
%
The {\em second chapter} is about effective learnability and discusses in great detail what kind of computational content can be extracted, and under which circumstances, from convergence proofs. 
It is based on the most recent joint work with U. Kohlenbach~\cite{KS13}, where both authors contributed
equally~\footnote{though the proof of Proposition~\ref{p:nonLearnablePhi} is solely due to Kohlenbach} and I wrote the parts presented in this thesis. We start with this topic, because the analysis
deals with the lowest level of computational complexity (compared to the remainder of this thesis).
The goal is to precisely characterize when can we obtain more than a (computable) rate of metastability 
from a proof of a convergence statement in analysis. Of course, it is known when 
(computable) full rate of convergence is extractable. Here, we are concerned
with computational content strictly in between. Namely a (computable) bound on fluctuations or at least an (effective) learning procedure for the limit point.\\
%
We take this, so to say, a step further in the {\em third chapter}, where we dive into a
case study on a proof of a strong non-linear ergodic theorem due to R. Wittmann. 
The situation here, closely resembles our abstract separating example for a convergence statement which is non-learnable but has an effective rate of metastability from second chapter. This is due to a specific nested structure of the extracted realizer, which is unprecedented by similar results in proof mining so far.
It is a rare example, where a proof of strong convergence doesn't lead to a realizer of the specific
form which is typically guaranteed for effectively learnable statements. This chapter is based on~\cite{Safarik(11)} extended by an in depth analysis of an arithmetized version of Wittmann's proof.\\
%
In the {\em fourth chapter} we go even further and investigate how a specific compactness principle, the theorem of Bolzano-Weierstrass, contributes to the complexity of realizers when used
as the main principle in proofs of logical statements in a particular, but rather general, logical form.
To achieve this, we need to use advanced techniques like bar-recursion, but we demonstrate that even in systems formalizing large parts of full classical analysis, powerful principles like $\BW$
contribute by the minimal amount possible, when they are used naturally and are treated correctly.
The results in this chapter improve, generalize and extend preliminary results going back to my Diplom-Thesis \cite{Safarik08} and
have been published in~\cite{KS10}. In particular, we now treat the Bolzano-Weierstrass principle for general
Polish spaces of the form $\PP^b$ (see Definition~\ref{d:PPs}) which subsequently has been crucially used
by U. Kohlenbach in his proof-theoretic analysis of weak compactness in~\cite{Kohlenbach2012(weak)}.\\ 
%
The {\em last chapter} surveys my joint work with B. van den Berg and E. Briseid in~\cite{BBS12} presenting in 
detail my contributions (sections~\ref{ss:dst:negative} and~\ref{s:Shoenfield}), here actually extended by an example (interpretation of $\DNS{\st}$). While in the chapters three and four we use established techniques already fine-tuned for proof-mining, here we want to contribute to
the field in general and provide extraction procedures for non-standard analysis. Such a challenging task certainly needs more than a single publication on a system formalizing the non-standard methods and a corresponding (negative) functional interpretation, but we believe we made a very solid first step. In this thesis we go actually further and discuss a yet unpublished successful part of a larger unfinished attempt on interpreting the principle of countable saturation, which is the basic step needed for interpreting proofs based on Loeb measures (see section~\ref{ss:dst:csat}). This part is of particular interest in the context of this thesis, since it was achieved
using much of the ideas and techniques needed for our previous results.

\mySubsection{Notation and common Expressions}
%
By "$\equiv$" we mean the syntactical identity.
We will write \defkeyn{$\PiL$} and \defkeyn{$\SiL$} for the
purely universal arithmetic formulas, i.e. $\forall n^0\ \phi_{_\QF}(n)$, and
the purely existential arithmetic formulas, i.e. $\exists n^0\ \phi_{_\QF}(n)$, where
in both cases $\phi_{_\QF}$ denotes a quantifier-free formula, 
which may contain parameters of arbitrary type. 
However, in general, we allow quantification over variables of any finite type.\\
For the encoding of a given finite sequence $s$ of natural numbers we 
write \defkey{$\lh(s)$} for the length of $s$ and
denote by \defkey{$[s]$} the type one function:
\[
  [s](i^0)\ :=_0\ \begin{cases}s(i)&\text{if}\ \ i<_0\lh(s)\\0&\text{else}\end{cases}
\text{.}\]
For a type one function $f$ and a natural number $n$ we define the corresponding encoding of
the finite sequence \defkey{$\overline{f}n$} of length $n$
as follows:
\[
  \overline{f}n\ :=\ \big\langle f(0), f(1), \cdots, f(n-1) \big\rangle
\text{.} \]
Given two finite sequences $s$ and $t$ we write $s*t$ for the concatenation of $s$ and $t$. 
We write shortly $s*\langle 0\rangle$ and $s*\langle 1\rangle$ as
$s*0$ and $s*1$. Following the notation of Avigad and Feferman \cite{AF98} by
$s\subseteq t$ we mean $t$ is an extension of $s$ (i.e. the sequence $t$ starts
with the sequence $s$, or is $s$). We denote the empty sequence by \defkeyn{$\emptyset$}.\\
For finite tuples of variables (not necessarily of the same type) $x_1,x_2,\ldots,x_k$ we
write $\tup x$. By $\tup x^{\tup \rho}$ we 
mean $x_1^{\rho_1},x_2^{\rho_2},\ldots,x_k^{\rho_k}$.\\
Generally we will use the Greek letters $\phi$,$\psi$, $\chi$ to 
denote formulas (as an exception
a predicate in section~\ref{s:bw} is named $I$), the lower case Latin letters $f$,$g$,$h$
for functions, the letters $a$,$b$,$i$,$j$,$k$,$\ldots$ for 
natural numbers and encodings, and the capitals like $A$,$B$,$\ldots$ for functionals.\\
We denote $\lambda n^0.1^0$, $\lambda f^1.1^0$ and so on by $\one\equiv\one^1$, 
$\one^2$ and so on.
We use bold numbers to indicate the type level of a term, e.g. we would write 
$t\tp 1$ for $t^{1(0)}$. In general, we use this superscript as a shortcut for a specific type having
the given type level. So, by $\forall X\tp2$ we mean for all $X$ of an appropriate type,
e.g. $2(1(0))$, not for all $X$ which are of any type with level $2$.\\
We will write ``holds classically/intuition\-istic\-ally'' or 
"is equivalent classically/intuition\-istic\-ally"
and so on meaning in fact "is provable" or "the equivalence is provable" and so on
in the classical, i.e. in $\wepa + \QFm\AC$, or intuitionistic, i.e. in
$\weha + \M + \AC$, system respectively.\\
We will treat tuples and pairs somewhat special in the second chapter, but we will clarify this in its first section.\\
Finally, in our last chapter on non-standard systems, we will need some more elaborate conventions, however we define those
there at the beginning, since we don't need them anywhere else.

%% Sysint (1)
%The basic notions of Proof Mining, some of which we mentioned already in this introduction, are given in section~\ref{s:si}.
%% Basics (2)
%We present the functional interpretations of some principles we will need, mainly the schema 
%of arithmetical comprehension, in section~\ref{s:basicInterpretations}.\\
%% CompPM (3)
%In section \ref{s:compPM}, we give a basic introduction to some computability theory related to proof mining.
%% Schema/rule (4)
%Section \ref{s:sr} is a brief discussion on the different effects on
%the ND-interpretation of proofs based on full schemas and proofs based only 
%on their concrete instances.\\
%% WKL (5)
%In section \ref{s:wkl}, we present and extend, by
%giving a few more details, Howard's results on the
%functional interpretation of Weak K\"onigs lemma.\\
%% BW (6)
%We use this lemma  together with the results from section \ref{s:basicInterpretations} 
%in section \ref{s:bw} to give
%a proof of $\BW$ and its bar-recursive
%functional interpretation. Also, we tie up to sections \ref{s:compPM} and \ref{s:sr} by analyzing
%the complexity of the realizers in both cases: the general schema and its
%concrete instance.\\

\mySubsection{Acknowledgments}
First of all, I would like to thank my supervisor Prof. Dr. Ulrich
Kohlenbach. Professor Kohlenbach not only proposed the topic for this thesis and supported
my research in a very flexible way depending on the most recent
results, but also went out of his way and accorded me his time whenever I
encountered a problem providing hints, explanations and encouragement. 
Moreover, he led me to a large field
of applied logic far out of the bounds of the mere subject at hand.\\
I am also grateful to his assistants Dr. Eyvind Briseid, Dr. Alexander Kreuzer and Dr. Laurentiu
Leustean for their direct advice as well as for pointing me to some additional 
literature. I thank Prof. Dr. Benno van den Berg and Prof. Dr. Thomas Streicher
for several fruitful discussions and tips.\\
I also appreciate the continuing support of my family and friends. %A lot.



\subsection{A general bound existence theorem}\label{s:Meta}

The main result of this paper, Corollary~\ref{c:fin22}, is a quantitative version
of a nonlinear strong ergodic theorem for 
operators satisfying Wittmann's condition~\eqref{e:W-assym} 
on an arbitrary subset of a Hilbert space. In this section
we outline how for this type of theorems the existence of such uniform bounds 
can be obtained by means of a general logical metatheorem. This sort of 
metatheorems was developed in~\cite{Kohlenbach05meta} and~\cite{GK08} (see also~\cite{Kohlenbach08})
and are applicable to many theorems concerning a wide range of classes of maps and abstract spaces. 
%Many convergence theorems typically meet both conditions. 
%Apart form ergodic theory, where at least all
%the ergodic theorems mentioned in Figure~\ref{f:METtree} meet both of these conditions, such metatheorems were
%successfully applied for asymptotic regularity theorems in metric fixed point theory~\cite{Kohlenbach2010}. 
For example, they were successfully applied to 
the ergodic theorems mentioned in Figure~\ref{f:METtree} or to 
asymptotic regularity theorems in metric fixed point theory~\cite{kohlenbachleustean10}.
In the last mentioned example, as in this paper, 
the authors infer from the metatheorems that uniform bounds exist and derive them explicitly.\\
To apply the metatheorems, we need the analyzed theorem to meet only two conditions:
\begin{enumerate}
\item The proof does not use axioms or rules which are too strong.
\item The analyzed theorem in its logical form is not too complex in terms of quantification.
\end{enumerate}
%spaces:
To formalize the first condition we start with a logical system for so called full classical analysis
introduced by Spector in \cite{Spector62}.\footnote{In particular this system covers full comprehension
over numbers, including also full second order arithmetic.} Kohlenbach
extended this system by an additional basic type and its defining axioms representing a given abstract space
and its properties. Kohlenbach also considers cases, where a specific subset of such a space 
(or rather its characteristic function) has to exist as a constant. For instance in~\cite{Kohlenbach08} Kohlenbach defines 
such systems for the theory of metric, hyperbolic, normed, uniformly convex or Hilbert spaces -- 
if required -- together with a (bounded) convex subset.\footnote{In any such abstract space,
its metric plays a major role as two objects are defined to be equal, if and only if their 
distance is zero.} In our case (i.e. simply for a pre-Hilbert space) this extended system is denoted by $\AHilb$.
In general, the system can be extended to arbitrary Hilbert spaces, however it turns out that the completeness
is not necessary for the proof that we analyze.\\
%functions
The second condition has to be investigated for each theorem specifically, depending on
the given theorem and the metatheorem we wish to use. Examples are metastable versions of formulas 
expressing the convergence or fixed point properties
of nonexpansive, Lipschitz, weakly quasi-nonexpansive or uniformly continuous functions even simply functions 
which are majorizable (see Corollary 6.6 in~\cite{GK08} and Theorem~\ref{t:GKmeta1} below).
%(on the corresponding spaces)
The metatheorem applicable
in our scenario follows from Corollary 6.6.7) in~\cite{GK08}. In particular with 
the theory $\AHilb$ with an additional parameter for an arbitrary subset $S$ of the
abstract Hilbert space $X$.\footnote{This is analogous to the case where we add $C$ to the
theory for normed space, but this time without any additional axioms.}\\

\begin{thm}[Gerhardy-Kohlenbach~\cite{GK08} - specific case 1]\label{t:GKmeta1}
Let $\varphi_\forall$, resp. $\psi_\exists$, be $\forall$-
resp. $\exists$-formulas that contain only $x,z,f$ free, resp. $x,z,f$, $v$ free. Assume that
$\mathcal{A}^\omega[X,\langle\cdot,\cdot\rangle,S]$ proves the following sentence:
\[
\forall  x\in\NN^\NN, z\in S, f\in{S^S} 
	\big( \varphi_\forall(x, z, f)\rightarrow\exists v\in\NN\ \psi_\exists(x, z, f, v)\big).
\]
Then there is a computable functional $F : \NN^\NN\times\NN\times\NN^\NN\to\NN$ s. t. the following holds
in all non-trivial (real) inner product spaces $(X,\langle\cdot,\cdot\rangle)$ 
and for any subset $S\subseteq X$
\begin{align*}
\forall  &x\in\NN^\NN, z\in S, b\in\NN, f\in{S^S},f^*\in\NN^\NN\\
	&\big( \TMaj(f^*,f)\ \wedge\ \|z\|\leq b\ \wedge\ \varphi_\forall(x, z, f) \rightarrow 
	\exists v\leq F(x,b,f^*)\ \psi_\exists(x, z, f, v) \big),
\end{align*}
where %$0_X$ does not occur in $\varphi_\forall$ and $\psi_\exists$ and 
\[
\TMaj(f^*,f):\equiv \forall n\in\NN\forall z\in S \big( \|z\|\leq_\RR n \rightarrow \|f(z)\|\leq_\RR f^*(n)\big).
\]
The theorem holds analogously for finite tuples. % $\tup x\in \prod^n_{i=0}\NN^\NN$.
\end{thm}
Consider the following metastable version of Wittmann's Theorem 2.1~\cite{Wittmann90}:
\begin{thm}[Theorem 2.1 in~\cite{Wittmann90}] \label{t:W21}
Let $S$ be a subset of a Hilbert space and $T:S\to S$
be a mapping satisfying 
\[
\forall x,y\in S\ (\| Tx + Ty \| \leq \|x + y\|).\tag{$\TAN$}\label{e:W}
\]
Then for any $x\in S$ the sequence of the Ces{\`a}ro means,
\[
A_nx:=\frac{1}{n+1}\sum^{n}_{i=0} T^i x,
\]
is norm convergent.
\end{thm}
This theorem has the following form:
\begin{align*}
\forall l\in\NN,g\in\NN^\NN, &\ x\in S, T\in S^S \tag{+}\label{e:w21meta}  
\ \big( \TAN(T)\rightarrow 
	\exists m\in\NN \ (\|A_mx-A_{m+g(m)}x\|< 2^{-l})\big). 
\end{align*}
Obviously the conclusion, i.e. 
$
\exists m\ \big( \|A_mx-A_{m+g(m)}x\| < 2^{-l}\big),
$
has the form $\exists m\ \psi_\exists(m,l,g)$ 
%(or $\exists m\psi_\exists(m,\langle l,g\rangle)$
%for any suitable encoding of $l$ and $g$ as an element $\langle l,g\rangle$ 
%of $\NN^\NN$)\footnote{Encoding and decoding of finite tuples of numbers, 
%functions and even functionals are definable in $\AHilb$ as primitive recursive operations.}
and the assumption $\TAN(T)$, i.e.
$
\forall x,y\in S \big(\| Tx + Ty \| \leq \|x + y\|\big),
$
has the form $\varphi_\forall(T)$.\\
Moreover, $\TAN(T)$ already implies $\TMaj(\Id,T)$ (here $\Id$ stands 
simply for the identity function on $\NN$), since $\TAN(T)$ applied to $x=y=z$
implies
$
\forall z\in S \big( \|T(z)\| \leq \|z\| \big).
$\\
Hence we can apply Theorem~\ref{t:GKmeta1} to~\eqref{e:w21meta} by setting
\[
\tup x:=_{\NN\times\NN^\NN} l,g,
\ z:=_{S}x,
\ f:=_{S\to S}T,
\ f^*:=_{\NN\to \NN}\Id,
\] and
\[
\varphi_\forall(x, z, f):\equiv\TAN(T),\ 
\exists v\in\NN\ \psi_\exists(x, z, f, v):\equiv\exists m\in\NN\ \big( \|A_mx-A_{m+g(m)}x\| < 2^{-l}\big),
\]
to obtain that there is a computable bound $M:\NN\times\NN^\NN\times\NN\to\NN$, s.t.
\begin{align*}
\forall l\in\NN,g\in\NN^\NN&, x\in S, T\in S^S \\
&\big( \TAN(T) \wedge \|x\| \leq b\ \rightarrow \exists m\leq_\NN M(l,g,b)\ ( \|A_mx-A_{m+g(m)}x\|\leq 2^{-l}) \big). 
\end{align*}
It is rather easy to see that the proof can be formalized in $\AHilbS$, except for the question of the use
of the axiom of extensionality (full extensionality
is in general unavailable in any proof-theoretic extraction of computational bounds). 
% unless one works with extremely weak systems)
Generally, one can avoid the use of full extensionality in proofs of statements
about continuous objects. Note that in particular any nonexpansive operator
is also continuous. However, in our case, the operator $T$ may be discontinuous. 
Fortunately, Wittmann proves his main results as a consequence of
a statement about a simple sequence of elements in $S$, 
which as such is independent of $T$ (see Theorem 2.3 in~\cite{Wittmann90} or Theorem~\ref{t:fin23l} below),
whereby all relevant equalities are provable directly. Therefore the rule of extensionality
suffices to formalize his proof.\\
Hence the existence of a {\em uniform computable bound} for the metastable version
can be inferred from the metatheorem in~\cite{GK08}. Furthermore, since the metatheorem is established
by proof-theoretic reasoning, it provides not only the existence of a uniform bound but also
a procedure for its extraction.\\
Now, in general such a bound might need so called
bar-recursion ($\BR$), which is required to interpret the schema of full comprehension over numbers
in Spector's system (see~\cite{Spector62}). However, once more due to the
way how Wittmann proved the analyzed theorem, it is easy to see 
that the only proof-theoretically non-trivial principles needed in the proof are 
the existence of the infimum/supremum of bounded sequences and 
the principle of convergence for bounded monotone sequences.
Both of these principles need only bar-recursion restricted to numbers and functions ($\BR_{0,1}$)
and not full $\BR$. (Kohlenbach shows in~\cite{Kohlenbach08, Kohlenbach00} that both principles
are provable from arithmetical comprehension which is interpreted in $\T_0+\BR_{0,1}$.) 
Moreover, since the bound itself has only functions and numbers as arguments, 
it follows from~\cite{Schwichtenberg79, Kohlenbach99} that
the bound is not only computable, but that the {\em bound
is a primitive recursive functional in the sense of G\"odel's $\T$}.\\
These observations can be made a priori, without any in-depth analysis of the proof. In addition,
one more conclusion can be drawn before one actually extracts the bound. In general, it is helpful, and sometimes
even necessary, to simplify (in the sense of proof-theoretic strength) the analyzed proof. We do so
in Section~\ref{s:ArProof}. As one can see, in fact the proof uses only arithmetical versions of
the non-trivial principles (which can be proved by $\SiLm\IA$) and therefore we 
know that the use of $\BR_{0,1}$ can be eliminated as well. 
In fact, for these arithmetic versions the bounds for the witnesses for the metastable
formulations are already known and rather simple.
\begin{prop}[Kohlenbach~\cite{Kohlenbach08}]\label{p:Ulrich}
Let $(a_n)$ be a nonincreasing sequence in $[0,C]$ for some constant $C\in\NN$, then
\[ \forall k\in\NN,g\in\NN^\NN\exists n\leq F(g,k,C)\forall i,j\in[n;n+g(n)]\ \big(|a_i-a_j|<2^{-k}\big), \]
where $F(g,k,C):={\tilde g}^{C\cdot 2^k}(0)$ with $\tilde g(n):=n+g(n)$.
\end{prop}
\begin{proof}
See Propositions 2.27 and Remark 2.29 in~\cite{Kohlenbach08}.\\
\end{proof}\\
Hence, we can infer that there is actually {\em an ordinary primitive recursive bound} (a bound in $\T_0$)
which we give explicitly in Section~\ref{s:Main}, Corollary~\ref{c:fin21}.\\
We should point out that the original Corollary in~\cite{GK08} can be used
in a more general context than the particular example we just discussed. For instance, it can be 
applied to both, Theorem 2.2 and Theorem 2.3 in~\cite{Wittmann90}. Take for example the metastable 
formulation of Theorem 2.3 in~\cite{Wittmann90}:
\begin{thm}\label{t:fin23l}
Let $X_{(\cdot)}$ be a sequence in a Hilbert space s.t. 
\[
\forall m,n,k\in\NN\quad \big(\|X_{n+k}+X_{m+k}\|^2 \leq \|X_{n}+X_{m}\|^2 + \delta_k\big), % label{e:LogOne}
\] and $\exists K\in\NN^\NN\forall n\in\NN \forall i\geq Kn\quad (\delta_i\leq 2^{-n})$.
Then
\[
\forall l\in\NN, g:\NN\to\NN\ \exists m\quad \big( \|A_mx-A_{m+g(m)}x\|\leq 2^{-l} \big).
\]
\end{thm}
For simplicity, let us
here assume that the sequence $\delta_{(\cdot)}$ is in the real unit interval.
In this case we have the following additional parameters: $K:\NN\to\NN$, $\delta:\NN\to[0,1]$ and
a sequence $z_{(\cdot)}:=X_{(\cdot)}$ (rather than a starting point $z:=x$). 
We can apply Corollary 6.6.7) in~\cite{GK08} for the theory $\AHilbS$ again. 
Additionally we use point 6.6.3). On the other hand,
this time we don't need the function parameter $f$:
\begin{thm}[Gerhardy-Kohlenbach~\cite{GK08} - specific case 2]\label{t:GKmeta2}
Let $P$ be a $\mathcal{A}^\omega$-definable Polish space and let $\varphi_\forall$, resp. $\psi_\exists$, be a $\forall$-
resp. an $\exists$-formula that contains only $x,z,f$ free, resp. $x,z,f$, $v$ free. Assume that
$\mathcal{A}^\omega[X,\langle\cdot,\cdot\rangle,S]$ proves the following sentence:
\[
\forall  x\in\NN^\NN, y\in P,z_{(\cdot)}\in\NN^S 
	\big( \varphi_\forall(x, z)\rightarrow\exists v\in\NN\ \psi_\exists(x, z, v)\big),
\]
Then there is a computable functional $F : \NN^\NN\times \NN^\NN\to \NN$ s. t. the following holds
in all non-trivial (real) inner product spaces $(X,\langle\cdot,\cdot\rangle)$ and for any subset $S\subseteq X$
\begin{align*}
\forall  x\in\NN^\NN, &z_{(\cdot)}\in\NN^S, b_{(\cdot)}\in\NN^\NN \\
	&\big( \forall n\in\NN\ ( \|z_n\|\leq b_n)\wedge\varphi_\forall(x, z) \rightarrow
	\exists v\leq F(x,b_{(\cdot)})\psi_\exists(x, z, v)\big).
\end{align*}
The theorem holds analogously for finite tuples.
\end{thm}
Given any rate of convergence for the sequence $\delta_{(\cdot)}$, the metastable version of the assumptions 
in Theorem~\ref{t:fin23l} is purely universal:
\begin{align*}
\forall m,n,k,j\in\NN\ \forall i\geq K(j) 
 \ \big( \|X_{n+k}+X_{m+k}\|^2 \leq \|X_{n}+X_{m}\|^2 + \delta_k\ \wedge\ \delta_i\leq 2^{-n} \big). \tag{$\TAN'$}
\end{align*}
Moreover, the Polish space $[0,1]^\NN$ is naturally definable in $\mathcal{A}^\omega$ so we obtain
by Theorem~\ref{t:GKmeta2} that
\begin{align*}
\forall l\in\NN,g\in\NN^\NN,K\in\NN^\NN&,\ \delta_{(\cdot)}\in[0,1]^\NN, X_{(\cdot)}\in S^\NN, b_{(\cdot)}\in\NN^\NN\\ 
  \big(\ &\forall i\in\NN\ ( X_i\leq b_i)\ \wedge\ \TAN'(X,K,\delta)\ \rightarrow \\
	      &\exists m\leq_\NN M(l,g,b_{(\cdot)},K)\ ( \|A_mx-A_{m+g(m)}x\|\leq 2^{-l}) \ \big).
\end{align*}
Similarly as before, $\TAN'$ implies $\forall i\in\NN\ ( X_i\leq b_i)$ for a suitable $b_{(\cdot)}$.
We give such a bound explicitly in Section~\ref{s:Main}. To be precise,
%in Theorem~\ref{t:fin}
we give a bound $M(l,g, b_0, b_1, \ldots, b_{K(0)},K)$ for 
an arbitrary sequence $\delta_{(\cdot)}\in\RR^\NN$ converging to zero with the rate $K$. In the simplified case for 
the unit interval, it is straightforward to see that the bound also
simplifies to an even more uniform bound $M(l,g,b_0,K)$.\\
To repeat, these are very specific scenarios. We should emphasize that the 
Corollaries in~\cite{GK08}, and 
the metatheorem(s) even more so, have a much wider range of application.


\subsection{Arithmetizing Wittmann's proof}\label{s:ArProof}
%\section{Wittmann's proof without Skolem functions}\label{s:ArProof}
The first step of a proof mining process  is to investigate the proof of the 
theorem we want to analyze. 
For a discussion on proof mining techniques in connection with ergodic theory 
see~\cite{Gerhardy2010} and the last section of~\cite{AGT10} and for
an in-depth analysis of applied proof theory see~\cite{Kohlenbach08}.
The nonconstructive, or ineffective, content of Wittmann's proof are the principle
of convergence for bounded monotone sequences of real numbers and the existence of
infimum for bounded sequences of real numbers. 
Formulated in the usual way, both principles state the existence
of a real number, which we represent as fast converging 
Cauchy sequences of rationals\footnote{We represent rational numbers as pairs of natural numbers. For further details on representations in this context, see~\cite{Kohlenbach08}.} encoded as number theoretic functions (i.e. functions in $\NN^\NN$).
However, for a given sequence, both principles can be replaced by 
weaker statements about natural numbers only (as opposed to statements about objects in $\NN^\NN$).
In the presence of arithmetical comprehension, these weaker (arithmetical) statements are
equivalent to the original (analytical) principles.\footnote{
While the analytical principles are actually known to be equivalent to
arithmetic comprehension (see~\cite{Simpson1999} and -- for more
detailed results --~\cite{Kohlenbach00}), the arithmetic versions are equivalent
to $\SiL$-induction and hence have a functional interpretation
by ordinarily primitive recursive functionals (see~\cite{Kohlenbach08}).
}
For the convergence we work with the arithmetic Cauchy property
and for infimum we give for any precision an approximate infimum.
\begin{enumerate}
	\item Arithmetized convergence of a monotone bounded sequence $a_{(\cdot)}$:
	\[\forall l\exists n\forall m\geq n\quad \big( |a_n-a_m|\leq 2^{-l}\big).\]
	\item Arithmetized existence of the infimum of a bounded sequence $a_{(\cdot)}$:
	\[\forall l\exists n\forall m\quad \big( a_n-a_m\leq 2^{-l}\big).\]
\end{enumerate}
Of course, in this way we don't get a single point which {\em is} the limit point or infimum. Therefore
we have to analyze the proof and see whether such points are actually needed or whether
these arithmetical versions suffice. Here, fortunately, it turns out that 
the latter is the case (see~\cite{Kohlenbach1998} for a general discussion of this point).\\
Following~\cite{Kohlenbach08}, we show that we can rewrite Wittmann's proof (see~\cite{Wittmann90}) 
in the language of $\AHilb$, carefully using only weak (arithmetized) principles at the relevant places.
\begin{thm}[Wittmann 1990,~\cite{Wittmann90}]\label{t:fin21ar}
Let $X_{(\cdot)}$ be a sequence in a Hilbert space s.t. for all $m,n,k\in\NN$
\[
||X_{n+k} + X_{m+k}||^2 \leq ||X_{n} + X_{m}||^2+\delta_k\tag{i}
\]
with
\[
\lim_{k\to\infty} \delta_{k}=0.\tag{ii}
\]
Then the sequence $A_{(\cdot)}$ defined by
\[
A_n:=\frac{1}{n}\sum^n_{i=1}X_i,
\]
is norm convergent.
\end{thm}
We follow Wittmann's notation and use $X_{(\cdot)}$ to denote the sequence in the
Hilbert space (not to be confused with the Hilbert space itself, which might be implied by the notation $\AHilb$). 
Also, to keep the proof more readable, we refer the more technical steps to later sections. There we need to do 
a thorough analysis of those steps to obtain the precise bounds of the realizers, while here we 
can settle for their existence.\\
We formulate conditions (i) and (ii) from the theorem as arithmetical statements as follows (except for
the variable $K$, which we will treat as a given parameter):
\begin{align}
&\forall m,n,k\in\NN \quad \big( ||X_{n+k}+X_{m+k}||^2 \leq ||X_{n}+X_{m}||^2 + \delta_k\big),\label{l:one}\\
&\exists K\in\NN^\NN\forall n\in\NN \forall i\geq K(n)\quad \big( |\delta_i|\leq 2^{-n}\big).\label{l:two}
\end{align}
From now on, let $K$ always denote a rate of convergence of the sequence $\delta_{(\cdot)}$ 
(i.e. a function satisfying~\eqref{two}) and $B$ the upper bound of $X_{(\cdot)}$ (such a bound
can be defined primitive recursively in $K(0)$ and some elements of $X_{(\cdot)}$, see Proposition~\ref{p:fin}).\\
It is easy to show that the sequence $(||X_n||)_n$ is a Cauchy sequence 
(see proof of Lemma~\ref{l:N0} below). Such an arithmetical formulation
of the convergence of $(||X_n||)_n$ is sufficient
to infer that in particular we have that
\begin{align}\forall l^0\exists n^0 \forall m_2^0\geq m_1^0\geq n\quad 
\big(\ ||X_{m_1}||^2 \leq ||X_{m_2}||^2 + 2^{-(l+1)}\ \big).\label{n_epsilon}\end{align}
The Skolem function realizing $n^0$ would correspond to $n_\epsilon$ in Wittmann's proof
(where he used the standard convergence), however we work only with 
the fact that such an $n$ exists for any given precision $l$. %We denote such an $n$ by $n_l$.
From~\ref{n_epsilon} we infer that
\[\forall l\exists n_l \forall m,n\geq n_l\ \forall k\geq K(l)\quad
 \big(\ \langle X_{n+k}, X_{m+k} \rangle \leq \langle X_{n}, X_{m} \rangle + 2^{-l}\ \big),\tag{W1}\label{w:1}\]
the same way as Wittmann in~\cite{Wittmann90}, see also the end of the proof of Lemma~\ref{l:scpb}.
From now on, by $n_l$ we always denote an $n_l$ which satisfies~\eqref{w:1}.\\
Since the norm of any convex combination of the elements of the sequence is bounded from below by $0$, the
set of all such convex combinations has an infimum. To give an arithmetic
formulation of this fact, it is useful to have a primitive recursive functional 
$C$ which gives us a $2^{-l}$ approximation of the smallest 
convex combination of $X_n, X_{n+1}, \ldots, X_p$
(more precisely, of the square of the infimum of the set of the norms of these convex combinations). 
Clearly, there are many ways to define such a functional. We do so in a straightforward 
way in Definition~\ref{d:C} below. Having such a $C$ in place,
the following arithmetical formula
\[
\forall l\exists p_l\forall p\geq p_l \quad  
\big(\ C(l,n_l,p_l) \leq C(l,n_l,p) + 2^{-l}\ \big),
 \tag{W2}\label{w:2} \]
states the existence of an approximate infimum of all convex combinations of $X_{(\cdot)}$
(see also Lemma~\ref{l:newC}). Similarly as with $n_l$, from now on by $p_l$ we mean
a number satisfying~\eqref{w:2}.\\
Wittmann introduces a specific point $z_{\epsilon,m}$, which we denote by $Z(l,n,p,m)$
and define using our notation in Definition~\ref{d:Z} below. However, here
the precise definition is not as important as the properties of this point.
%, that for
%an $A_m$ with sufficiently large index $m$, the functional
%$Z$ defines a point which is close to $A_m$. Moreover, all such
%points (i.e. points given by $Z$ for some $m$) are close to each other.\\
Firstly, we show that for any two natural numbers $i$ and $j$, the distance between $Z(l,n_l,p_l,i)$ and
$Z(l,n_l,p_l,j)$ is arbitrary small for sufficiently large $l$. Together with the convexity of the 
square function we have (again, see~\cite{Wittmann90} or the proof of Lemma~\ref{l:Zs}.\eqref{e:Zdown} below) that
\[ \forall l^0,m^0\quad 
\big(\ ||Z(l,n_l,p_l,m)||^2\leq C(l, n_l, p_l) + 2^{-l}\ \big).
\tag{W3}\label{w:3}\]
From~\eqref{w:2} we can infer (see the proof of Lemma~\ref{l:Zs}.\eqref{e:Zup} below) that
\[
\forall l,i,j\ \Big(\big\|\frac{1}{2}(Z(l,n_l,p_l,i)+Z(l,n_l,p_l,j))\big\|^2 + 2^{-l} \geq C(l, n_l, p_l)\Big),
\]
and together with~\eqref{w:3} and the parallelogram identity we can conclude that
\[
\forall l,i,j\ \Big(\big\|Z(l,n_l,p_l,i)-Z(l,n_l,p_l,j)\big\|^2 \leq  2^{-l+4}\Big).
\]\\
Secondly, it follows from the definition of $Z$ (yet again see~\cite{Wittmann90} or  
proof of Lemma~\ref{l:ZA}) that 
%\begin{align*} \forall &l,m\geq p_l\quad\\ &\Big( \|Z(l,n_l,p_l,m)-A_{m+1}\|\leq \frac{
% 2p_l\sup_{n\in\NN}\|X_n\|+2(n_l+K(l))\sup_{n\in\NN}\|X_n\|}{m+1} \Big),\end{align*}
 \begin{align*} \forall l\forall m\geq p_l\ \Big( \|Z(l,n_l,p_l,m)-A_{m+1}\|\leq \frac{
 2(p_l+n_l+K(l))\sup_{n\in\NN}\|X_n\|}{m+1} \Big),\end{align*}
which means that we can make the distance between $A_{m+1}$ and $Z(l,n_l,p_l,m)$ arbitrarily small by
choosing $m$ sufficiently large.\\
In particular we have shown that the distance between any $A_i$ and any $A_j$ is arbitrarily small 
once $i$ and $j$ are sufficiently large. Note that we can choose an arbitrarily large $l$ first and then we 
are still free to choose sufficiently large $i$ and $j$ after $n_l$ and $p_l$ are fixed.
This concludes the sketch of the proof that $A_{(\cdot)}$ is a Cauchy sequence.\\


\subsection{Obtaining a bound}\label{s:terms}
%\begin{dfn}
%By a formula $\phi$ containing
%\[
%t_1 \leq_l t_2,
%\]
%we mean that there is a small natural number $c$, s.t. for all $l$ the formula $\phi$ holds
%with\footnote{instead of $t_1 \leq_l t_2$} $t_1 \leq t_2 + 2^{-l}$ (note that we require the existance of
%a fixed $c$ independently on all other parameters and/or variables).
%\end{dfn}
The goal of this section is to roughly describe how to find a bound for $m$ in
\[
\forall l,g \exists m\ \big( \|A_m-A_{m+g(m)}\|\leq 2^{-l} \big).
\]
Similarly as before, for better understanding we leave some technical details
for later sections and disregard the monotonicity of the bounds as well as some
small corrections needed in the exponent of $2^{-l}$.
We handle these two aspects very carefully in the sections~\ref{s:Main} and~\ref{s:Lemmas}, where we make also
all of the following steps more explicit.\\
Let the assumptions in the proof of Theorem~\ref{t:fin21ar} hold and let $K$, 
$B$ and $C$ be as before as well.\\
Furthermore, we assume that $N_0$ and $P_0$ are the witnessing terms for~\eqref{w:1} (note that in
fact this means rather that we have a witness for Kreisel's no-counterexample interpretation -- n.c.i. see~\cite{Kreisel51, Kreisel1959} -- of the
convergence of $\|X_n\|$ in the first place) and~\eqref{w:2} as given in~\cite{Kohlenbach08}, i.e. we have that:
\begin{align*}
\forall l,h \forall m,n\in[N_0(l,h) ; N_0(l,h)&+g(N_0(l,h))]\ \big( 
\langle X_{m+k},X_{n+k}\rangle \leq \langle X_{m},X_{n}\rangle + 2^{-l} \big)
\end{align*}
and
\[
\forall l,f \ \big( C(l,n_l,P_0(l,f))\leq C(l,n_l,f(P_0(l,f))) + 2^{-l} \big)
\]
for any $n_l$ satisfying~\eqref{w:1}.\\ This structure accounts for the already mentioned nested iteration.  
Eventually, we have to specify a counterfunction $f$ s.t. it is sufficient that this inequality holds for that particular $f$. 
This $f$ has to depend on $n_l$, which will obviously be defined as an iteration, and $f$ itself will be iterated by $P_0$.\\
%Also, from now on, we will use the more formal $\lambda$-abstraction rather than the notation we used above. 
%For example we write $F(\tup x)=_1\lambda n.n^2$ rather than $F(\tup x)(n)=n^2$, to express that $F$ is a functional, 
%which given the arguments $\tup x$ returns the number theoretic square function
%(see e.g. Lemma 3.15 in~\cite{Kohlenbach08} for a precise definition within a subsystem of $\AHilb$). \\
To obtain a bound for $m$, we follow the proof from the last section backwards. We define 
\[
M_0(l,n,p):=2(p + n + K(l))B2^l,
\]
since then we have that $\|Z(l,n_l,p_l,M_0(l,n_l,p_l))-A_{m+1}\|\leq 2^{-l}$, for the 
right values of $l$, $n_l$ and $p_l$ (which we don't know yet).\\
From the proof we can infer that the largest $p$ needed in~\eqref{w:2} is $K(l)+m+g(m)+n_l+p_l$ (for details
see proof of Lemma~\ref{l:Zs}.\eqref{e:Zup}), therefore we need that
\[f(P_0(l,f))=K(l)+m+g(m)+n_l+P_0(l,f),\]
(where $m$ and $m+g(m)$ correspond to the $m$ and $n$ in~\eqref{w:1}). \\
Moreover, we can see that the largest $m$ or $n$ needed in~\eqref{w:1} is $n_l+p_l+m+g(m)+K(l)$ (for details
see proof of Lemma~\ref{l:scpb} and its application in proof of Lemma~\ref{l:Zs}.\eqref{e:Zdown}). So we need that
$h(N_0(h,l))=p_l+m+g(m)+K(l)$. \\
Keeping in mind that $N_0(l,h)$ corresponds to $n_l$ (and to $n$ below) and $P_0(l,f)$ to $p_l$ (and to $p$ below) 
we obtain\footnote{
Here and in the following, when considering a complex term $t[a_1,\ldots,a_n]$ in several variables $a_1,\ldots,a_n$,
the notation $\lambda a_i.t[a_1,\ldots,a_n]$ defines (as usual in logic) the function $a_i\mapsto t[a_1,\ldots,a_n]$.}
\begin{align*}
 h(l,g) = \lambda n\ .\ &\underbrace{P_0(l,\lambda p\ .\ K(l)+M_0(l,n,p)+g(M_0(l,n,p))+p+n)}_{p_l} + \\
&+M_0(p_l,n,l) + g(M_0(p_l,n,l))+K(l),
\end{align*}
and define
\[
N(l,g):=N_0(l, h(l,g)).
\]
Given $N(l,g)$, which corresponds to $n_l$, and again keeping in mind that $P_0(l,f)$ corresponds to $p_l$ (and to $p$ below),
we can define $P(l,g)$ as well:
\[
P(l,g):=P_0(l, f(l,g)),
\]
where
$ f(l,g) = \lambda p\ .\ K(l)+M_0(p,N(l,g),l)+g(M_0(p,N(l,g),l))+p+N(l,g).$
Finally we can define the desired witness for $m$ as follows:
\[
M(l,g):=M_0( l, N(l,g), P(l,g) ).
\]





\section{Uniform bounds for Wittmann's ergodic theorems}\label{s:Main}
%\section{The witness is correct}\label{s:Main}

We give a bound for a finitary version of Wittmann's convergence result for
a general series in a Hilbert space satisfying a suitable formulation
of the condition~\eqref{e:W-assym} first 
(see Theorem~\ref{t:fin} and Corollary~\ref{c:finInt}) to derive the bounds
for the finitary versions of the actual ergodic theorems later
(see Corollary~\ref{c:fin22} and Corollary~\ref{c:fin21}).\\
We already discussed how to obtain the first bound in Sections~\ref{s:terms} and~\ref{s:ArProof},
here we concentrate on a formal and precise definition of the bounds as well as on the proof
that these bounds are correct.
\begin{thm}[Finitary version of Theorem 2.3 in~\cite{Wittmann90}] \label{t:fin}
Let $K:\NN\to\NN$ be a function and $X_{(\cdot)}$ a sequence in a Hilbert space s.t. for all $m,n,k\in\NN$
\[
\|X_{n+k} + X_{m+k}\|^2 \leq \|X_{n} + X_{m}\|^2+\delta_k,
\]
with
$
\forall l\in\NN\forall n\geq K(l)\ \big( |\delta_{n}|<2^{-l}\big),
$
in other words $K$ is a rate of convergence of $(\delta_k)$ towards 0.
Then the sequence $A_{(\cdot)}$, defined by
$
A_n:=\frac{1}{n}\sum^n_{i=1}X_i,
$
is a Cauchy sequence and we have that
\[
\forall l,g \exists m\leq M'(l,g^M)\ \big( \|A_{m}-A_{m+g(m)}\|\leq 2^{-l} \big),
\]
with $g^M(n):=\max_{i\leq n} g(n)$ and $M'(l,g,K)$ defined as follows (we ommit the dependencies whenever the arguments are trivial):
\begin{align*}
M'&:=M(2l+6, \lambda n\ .\ g(n+1))+1,\\
M&:=M_0( P(l,g), N(l,g), l),\\
P&:=P_0(l,F(l,g,N(l,g))),\\
F(p)&:=p+n+K^M(l)+M_0+g(M_0), F:\NN\to\NN\\
N&:=N_0(l+1,H(l,g)), \\
H(n)&:= H_0(l,g,n,P_0(l,F)), H:\NN\to\NN,
%M'(l,g)&:=M(2l+6, \lambda n\ .\ g(n+1))+1,\\
%M(l,g)&:=M_0( P(l,g), N(l,g), l),\\
%P(l,g)&:=P_0(l,F(l,g,N(l,g))),\\
%F(l,g,n) &:=_1 \lambda p\ .\ p+n+K^M(l)+M_0(l,n,p)+g(M_0(l,n,p)),\\
%N(l,g)&:=N_0(l+1,H(l,g)), \\
%H(l,g)&:=_1 \lambda n\ .\  H_0(l,g,n,P_0(l,F(l,g,n))),
\end{align*}
where (recall $K^M(n):=\max_{i\leq n} g(n)$)
\begin{align*}
 H_0&:= p + M_0+g(M_0)+K^M(l),\\
 M_0&:=(2n + 2p + 2K^M(l))B2^{l},\\
 P_0&:=\tilde f^{B^22^l}(0),\quad \tilde f(n):=n + f(n),\\
 N_0&:=R(l+1,U)+K^M(l+1), \\
 R&:=\tilde u^{\lceil \|X_0\| \rceil^22^l}(0),\\
 U(n)&:=(n + K^M(l+1))+h^M(n + K^M(l+1)), U:\NN\to\NN,\\
 B&:=\max_{1\leq i \leq K^M(0)} \lceil \|X_i\| \rceil+1.\\
% H_0(l,g,n,p)&:= p + M_0(l,n, p )+g(M_0(l,n, p ))+K^M(l),\\
% M_0(l,n,p)&:=(2n + 2p + 2K^M(l))B2^{l},\\
% P_0(l,f)&:=\tilde f^{B^22^l}(0),\quad \tilde f(n):=n + f(n),\\
% N_0(l,h)&:=R(l+1,U(l,h))+K^M(l+1), \\
% R(l,u)&:=\tilde u^{\lceil \|X_0\| \rceil^22^l}(0),\\
% U(l,h)&:=_1\lambda n\ .\ (n + K^M(l+1))+h^M(n + K^M(l+1)),\\
% B&:=\max_{1\leq i \leq K^M(0)} \lceil \|X_i\| \rceil+1.\\
\end{align*}
\end{thm}

%\begin{proof}
%Given $\tup s$ choose $\tup s'\in S_{p,l}$ s.t. $|s'_i-\tilde s_i|\leq \frac{2^{-(l+1)}}{pB^2}$. 
%We have
%\begin{align*}
%\left\|\sum_{i=0}^{p} \widetilde{s}_i X_{n+i} - \sum_{i=0}^{p} {s'}_i X_{n+i}\right\| = 
%\left\|\sum_{i=0}^{p} |\widetilde{s}_i-s'_i| X_{n+i} \right\| \leq
%\frac{2^{-(l+1)}}{pB^2}pB = \frac{2^{-(l+1)}}{B}.
%\end{align*}
%therefore
%\[
%\left|\ \left\|\sum_{i=0}^{p} \widetilde{s}_i X_{n+i}\right\| - \left\|\sum_{i=0}^{p} {s'}_i X_{n+i}\right\| \ \right|\leq\frac{2^{-(l+1)}}{B}
%\]
%and finally
%\[
%\left|\ \left\|\sum_{i=0}^{p} \widetilde{s}_i X_{n+i}\right\|^2 - \left\|\sum_{i=0}^{p} {s'}_i X_{n+i}\right\|^2\  \right| \leq \frac{2^{-(l+1)}}{B} ( B + \frac{2^{-(l+1)}}{B} ) \leq 2^{-l}.
%\]
%\end{proof}

We will also use that $M$ majorizes itself and that $N_0$ and $P_0$ are the right witnesses for the two main assumptions needed in Wittmann's proof. For better readability, we prove these lemmas in Section~\ref{s:Lemmas} at the end of the paper.

\begin{lemma}[$M$ is a majorant]\label{l:M}
Each of the terms $M$,$P$,$N$,$M_0$,$P_0$,$N_0$,$R$ majorizes itself.
In particular we have:
\[ \forall l'\geq l\forall h',h ( \forall n (h(n)\geq h'(n)) \rightarrow N_0(l',h^M)\geq N_0(l,h') ) \]
and
\begin{align*}
 \forall l,g \forall n\leq N(l,g^M) &\forall p\leq P(l,g^M)\\
  &( P(l,g^M)\geq P_0(l,F(l,g^M,n))\ \wedge\ M(l,g^M)\geq M_0(l,n,p) ) . 
\end{align*}
%and  
%\[ \forall l \forall g \forall n'\geq n (\ (H(l,g))(n') \geq (H(l,g))(n)\ ). \]
\end{lemma}

\begin{lemma}[$N_0$ is correct]\label{l:N0}
\[
\forall l,h \exists n \leq N_0(l,h) \forall i,j\in[n;n+h(n)]\ i\leq j\rightarrow \|X_i\|^2-\|X_j\|^2\leq 2^{-l}.
\]
\end{lemma}

\begin{lemma}[$P_0$ is correct]\label{l:P0}
$
\forall l,f,n\exists p\leq P_0(l,f)\ \big( C(l,n,p)\leq C(l,n,f(p)) + 2^{-l} \big).
$
\end{lemma}

Next three lemmas give a quantitative analysis of the original proof in~\cite{Wittmann90}. Again, for better readability, we 
give the proofs in Section~\ref{s:Lemmas}. 

\begin{lemma}[The scalar product increase is bounded]\label{l:scpb}
For any $l$ and any $g$, consider $h:= H(l,g^M)$. Let
$n$ be a witness for Lemma~\ref{l:N0}, i.e. 
\[
n\leq N(l,h)\ \wedge\ \forall i,j\in[n;n+h(n)]\ 
\big( i\leq j\rightarrow \|X_i\|^2-\|X_j\|^2\leq 2^{-l-1} \big) . \tag{N} 
\]
Moreover let $f:=F(l,g^M,n)$,
$p$ be a number smaller than $P_0(l,f)$ 
and  $m:=M_0(l,n,p)$. Then we have that
\[ 
\langle X_{a+k},X_{b+k} \rangle \leq \langle X_{a},X_{b} \rangle + 2^{-l}
\]
holds for all $k,a,b$ s.t. $K^M(l)\leq k \leq K^M(l)+m+g^M(m)$ and $ n\leq a,b \leq n+p$.
\end{lemma} 

Analogously to Wittmann~\cite{Wittmann90} we define
\begin{dfn}[$Z$]\label{d:Z}
$Z(l,n,p,m):=\frac{1}{m+1}\sum^{K^M(l)+m}_{k=K^M(l)}\sum^{p}_{i=0}  \widetilde{\tup s}_i X_{n+k+i},$
with $\tup s$ corresponding to the tuple in the definition of $C(l,n,p)$ (see
Definition~\ref{d:C} above). 
\end{dfn}

\begin{lemma}[$Z$s are close]\label{l:Zs}
For any $l$ and any $g$, consider $h:= H(l,g^M)$. Let
$n$ be a witness for Lemma~\ref{l:N0}, i.e. 
\[
n\leq N(l,g^M)\ \wedge\ \forall i,j\in[n;n+h(n)]\ 
\big( i\leq j\rightarrow \|X_i\|^2-\|X_j\|^2\leq 2^{-l-1} \big) . \tag{N} 
\]
Moreover, let $m:=M_0(l,n,p)$, $f:=F(l,g^M,n)$ and 
$p$ be a witness for Lemma~\ref{l:P0}, i.e. 
\begin{align*}
p\leq P_0(l,f)\ \wedge\ ( C(l,n,p)\leq C(l,n,f(p)) + 2^{-l}  ),  \tag{P}\\
\end{align*} 
Then we have that
$ \|Z(l,n,p,m) - Z( l,n,p,m+g^M(m) ) \|^2 \leq 2^{-l+4}.$
%holds for $m=M_0(l,n,p)$.
\end{lemma}

\begin{lemma}[$Z$s and $A$s are close]\label{l:ZA}
For any $l$ and any $g$, consider $h:= H(l,g^M)$. Let
$n$ be a witness for Lemma~\ref{l:N0}, i.e. 
\[
n\leq N(l,g^M)\ \wedge\ \forall i,j\in[n;n+h(n)]\ 
\big( i\leq j\rightarrow \|X_i\|^2-\|X_j\|^2\leq 2^{-l-1} \big) . \tag{N} 
\]
Moreover let $f:=F(l,g^M,n)$,
$p$ be a witness for Lemma~\ref{l:P0}, i.e. 
\begin{align*}
p\leq P_0(l,f)\ \wedge\ ( C(l,n,p)\leq C(l,n,f(p)) + 2^{-l}  ),  \tag{P}\\
\end{align*} 
and  $m:=M_0(l,n,p)$, $m':=m+g(m)$.
Then we have that
\[
 \|A_{m+1} - Z( l,n,p,m )\|\leq \frac{1}{m+1}(2n + 2p + 2K(l))B +2^{-l}
\]
and
\[
 \|A_{m'+1} - Z( l,n,p,m' )\|\leq \frac{1}{m'+1}(2n + 2p + 2K(l))B +2^{-l}.
\]
\end{lemma}


\begin{proof}[ of Theorem~\ref{t:fin}]
Fix arbitrary $l$ and $g$. Set $h:= H(l,g^M)$.
By Lemma~\ref{l:N0} we know there is an $n$ s.t.  
\[
n\leq N(l,g^M)\ \wedge\ \forall i,j\in[n;n+h(n)]\ 
\big( i\leq j\rightarrow \|X_i\|^2-\|X_j\|^2\leq 2^{-l-1} \big). %\tag{N}\label{e:N1}
\]
Let $f:=F(l,g^M,n)$. By Lemma~\ref{l:P0} we know that there is a $p$ s.t.
\begin{align*}
p\leq P_0(l,f)\ \wedge\ ( C(l,n,p)\leq C(l,n,f(p)) + 2^{-l}  ).  %\tag{P}\label{e:P1}\\
\end{align*}
Note that by Lemma~\ref{l:M} we have that $p\leq P(l,g^M)$. We set $m:=M_0(l,n,p)$. By Lemma~\ref{l:M} we get that $m\leq M(l,g^M)$.
Finally, it follows from lemmas~\ref{l:Zs} and~\ref{l:ZA} that
\begin{align*}
\|A_{m+1} - A_{m+g(m)+1}\| &\leq \|Z( l,n,p,m ) - Z( l,n,p,m+g(m) )\|\\
		&\quad\quad\quad + 2\Big(\frac{1}{m+1}(2n + 2p + 2K(l))B +2^{-l}\Big)\\
		&\leq \sqrt{2^{-l+4}} + 2^{-l+1} + \frac{2(2n + 2p + 2K(l))B}{m+1} \\
		&= \sqrt{2^{-l+4}} + 2^{-l+1} + \frac{2(2n + 2p + 2K(l))B}{(2n + 2p + 2K(l))B2^{l}+1} \\
		&< \sqrt{2^{-l+4}} + 2^{-l+1} + 2^{-l+1} \leq 2^{-\frac{l}{2}+3}.
\end{align*}
This proves 
\[
\forall l,g \exists m\leq M(l,g^M)\ \big( \|A_{m+1}-A_{m+g(m)+1}\|\leq 2^{-\frac{l}{2}+3}\big),
\]
from which the claim follows immediately by the definition of $M'$.
\end{proof}

Sometimes it is useful to work with the following version of the previous theorem, though
both these formulations are equivalent (even in weaker systems than we used to
formalize the original proof itself).

\begin{cor}[Finitary version of Theorem 2.3 in~\cite{Wittmann90} for intervals] \label{c:finInt}
Let $K$ be a function and $X_{(\cdot)}$ a sequence in a Hilbert space s.t. for all $m,n,k\in\NN$
\[
\|X_{n+k} + X_{m+k}\|^2 \leq \|X_{n} + X_{m}\|^2+\delta_k,
\]
with
\[
\forall l\in\NN\forall n\geq K(l)\ \big( |\delta_{n}|<2^{-l}\big).
\]
Then the sequence $A_{(\cdot)}$ defined by
\[
A_n:=\frac{1}{n}\sum^n_{i=1}X_i,
\]
is a Cauchy sequence and we have that
\[
\forall l,g \exists m\leq M'(l+1,g^M)\forall i,j\in[m;m+g(m)]\ \big( \|A_{i}-A_{j}\|\leq 2^{-l} \big),
\]
with $M'$ defined as in Theorem~\ref{t:fin}.
\end{cor}
\begin{proof}
Given any $l$ and $g$, apply Theorem~\ref{t:fin} to the number $l+1$ and to the function
\begin{align*}
h(n):=\min\Big\{\ &i\in [0;g(n)] \text{ s.t. }\\
 &\forall j\in[0;g(n)]\quad \Big(\big| \|A_{n+i}\| - \|A_n\| \big| \geq \big| \|A_{n+j}\|- \|A_n\|\big|\Big) \Big\}.
\end{align*}
It follows that 
\[
\exists m\leq M'(l+1,h^M)\ \big( \|A_{m}-A_{m+h(m)}\|\leq 2^{-l-1} \big).
\]
We fix such an $m$ and conclude (by the triangle inequality) that
\[
\forall i,j\in[m;m+g(m)]\ \big( \|A_{i}-A_{j}\|\leq 2^{-l} \big).
\]
Moreover, since $\forall n\in\NN\ (h^M(n)\leq g^M(n))$ we have that \[m\leq M'(l+1,g^M)\] due to Lemma~\ref{l:M}.
\end{proof}

Now, we obtain the bound for the metastable version of Theorem 2.2 in Wittmann's paper~\cite{Wittmann90} as a simple conclusion:

\begin{cor}[Finitary version of Theorem 2.2 in~\cite{Wittmann90}] \label{c:fin22}
Let $S$ be a subset of a Hilbert space and $T:S\to S$
be a mapping satisfying
\begin{align*}
&\forall n\in\NN\forall x,y\in S\ \big(\| T^nx + T^ny \| \leq \alpha_n\|x + y\|\big),\\ % \label{e:22.a1} \\
&\forall l\in\NN\forall n\geq K'(l)\ \big( |1-\alpha_{n}|<2^{-l}\big). % \label{e:22.a2}
\end{align*}
Then for any $x\in S$ the sequence of the Ces{\`a}ro means
\[
A_nx:=\frac{1}{n+1}\sum^{n}_{i=0} T^i x
\]
is norm convergent and the following holds:
\[
\forall l,g \exists m\leq M'(l+1,g^M)\forall i,j\in[m;m+g(m)]\ \big( \|A_{i}x-A_{j}x\|\leq 2^{-l}\big ),
%\forall l\in\NN,g\in\NN^\NN \exists m\leq_\NN M(l,g^M)\ \big( \|A_{m}x-A_{m+g(m)}x\|\leq 2^{-l}\big ).
\]
with $M'$ defined as in Theorem~\ref{t:fin}.
\end{cor}
\begin{proof}
Fix an arbitrary $x\in S$ and set $B':=\max_{i\leq K'(0)}(T^ix)+1.$
The claim follows from Corollary~\ref{c:finInt} with 
$X_i:=T^ix$, $\delta_n:=4B'^2(\alpha_n^2-1)$, $K(l):=K'(l+2\lceil\log_2 B' \rceil+2)$.
\end{proof}

The following corollary follows immediately:
%\newpage
\begin{cor}[Finitary version of Theorem 2.1 in~\cite{Wittmann90}] \label{c:fin21}
Let $S$ be a subset of a Hilbert space and $T:S\to S$
be a mapping satisfying
\[
\forall x,y\in S\ \big(\| Tx + Ty \| \leq \|x + y\|\big).
\]
Then for any $x\in S$ the sequence of the Ces{\`a}ro means
$
A_nx:=\frac{1}{n+1}\sum^{n}_{i=0} T^i x
$
is norm convergent and the following holds:
\[
\forall l,g \exists m\leq M(l,g^M)\forall i,j\in[m;m+g(m)]\ \big( \|A_{i}x-A_{j}x\|\leq 2^{-l}\big ),
%\forall l\in\NN,g\in\NN^\NN \exists m\leq_\NN M(l,g^M)\ \big( \|A_{m}x-A_{m+g(m)}x\|\leq 2^{-l}\big ),
\]
%and there is a witness
%\[
%%m \leq 1+M(2l+6,(\breve g)^M),\quad \breve g(n+1):=g(n), \breve g(0):=0
%m \leq 1+M(2l+6,(\breve g)^M),\quad \breve g(n+1):=g(n), \breve g(0):=0
%\]
with $M$ defined as follows:
\begin{align*}
%M(l,g)&:=(N( 2l+6, g) + P( 2l+6, g)){\lceil \|x\| \rceil}2^{2l+7}+1,\\
M(l,g)&:=(N( 2l+7, g) + P( 2l+7, g)){\lceil \|x\| \rceil}2^{2l+8}+1,\\
P(l,g)&:=P_0(l,F(l,g,N(l,g))),\\
F(l,g,n) &:=_1 \lambda p\ .\ p+n+\tilde g((n + p){\lceil \|x\| \rceil}2^{l+1}),\\
N(l,g)&:=\big({\lambda n\ .\ n+P_0(l,F(l,g,n))+\tilde g((n + P_0(l,F(l,g,n))){\lceil \|x\| \rceil}2^{l+1}) } \big)^{{\lceil \|x\| \rceil}^22^{l+2}}(0),
\end{align*}
where $P_0(l,f):=\tilde f^{ {\lceil \|x\| \rceil}^2 2^l}(0)$, $\tilde f(n):=n+f(n)$, $f^M(n):=\max_{i\leq n+1} f(i)$.\\

Note that (due to Lemma~\ref{l:M}) the bound for $m$ depends only on a bound for the norm of the
parameter $x$ and not directly on the starting point.
\end{cor}



\subsection{Lemmas}\label{s:Lemmas}

Here, we assume that the assumptions of Theorem~\ref{t:fin} hold and use the terms as they are defined in that theorem.
We use also the definitions~\ref{d:C} and \ref{d:Z} for $C$ and $Z$.
Moreover, w.l.o.g we can assume $K^M=K$, since the original assumption implies
\[ \forall l\in\NN\forall n\geq K^M(l) \ \big(\|\delta_n\|<2^{-l}\big). \]
We prove Lemma~\ref{l:newC} first and then the two lemmas which show 
that $N_0$ and $P_0$ are the right
witnesses for the two main assumptions needed in Wittmann's proof. The fact
that $M$ majorizes itself (and therefore so does $M'$) 
follows simply from its definition in Theorem~\ref{t:fin}.\\

\begin{lemma*}[$C$ approximates the smallest convex combination]{l:newC}
\[
\forall l,n,p,f \forall \tup s\ \big( C'(\tup s,l,n,p,f)+2^{-l}\geq C(l,n,p) \big).
\]
\end{lemma*}
\begin{proof}
Given $\tup s$ choose $\tup s'\in S_{p,l}$ s.t. $|s'_i-\tilde {\tup s}(i)|\leq \frac{2^{-(l+1)}}{pB^2}$ for all $i\in[0;p]$. 
Then we have that
\begin{align*}
\left\|\sum_{i=0}^{p} \widetilde{\tup s}(i) X_{n+i} - \sum_{i=0}^{p} {s'}_i X_{n+i}\right\| = 
\left\|\sum_{i=0}^{p} |\widetilde{\tup s}(i)-s'_i| X_{n+i} \right\| \leq
\frac{2^{-(l+1)}}{pB^2}pB = \frac{2^{-(l+1)}}{B},
\end{align*}
and therefore also that
$
\left|\ \left\|\sum_{i=0}^{p} \widetilde{\tup s}(i) X_{n+i}\right\| - \left\|\sum_{i=0}^{p} {s'}_i X_{n+i}\right\| \ \right|\leq\frac{2^{-(l+1)}}{B},
$
so finally we get that
\[
\left|\ \left\|\sum_{i=0}^{p} \widetilde{\tup s}(i) X_{n+i}\right\|^2 - \left\|\sum_{i=0}^{p} {s'}_i X_{n+i}\right\|^2\  \right| \leq \frac{2^{-(l+1)}}{B} ( B + \frac{2^{-(l+1)}}{B} ) \leq 2^{-l}.
\]
\end{proof}

%\begin{lemma*}[$M$ is a majorant]{l:M}
%Each of the terms $M$,$P$,$N$,$M_0$,$P_0$,$N_0$,$R$ majorizes itself.
%In particular we have:
%\[ \forall l'\geq l\forall h',h ( \forall n (h(n)\geq h'(n)) \rightarrow N_0(l',h^M)\geq N_0(l,h') ) \]
%and
%\[ \forall l,g \forall n\leq N(l,g^M) \forall p\leq P(l,g^M)\ ( P(l,g^M)\geq P_0(l,F(l,g^M,n))\ \wedge\ M(l,g^M)\geq M_0(l,n,p) ) . \]
%%and  
%%\[ \forall l \forall g \forall n'\geq n (\ (H(l,g))(n') \geq (H(l,g))(n)\ ). \]
%\end{lemma*}

\begin{lemma*}[$N_0$ is correct]{l:N0}
The sentence
\[
\forall l,h \exists n \forall i,j\in[n;n+h(n)]\ \big( i\leq j\rightarrow \|X_i\|^2-\|X_j\|^2\leq 2^{-l}\big)
\]
is witnessed by an $n\leq N_0(l,h)$.
\end{lemma*}
\begin{proof}
The sequence $(\|X_n\|)$ is bounded from below therefore we have:
\[ \forall l \exists r \forall k \ \big( \|X_k\|^2 + 2^{-l}\geq \|X_r\|^2\big).\]
%and in particular
%\[ \forall l \exists r \forall k\geq r \|X_k\| + 2^{-l}\geq \|X_r\|\]
For any given $l$ we fix such an $r$. The following statements 
imply that $(\|X_n\|)$ is a Cauchy sequence:
\begin{enumerate}
	\item\label{e1} $\exists k_0 \forall k\geq k_0 \quad \big(\|X_k\|^2\leq \|X_{r}\|^2 + 2^{-l}\big)$, and
	\item\label{e2} $\forall k \quad \big(\|X_k\|^2 + 2^{-l}\geq \|X_{r}\|^2\big) $,\\
\end{enumerate}
since \eqref{e1}$\wedge$\eqref{e2} means that there is an index
from which on the norm of any of the remaining elements of the sequence is $2^{-l}$ close to
a fixed number, namely $\|X_{r}\|$, and therefore the norms
of any two such elements can't differ form each other by more than $2^{-l+1}$.
While the second condition follows immediately, to prove the first condition, assume towards contradiction
\[\forall k_0 \exists k\geq k_0 \quad \big( \|X_k\|^2> \|X_{r}\|^2 + 2^{-l}\big).\]
Applied to $k_0 = r + K(l)$ this implies
\[ \exists k\geq r+K(l) \quad \big( \|X_k\|^2 > \|X_{r}\|^2 + 2^{-l}\big),\]
which is a contradiction to \eqref{one} applied to $m=n$ and  \eqref{two}:
\[ \forall n^0,k^0\quad \big( \|X_{n+k}\|^2 \leq \|X_{n}\|^2 + \frac{\delta_k}{4}\big) \quad\wedge\quad 
\forall n^0 \forall i^0\geq K(n)\quad  \big( \delta_i\leq 2^{-n}\big). \]
This concludes the proof that $(\|X_n\|)$ is a Cauchy sequence:
\[\forall l \exists k_0 \forall k\geq k_0 \quad  \Big(\ \big|\ \|X_k\|^2 - \|X_{k_0}\|^2 \big| \leq 2^{-l}\ \Big). \]
We rewrite this using the n.c.i. Let $R$ denote a bound for the n.c.i. of the existence
of the approximate infimum of the sequence, i.e.:
\[ \forall l,u \exists r\leq R(l,u) \forall i\leq u(r) 
	\quad \big( \|X_{i}\|^2 + 2^{-l}\geq \|X_{r}\|^2\big)\tag{R}\label{e:R},\]
as it is defined in~\cite{Kohlenbach08} (note that since $R$ does not depend on the sequence, it does not
matter, whether we consider $(\|X_n\|)$ or $(\|X_n\|^2)$, except that we have to consider
a bound for $(\|X_0\|^2)$ rather than $(\|X_0\|)$).
We have ($N_0(l,h)=R(l+1,u)+K(l)$ with $ u:\equiv\lambda n . (n + K(l))+h(n + K(l))$)
\begin{align*}
\forall l,h \exists n\leq N_0(l,h) \quad  \Big(\ \big|\ \|X_{n+h(n)}\|^2 - \|X_{n}\|^2\big| \leq 2^{-l}\ \Big),  \tag{N0}\label{e:N_0} 
\end{align*}
since the following holds (here $N_0'(l,h,r) = r+K(l)$):
\begin{align*}
\forall l,h \exists r\leq R(l+1,u)\ \big(
		  & \|X_{N_0'(l,h,r)+h(N_0'(l,h,r))}\|^2\leq \|X_{r}\|^2 + 2^{-l-1}\ \wedge\\
		  & \|X_{N_0'(l,h,r)+h(N_0'(l,h,r))}\|^2 + 2^{-l-1}\geq \|X_{r}\|^2 \big).
\end{align*}
The second inequality follows from~\eqref{e:R} (for $u$ as above) since
\begin{align*}
u(r)&=(\lambda n . (n + K(l))+h(n + K(l))) (r) = r + K(l) + h(r + K(l)) \\
&=N_0'(l,h,r)+h(N_0'(l,h,r)).
\end{align*}
The first condition follows from
\begin{align*}
\|X_{N_0'(l,h,r)+h(N_0'(l,h,r))}\|^2&=\|X_{r+K(l)+h(N_0'(l,h,r))}\|^2\\
&\leq\|X_{r}\|^2+\frac{\delta_{K(l)+h(N_0'(l,h,r))}}{4}\leq\|X_{r}\|^2 + 2^{-l-1}.
\end{align*}
Note that for all $r\leq R(l,u)$ we have that $N_0'(l,h,r)\leq N_0'(l,h,R(l,u)) = N_0(l,h)$.
Finally, given any $h$ in the claim, we can define 
\begin{align*}
h'(n):=&\min\Big\{ i\in [0;h(n)]\, \Big|\, 
 \forall j\in[0;h(n)]\ \Big(\big|\, \|X_{n+i}\|^2 - \|X_n\|^2 \big| \geq \big|\, \|X_{n+j}\|^2- \|X_n\|^2 \big|\Big) \Big\}.
\end{align*}
Now the claim follows from~\eqref{e:N_0} applied to $h'$, the triangle inequality and the fact
that we actually prove not only that  
$\|X_{i}\|^2 - \|X_j\|^2\leq 2^{-l}$ 
but also
$|\ \|X_{i}\|^2 - \|X_j\|^2 |\leq 2^{-l}$, and 
$
N_0(l,h')\leq N_0(l,h^M),
$
which follows from lemma~\ref{l:M} (we discuss a similar argument in the proof of Corollary~\ref{c:finInt} in more detail). \\
\end{proof}

\begin{lemma*}[$P_0$ is correct]{l:P0}
\[
\forall l,f,n\exists p\leq P_0(l,f)\ ( C(l,n,p)\leq C(l,n,f(p)) + 2^{-l} ).
\]
\end{lemma*}
\begin{proof}
Given any $n$ and any $l$, the sequence $(a_i)$ defined by
%\footnote{Note that
%we can consider $a_i:=C(i,n,n)$ if $n>>l$ which is always true in our case.}
$
a_i:=C(l,n,i)
$
is monotone, since for $i<j$ we have 
$\widetilde{S_{i,l}} \subseteq \widetilde{S_{j,l}}$ (in the sense that$\forall \tup s \in S_{i,l}\exists \tup s'\in S_{j,l}\ \widetilde{\tup s}=\widetilde{\tup s'}$) 
and therefore also $C(l,n,i)\geq C(l,n,j)$. Hence the claim follows from Proposition 2.26 in~\cite{Kohlenbach08}.
\end{proof}

Next we prove the three lemmas, which give a quantitative analysis of the original proof in~\cite{Wittmann90}.\\

\begin{lemma*}[The scalar product increase is bounded]{l:scpb}
For any $l$ and any $g$, consider $h:= H(l,g^M)$. Let
$n$ be a witness for Lemma~\ref{l:N0}, i.e. 
\[
n\leq N(l,h)\ \wedge\ \forall i,j\in[n;n+h(n)]\ 
\big( i\leq j\rightarrow \|X_i\|^2-\|X_j\|^2\leq 2^{-l-1} \big) . \tag{N}\label{e:N1}
\]
Moreover let $f:=F(l,g^M,n)$,
%$p$ be a witness for Lemma~\ref{l:P0}, i.e. 
%\begin{align*}
%p\leq P_0(l,f)\ \wedge\ ( C(l,n,p)\leq C(l,n,f(p)) + 2^{-l}  ),  \tag{P}\label{e:P1}\\
%\end{align*} 
$p$ be a number smaller than $P_0(l,f)$
and  $m:=M_0(l,n,p)$. Then we have that
\[ 
\langle X_{a+k},X_{b+k} \rangle \leq \langle X_{a},X_{b} \rangle + 2^{-l}
\]
holds for all $k,a,b$ s.t. $K(l)\leq k \leq K(l)+m+g^M(m)$ and $ n\leq a,b \leq n+p$.
\end{lemma*}
\begin{proof}
We have
\[
\|X_{a+k}+X_{b+k}\|^2 \leq \|X_{a}+X_{b}\|^2 + 2^{-l} \tag{1}\label{e:sc1}
\] since $k\geq K(l)$. Moreover we can infer
\[
\|X_{a+k}\|^2 \geq \|X_{a}\|^2 - 2^{-l-1} \wedge \|X_{b+k}\|^2 \geq \|X_{b}\|^2 - 2^{-l-1} \tag{2}\label{e:sc2}
\] from~\eqref{e:N1}, $a\geq n$, $b \geq n$, and 
\begin{align*}
 a+k,b+k &\leq n+p+m+g^M(m)+K(l) \\
 &= n+p+M_0(l,n,p)+g^M(M_0(l,n,p))+K(l) \\
% &\leq n + \max\{ p+M_0(l,n,p)+g(M_0(l,n,p))+K(l) | p\leq P_0(l,F(l,g,n)) \} \\
% &=n+H(l,g)(n) = n+h(n).
 &\leq n+H(l,g^M)(n) = n+h(n). 
\end{align*}
Therefore
\begin{align*}
\langle X_{a+k},X_{b+k} \rangle &= \frac{1}{2}( \|X_{a+k}+X_{b+k}\|^2 - \|X_{a+k}\|^2 - \|X_{b+k}\|^2 )\\
&\leq \frac{1}{2}( \|X_{a}+X_{b}\|^2 + 2^{-l} - \|X_{a}\|^2 + 2^{-l-1} - \|X_{b}\|^2  + 2^{-l-1})\\
&= \langle X_{a},X_{b} \rangle + 2^{-l}.
\end{align*}
\end{proof}

\begin{lemma*}[$Z$s are close]{l:Zs}
For any $l$ and any $g$, consider $h:= H(l,g^M)$. Let
$n$ be a witness for Lemma~\ref{l:N0}, i.e. 
\[
n\leq N(l,g^M)\ \wedge\ \forall i,j\in[n;n+h(n)]\ 
\big( i\leq j\rightarrow \|X_i\|^2-\|X_j\|^2\leq 2^{-l-1} \big) . \tag{N} \label{e:N2}
\]
Moreover let $m:=M_0(l,n,p)$, $f:=F(l,g^M,n)$ and 
$p$ be a witness for Lemma~\ref{l:P0}, i.e. 
\begin{align*}
p\leq P_0(l,f)\ \wedge\ ( C(l,n,p)\leq C(l,n,f(p)) + 2^{-l}  ),  \tag{P}\label{e:P2}\\
\end{align*} 
Then we have that
$
\|Z(l,n,p,m) - Z( l,n,p,m+g(m) ) \|^2 \leq 2^{-l+4}.
$
%holds for $m=M_0(l,n,p)$.
\end{lemma*}
\begin{proof}
Firstly, we will show that
\[
\|\frac{1}{2}( Z(l,n,p,m) + Z( l,n,p,m+g(m) )  )\|^2 + 2^{-l+1} \geq C(l,n,p). \tag{1}\label{e:Zup}
\]
Since $\frac{1}{2}( Z(l,n,p,m) + Z( l,n,p,m+g(m) )  )$ is a convex
combination of \[ X_{n+K(l)},\ldots,X_{n+K(l)+p+m+g(m)},\] 
we obtain by Lemma~\ref{l:newC} that
\begin{align*}
\Big\|\frac{1}{2}( Z(l,n,p,m) + Z( l,n,p,m&+g(m) )  )\Big\|^2 + 2^{-l} \geq\\ &C( l,n,n+K(l)+p+m+g^M(m) ). 
\end{align*}
Now, because of 
\begin{align*}
f(p)&=p+n+K(l)+M_0(l,n,p)+g^M(M_0(l,n,p))\\&=n+K(l)+p+m+g^M(m), 
\end{align*}
it follows from~\eqref{e:P2} that
\[
 C( l,n,n+K(l)+p+m+g^M(m) ) \geq  C(l,n,p) - 2^{-l}, % \tag{1.1}\label{e:ZupPrime}
\]
which concludes the proof of~\eqref{e:Zup}.
Secondly, we will show that \[
\forall o\leq m+g(m)\ \big( \big\| Z( l,n,p,o ) \big\|^2\leq C(l,n,p) + 2^{-l} \big). \tag{2}\label{e:Zdown}
\] 
Let $\tup s$ be the tuple corresponding to the tuple in the definition of $C(l,n,p)$ (note that $\tilde{\tup s}=\tup s$).
By Lemma~\ref{l:scpb} we have
\begin{align*}
\bigg\|\sum^{p}_{i=0} s_i X_{n+k+i}\bigg\|^2&=\sum^{p}_{i,j=0}  s_i s_j \langle X_{n+k+i},X_{n+k+j} \rangle \\
&\leq \sum^{p}_{i,j=0}  s_i s_j \langle X_{n+i},X_{n+j} \rangle + \sum^{p}_{i,j=0}  s_i s_j 2^{-l}=\bigg\|\sum^{p}_{i=0} s_i X_{n+i}\bigg\|^2+2^{-l},
\end{align*}
for all $K(l)\leq k \leq K(l)+m+g^M(m)$, since $n\leq n+i,n+j\leq n+p$. Together with the convexity of the square function (and the definition of $Z$) this implies~\eqref{e:Zdown}.\\
Finally, the claim follows from \eqref{e:Zup}
%, \eqref{e:ZupPrime} 
and \eqref{e:Zdown} by the parallelogram identity:
\begin{align*}
\big\|Z&(l,n,p,m) - Z( l,n,p,\tilde g(m) ) \big\|^2 =\\ 
&=2\big\|Z( l,n,p,m)\big\|^2 + 2\big\|Z( l,n,p,\tilde g(m) )\big\|^2  - \big\|Z(l,n,p,m) + Z( l,n,p,\tilde g(m) )\big\|^2 \\
&\leq 4( C(l,n,p) + 2^{-l} ) - 4( C(l,n,p) - 2^{-l+1} ) = 2^{-l+2} + 2^{-l+3} \leq 2^{-l+4}.
\end{align*}
\end{proof}

\begin{lemma*}[$Z$s and $A$s are close]{l:ZA}
For any $l$ and any $g$ let $h:= H(l,g^M)$ and
$n$ be a witness for Lemma~\ref{l:N0}, i.e. 
\[
n\leq N(l,g^M)\ \wedge\ \forall i,j\in[n;n+h(n)]\ \big( \ i\leq j\rightarrow \|X_i\|^2-\|X_j\|^2\leq 2^{-l-1}\big). \tag{N}\label{e:N3}
\]
Moreover let $f:=F(l,g^M,n)$,
$p$ be a witness for Lemma~\ref{l:P0}, i.e. 
\begin{align*}
p\leq P_0(l,f)\ \wedge\ ( C(l,n,p)\leq C(l,n,f(p)) + 2^{-l}  ),  \tag{P}\label{e:P3}\\
\end{align*} 
and set $m:=M_0(l,n,p)$, $m':=m+g(m)$.
Then we have that
\[
 \big\|A_{m+1} - Z( l,n,p,m )\big\|\leq \frac{1}{m+1}(2n + 2p + 2K(l))B +2^{-l}
\]
and
\[
 \big\|A_{m'+1} - Z( l,n,p,m' )\big\|\leq \frac{1}{m'+1}(2n + 2p + 2K(l))B +2^{-l}.
\]
\end{lemma*}

\begin{proof}
From the definition of $Z$ we see that (note that $m,m'\geq p$):
\begin{align*}
(m+1)Z(l,n,p,m) - \sum^{n+K(l)+m}_{i=n+p+K(l)} X_{i} = 
\sum^{p-1}_{i=0} t_iX_{n+K(l)+i} + \sum^{p}_{i=l} r_iX_{n+K(l)+m+i},
\end{align*}
for suitable $\tup t$ and $\tup r$ with $0\leq t_i,r_i\leq 1$. Hence (note that $m,m'\geq K(l)+n+p$)
\begin{align*}
(m&+1)\big\|Z( l,n,p,m ) - A_{m+1}\big\| =
	 \bigg\| \sum^{K(l)+m}_{k=K(l)}\sum^{p}_{i=0}  \widetilde{\tup s}_i X_{n+k+i} - \sum^{m+1}_{i=1}X_i \bigg\|	\\
	 &=\bigg\| \sum^{p-1}_{i=0} t_iX_{n+K(l)+i} + \sum^{p}_{i=1} r_iX_{n+K(l)+m+i} + \sum^{n+K(l)+m}_{i=n+p+K(l)} X_{i} 
	      - \sum^{m+1}_{i=1}X_i \bigg\| \\
	 &= \bigg\| \sum^{p-1}_{i=0} t_iX_{n+K(l)+i} + \sum^{p}_{i=1} r_iX_{n+K(l)+m+i} + \sum^{n+K(l)+m}_{i=m+2} X_{i} - \sum^{n+p+K(l)-1}_{i=1} X_{i}  \bigg\| \\
	 &\leq \bigg\| \sum^{p-1}_{i=0} t_iX_{n+K(l)+i} - \sum^{n+p+K(l)-1}_{i=1} X_{i} \bigg\| + \bigg\|\sum^{p}_{i=1} r_iX_{n+K(l)+m+i} \bigg\| +  \bigg\| \sum^{n+K(l)+m}_{i=m+2} X_{i} \bigg\|  \\
	 &\leq (n+p+K(l)-1)B + pB + (n+K(l)-1)B \leq (2n + 2p + 2K(l))B.
\end{align*}
Obviously, same holds for $m'$.\\
\end{proof}


\subsection*{Acknowledgements}
I wish to thank Ulrich Kohlenbach for helpful discussions and for suggesting the topic.
This research was supported by the German Science Foundation (DFG Project KO 1737/5-1). 

\nocite{*}
\nocommandslikethis[}
\bibliographystyle{plain}
\addcontentsline{toc}{section}{Bibliography}
\renewcommand{\refname}{Bibliography}
\markright{\textsc{Bibliography}}
\bibliography{diplomarbeit}
\markright{\textsc{Bibliography}}


\end{document}
