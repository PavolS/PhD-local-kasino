\subsection{Lemmas}\label{s:Lemmas}

Here, we assume that the assumptions of Theorem~\ref{t:fin} hold and use the terms as they are defined in that theorem.
We use also the definitions~\ref{d:C} and \ref{d:Z} for $C$ and $Z$.
Moreover, w.l.o.g we can assume $K^M=K$, since the original assumption implies
\[ \forall l\in\NN\forall n\geq K^M(l) \ \big(\|\delta_n\|<2^{-l}\big). \]
We prove Lemma~\ref{l:newC} first and then the two lemmas which show 
that $N_0$ and $P_0$ are the right
witnesses for the two main assumptions needed in Wittmann's proof. The fact
that $M$ majorizes itself (and therefore so does $M'$) 
follows simply from its definition in Theorem~\ref{t:fin}.\\

\begin{lemma*}[$C$ approximates the smallest convex combination]{l:newC}
\[
\forall l,n,p,f \forall \tup s\ \big( C'(\tup s,l,n,p,f)+2^{-l}\geq C(l,n,p) \big).
\]
\end{lemma*}
\begin{proof}
Given $\tup s$ choose $\tup s'\in S_{p,l}$ s.t. $|s'_i-\tilde {\tup s}(i)|\leq \frac{2^{-(l+1)}}{pB^2}$ for all $i\in[0;p]$. 
Then we have that
\begin{align*}
\left\|\sum_{i=0}^{p} \widetilde{\tup s}(i) X_{n+i} - \sum_{i=0}^{p} {s'}_i X_{n+i}\right\| = 
\left\|\sum_{i=0}^{p} |\widetilde{\tup s}(i)-s'_i| X_{n+i} \right\| \leq
\frac{2^{-(l+1)}}{pB^2}pB = \frac{2^{-(l+1)}}{B},
\end{align*}
and therefore also that
$
\left|\ \left\|\sum_{i=0}^{p} \widetilde{\tup s}(i) X_{n+i}\right\| - \left\|\sum_{i=0}^{p} {s'}_i X_{n+i}\right\| \ \right|\leq\frac{2^{-(l+1)}}{B},
$
so finally we get that
\[
\left|\ \left\|\sum_{i=0}^{p} \widetilde{\tup s}(i) X_{n+i}\right\|^2 - \left\|\sum_{i=0}^{p} {s'}_i X_{n+i}\right\|^2\  \right| \leq \frac{2^{-(l+1)}}{B} ( B + \frac{2^{-(l+1)}}{B} ) \leq 2^{-l}.
\]
\end{proof}

%\begin{lemma*}[$M$ is a majorant]{l:M}
%Each of the terms $M$,$P$,$N$,$M_0$,$P_0$,$N_0$,$R$ majorizes itself.
%In particular we have:
%\[ \forall l'\geq l\forall h',h ( \forall n (h(n)\geq h'(n)) \rightarrow N_0(l',h^M)\geq N_0(l,h') ) \]
%and
%\[ \forall l,g \forall n\leq N(l,g^M) \forall p\leq P(l,g^M)\ ( P(l,g^M)\geq P_0(l,F(l,g^M,n))\ \wedge\ M(l,g^M)\geq M_0(l,n,p) ) . \]
%%and  
%%\[ \forall l \forall g \forall n'\geq n (\ (H(l,g))(n') \geq (H(l,g))(n)\ ). \]
%\end{lemma*}

\begin{lemma*}[$N_0$ is correct]{l:N0}
The sentence
\[
\forall l,h \exists n \forall i,j\in[n;n+h(n)]\ \big( i\leq j\rightarrow \|X_i\|^2-\|X_j\|^2\leq 2^{-l}\big)
\]
is witnessed by an $n\leq N_0(l,h)$.
\end{lemma*}
\begin{proof}
The sequence $(\|X_n\|)$ is bounded from below therefore we have:
\[ \forall l \exists r \forall k \ \big( \|X_k\|^2 + 2^{-l}\geq \|X_r\|^2\big).\]
%and in particular
%\[ \forall l \exists r \forall k\geq r \|X_k\| + 2^{-l}\geq \|X_r\|\]
For any given $l$ we fix such an $r$. The following statements 
imply that $(\|X_n\|)$ is a Cauchy sequence:
\begin{enumerate}
	\item\label{e1} $\exists k_0 \forall k\geq k_0 \quad \big(\|X_k\|^2\leq \|X_{r}\|^2 + 2^{-l}\big)$, and
	\item\label{e2} $\forall k \quad \big(\|X_k\|^2 + 2^{-l}\geq \|X_{r}\|^2\big) $,\\
\end{enumerate}
since \eqref{e1}$\wedge$\eqref{e2} means that there is an index
from which on the norm of any of the remaining elements of the sequence is $2^{-l}$ close to
a fixed number, namely $\|X_{r}\|$, and therefore the norms
of any two such elements can't differ form each other by more than $2^{-l+1}$.
While the second condition follows immediately, to prove the first condition, assume towards contradiction
\[\forall k_0 \exists k\geq k_0 \quad \big( \|X_k\|^2> \|X_{r}\|^2 + 2^{-l}\big).\]
Applied to $k_0 = r + K(l)$ this implies
\[ \exists k\geq r+K(l) \quad \big( \|X_k\|^2 > \|X_{r}\|^2 + 2^{-l}\big),\]
which is a contradiction to \eqref{one} applied to $m=n$ and  \eqref{two}:
\[ \forall n^0,k^0\quad \big( \|X_{n+k}\|^2 \leq \|X_{n}\|^2 + \frac{\delta_k}{4}\big) \quad\wedge\quad 
\forall n^0 \forall i^0\geq K(n)\quad  \big( \delta_i\leq 2^{-n}\big). \]
This concludes the proof that $(\|X_n\|)$ is a Cauchy sequence:
\[\forall l \exists k_0 \forall k\geq k_0 \quad  \Big(\ \big|\ \|X_k\|^2 - \|X_{k_0}\|^2 \big| \leq 2^{-l}\ \Big). \]
We rewrite this using the n.c.i. Let $R$ denote a bound for the n.c.i. of the existence
of the approximate infimum of the sequence, i.e.:
\[ \forall l,u \exists r\leq R(l,u) \forall i\leq u(r) 
	\quad \big( \|X_{i}\|^2 + 2^{-l}\geq \|X_{r}\|^2\big)\tag{R}\label{e:R},\]
as it is defined in~\cite{Kohlenbach08} (note that since $R$ does not depend on the sequence, it does not
matter, whether we consider $(\|X_n\|)$ or $(\|X_n\|^2)$, except that we have to consider
a bound for $(\|X_0\|^2)$ rather than $(\|X_0\|)$).
We have ($N_0(l,h)=R(l+1,u)+K(l)$ with $ u:\equiv\lambda n . (n + K(l))+h(n + K(l))$)
\begin{align*}
\forall l,h \exists n\leq N_0(l,h) \quad  \Big(\ \big|\ \|X_{n+h(n)}\|^2 - \|X_{n}\|^2\big| \leq 2^{-l}\ \Big),  \tag{N0}\label{e:N_0} 
\end{align*}
since the following holds (here $N_0'(l,h,r) = r+K(l)$):
\begin{align*}
\forall l,h \exists r\leq R(l+1,u)\ \big(
		  & \|X_{N_0'(l,h,r)+h(N_0'(l,h,r))}\|^2\leq \|X_{r}\|^2 + 2^{-l-1}\ \wedge\\
		  & \|X_{N_0'(l,h,r)+h(N_0'(l,h,r))}\|^2 + 2^{-l-1}\geq \|X_{r}\|^2 \big).
\end{align*}
The second inequality follows from~\eqref{e:R} (for $u$ as above) since
\begin{align*}
u(r)&=(\lambda n . (n + K(l))+h(n + K(l))) (r) = r + K(l) + h(r + K(l)) \\
&=N_0'(l,h,r)+h(N_0'(l,h,r)).
\end{align*}
The first condition follows from
\begin{align*}
\|X_{N_0'(l,h,r)+h(N_0'(l,h,r))}\|^2&=\|X_{r+K(l)+h(N_0'(l,h,r))}\|^2\\
&\leq\|X_{r}\|^2+\frac{\delta_{K(l)+h(N_0'(l,h,r))}}{4}\leq\|X_{r}\|^2 + 2^{-l-1}.
\end{align*}
Note that for all $r\leq R(l,u)$ we have that $N_0'(l,h,r)\leq N_0'(l,h,R(l,u)) = N_0(l,h)$.
Finally, given any $h$ in the claim, we can define 
\begin{align*}
h'(n):=&\min\Big\{ i\in [0;h(n)]\, \Big|\, 
 \forall j\in[0;h(n)]\ \Big(\big|\, \|X_{n+i}\|^2 - \|X_n\|^2 \big| \geq \big|\, \|X_{n+j}\|^2- \|X_n\|^2 \big|\Big) \Big\}.
\end{align*}
Now the claim follows from~\eqref{e:N_0} applied to $h'$, the triangle inequality and the fact
that we actually prove not only that  
$\|X_{i}\|^2 - \|X_j\|^2\leq 2^{-l}$ 
but also
$|\ \|X_{i}\|^2 - \|X_j\|^2 |\leq 2^{-l}$, and 
$
N_0(l,h')\leq N_0(l,h^M),
$
which follows from lemma~\ref{l:M} (we discuss a similar argument in the proof of Corollary~\ref{c:finInt} in more detail). \\
\end{proof}

\begin{lemma*}[$P_0$ is correct]{l:P0}
\[
\forall l,f,n\exists p\leq P_0(l,f)\ ( C(l,n,p)\leq C(l,n,f(p)) + 2^{-l} ).
\]
\end{lemma*}
\begin{proof}
Given any $n$ and any $l$, the sequence $(a_i)$ defined by
%\footnote{Note that
%we can consider $a_i:=C(i,n,n)$ if $n>>l$ which is always true in our case.}
$
a_i:=C(l,n,i)
$
is monotone, since for $i<j$ we have 
$\widetilde{S_{i,l}} \subseteq \widetilde{S_{j,l}}$ (in the sense that$\forall \tup s \in S_{i,l}\exists \tup s'\in S_{j,l}\ \widetilde{\tup s}=\widetilde{\tup s'}$) 
and therefore also $C(l,n,i)\geq C(l,n,j)$. Hence the claim follows from Proposition 2.26 in~\cite{Kohlenbach08}.
\end{proof}

Next we prove the three lemmas, which give a quantitative analysis of the original proof in~\cite{Wittmann90}.\\

\begin{lemma*}[The scalar product increase is bounded]{l:scpb}
For any $l$ and any $g$, consider $h:= H(l,g^M)$. Let
$n$ be a witness for Lemma~\ref{l:N0}, i.e. 
\[
n\leq N(l,h)\ \wedge\ \forall i,j\in[n;n+h(n)]\ 
\big( i\leq j\rightarrow \|X_i\|^2-\|X_j\|^2\leq 2^{-l-1} \big) . \tag{N}\label{e:N1}
\]
Moreover let $f:=F(l,g^M,n)$,
%$p$ be a witness for Lemma~\ref{l:P0}, i.e. 
%\begin{align*}
%p\leq P_0(l,f)\ \wedge\ ( C(l,n,p)\leq C(l,n,f(p)) + 2^{-l}  ),  \tag{P}\label{e:P1}\\
%\end{align*} 
$p$ be a number smaller than $P_0(l,f)$
and  $m:=M_0(l,n,p)$. Then we have that
\[ 
\langle X_{a+k},X_{b+k} \rangle \leq \langle X_{a},X_{b} \rangle + 2^{-l}
\]
holds for all $k,a,b$ s.t. $K(l)\leq k \leq K(l)+m+g^M(m)$ and $ n\leq a,b \leq n+p$.
\end{lemma*}
\begin{proof}
We have
\[
\|X_{a+k}+X_{b+k}\|^2 \leq \|X_{a}+X_{b}\|^2 + 2^{-l} \tag{1}\label{e:sc1}
\] since $k\geq K(l)$. Moreover we can infer
\[
\|X_{a+k}\|^2 \geq \|X_{a}\|^2 - 2^{-l-1} \wedge \|X_{b+k}\|^2 \geq \|X_{b}\|^2 - 2^{-l-1} \tag{2}\label{e:sc2}
\] from~\eqref{e:N1}, $a\geq n$, $b \geq n$, and 
\begin{align*}
 a+k,b+k &\leq n+p+m+g^M(m)+K(l) \\
 &= n+p+M_0(l,n,p)+g^M(M_0(l,n,p))+K(l) \\
% &\leq n + \max\{ p+M_0(l,n,p)+g(M_0(l,n,p))+K(l) | p\leq P_0(l,F(l,g,n)) \} \\
% &=n+H(l,g)(n) = n+h(n).
 &\leq n+H(l,g^M)(n) = n+h(n). 
\end{align*}
Therefore
\begin{align*}
\langle X_{a+k},X_{b+k} \rangle &= \frac{1}{2}( \|X_{a+k}+X_{b+k}\|^2 - \|X_{a+k}\|^2 - \|X_{b+k}\|^2 )\\
&\leq \frac{1}{2}( \|X_{a}+X_{b}\|^2 + 2^{-l} - \|X_{a}\|^2 + 2^{-l-1} - \|X_{b}\|^2  + 2^{-l-1})\\
&= \langle X_{a},X_{b} \rangle + 2^{-l}.
\end{align*}
\end{proof}

\begin{lemma*}[$Z$s are close]{l:Zs}
For any $l$ and any $g$, consider $h:= H(l,g^M)$. Let
$n$ be a witness for Lemma~\ref{l:N0}, i.e. 
\[
n\leq N(l,g^M)\ \wedge\ \forall i,j\in[n;n+h(n)]\ 
\big( i\leq j\rightarrow \|X_i\|^2-\|X_j\|^2\leq 2^{-l-1} \big) . \tag{N} \label{e:N2}
\]
Moreover let $m:=M_0(l,n,p)$, $f:=F(l,g^M,n)$ and 
$p$ be a witness for Lemma~\ref{l:P0}, i.e. 
\begin{align*}
p\leq P_0(l,f)\ \wedge\ ( C(l,n,p)\leq C(l,n,f(p)) + 2^{-l}  ),  \tag{P}\label{e:P2}\\
\end{align*} 
Then we have that
$
\|Z(l,n,p,m) - Z( l,n,p,m+g(m) ) \|^2 \leq 2^{-l+4}.
$
%holds for $m=M_0(l,n,p)$.
\end{lemma*}
\begin{proof}
Firstly, we will show that
\[
\|\frac{1}{2}( Z(l,n,p,m) + Z( l,n,p,m+g(m) )  )\|^2 + 2^{-l+1} \geq C(l,n,p). \tag{1}\label{e:Zup}
\]
Since $\frac{1}{2}( Z(l,n,p,m) + Z( l,n,p,m+g(m) )  )$ is a convex
combination of \[ X_{n+K(l)},\ldots,X_{n+K(l)+p+m+g(m)},\] 
we obtain by Lemma~\ref{l:newC} that
\begin{align*}
\Big\|\frac{1}{2}( Z(l,n,p,m) + Z( l,n,p,m&+g(m) )  )\Big\|^2 + 2^{-l} \geq\\ &C( l,n,n+K(l)+p+m+g^M(m) ). 
\end{align*}
Now, because of 
\begin{align*}
f(p)&=p+n+K(l)+M_0(l,n,p)+g^M(M_0(l,n,p))\\&=n+K(l)+p+m+g^M(m), 
\end{align*}
it follows from~\eqref{e:P2} that
\[
 C( l,n,n+K(l)+p+m+g^M(m) ) \geq  C(l,n,p) - 2^{-l}, % \tag{1.1}\label{e:ZupPrime}
\]
which concludes the proof of~\eqref{e:Zup}.
Secondly, we will show that \[
\forall o\leq m+g(m)\ \big( \big\| Z( l,n,p,o ) \big\|^2\leq C(l,n,p) + 2^{-l} \big). \tag{2}\label{e:Zdown}
\] 
Let $\tup s$ be the tuple corresponding to the tuple in the definition of $C(l,n,p)$ (note that $\tilde{\tup s}=\tup s$).
By Lemma~\ref{l:scpb} we have
\begin{align*}
\bigg\|\sum^{p}_{i=0} s_i X_{n+k+i}\bigg\|^2&=\sum^{p}_{i,j=0}  s_i s_j \langle X_{n+k+i},X_{n+k+j} \rangle \\
&\leq \sum^{p}_{i,j=0}  s_i s_j \langle X_{n+i},X_{n+j} \rangle + \sum^{p}_{i,j=0}  s_i s_j 2^{-l}=\bigg\|\sum^{p}_{i=0} s_i X_{n+i}\bigg\|^2+2^{-l},
\end{align*}
for all $K(l)\leq k \leq K(l)+m+g^M(m)$, since $n\leq n+i,n+j\leq n+p$. Together with the convexity of the square function (and the definition of $Z$) this implies~\eqref{e:Zdown}.\\
Finally, the claim follows from \eqref{e:Zup}
%, \eqref{e:ZupPrime} 
and \eqref{e:Zdown} by the parallelogram identity:
\begin{align*}
\big\|Z&(l,n,p,m) - Z( l,n,p,\tilde g(m) ) \big\|^2 =\\ 
&=2\big\|Z( l,n,p,m)\big\|^2 + 2\big\|Z( l,n,p,\tilde g(m) )\big\|^2  - \big\|Z(l,n,p,m) + Z( l,n,p,\tilde g(m) )\big\|^2 \\
&\leq 4( C(l,n,p) + 2^{-l} ) - 4( C(l,n,p) - 2^{-l+1} ) = 2^{-l+2} + 2^{-l+3} \leq 2^{-l+4}.
\end{align*}
\end{proof}

\begin{lemma*}[$Z$s and $A$s are close]{l:ZA}
For any $l$ and any $g$ let $h:= H(l,g^M)$ and
$n$ be a witness for Lemma~\ref{l:N0}, i.e. 
\[
n\leq N(l,g^M)\ \wedge\ \forall i,j\in[n;n+h(n)]\ \big( \ i\leq j\rightarrow \|X_i\|^2-\|X_j\|^2\leq 2^{-l-1}\big). \tag{N}\label{e:N3}
\]
Moreover let $f:=F(l,g^M,n)$,
$p$ be a witness for Lemma~\ref{l:P0}, i.e. 
\begin{align*}
p\leq P_0(l,f)\ \wedge\ ( C(l,n,p)\leq C(l,n,f(p)) + 2^{-l}  ),  \tag{P}\label{e:P3}\\
\end{align*} 
and set $m:=M_0(l,n,p)$, $m':=m+g(m)$.
Then we have that
\[
 \big\|A_{m+1} - Z( l,n,p,m )\big\|\leq \frac{1}{m+1}(2n + 2p + 2K(l))B +2^{-l}
\]
and
\[
 \big\|A_{m'+1} - Z( l,n,p,m' )\big\|\leq \frac{1}{m'+1}(2n + 2p + 2K(l))B +2^{-l}.
\]
\end{lemma*}

\begin{proof}
From the definition of $Z$ we see that (note that $m,m'\geq p$):
\begin{align*}
(m+1)Z(l,n,p,m) - \sum^{n+K(l)+m}_{i=n+p+K(l)} X_{i} = 
\sum^{p-1}_{i=0} t_iX_{n+K(l)+i} + \sum^{p}_{i=l} r_iX_{n+K(l)+m+i},
\end{align*}
for suitable $\tup t$ and $\tup r$ with $0\leq t_i,r_i\leq 1$. Hence (note that $m,m'\geq K(l)+n+p$)
\begin{align*}
(m&+1)\big\|Z( l,n,p,m ) - A_{m+1}\big\| =
	 \bigg\| \sum^{K(l)+m}_{k=K(l)}\sum^{p}_{i=0}  \widetilde{\tup s}_i X_{n+k+i} - \sum^{m+1}_{i=1}X_i \bigg\|	\\
	 &=\bigg\| \sum^{p-1}_{i=0} t_iX_{n+K(l)+i} + \sum^{p}_{i=1} r_iX_{n+K(l)+m+i} + \sum^{n+K(l)+m}_{i=n+p+K(l)} X_{i} 
	      - \sum^{m+1}_{i=1}X_i \bigg\| \\
	 &= \bigg\| \sum^{p-1}_{i=0} t_iX_{n+K(l)+i} + \sum^{p}_{i=1} r_iX_{n+K(l)+m+i} + \sum^{n+K(l)+m}_{i=m+2} X_{i} - \sum^{n+p+K(l)-1}_{i=1} X_{i}  \bigg\| \\
	 &\leq \bigg\| \sum^{p-1}_{i=0} t_iX_{n+K(l)+i} - \sum^{n+p+K(l)-1}_{i=1} X_{i} \bigg\| + \bigg\|\sum^{p}_{i=1} r_iX_{n+K(l)+m+i} \bigg\| +  \bigg\| \sum^{n+K(l)+m}_{i=m+2} X_{i} \bigg\|  \\
	 &\leq (n+p+K(l)-1)B + pB + (n+K(l)-1)B \leq (2n + 2p + 2K(l))B.
\end{align*}
Obviously, same holds for $m'$.\\
\end{proof}
