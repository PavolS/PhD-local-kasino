\section{Introduction}


%To generalize MET to nonlinear mappings one has
%to add additional requirements (to gain at least``some linearity'') or one obtains a weaker conclusion (such as different
%means instead of the Caesaro means or just weak convergence).
The Riesz version of the von Neumann mean ergodic theorem~\cite{Neumann1932} asserts that for any linear operator $T$ on a Hilbert space $X$, which is nonexpansive, i.e. 
\[
\forall u,v\in X\ \big(\|Tu-Tv\|\leq\|u-v\|\big),
%\forall u,v\in C \|Tu-Tv\|\leq\|u-v\|,
\]
the sequence of the  Ces{\`a}ro means
\[
A_nx:=\frac{1}{n+1}\sum^{n}_{i=0} T^i x,
\]
converges in norm for any starting point $x$. 
%It is known that this in general need not  
%be the case without linearity. 
It follows from an example 
by Genel and Lindenstrauss~\cite{Genel1975} that there is a nonexpansive operator on
the unit ball of $\ell_2$, for which the sequence of the  Ces{\`a}ro means does not converge strongly (see also~\cite{Krengel1985}).
So in comparison with von Neumann's linear mean ergodic theorem, in nonlinear ergodic theory 
one obtains either a weaker conclusion (such as weak convergence or convergence of a different
iteration scheme instead of the Ces{\`a}ro means) or one has to add additional requirements 
(to preserve at least some linearity).\\ 
Let $H$ be a Hilbert space, $C$ a subset of $H$ and
$T:C\to C$ a (possibly nonlinear) mapping. In 1975, Baillon~\cite{Baillon1975} showed  that if $C$ is convex and closed,
and $T$ is nonexpansive and has a fixed point, then the sequence of the Ces{\`a}ro means is weakly convergent to a fixed point of $T$.
%, i.e.
%\[
%\forall x\in C\exists w,y\in H\forall k\in\NN\ \big( \lim_{n\to\infty}\frac{1}{n}\sum^{n-1}_0T^{i+k}\wedge Ty=y\big).
%\]
A year later, Baillon~\cite{Baillon1976} also proved that if in addition $T$ is {\em odd}, i.e.
\[
-C = C\text{ and } \forall u\in C\ \big( T(-u)=-Tu\big),
\]
then the sequence of the Ces{\`a}ro means converges to a fixed point in norm. Shortly after this,
Br\'ezis and Browder~\cite{Brezis1976} showed that 
Baillon's first result is also true for a more general averaging process than the usual Ces{\`a}ro means
and that
Baillon's second
result remains valid if $0\in C$ and $T$ is not necessarily odd but satisfies the following, weaker condition:
\begin{align*}
\exists c&\in\RR\ \forall u,v \in C 
 \big(\|Tu+Tv\|^2\leq\|u+v\|^2+c(\|u\|^2-\|Tu\|^2+\|v\|^2-\|Tv\|^2)\big). \tag{BB}\label{e:BB}
\end{align*}
On the other hand, in 1979, Hirano and Takahashi~\cite{Hirano1979} showed that Baillon's weak convergence result remains true if the mapping is just {\em asymptotically} nonexpansive, i.e. 
\[
\forall u,v\in C\ \forall n\in\NN\ \big( \|T^nu-T^nv\|\leq \alpha_n\|u-v\|\big),
\]
for some sequence $(\alpha_n)_n$ of nonnegative real numbers which converges to $1$. 
Moreover, an odd and nonexpansive
mapping satisfies the following condition
\[
\forall u,v\in C\ (\| T^nu + T^nv \| \leq \|u + v\|) \tag{$W$}\label{e:W-intro}
\]
and analogously an odd and asymptotically nonexpansive mapping satifies the assymptotic version
\[
\forall n\in\NN\ \forall u,v\in C\ (\| T^nu + T^nv \| \leq \alpha_n\|u + v\|), \tag{$W^-$}\label{e:W-assym}
\]
for some sequence $(\alpha_n)_n$ of nonnegative real numbers which converges to $1$. In 1990, Wittmann~\cite{Wittmann90}
proved a generalization of Baillon's strong convergence theorem 
to an arbitrary $C$ and a mapping satisfying the condition~\eqref{e:W-assym} (see also Theorem 2.2 in~\cite{Wittmann90} and Theorem~\ref{t:fin21ar} below).
%\begin{thm}[Theorem 2.2 in~\cite{Wittmann90}] \label{t:W22}
%Let $S$ be a subset of a Hilbert space and $T:S\to S$
%be a mapping satisfying 
%\[
%\forall u,v\in C\ (\| T^nu + T^nv \| \leq \alpha_n\|x + y\|).
%\]
%Then for any $x\in S$ the sequence of the Ces{\`a}ro means
%\[
%A_nx:=\frac{1}{n+1}\sum^{n}_{i=0} T^i x
%\]
%is norm convergent.
%\end{thm}
Two years later, Wittmann~\cite{Wittmann1992} also showed that for a nonexpansive $T$ which has a fixed point, and a convex and closed $C$, the averaging sequence $(x_n)$, first defined by Halpern~\cite{Halpern1967} (for $x=0$) as
\[
x_0:=x,\quad x_{n+1}:=\alpha_{n+1}x+(1-\alpha_{n+1})T(x_n),
\] converges to the closest fixed point of $T$ in norm. The Halpern iteration coincides
with the Ces{\`a}ro means for linear maps and $\alpha_n=\frac{1}{n+1}$.
We depict this development in Figure~\ref{f:METtree} (the references in parentheses refer to
quantitative versions of the respective theorems, which we discuss below).
\begin{figure}[htb]%
\[
\begin{xy}
  \xymatrix@C=-50pt@M=3pt{
  		  & \txt{ MET : von Neumann  1931 \\ linear, nonexpansive operator $\ |\ $ strong convergence\\(Avigad, Gerhardy, Towsner 2010 in~\cite{AGT08})\footnotemark\\ ( Leu{\c{s}}tean, Kohlenbach 2009 in~\cite{Kohlenbach2009})
 		  } \ar[dl]_-{\txt{\tiny weak\\\tiny convergence}} \ar[d]^{\txt{\tiny odd}} \ar[dr]^-{\txt{\tiny $ $\hspace{2ex} Halpern\\ \tiny $ $\hspace{2ex} iteration}} &  \\
        \txt{ 					Baillon 1975 in~\cite{Baillon1975} \\ (Kohlenbach 2010 in~\cite{Kohlenbach_ComContemporaryMath})} \ar[d]^{\txt{\tiny more general\hspace{5ex}$ $\\ \vspace{-7pt}\tiny averaging process}} &  
        \txt{ Baillon 1976 in~\cite{Baillon1976} \\ (this paper) } \ar[d]_{\txt{\tiny \eqref{e:BB}}} \ar[dr]^{\txt{\tiny \eqref{e:W-assym}}} &  
        \txt{ Wittman 1992 in~\cite{Wittmann1992} \\ (Kohlenbach 2011 in~\cite{Kohlenbach2011})}        \\
         \txt{ Br\'ezis, Browder 1976 in~\cite{Brezis1976} \\ (Kohlenbach 2010 in~\cite{Kohlenbach_ComContemporaryMath}) }  & \txt{ Br\'ezis, Browder 1976 in~\cite{Brezis1976} \\ (unknown) } & \txt{Wittmann 1990 in~\cite{Wittmann90} \\ (this paper)} &   
   }
\end{xy}
\]
\caption{Some nonlinear ergodic theorems (for Hilbert spaces) and their finitisations.}%
\label{f:METtree}
\end{figure}
\footnotetext{While the results were essentially available on arxive since 2007, the paper as such was submitted in 2008. Thereafter Kohlenbach and Leu{\c{s}}tean 
extended the result to uniformly convex Banach spaces and gave a better bound.} \\					
There are many further results and generalizations
in the field of nonlinear ergodic theory 
(regarding different spaces see e.g.~\cite{Bruck1979,Hirano1982}, even weaker "linearity" conditions see e.g.~\cite{Miyadera1997,Rouhani2001}, and other improvements) 
and it is subject to ongoing research.\\
In this paper we investigate the computational content of Wittmann's nonlinear strong ergodic
theorem: 
\begin{thm}[Wittmann 1990,~\cite{Wittmann90}]\label{t:22}
Let $S$ be a subset of a Hilbert space and $T:S\to S$
be a mapping satisfying 
\[
\forall x,y\in S\ (\| T^nx + T^ny \| \leq \alpha\|x + y\|),
\lim_{n\to\infty}\alpha_n=1.
\]
Then for any $x\in S$ the sequence of the Ces{\`a}ro means,
\[
A_nx:=\frac{1}{n+1}\sum^{n}_{i=0} T^i x,
\]
is norm convergent.
\end{thm}
Although in general the sequence of the ergodic averages does not have a
computable rate of convergence (even for the 
von Neumann's mean ergodic theorem for a separable space and computable $x$ and $T$),
as was shown by Avigad, Gerhardy and Towsner in \cite{AGT08},
the so called metastable version nevertheless has a primitive recursive bound.
In our case this means that given the assumptions from Wittmann's strong ergodic theorem,
the following holds
\begin{align*}
\forall b, l\in\NN, g:\NN\to\NN, x\in S\ &\exists m\leq M(l,g,b,K)
\big( \|x\|\leq b \rightarrow \|A_mx-A_{m+g(m)}x\|\leq 2^{-l} \big),
\end{align*}
for a primitive recursive $M$, where $K$ is a rate of convergence for the sequence $\alpha$ in
the assumption~\eqref{e:W-assym}. 
%In this paper, we not only show the existence of such an $M$,
%but also we give such a bound explicitly in Corollary~\ref{c:fin22}.
We will not only prove the existence of such an $M$ but also give such a bound explicitly in Corollary~\ref{c:fin22}.
For the specific case where $(\alpha_n)_n$ is a constant $1$ sequence (i.e. $T$ satisfies~\eqref{e:W-intro} rather than~\eqref{e:W-assym},
see also Theorem 2.2 in~\cite{Wittmann90} and Corollary~\ref{c:fin21} below),
$M$ can be defined as follows:
\begin{align*}
M(l,g,b)&:=(N( 2l+7, g^M,b) + P( 2l+7, g^M, b)){b}2^{2l+8}+1,\\
P(l,g,b)&:=P_0(l,F(l,g,N(l,g,b),b),b),\\
F(l,g,n,b)(p) &:= p+n+\tilde g((n + p){b}2^{l+1}),\\
N(l,g,b)&:= \big(H(l,g,b) \big)^{{b}^22^{l+2}}(0), \\
H(l,g,b)(n) &:= n+P_0(l,F(l,g,n,b),b)+\tilde g((n + P_0(l,F(l,g,n,b),b)){b}2^{l+1}),
\end{align*}
where (for $f:\NN\to\NN$)
\begin{align*}
 P_0(l,f,b)&:=\tilde f^{ {b}^2 2^l}(0),\quad \tilde g(n):=n+g(n),\ g^M(n):=\max_{i\leq n+1} g(i).\footnote{The slightly unintuitive definition of $g^M$ makes the definition of $M(l,g,b)$ shorter.}\\
\end{align*}
Note that apart from the counterfunction $g$ and the precision $l$,
this bound depends only on $b$ and not on $S$, $T$ or $x$.
For another quantitative result on operators satisfying the condition~\eqref{e:W-intro} see~\cite{Kohlenbach2011odd}.\\
%
These results, along with those 
by Avigad, Gerhardy, Towsner~\cite{AGT08} and Kohlenbach,  
Leu{\c{s}}tean~\cite{Kohlenbach2009} for the finitary version of the von Neumann ergodic theorem 
as well as Kohlenbach's bounds for the finitary versions of Baillon's weak ergodic theorem~\cite{Kohlenbach_ComContemporaryMath} and Wittmann's 
convergence result for Halpern means~\cite{Kohlenbach2011} 
%(for quantitative versions of convergence theorems in other spaces, such as uniformly smooth Banach spaces or CAT(0) spaces, see e.g. \cite{KL11bl} or \cite{KL11} respectively)
, 
can be seen as instances of `hard analysis' in the sense of T. Tao; see ~\cite{Tao07, tao-2007}, where he
discusses the uses and benefits of (the existence of) uniform bounds for such finitary formulations of well-known
theorems.
%The uniformity of these bounds is thereby most important.\\
%In some cases we even improve their uniformity in comparison to the results guaranteed by the metatheorem. 
It is one of the goals of this paper to demonstrate that there are proof-theoretic means to systematically 
obtain such uniform bounds.
In fact, for many theorems the existence of a uniform bound is guaranteed by Kohlenbach's metatheorems
introduced in~\cite{Kohlenbach05meta} and refined in~\cite{GK08}. Additionally, 
%due to the proof-theoretic nature of these metatheorems, 
proof theoretic methods such as Kohlenbach's monotone functional interpretation (see~\cite{Kohlenbach96mfi}) can be used to
systematically obtain these effective bounds.
The paper at hand is a case study in applying such proof mining techniques. \\
%As a result,
We improve results in the area of nonlinear
generalizations of the mean ergodic theorem and their corresponding finitisations (see Figure~\ref{f:METtree}). Moreover, 
we have here a rare example of an application of these techniques to not necessarily continuous operators. In logical terms this amounts
to the subtlety that only a weak version of extensionality is available. Also, for the first time,
we obtain a bound which in fact makes use of nested iteration. One can
see this quickly on the term $M$ above. While $F$ as a function is defined via iteration of the counterfunction
$g$, it itself is being iterated by $P$. This is a direct consequence
of the logical form of Wittmann's original proof.\\
It is a surprising observation that so far for all metastable versions of strong ergodic theorems
primitive recursive bounds could be obtained.\\ 
We discuss the application of general logical metatheorems in more detail in Section~\ref{s:Meta} which is not
necessary to understand and verify our main results. We present these, namely the explicit bounds
for all three theorems in Wittmann's paper~\cite{Wittmann90}, in Section~\ref{s:Main}.
We explain how to obtain the explicit bounds in Section~\ref{s:terms},
after formalizing Wittmann's proof in Section~\ref{s:ArProof}. Both of these sections, though inspired by proof theoretic methods, 
require no facts from logic.
					
