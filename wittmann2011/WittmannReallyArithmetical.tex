\subsection{Arithmetizing Wittmann's proof}\label{s:ArProof}
%\section{Wittmann's proof without Skolem functions}\label{s:ArProof}
The first step of a proof mining process  is to investigate the proof of the 
theorem we want to analyze. 
For a discussion on proof mining techniques in connection with ergodic theory 
see~\cite{Gerhardy2010} and the last section of~\cite{AGT10} and for
an in-depth analysis of applied proof theory see~\cite{Kohlenbach08}.
The nonconstructive, or ineffective, content of Wittmann's proof are the principle
of convergence for bounded monotone sequences of real numbers and the existence of
infimum for bounded sequences of real numbers. 
Formulated in the usual way, both principles state the existence
of a real number, which we represent as fast converging 
Cauchy sequences of rationals\footnote{We represent rational numbers as pairs of natural numbers. For further details on representations in this context, see~\cite{Kohlenbach08}.} encoded as number theoretic functions (i.e. functions in $\NN^\NN$).
However, for a given sequence, both principles can be replaced by 
weaker statements about natural numbers only (as opposed to statements about objects in $\NN^\NN$).
In the presence of arithmetical comprehension, these weaker (arithmetical) statements are
equivalent to the original (analytical) principles.\footnote{
While the analytical principles are actually known to be equivalent to
arithmetic comprehension (see~\cite{Simpson1999} and -- for more
detailed results --~\cite{Kohlenbach00}), the arithmetic versions are equivalent
to $\SiL$-induction and hence have a functional interpretation
by ordinarily primitive recursive functionals (see~\cite{Kohlenbach08}).
}
For the convergence we work with the arithmetic Cauchy property
and for infimum we give for any precision an approximate infimum.
\begin{enumerate}
	\item Arithmetized convergence of a monotone bounded sequence $a_{(\cdot)}$:
	\[\forall l\exists n\forall m\geq n\quad \big( |a_n-a_m|\leq 2^{-l}\big).\]
	\item Arithmetized existence of the infimum of a bounded sequence $a_{(\cdot)}$:
	\[\forall l\exists n\forall m\quad \big( a_n-a_m\leq 2^{-l}\big).\]
\end{enumerate}
Of course, in this way we don't get a single point which {\em is} the limit point or infimum. Therefore
we have to analyze the proof and see whether such points are actually needed or whether
these arithmetical versions suffice. Here, fortunately, it turns out that 
the latter is the case (see~\cite{Kohlenbach1998} for a general discussion of this point).\\
Following~\cite{Kohlenbach08}, we show that we can rewrite Wittmann's proof (see~\cite{Wittmann90}) 
in the language of $\AHilb$, carefully using only weak (arithmetized) principles at the relevant places.
\begin{thm}[Wittmann 1990,~\cite{Wittmann90}]\label{t:fin21ar}
Let $X_{(\cdot)}$ be a sequence in a Hilbert space s.t. for all $m,n,k\in\NN$
\[
||X_{n+k} + X_{m+k}||^2 \leq ||X_{n} + X_{m}||^2+\delta_k\tag{i}
\]
with
\[
\lim_{k\to\infty} \delta_{k}=0.\tag{ii}
\]
Then the sequence $A_{(\cdot)}$ defined by
\[
A_n:=\frac{1}{n}\sum^n_{i=1}X_i,
\]
is norm convergent.
\end{thm}
We follow Wittmann's notation and use $X_{(\cdot)}$ to denote the sequence in the
Hilbert space (not to be confused with the Hilbert space itself, which might be implied by the notation $\AHilb$). 
Also, to keep the proof more readable, we refer the more technical steps to later sections. There we need to do 
a thorough analysis of those steps to obtain the precise bounds of the realizers, while here we 
can settle for their existence.\\
We formulate conditions (i) and (ii) from the theorem as arithmetical statements as follows (except for
the variable $K$, which we will treat as a given parameter):
\begin{align}
&\forall m,n,k\in\NN \quad \big( ||X_{n+k}+X_{m+k}||^2 \leq ||X_{n}+X_{m}||^2 + \delta_k\big),\label{l:one}\\
&\exists K\in\NN^\NN\forall n\in\NN \forall i\geq K(n)\quad \big( |\delta_i|\leq 2^{-n}\big).\label{l:two}
\end{align}
From now on, let $K$ always denote a rate of convergence of the sequence $\delta_{(\cdot)}$ 
(i.e. a function satisfying~\eqref{two}) and $B$ the upper bound of $X_{(\cdot)}$ (such a bound
can be defined primitive recursively in $K(0)$ and some elements of $X_{(\cdot)}$, see Proposition~\ref{p:fin}).\\
It is easy to show that the sequence $(||X_n||)_n$ is a Cauchy sequence 
(see proof of Lemma~\ref{l:N0} below). Such an arithmetical formulation
of the convergence of $(||X_n||)_n$ is sufficient
to infer that in particular we have that
\begin{align}\forall l^0\exists n^0 \forall m_2^0\geq m_1^0\geq n\quad 
\big(\ ||X_{m_1}||^2 \leq ||X_{m_2}||^2 + 2^{-(l+1)}\ \big).\label{n_epsilon}\end{align}
The Skolem function realizing $n^0$ would correspond to $n_\epsilon$ in Wittmann's proof
(where he used the standard convergence), however we work only with 
the fact that such an $n$ exists for any given precision $l$. %We denote such an $n$ by $n_l$.
From~\ref{n_epsilon} we infer that
\[\forall l\exists n_l \forall m,n\geq n_l\ \forall k\geq K(l)\quad
 \big(\ \langle X_{n+k}, X_{m+k} \rangle \leq \langle X_{n}, X_{m} \rangle + 2^{-l}\ \big),\tag{W1}\label{w:1}\]
the same way as Wittmann in~\cite{Wittmann90}, see also the end of the proof of Lemma~\ref{l:scpb}.
From now on, by $n_l$ we always denote an $n_l$ which satisfies~\eqref{w:1}.\\
Since the norm of any convex combination of the elements of the sequence is bounded from below by $0$, the
set of all such convex combinations has an infimum. To give an arithmetic
formulation of this fact, it is useful to have a primitive recursive functional 
$C$ which gives us a $2^{-l}$ approximation of the smallest 
convex combination of $X_n, X_{n+1}, \ldots, X_p$
(more precisely, of the square of the infimum of the set of the norms of these convex combinations). 
Clearly, there are many ways to define such a functional. We do so in a straightforward 
way in Definition~\ref{d:C} below. Having such a $C$ in place,
the following arithmetical formula
\[
\forall l\exists p_l\forall p\geq p_l \quad  
\big(\ C(l,n_l,p_l) \leq C(l,n_l,p) + 2^{-l}\ \big),
 \tag{W2}\label{w:2} \]
states the existence of an approximate infimum of all convex combinations of $X_{(\cdot)}$
(see also Lemma~\ref{l:newC}). Similarly as with $n_l$, from now on by $p_l$ we mean
a number satisfying~\eqref{w:2}.\\
Wittmann introduces a specific point $z_{\epsilon,m}$, which we denote by $Z(l,n,p,m)$
and define using our notation in Definition~\ref{d:Z} below. However, here
the precise definition is not as important as the properties of this point.
%, that for
%an $A_m$ with sufficiently large index $m$, the functional
%$Z$ defines a point which is close to $A_m$. Moreover, all such
%points (i.e. points given by $Z$ for some $m$) are close to each other.\\
Firstly, we show that for any two natural numbers $i$ and $j$, the distance between $Z(l,n_l,p_l,i)$ and
$Z(l,n_l,p_l,j)$ is arbitrary small for sufficiently large $l$. Together with the convexity of the 
square function we have (again, see~\cite{Wittmann90} or the proof of Lemma~\ref{l:Zs}.\eqref{e:Zdown} below) that
\[ \forall l^0,m^0\quad 
\big(\ ||Z(l,n_l,p_l,m)||^2\leq C(l, n_l, p_l) + 2^{-l}\ \big).
\tag{W3}\label{w:3}\]
From~\eqref{w:2} we can infer (see the proof of Lemma~\ref{l:Zs}.\eqref{e:Zup} below) that
\[
\forall l,i,j\ \Big(\big\|\frac{1}{2}(Z(l,n_l,p_l,i)+Z(l,n_l,p_l,j))\big\|^2 + 2^{-l} \geq C(l, n_l, p_l)\Big),
\]
and together with~\eqref{w:3} and the parallelogram identity we can conclude that
\[
\forall l,i,j\ \Big(\big\|Z(l,n_l,p_l,i)-Z(l,n_l,p_l,j)\big\|^2 \leq  2^{-l+4}\Big).
\]\\
Secondly, it follows from the definition of $Z$ (yet again see~\cite{Wittmann90} or  
proof of Lemma~\ref{l:ZA}) that 
%\begin{align*} \forall &l,m\geq p_l\quad\\ &\Big( \|Z(l,n_l,p_l,m)-A_{m+1}\|\leq \frac{
% 2p_l\sup_{n\in\NN}\|X_n\|+2(n_l+K(l))\sup_{n\in\NN}\|X_n\|}{m+1} \Big),\end{align*}
 \begin{align*} \forall l\forall m\geq p_l\ \Big( \|Z(l,n_l,p_l,m)-A_{m+1}\|\leq \frac{
 2(p_l+n_l+K(l))\sup_{n\in\NN}\|X_n\|}{m+1} \Big),\end{align*}
which means that we can make the distance between $A_{m+1}$ and $Z(l,n_l,p_l,m)$ arbitrarily small by
choosing $m$ sufficiently large.\\
In particular we have shown that the distance between any $A_i$ and any $A_j$ is arbitrarily small 
once $i$ and $j$ are sufficiently large. Note that we can choose an arbitrarily large $l$ first and then we 
are still free to choose sufficiently large $i$ and $j$ after $n_l$ and $p_l$ are fixed.
This concludes the sketch of the proof that $A_{(\cdot)}$ is a Cauchy sequence.\\
