\subsection{A general bound existence theorem}\label{s:Meta}

The main result of this paper, Corollary~\ref{c:fin22}, is a quantitative version
of a nonlinear strong ergodic theorem for 
operators satisfying Wittmann's condition~\eqref{e:W-assym} 
on an arbitrary subset of a Hilbert space. In this section
we outline how for this type of theorems the existence of such uniform bounds 
can be obtained by means of a general logical metatheorem. 
%(on the corresponding spaces)
The metatheorem applicable
in our scenario follows from Corollary 6.6.7) in~\cite{GK08} and we introduced
it in the first chapter (Theorem~\ref{t:GKmeta1}). However, let us recall it here, for reader's convenience.\\

\begin{thm}*[Gerhardy-Kohlenbach~\cite{GK08} - specific case 1]%\label{t:GKmeta1}
Let $\varphi_\forall$, resp. $\psi_\exists$, be $\forall$-
resp. $\exists$-formulas that contain only $x,z,f$ free, resp. $x,z,f$, $v$ free. Assume that
$\mathcal{A}^\omega[X,\langle\cdot,\cdot\rangle,S]$ proves the following sentence:
\[
\forall  x\in\NN^\NN, z\in S, f\in{S^S} 
	\big( \varphi_\forall(x, z, f)\rightarrow\exists v\in\NN\ \psi_\exists(x, z, f, v)\big).
\]
Then there is a computable functional $F : \NN^\NN\times\NN\times\NN^\NN\to\NN$ s. t. the following holds
in all non-trivial (real) inner product spaces $(X,\langle\cdot,\cdot\rangle)$ 
and for any subset $S\subseteq X$
\begin{align*}
\forall  &x\in\NN^\NN, z\in S, b\in\NN, f\in{S^S},f^*\in\NN^\NN\\
	&\big( \TMaj(f^*,f)\ \wedge\ \|z\|\leq b\ \wedge\ \varphi_\forall(x, z, f) \rightarrow 
	\exists v\leq F(x,b,f^*)\ \psi_\exists(x, z, f, v) \big),
\end{align*}
where %$0_X$ does not occur in $\varphi_\forall$ and $\psi_\exists$ and 
\[
\TMaj(f^*,f):\equiv \forall n\in\NN\forall z\in S \big( \|z\|\leq_\RR n \rightarrow \|f(z)\|\leq_\RR f^*(n)\big).
\]
The theorem holds analogously for finite tuples. % $\tup x\in \prod^n_{i=0}\NN^\NN$.
\end{thm}
Consider the following metastable version of Wittmann's Theorem 2.1~\cite{Wittmann90}:
\begin{thm}[Theorem 2.1 in~\cite{Wittmann90}] \label{t:W21}
Let $S$ be a subset of a Hilbert space and $T:S\to S$
be a mapping satisfying 
\[
\forall x,y\in S\ (\| Tx + Ty \| \leq \|x + y\|).\tag{$\TAN$}\label{e:W}
\]
Then for any $x\in S$ the sequence of the Ces{\`a}ro means,
\[
A_nx:=\frac{1}{n+1}\sum^{n}_{i=0} T^i x,
\]
is norm convergent.
\end{thm}
This theorem has the following form:
\begin{align*}
\forall l\in\NN,g\in\NN^\NN, &\ x\in S, T\in S^S \tag{+}\label{e:w21meta}  
\ \big( \TAN(T)\rightarrow 
	\exists m\in\NN \ (\|A_mx-A_{m+g(m)}x\|< 2^{-l})\big). 
\end{align*}
Obviously the conclusion, i.e. 
$
\exists m\ \big( \|A_mx-A_{m+g(m)}x\| < 2^{-l}\big),
$
has the form $\exists m\ \psi_\exists(m,l,g)$ 
%(or $\exists m\psi_\exists(m,\langle l,g\rangle)$
%for any suitable encoding of $l$ and $g$ as an element $\langle l,g\rangle$ 
%of $\NN^\NN$)\footnote{Encoding and decoding of finite tuples of numbers, 
%functions and even functionals are definable in $\AHilb$ as primitive recursive operations.}
and the assumption $\TAN(T)$, i.e.
$
\forall x,y\in S \big(\| Tx + Ty \| \leq \|x + y\|\big),
$
has the form $\varphi_\forall(T)$.\\
Moreover, $\TAN(T)$ already implies $\TMaj(\Id,T)$ (here $\Id$ stands 
simply for the identity function on $\NN$), since $\TAN(T)$ applied to $x=y=z$
implies
$
\forall z\in S \big( \|T(z)\| \leq \|z\| \big).
$\\
Hence we can apply Theorem~\ref{t:GKmeta1} to~\eqref{e:w21meta} by setting
\[
\tup x:=_{\NN\times\NN^\NN} l,g,
\ z:=_{S}x,
\ f:=_{S\to S}T,
\ f^*:=_{\NN\to \NN}\Id,
\] and
\[
\varphi_\forall(x, z, f):\equiv\TAN(T),\ 
\exists v\in\NN\ \psi_\exists(x, z, f, v):\equiv\exists m\in\NN\ \big( \|A_mx-A_{m+g(m)}x\| < 2^{-l}\big),
\]
to obtain that there is a computable bound $M:\NN\times\NN^\NN\times\NN\to\NN$, s.t.
\begin{align*}
\forall l\in\NN,g\in\NN^\NN&, x\in S, T\in S^S \\
&\big( \TAN(T) \wedge \|x\| \leq b\ \rightarrow \exists m\leq_\NN M(l,g,b)\ ( \|A_mx-A_{m+g(m)}x\|\leq 2^{-l}) \big). 
\end{align*}
It is rather easy to see that the proof can be formalized in $\AHilbS$, except for the question of the use
of the axiom of extensionality (full extensionality
is in general unavailable in any proof-theoretic extraction of computational bounds). 
% unless one works with extremely weak systems)
Generally, one can avoid the use of full extensionality in proofs of statements
about continuous objects. Note that in particular any nonexpansive operator
is also continuous. However, in our case, the operator $T$ may be discontinuous. 
Fortunately, Wittmann proves his main results as a consequence of
a statement about a simple sequence of elements in $S$, 
which as such is independent of $T$ (see Theorem 2.3 in~\cite{Wittmann90} or Theorem~\ref{t:fin23l} below),
whereby all relevant equalities are provable directly. Therefore the rule of extensionality
suffices to formalize his proof.\\
Hence the existence of a {\em uniform computable bound} for the metastable version
can be inferred from the metatheorem in~\cite{GK08}. Furthermore, since the metatheorem is established
by proof-theoretic reasoning, it provides not only the existence of a uniform bound but also
a procedure for its extraction.\\
Now, in general such a bound might need so called
bar-recursion ($\BR$), which is required to interpret the schema of full comprehension over numbers
in Spector's system (see~\cite{Spector62}). However, once more due to the
way how Wittmann proved the analyzed theorem, it is easy to see 
that the only proof-theoretically non-trivial principles needed in the proof are 
the existence of the infimum/supremum of bounded sequences and 
the principle of convergence for bounded monotone sequences.
Both of these principles need only bar-recursion restricted to numbers and functions ($\BR_{0,1}$)
and not full $\BR$. (Kohlenbach shows in~\cite{Kohlenbach08, Kohlenbach00} that both principles
are provable from arithmetical comprehension which is interpreted in $\T_0+\BR_{0,1}$.) 
Moreover, since the bound itself has only functions and numbers as arguments, 
it follows from~\cite{Schwichtenberg79, Kohlenbach99} that
the bound is not only computable, but that the {\em bound
is a primitive recursive functional in the sense of G\"odel's $\T$}.\\
These observations can be made a priori, without any in-depth analysis of the proof. In addition,
one more conclusion can be drawn before one actually extracts the bound. In general, it is helpful, and sometimes
even necessary, to simplify (in the sense of proof-theoretic strength) the analyzed proof. We do so
in Section~\ref{s:ArProof}. As one can see, in fact the proof uses only arithmetical versions of
the non-trivial principles (which can be proved by $\SiLm\IA$) and therefore we 
know that the use of $\BR_{0,1}$ can be eliminated as well. 
In fact, for these arithmetic versions the bounds for the witnesses for the metastable
formulations are already known and rather simple.
\begin{prop}[Kohlenbach~\cite{Kohlenbach08}]\label{p:Ulrich}
Let $(a_n)$ be a nonincreasing sequence in $[0,C]$ for some constant $C\in\NN$, then
\[ \forall k\in\NN,g\in\NN^\NN\exists n\leq F(g,k,C)\forall i,j\in[n;n+g(n)]\ \big(|a_i-a_j|<2^{-k}\big), \]
where $F(g,k,C):={\tilde g}^{C\cdot 2^k}(0)$ with $\tilde g(n):=n+g(n)$.
\end{prop}
\begin{proof}
See Propositions 2.27 and Remark 2.29 in~\cite{Kohlenbach08}.\\
\end{proof}\\
Hence, we can infer that there is actually {\em an ordinary primitive recursive bound} (a bound in $\T_0$)
which we give explicitly in Section~\ref{s:Main}, Corollary~\ref{c:fin21}.\\
We should point out that the original Corollary in~\cite{GK08} can be used
in a more general context than the particular example we just discussed. For instance, it can be 
applied to both, Theorem 2.2 and Theorem 2.3 in~\cite{Wittmann90}. Take for example the metastable 
formulation of Theorem 2.3 in~\cite{Wittmann90}:
\begin{thm}\label{t:fin23l}
Let $X_{(\cdot)}$ be a sequence in a Hilbert space s.t. 
\[
\forall m,n,k\in\NN\quad \big(\|X_{n+k}+X_{m+k}\|^2 \leq \|X_{n}+X_{m}\|^2 + \delta_k\big), % label{e:LogOne}
\] and $\exists K\in\NN^\NN\forall n\in\NN \forall i\geq Kn\quad (\delta_i\leq 2^{-n})$.
Then
\[
\forall l\in\NN, g:\NN\to\NN\ \exists m\quad \big( \|A_mx-A_{m+g(m)}x\|\leq 2^{-l} \big).
\]
\end{thm}
For simplicity, let us
here assume that the sequence $\delta_{(\cdot)}$ is in the real unit interval.
In this case we have the following additional parameters: $K:\NN\to\NN$, $\delta:\NN\to[0,1]$ and
a sequence $z_{(\cdot)}:=X_{(\cdot)}$ (rather than a starting point $z:=x$). 
We can apply Corollary 6.6.7) in~\cite{GK08} for the theory $\AHilbS$ again. 
Additionally we use point 6.6.3). On the other hand,
this time we don't need the function parameter $f$:
\begin{thm}[Gerhardy-Kohlenbach~\cite{GK08} - specific case 2]\label{t:GKmeta2}
Let $P$ be a $\mathcal{A}^\omega$-definable Polish space and let $\varphi_\forall$, resp. $\psi_\exists$, be a $\forall$-
resp. an $\exists$-formula that contains only $x,z,f$ free, resp. $x,z,f$, $v$ free. Assume that
$\mathcal{A}^\omega[X,\langle\cdot,\cdot\rangle,S]$ proves the following sentence:
\[
\forall  x\in\NN^\NN, y\in P,z_{(\cdot)}\in\NN^S 
	\big( \varphi_\forall(x, z)\rightarrow\exists v\in\NN\ \psi_\exists(x, z, v)\big),
\]
Then there is a computable functional $F : \NN^\NN\times \NN^\NN\to \NN$ s. t. the following holds
in all non-trivial (real) inner product spaces $(X,\langle\cdot,\cdot\rangle)$ and for any subset $S\subseteq X$
\begin{align*}
\forall  x\in\NN^\NN, &z_{(\cdot)}\in\NN^S, b_{(\cdot)}\in\NN^\NN \\
	&\big( \forall n\in\NN\ ( \|z_n\|\leq b_n)\wedge\varphi_\forall(x, z) \rightarrow
	\exists v\leq F(x,b_{(\cdot)})\psi_\exists(x, z, v)\big).
\end{align*}
The theorem holds analogously for finite tuples.
\end{thm}
Given any rate of convergence for the sequence $\delta_{(\cdot)}$, the metastable version of the assumptions 
in Theorem~\ref{t:fin23l} is purely universal:
\begin{align*}
\forall m,n,k,j\in\NN\ \forall i\geq K(j) 
 \ \big( \|X_{n+k}+X_{m+k}\|^2 \leq \|X_{n}+X_{m}\|^2 + \delta_k\ \wedge\ \delta_i\leq 2^{-n} \big). \tag{$\TAN'$}
\end{align*}
Moreover, the Polish space $[0,1]^\NN$ is naturally definable in $\mathcal{A}^\omega$ so we obtain
by Theorem~\ref{t:GKmeta2} that
\begin{align*}
\forall l\in\NN,g\in\NN^\NN,K\in\NN^\NN&,\ \delta_{(\cdot)}\in[0,1]^\NN, X_{(\cdot)}\in S^\NN, b_{(\cdot)}\in\NN^\NN\\ 
  \big(\ &\forall i\in\NN\ ( X_i\leq b_i)\ \wedge\ \TAN'(X,K,\delta)\ \rightarrow \\
	      &\exists m\leq_\NN M(l,g,b_{(\cdot)},K)\ ( \|A_mx-A_{m+g(m)}x\|\leq 2^{-l}) \ \big).
\end{align*}
Similarly as before, $\TAN'$ implies $\forall i\in\NN\ ( X_i\leq b_i)$ for a suitable $b_{(\cdot)}$.
We give such a bound explicitly in Section~\ref{s:Main}. To be precise,
%in Theorem~\ref{t:fin}
we give a bound $M(l,g, b_0, b_1, \ldots, b_{K(0)},K)$ for 
an arbitrary sequence $\delta_{(\cdot)}\in\RR^\NN$ converging to zero with the rate $K$. In the simplified case for 
the unit interval, it is straightforward to see that the bound also
simplifies to an even more uniform bound $M(l,g,b_0,K)$.\\
To repeat, these are very specific scenarios. We should emphasize that the 
Corollaries in~\cite{GK08}, and 
the metatheorem(s) even more so, have a much wider range of application.