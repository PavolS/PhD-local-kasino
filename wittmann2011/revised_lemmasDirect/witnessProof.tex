\section{Uniform bounds for Wittmann's ergodic theorems}\label{s:Main}
%\section{The witness is correct}\label{s:Main}

We give a bound for a finitary version of Wittmann's convergence result for
a general series in a Hilbert space satisfying a suitable formulation
of the condition~\eqref{e:W-assym} first 
(see Theorem~\ref{t:fin} and Corollary~\ref{c:finInt}) to derive the bounds
for the finitary versions of the actual ergodic theorems later
(see Corollary~\ref{c:fin22} and Corollary~\ref{c:fin21}).\\
We already discussed how to obtain the first bound in Sections~\ref{s:terms} and~\ref{s:ArProof},
here we concentrate on a formal and precise definition of the bounds as well as on the proof
that these bounds are correct.
\begin{thm}[Finitary version of Theorem 2.3 in~\cite{Wittmann90}] \label{t:fin}
Let $K:\NN\to\NN$ be a function and $X_{(\cdot)}$ a sequence in a Hilbert space s.t. for all $m,n,k\in\NN$
\[
\|X_{n+k} + X_{m+k}\|^2 \leq \|X_{n} + X_{m}\|^2+\delta_k,
\]
with
$
\forall l\in\NN\forall n\geq K(l)\ \big( |\delta_{n}|<2^{-l}\big),
$
in other words $K$ is a rate of convergence of $(\delta_k)$ towards 0.
Then the sequence $A_{(\cdot)}$, defined by
$
A_n:=\frac{1}{n}\sum^n_{i=1}X_i,
$
is a Cauchy sequence and we have that
\[
\forall l,g \exists m\leq M'(l,g^M)\ \big( \|A_{m}-A_{m+g(m)}\|\leq 2^{-l} \big),
\]
with $g^M(n):=\max_{i\leq n} g(n)$ and $M'(l,g,K)$ defined as follows (we ommit the dependencies whenever the arguments are trivial):
\begin{align*}
M'&:=M(2l+6, \lambda n\ .\ g(n+1))+1,\\
M&:=M_0( P(l,g), N(l,g), l),\\
P&:=P_0(l,F(l,g,N(l,g))),\\
F(p)&:=p+n+K^M(l)+M_0+g(M_0), F:\NN\to\NN\\
N&:=N_0(l+1,H(l,g)), \\
H(n)&:= H_0(l,g,n,P_0(l,F)), H:\NN\to\NN,
%M'(l,g)&:=M(2l+6, \lambda n\ .\ g(n+1))+1,\\
%M(l,g)&:=M_0( P(l,g), N(l,g), l),\\
%P(l,g)&:=P_0(l,F(l,g,N(l,g))),\\
%F(l,g,n) &:=_1 \lambda p\ .\ p+n+K^M(l)+M_0(l,n,p)+g(M_0(l,n,p)),\\
%N(l,g)&:=N_0(l+1,H(l,g)), \\
%H(l,g)&:=_1 \lambda n\ .\  H_0(l,g,n,P_0(l,F(l,g,n))),
\end{align*}
where (recall $K^M(n):=\max_{i\leq n} g(n)$)
\begin{align*}
 H_0&:= p + M_0+g(M_0)+K^M(l),\\
 M_0&:=(2n + 2p + 2K^M(l))B2^{l},\\
 P_0&:=\tilde f^{B^22^l}(0),\quad \tilde f(n):=n + f(n),\\
 N_0&:=R(l+1,U)+K^M(l+1), \\
 R&:=\tilde u^{\lceil \|X_0\| \rceil^22^l}(0),\\
 U(n)&:=(n + K^M(l+1))+h^M(n + K^M(l+1)), U:\NN\to\NN,\\
 B&:=\max_{1\leq i \leq K^M(0)} \lceil \|X_i\| \rceil+1.\\
% H_0(l,g,n,p)&:= p + M_0(l,n, p )+g(M_0(l,n, p ))+K^M(l),\\
% M_0(l,n,p)&:=(2n + 2p + 2K^M(l))B2^{l},\\
% P_0(l,f)&:=\tilde f^{B^22^l}(0),\quad \tilde f(n):=n + f(n),\\
% N_0(l,h)&:=R(l+1,U(l,h))+K^M(l+1), \\
% R(l,u)&:=\tilde u^{\lceil \|X_0\| \rceil^22^l}(0),\\
% U(l,h)&:=_1\lambda n\ .\ (n + K^M(l+1))+h^M(n + K^M(l+1)),\\
% B&:=\max_{1\leq i \leq K^M(0)} \lceil \|X_i\| \rceil+1.\\
\end{align*}
\end{thm}

%\begin{proof}
%Given $\tup s$ choose $\tup s'\in S_{p,l}$ s.t. $|s'_i-\tilde s_i|\leq \frac{2^{-(l+1)}}{pB^2}$. 
%We have
%\begin{align*}
%\left\|\sum_{i=0}^{p} \widetilde{s}_i X_{n+i} - \sum_{i=0}^{p} {s'}_i X_{n+i}\right\| = 
%\left\|\sum_{i=0}^{p} |\widetilde{s}_i-s'_i| X_{n+i} \right\| \leq
%\frac{2^{-(l+1)}}{pB^2}pB = \frac{2^{-(l+1)}}{B}.
%\end{align*}
%therefore
%\[
%\left|\ \left\|\sum_{i=0}^{p} \widetilde{s}_i X_{n+i}\right\| - \left\|\sum_{i=0}^{p} {s'}_i X_{n+i}\right\| \ \right|\leq\frac{2^{-(l+1)}}{B}
%\]
%and finally
%\[
%\left|\ \left\|\sum_{i=0}^{p} \widetilde{s}_i X_{n+i}\right\|^2 - \left\|\sum_{i=0}^{p} {s'}_i X_{n+i}\right\|^2\  \right| \leq \frac{2^{-(l+1)}}{B} ( B + \frac{2^{-(l+1)}}{B} ) \leq 2^{-l}.
%\]
%\end{proof}

We will also use that $M$ majorizes itself and that $N_0$ and $P_0$ are the right witnesses for the two main assumptions needed in Wittmann's proof. For better readability, we prove these lemmas in Section~\ref{s:Lemmas} at the end of the paper.

\begin{lemma}[$M$ is a majorant]\label{l:M}
Each of the terms $M$,$P$,$N$,$M_0$,$P_0$,$N_0$,$R$ majorizes itself.
In particular we have:
\[ \forall l'\geq l\forall h',h ( \forall n (h(n)\geq h'(n)) \rightarrow N_0(l',h^M)\geq N_0(l,h') ) \]
and
\begin{align*}
 \forall l,g \forall n\leq N(l,g^M) &\forall p\leq P(l,g^M)\\
  &( P(l,g^M)\geq P_0(l,F(l,g^M,n))\ \wedge\ M(l,g^M)\geq M_0(l,n,p) ) . 
\end{align*}
%and  
%\[ \forall l \forall g \forall n'\geq n (\ (H(l,g))(n') \geq (H(l,g))(n)\ ). \]
\end{lemma}

\begin{lemma}[$N_0$ is correct]\label{l:N0}
\[
\forall l,h \exists n \leq N_0(l,h) \forall i,j\in[n;n+h(n)]\ i\leq j\rightarrow \|X_i\|^2-\|X_j\|^2\leq 2^{-l}.
\]
\end{lemma}

\begin{lemma}[$P_0$ is correct]\label{l:P0}
$
\forall l,f,n\exists p\leq P_0(l,f)\ \big( C(l,n,p)\leq C(l,n,f(p)) + 2^{-l} \big).
$
\end{lemma}

Next three lemmas give a quantitative analysis of the original proof in~\cite{Wittmann90}. Again, for better readability, we 
give the proofs in Section~\ref{s:Lemmas}. 

\begin{lemma}[The scalar product increase is bounded]\label{l:scpb}
For any $l$ and any $g$, consider $h:= H(l,g^M)$. Let
$n$ be a witness for Lemma~\ref{l:N0}, i.e. 
\[
n\leq N(l,h)\ \wedge\ \forall i,j\in[n;n+h(n)]\ 
\big( i\leq j\rightarrow \|X_i\|^2-\|X_j\|^2\leq 2^{-l-1} \big) . \tag{N} 
\]
Moreover let $f:=F(l,g^M,n)$,
$p$ be a number smaller than $P_0(l,f)$ 
and  $m:=M_0(l,n,p)$. Then we have that
\[ 
\langle X_{a+k},X_{b+k} \rangle \leq \langle X_{a},X_{b} \rangle + 2^{-l}
\]
holds for all $k,a,b$ s.t. $K^M(l)\leq k \leq K^M(l)+m+g^M(m)$ and $ n\leq a,b \leq n+p$.
\end{lemma} 

Analogously to Wittmann~\cite{Wittmann90} we define
\begin{dfn}[$Z$]\label{d:Z}
$Z(l,n,p,m):=\frac{1}{m+1}\sum^{K^M(l)+m}_{k=K^M(l)}\sum^{p}_{i=0}  \widetilde{\tup s}_i X_{n+k+i},$
with $\tup s$ corresponding to the tuple in the definition of $C(l,n,p)$ (see
Definition~\ref{d:C} above). 
\end{dfn}

\begin{lemma}[$Z$s are close]\label{l:Zs}
For any $l$ and any $g$, consider $h:= H(l,g^M)$. Let
$n$ be a witness for Lemma~\ref{l:N0}, i.e. 
\[
n\leq N(l,g^M)\ \wedge\ \forall i,j\in[n;n+h(n)]\ 
\big( i\leq j\rightarrow \|X_i\|^2-\|X_j\|^2\leq 2^{-l-1} \big) . \tag{N} 
\]
Moreover, let $m:=M_0(l,n,p)$, $f:=F(l,g^M,n)$ and 
$p$ be a witness for Lemma~\ref{l:P0}, i.e. 
\begin{align*}
p\leq P_0(l,f)\ \wedge\ ( C(l,n,p)\leq C(l,n,f(p)) + 2^{-l}  ),  \tag{P}\\
\end{align*} 
Then we have that
$ \|Z(l,n,p,m) - Z( l,n,p,m+g^M(m) ) \|^2 \leq 2^{-l+4}.$
%holds for $m=M_0(l,n,p)$.
\end{lemma}

\begin{lemma}[$Z$s and $A$s are close]\label{l:ZA}
For any $l$ and any $g$, consider $h:= H(l,g^M)$. Let
$n$ be a witness for Lemma~\ref{l:N0}, i.e. 
\[
n\leq N(l,g^M)\ \wedge\ \forall i,j\in[n;n+h(n)]\ 
\big( i\leq j\rightarrow \|X_i\|^2-\|X_j\|^2\leq 2^{-l-1} \big) . \tag{N} 
\]
Moreover let $f:=F(l,g^M,n)$,
$p$ be a witness for Lemma~\ref{l:P0}, i.e. 
\begin{align*}
p\leq P_0(l,f)\ \wedge\ ( C(l,n,p)\leq C(l,n,f(p)) + 2^{-l}  ),  \tag{P}\\
\end{align*} 
and  $m:=M_0(l,n,p)$, $m':=m+g(m)$.
Then we have that
\[
 \|A_{m+1} - Z( l,n,p,m )\|\leq \frac{1}{m+1}(2n + 2p + 2K(l))B +2^{-l}
\]
and
\[
 \|A_{m'+1} - Z( l,n,p,m' )\|\leq \frac{1}{m'+1}(2n + 2p + 2K(l))B +2^{-l}.
\]
\end{lemma}


\begin{proof}[ of Theorem~\ref{t:fin}]
Fix arbitrary $l$ and $g$. Set $h:= H(l,g^M)$.
By Lemma~\ref{l:N0} we know there is an $n$ s.t.  
\[
n\leq N(l,g^M)\ \wedge\ \forall i,j\in[n;n+h(n)]\ 
\big( i\leq j\rightarrow \|X_i\|^2-\|X_j\|^2\leq 2^{-l-1} \big). %\tag{N}\label{e:N1}
\]
Let $f:=F(l,g^M,n)$. By Lemma~\ref{l:P0} we know that there is a $p$ s.t.
\begin{align*}
p\leq P_0(l,f)\ \wedge\ ( C(l,n,p)\leq C(l,n,f(p)) + 2^{-l}  ).  %\tag{P}\label{e:P1}\\
\end{align*}
Note that by Lemma~\ref{l:M} we have that $p\leq P(l,g^M)$. We set $m:=M_0(l,n,p)$. By Lemma~\ref{l:M} we get that $m\leq M(l,g^M)$.
Finally, it follows from lemmas~\ref{l:Zs} and~\ref{l:ZA} that
\begin{align*}
\|A_{m+1} - A_{m+g(m)+1}\| &\leq \|Z( l,n,p,m ) - Z( l,n,p,m+g(m) )\|\\
		&\quad\quad\quad + 2\Big(\frac{1}{m+1}(2n + 2p + 2K(l))B +2^{-l}\Big)\\
		&\leq \sqrt{2^{-l+4}} + 2^{-l+1} + \frac{2(2n + 2p + 2K(l))B}{m+1} \\
		&= \sqrt{2^{-l+4}} + 2^{-l+1} + \frac{2(2n + 2p + 2K(l))B}{(2n + 2p + 2K(l))B2^{l}+1} \\
		&< \sqrt{2^{-l+4}} + 2^{-l+1} + 2^{-l+1} \leq 2^{-\frac{l}{2}+3}.
\end{align*}
This proves 
\[
\forall l,g \exists m\leq M(l,g^M)\ \big( \|A_{m+1}-A_{m+g(m)+1}\|\leq 2^{-\frac{l}{2}+3}\big),
\]
from which the claim follows immediately by the definition of $M'$.
\end{proof}

Sometimes it is useful to work with the following version of the previous theorem, though
both these formulations are equivalent (even in weaker systems than we used to
formalize the original proof itself).

\begin{cor}[Finitary version of Theorem 2.3 in~\cite{Wittmann90} for intervals] \label{c:finInt}
Let $K$ be a function and $X_{(\cdot)}$ a sequence in a Hilbert space s.t. for all $m,n,k\in\NN$
\[
\|X_{n+k} + X_{m+k}\|^2 \leq \|X_{n} + X_{m}\|^2+\delta_k,
\]
with
\[
\forall l\in\NN\forall n\geq K(l)\ \big( |\delta_{n}|<2^{-l}\big).
\]
Then the sequence $A_{(\cdot)}$ defined by
\[
A_n:=\frac{1}{n}\sum^n_{i=1}X_i,
\]
is a Cauchy sequence and we have that
\[
\forall l,g \exists m\leq M'(l+1,g^M)\forall i,j\in[m;m+g(m)]\ \big( \|A_{i}-A_{j}\|\leq 2^{-l} \big),
\]
with $M'$ defined as in Theorem~\ref{t:fin}.
\end{cor}
\begin{proof}
Given any $l$ and $g$, apply Theorem~\ref{t:fin} to the number $l+1$ and to the function
\begin{align*}
h(n):=\min\Big\{\ &i\in [0;g(n)] \text{ s.t. }\\
 &\forall j\in[0;g(n)]\quad \Big(\big| \|A_{n+i}\| - \|A_n\| \big| \geq \big| \|A_{n+j}\|- \|A_n\|\big|\Big) \Big\}.
\end{align*}
It follows that 
\[
\exists m\leq M'(l+1,h^M)\ \big( \|A_{m}-A_{m+h(m)}\|\leq 2^{-l-1} \big).
\]
We fix such an $m$ and conclude (by the triangle inequality) that
\[
\forall i,j\in[m;m+g(m)]\ \big( \|A_{i}-A_{j}\|\leq 2^{-l} \big).
\]
Moreover, since $\forall n\in\NN\ (h^M(n)\leq g^M(n))$ we have that \[m\leq M'(l+1,g^M)\] due to Lemma~\ref{l:M}.
\end{proof}

Now, we obtain the bound for the metastable version of Theorem 2.2 in Wittmann's paper~\cite{Wittmann90} as a simple conclusion:

\begin{cor}[Finitary version of Theorem 2.2 in~\cite{Wittmann90}] \label{c:fin22}
Let $S$ be a subset of a Hilbert space and $T:S\to S$
be a mapping satisfying
\begin{align*}
&\forall n\in\NN\forall x,y\in S\ \big(\| T^nx + T^ny \| \leq \alpha_n\|x + y\|\big),\\ % \label{e:22.a1} \\
&\forall l\in\NN\forall n\geq K'(l)\ \big( |1-\alpha_{n}|<2^{-l}\big). % \label{e:22.a2}
\end{align*}
Then for any $x\in S$ the sequence of the Ces{\`a}ro means
\[
A_nx:=\frac{1}{n+1}\sum^{n}_{i=0} T^i x
\]
is norm convergent and the following holds:
\[
\forall l,g \exists m\leq M'(l+1,g^M)\forall i,j\in[m;m+g(m)]\ \big( \|A_{i}x-A_{j}x\|\leq 2^{-l}\big ),
%\forall l\in\NN,g\in\NN^\NN \exists m\leq_\NN M(l,g^M)\ \big( \|A_{m}x-A_{m+g(m)}x\|\leq 2^{-l}\big ).
\]
with $M'$ defined as in Theorem~\ref{t:fin}.
\end{cor}
\begin{proof}
Fix an arbitrary $x\in S$ and set $B':=\max_{i\leq K'(0)}(T^ix)+1.$
The claim follows from Corollary~\ref{c:finInt} with 
$X_i:=T^ix$, $\delta_n:=4B'^2(\alpha_n^2-1)$, $K(l):=K'(l+2\lceil\log_2 B' \rceil+2)$.
\end{proof}

The following corollary follows immediately:
%\newpage
\begin{cor}[Finitary version of Theorem 2.1 in~\cite{Wittmann90}] \label{c:fin21}
Let $S$ be a subset of a Hilbert space and $T:S\to S$
be a mapping satisfying
\[
\forall x,y\in S\ \big(\| Tx + Ty \| \leq \|x + y\|\big).
\]
Then for any $x\in S$ the sequence of the Ces{\`a}ro means
$
A_nx:=\frac{1}{n+1}\sum^{n}_{i=0} T^i x
$
is norm convergent and the following holds:
\[
\forall l,g \exists m\leq M(l,g^M)\forall i,j\in[m;m+g(m)]\ \big( \|A_{i}x-A_{j}x\|\leq 2^{-l}\big ),
%\forall l\in\NN,g\in\NN^\NN \exists m\leq_\NN M(l,g^M)\ \big( \|A_{m}x-A_{m+g(m)}x\|\leq 2^{-l}\big ),
\]
%and there is a witness
%\[
%%m \leq 1+M(2l+6,(\breve g)^M),\quad \breve g(n+1):=g(n), \breve g(0):=0
%m \leq 1+M(2l+6,(\breve g)^M),\quad \breve g(n+1):=g(n), \breve g(0):=0
%\]
with $M$ defined as follows:
\begin{align*}
%M(l,g)&:=(N( 2l+6, g) + P( 2l+6, g)){\lceil \|x\| \rceil}2^{2l+7}+1,\\
M(l,g)&:=(N( 2l+7, g) + P( 2l+7, g)){\lceil \|x\| \rceil}2^{2l+8}+1,\\
P(l,g)&:=P_0(l,F(l,g,N(l,g))),\\
F(l,g,n) &:=_1 \lambda p\ .\ p+n+\tilde g((n + p){\lceil \|x\| \rceil}2^{l+1}),\\
N(l,g)&:=\big({\lambda n\ .\ n+P_0(l,F(l,g,n))+\tilde g((n + P_0(l,F(l,g,n))){\lceil \|x\| \rceil}2^{l+1}) } \big)^{{\lceil \|x\| \rceil}^22^{l+2}}(0),
\end{align*}
where $P_0(l,f):=\tilde f^{ {\lceil \|x\| \rceil}^2 2^l}(0)$, $\tilde f(n):=n+f(n)$, $f^M(n):=\max_{i\leq n+1} f(i)$.\\

Note that (due to Lemma~\ref{l:M}) the bound for $m$ depends only on a bound for the norm of the
parameter $x$ and not directly on the starting point.
\end{cor}

