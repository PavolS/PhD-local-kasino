\section{Introduction}

\newtheorem{definition}{Definition}[section]
\newtheorem{proposition}[definition]{Proposition}
%\newtheorem{remark}[definition]{Remark}
\newtheorem{theorem}[definition]{Theorem}
\newtheorem{corollary}[definition]{Corollary}
%\newtheorem{lemma}[definition]{Lemma}
\newtheorem{exercise}[definition]{Exercise}
\newtheorem{clm}[definition]{Claim}
%\newtheorem{prop}[definition]{Proposition}
\newtheorem{example}[definition]{Example}
\newtheorem{notation}[definition]{Notation}
\newtheorem{application}[definition]{Application} 


In this paper we investigate different levels of effective quantitative 
information on theorems stating the Cauchy property of some sequence 
$(x_n)$ in a metric space $(X,d)$
\[ (1) \ \forall k\in\NN\,\exists n\in\NN\,\forall m,\tilde{m}\ge n
\ \big( d(x_m,
x_{\tilde{m}})\le 2^{-k}\big) \] 
and also more general $\Pi^0_3$-theorems  
\[ (2) \ \varphi\equiv 
\forall k\in\NN\,\exists n\in\NN\,
\forall m\in\NN\,\varphi_0(k,n,m),\] where $\varphi_0$ is quantifier-free. 
Since we refer to real numbers as fast 
converging Cauchy sequences of rational numbers we have $\le_{\RR}\,\in\Pi^0_1$ 
so that $(1)$ has the form $(2).$
\\[2mm] 
Cauchy statements $(1)$ are special forms of finiteness statements expressing 
that there are only finitely many $2^{-k}$-fluctuations $(i_l,j_l)$ 
with 
\[ (3) \ j_l>i_l\wedge d(x_{i_l},x_{j_l})>2^{-k}. \] 
As with general finiteness statements one can ask for a bound on the height 
of $2^{-k}$-fluctuations, i.e. an $\rho(k)$ above which no such fluctuation 
occurs (i.e. $\rho$ is a rate of convergence) 
or for a weaker bound $F(k)$ on the number $l$ 
of such fluctuations $(i_0,j_0),\ldots,(i_l,j_l)$ with $i_{n+1}\le j_{n}$ 
for $n<l.$ 
As to be expected from standard recursion theoretic facts about finiteness 
statements, even primitive recursive Cauchy sequences $(x_n)$ in $\RR$ 
in general will not admit a computable (in $k$) bound $F$ on the fluctuations 
and even in cases where they do there will in general be no computable 
rate of convergence $\rho.$\\[2mm] 
A yet weaker information than a bound $F$ on the number of fluctuations is 
a bound on the Kreisel no-counterexample 
interpretation (called `metastability' by Tao) of $(x_n)$, namely a functional 
$\varphi(k,g)$ such that 
\[ (4) \ \forall k\in\NN\,\forall g:\NN\to\NN\, \exists n\le \Phi(k,g) 
\,\forall i,j \in [n;n+g(n)]\ \big( d(x_i,x_j)\le 2^{-k}\big). \]
Note that the underlying reformulation 
\[ (5) \ \forall k\in \NN\,\forall g:\NN\to\NN\, \exists n
\,\forall i,j \in [n;n+g(n)]\ \big( d(x_i,x_j)\le 2^{-k}\big) \]
of the Cauchy property 
still expresses the full Cauchy property of $(x_n).$ However, the proof 
of the latter from the former is ineffective corresponding to the fact 
there is no way to 
pass (even pointwise let alone uniformly) 
from an effective $\Phi$ in $(4)$ to an 
effective bound on fluctuations $F$ or an effective rate of convergence $\rho.$
\\[2mm] General logical metatheorems for strong systems of analysis based 
on full classical logic guarantee that the extractability of a (sub-)recursive 
(and highly uniform) rates of metastability $\Phi$ is always possible for 
large classes of convergence proofs. This has been applied extensively in 
the context of nonlinear analysis, fixed point theory and ergodic theory 
during the last 10 years. One of these results is the extraction of a 
uniform rate of metastability for the strong convergence in the mean 
ergodic theorem for uniformly convex Banach spaces $X$ carried out in 
\cite{Kohlenbach/Leustean4}. That a computable rate of convergence (even 
for an effective Hilbert space and a computable operator) in general 
is impossible has been shown in \cite{AGT10}. However, 
as recently observed by Avigad and Rute \cite{Avigad/Rute}, the analysis 
in \cite{Kohlenbach/Leustean4} can be used to obtain a simple effective (and 
also highly uniform) 
bound on the number of fluctuations (for the case of Hilbert spaces this 
was already obtained with an even better bound in \cite{Jones}). This 
raises the question whether there are general logical conditions on 
convergence proofs to guarantee the extractability of effective bounds 
on fluctuations. Obviously, any condition guaranteeing the extractability of 
a computable rate of convergence is a sufficient condition for this. Though 
not satisfied in the particular case just discussed, let us first consider 
this in order to see in what sense we might try to liberalize such conditions 
towards rates of fluctuations. To discuss things in somewhat more precise 
terms we fix a formal framework such as intuitionistic arithmetic 
HA$^{\omega}$ in all finite types (actually we use the so-called weakly 
extensional variant called WE-HA$^{\omega}$ in \cite{Kohlenbach08}) 
or its extension by an abstract (metric or) 
normed space $(X,\|\cdot\|)$ resulting 
in HA$^{\omega}[X,\|\cdot\|]$ possible with further axioms stating that 
$X$ is uniformly convex or even a Hilbert space (see \cite{Kohlenbach08} 
for details). 
As follows from the bound extraction theorem for monotone modified 
realizability from \cite{Kohlenbach08} (and for theories with 
abstract spaces $X$ in \cite{GerKoh06}) from a proof of $(1)$ (for 
some sequence $(x_n)$ definable by a term $t$ of the system having at most 
number and function parameters $a,f$) in 
\[ \mbox{HA$^{\omega} +$AC$+$LEM}_{\neg} \] 
(and in fact even stronger theories augmented with certain ineffective 
axioms $\Omega$), the extractability of a rate of convergence $\rho$ that 
is definable (in the same parameters as $t$) 
G\"odel's calculus of primitive recursive functionals of 
finite type is guaranteed.  Here AC is the full schema of choice and 
LEM$_{\neg}$ is the law-of-excluded-middle schema restricted to arbitrary 
negated formulas $\neg\psi$ (which, in particular, includes the case 
of existential-free formulas and so as a very special case $\Pi^0_1$-LEM, 
i.e. LEM restricted to $\Pi^0_1$-formulas). In the case of 
HA$^{\omega}[X,\| \cdot\|]$ even parameters of types such as $X, \NN\to X, 
X\to X$ are allowed in the definition of the sequence $(x_n)$ in $X$ where 
then $\rho$ depends additionally of majorants for these parameters (which are 
numbers in $\NN,$ in the case of the type $X,$ and number-theoretic 
functions, in the case of the types $\NN\to X$, $X\to X.$) \\[2mm] 
An important weak principle of classical logic not covered by this is 
the so-called Markov principle which, extended to all finite types, reads 
as follows 
\[ \mbox{M}^{\omega}\ :\ 
\neg\neg\exists \underline{x}^{\underline{\rho}} \ \varphi_0(\underline{x})\to 
\exists \underline{x}^{\underline{\rho}}\,\varphi_0(\underline{x}), \] 
where $\varphi_0$ is a quantifier-free formula and $\underline{\rho}$ an 
arbitrary tuple of types. However, M$^{\omega}$ becomes permissible once 
LEM$_{\neg}$ is weakened to the so-called lesser-limited-omniscience-principle 
LLPO (which is the precise amount of classical logic needed to prove the 
binary (`weak') K\"onig's lemma WKL; see \cite{Kohlenbach08} for details).
So instead of HA$^{\omega}([X,\|\cdot\|])+$AC$+$LEM$_{\neg}$ we may also have 
\[ \mbox{HA$^{\omega}([X,\|\cdot \| ])+$AC$+$M$^{\omega}+$LLPO}, \] 
where then the extraction of a rate of convergence uses the so-called 
monotone functional interpretation (see \cite{Kohlenbach08}).
\\[2mm] The in a sense weakest principle covered by neither of these systems 
(but provable in their union!) is LEM restricted to $\Sigma^0_1$-formulas, 
which we denote by $\Sigma^0_1$-LEM. 
While $\Sigma^0_1$-LEM in the presence of 
AC (even when restricted to numbers) 
creates highly noncomputable functions (in particular when function 
parameters are allowed to occur in $\Sigma^0_1$-LEM which then makes it 
possible to climb up the entire arithmetical hierarchy) it remains fairly 
weak over HA$^{\omega}.$ Nevertheless HA$^{\omega}+\Sigma^0_1$-LEM 
already allows 
to prove the Cauchyness of the Specker sequence, a primitive recursive 
monotone decreasing sequence of rational numbers in $[0,1]$ which does not 
have a computable rate of convergence. In fact, as shown in \cite{Toftdal}, 
the principle that every bounded monotone sequence of reals is Cauchy can be 
proven HA$^{\omega}+\Sigma^0_1$-LEM. This is not obvious and requires a novel 
proof as the usual argument uses the (by \cite{Akama}) 
strictly stronger principle 
\[ \Sigma^0_2\mbox{-DNE}\ :\ \neg\neg\exists n\in\NN\,\forall m\in\NN\,
\varphi_0(n,m)\to\exists n\in\NN\,\forall m\in\NN\,\varphi_0(n,m) \]
(`double-negation-elimination principle' for 
$\Sigma^0_2$-formulas) While $\Sigma^0_2$-DNE is limit computable in the 
sense of Hayashi and 
Nakana \cite{Hayashi/Nakata}, 
any single instance of $\Sigma^0_1$-LEM is even learnable 
with a single mind change. Note also that bounded monotone sequences of 
real numbers (say in $[0,1]$) always have the simple fluctuation bound 
$F(k):=2^k.$  
This at a first look might suggest to consider 
HA$^{\omega}+\Sigma^0_1$-LEM as a promising framework to guarantee 
(while in general not computable rates of convergence) computable bounds 
on the number of fluctuations for provable Cauchy sequence. However, this 
turns out to be mistaken as $\Sigma^0_1$-LEM is already the general case: 
let $(x_n)$ a sequence of real numbers definable by a term $t$ 
in HA$^{\omega}$ 
(which may have variables of arbitrary type as parameters. Suppose that 
\[ \mbox{PA}^{\omega}\vdash \forall k\,\exists n\,\forall m,\tilde{m}\ge n\,
(|x_m-x_{\tilde{m}}|\le_{\RR} 2^{-k}). \]
Then by negative translation (see \cite{Kohlenbach08}) 
\[ \mbox{HA}^{\omega}\vdash  \forall k\,\neg\neg 
\exists n\,\forall m,\tilde{m}\ge n\,
(|x_m-x_{\tilde{m}}|\le_{\RR} 2^{-k}). \]
Adapting Friedman's proof for the closure of HA$^{\omega}$ under the Markov 
rule one can show (this is stated for HA without proof in 
\cite{Hayashi/Nakata} and we include a proof below) that HA$^{\omega}+
\Sigma^0_1$-LEM  
is closed under the rule version of $\Sigma^0_2$-DNE so that we get 
\[ \mbox{HA$^{\omega}+\Sigma^0_1$-LEM} \ 
\vdash \forall k\,\exists n\,\forall m,\tilde{m}\ge n\,
(|x_m-x_{\tilde{m}}|\le_{\RR} 2^{-k}). \]
Moreover, if $t$ contains at most number parameters it also suffices to 
use the restriction $\Sigma^0_1$-LEM$^-$ of $\Sigma^0_1$-LEM to 
$\Sigma^0_1$-formulas with number parameters only. All this also holds for 
the systems HA$^{\omega}[X,\|\cdot\|]$ and PA$^{\omega}[X,\|\cdot\|]$ 
and sequences $(x_n)$ in $X$ defined by terms of these systems.\\[2mm] 
Looking more careful into the $\Sigma^0_1$-LEM based proof of the 
Cauchyness of bounded monotone sequences as given in \cite{Toftdal} reveals 
that one can define a sequence of instances $\Sigma^0_1$-LEM$(s(n))$) 
of $\Sigma^0_1$LEM$^-$ 
\[ \exists m\in\NN \,(s(n,m)=0)\vee \forall m\in\NN\,(s(n,m)\not= 0) \] 
such that to prove the Cauchy property with error $2^{-k}$ one only needs 
the first $n=0,\ldots,s(t(k))$-many instances of this sequence where $t$ is 
a simple primitive recursive function. So a more promising approach would be
to look at proofs of Cauchy statements which can be formalized as follows:
\[ \mbox{HA}^{\omega}\ \vdash \forall k \  \big( \forall l\le t(k) \ 
\Sigma^0_1\mbox{-LEM}(s(l))\to \exists n\,\forall m,\tilde{m}\ge n\,
(|x_m-x_{\tilde{m}}|\le_{\RR} 2^{-k})\big) \] 
or 
\[ \mbox{HA}^{\omega}[X,\|\cdot\|] \ \vdash \forall k \  \big( \forall l\le 
t(k) \ 
\Sigma^0_1\mbox{-LEM}(s(l))\to \exists n\,\forall m,\tilde{m}\ge n\,
(\|x_m-x_{\tilde{m}}\|\le_{\RR} 2^{-k})\big), \]
where $t$ may contain parameters of type $\NN, \NN\to\NN$ (and $X,\NN\to X, 
X\to X$ in the extended system). \\[2mm] 
We show that from such proofs one can always extract effective (and in 
fact primitive recursive in the sense of G\"odel's $T$) bounds on the 
effective learnability of a rate of convergence of $(x_n).$ 
This is, as we will show, a strictly stronger information than a 
rate of metastability as 
the latter can be obtained from the former (even by a uniform primitive 
recursive procedure) but there are primitive recursive 
Cauchy sequences with a primitive recursive rate of metastability which 
do not admit any computable bound for the 
learnability of a rate of convergence. \\[2mm]
The condition needed to assure the extractability of a primitive recursive 
(in the sense of G\"odel) bound on effective learnability is e.g. satisfied in 
the proof of the strong convergence of so-called Halpern iterations 
in CAT(0) (and so, in particular, in Hilbert) spaces as follows from the 
analysis given in \cite{Kohlenbach/Leustean6} where a primitive recursive 
(in the ordinary sense) rate of metastability is extracted. The strong 
convergence of Halpern iterations constitutes 
a far reaching nonlinear extension 
of the linear case of the mean ergodic theorem.
However, 
studying the argument from \cite{Avigad/Rute} for the existence of 
fluctuation bounds in the linear case reveals that this depends on 
highly specific numerical facts of the situation at hand which - least 
in the form used - does not seems to hold in the nonlinear case of 
Halpern iterations. This issue will be discussed in the final section 
of the paper. So the conclusion is that there is a robust and easy to check 
logical condition on a convergence proofs that guarantees the extractability 
of effective bounds on the learnability of the rate of convergence (which 
is better than metastability). Whether this, moreover, gives rise to a 
fluctuation bound, however, seems to rest on rather idiosyncratic features 
of the particular sequence in question.
