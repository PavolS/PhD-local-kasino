\section{
  Preliminaries
} \label{s:si}
\reviewQuote{bwint2010/review_bw_prelim}{KS10}{Nov 23, 2009}

\newcommand{\typeOfXZ}{0(10)(1)}

\mySubsection{Proof Interpretation} Generally, using a suitable proof interpretation, 
not necessarily the D-interpretation, 
 in the context of proof mining we can expect to extract 
programs and/or computable bounds 
from a given proof. However, the two interpretations we will use throughout this thesis, 
namely negative translation 
 and D-interpretation, were originally
presented with a different aim. The proof-theoretic gain of negative 
translation as a tool for a translation
from classical to intuitionistic systems is clear. The functional 
interpretation was first introduced
by G\"odel to obtain consistency results relative to $\T$ motivated by Hilbert's program and not
as a tool for proof mining. We give a 
very brief description of these interpretations right now,  see
e.g. \cite{Kohlenbach08} or \cite{AF98} and definitions \ref{d:NT} and 
\ref{d:FI} for further details.\\
%
As mentioned earlier, we will make use of Kuroda's negative translation, 
which assigns to every formula $\phi$ a 
logically equivalent formula $\phi':\equiv\neg\neg (\phi^*)$, where $\phi^*$ is just $\phi$ 
with a double negation behind each $\forall$-quantifier.\\
%
G\"odel's functional interpretation assigns to every formula $\phi$ an intuitionistically 
equivalent formula $\phi^D:\equiv\exists\tup x\forall\tup y \phi_D(\tup x,\tup y,\tup a)$,
where $\tup x$ and $\tup y$ are tuples of variables of arbitrary finite type, $\tup a$ is 
a tuple of parameters of arbitrary finite type and $\phi_D(\tup a)$ is quantifier-free. 
Moreover, by the soundness of
the functional interpretation we know there is a tuple $\tup t$ 
of closed terms of our intuitionistic system, such that 
$\forall y \phi_D(\tup t\atup a,\tup y,\tup a)$ holds intuitionistically and 
even provably in a quantifier-free calculus, provided
that $\phi$ is provable in $\weha+\M+\AC$.\\
We should mention 
that under the combination of both interpretations 
(i.e. under the ND-interpretation) for every formula provable in $\wepa$
with the help of quantifier-free axiom of choice and any additional universal 
axioms (usually denoted as $\wepa + \QFm\AC + \mathcal{U}$) the interpreted 
formula can be proved in $\weha+\mathcal{U}$ using the same purely universal
axioms $\mathcal{U}$ as we allowed in the original classical proof. 
%
%Furthermore, 
%since we can interpret and thereby eliminate the Markov principle under 
%ND-interpretations, we add this schema to our 
%intuitionistic system to be able to simplify the formulas we want to
%interpret without any significant effect on the 
%final interpretation of the formula which, of course,
%remains provable without this principle. 
%
Kohlenbach introduced in \cite{Kohlenbach96mfi}
the \defkey{monotone functional interpretation} or shortly \defkey{MD-interpretation} 
(see \ref{ss:mfi}). 
This interpretation is based on the D-interpretation and Howard's majorizability 
property (see definition \ref{d:maj}). 
Instead of delivering a concrete realizer as before, by the MD-interpretation 
we can construct a tuple of closed terms $\tup t$ which \defkey{majorize} 
some realizers. Unless combined with negative translation, we still need an 
intuitionistic proof, but, fortunately,
it can use some other ineffective principles which have
a direct MD-interpretation (e.g. Weak K\"onig's lemma, for general 
definition of these principles see theorem \ref{t:mfi}).\\

%
%  SYSTEMS
%
%
The first two interpretations in this section as well as the particular 
axiomatization of intuitionistic logic and arithmetic
go originally back to G\"odel (see \cite{Goedel33}, \cite{Goedel58}), 
the monotone functional interpretation was first introduced
by Kohlenbach (see \cite{Kohlenbach96mfi}).
The calculus of $\weha$ and Heyting Arithmetic in general was further
analyzed by Troelstra in \cite{Troelstra73}.\\
The herein presented description of the formal systems and representations
is based mainly on \cite{Kohlenbach08} and we use also the shortened version
extracted by E. M. Briseid (see \cite{Briseid05}).
We define the system $\weha$ directly
with its language, axioms, and rules rather than going the usual way of
defining an intuitionistic calculus for first order logic
 followed by the addition of arithmetic and
types.
%
%\section{Preliminaries}
%\subsection{All finite Types}
%
%
%
\begin{dfn}\label{d:Tp}
The \defkey{set of all finite types}, \defkeyn{$\Tp$}, is generated
inductively by
\begin{align*}
0 &\in \Tp \text{,} \\
\rho,\tau \in  \Tp \rightarrow\quad \tau(\rho) &\in  \Tp
\text{,}\end{align*} 
where the type $0$ is the type corresponding to the natural numbers $\NN$.
Objects of type $\tau(\rho)$ are functions mapping 
objects of type $\rho$ to objects of type $\tau$. We will sometimes 
omit brackets when specifying types, i.e. write $\rho(\tau \sigma)$ 
meaning in fact $\rho(\tau(\sigma))$.\\
The \defkey{set of pure types}, \defkeyn{$\PT$}, is generated
inductively by
\begin{align*}
0 &\in \PT \text{,}\\
\rho \in \PT \rightarrow\quad 0(\rho) &\in  \PT 
\text{.}\end{align*} 
We often denote pure types by natural numbers, with $1$ for
$00$, $2$ for $0(00)$, and $n+1$ for $0(n)$.\\
The \defkey{degree} or \defkey{type level} of $\rho$ is defined by
\begin{align*}
\emph{deg}(0) &: = 0 \text{,} \\
\emph{deg}(\tau(\rho)) &: = \max (\emph{deg}(\tau), \emph{deg}(\rho)+1)
\text{.}
\end{align*} 
\end{dfn}
We will use numbers in bold font for the type level of terms. I.e. for
a functional $A$ of type 
$(0\rightarrow 0) \rightarrow ( (0\rightarrow 0)\rightarrow((0\rightarrow 0)\rightarrow 0) ) $
we write also $A\tp{2}$, expressing that $A$ has degree $2$, 
instead of $A^{(21)1}$, where the type of $A$ is declared precisely.\\                                                         
%
We define higher order equality as follows:
\begin{dfn}\label{l:hte}
Two terms of type $\rho$ are equal, iff:
                        \[
         s^{\rho} =_{\rho} t^{\rho} : \equiv \forall y_{1}^{\rho_{1}},
         \dots, y_{k}^{\rho_{k}}(sy_{1} \dots y_{k} =_{0} ty_{1} \dots y_{k})\text{,}
                        \]
where $\rho = 0(\rho_{k})\dots (\rho_{1})$.
\end{dfn}

\subsection{The System $\weha$}
%
%
%
We call the system $\weha$, which we describe below, weakly
extensional intuitionistic ('Heyting'-) arithmetic in all finite
types. 
\begin{dfn}{\em Weakly extensional arithmetic in all finite types \defkeyn{$\weha$}}
  \label{d:weha}
  \begin{itemize}
  %
  % Language
  %
  \item {\em The language ${\mathcal{L}}(\weha)$ of $\weha$}:  
    \begin{itemize}
    \item {\em The signature of ${\mathcal{L}}(\weha)$}:  
      \begin{itemize}
      \item We use the following logical symbols (for any type $\rho$):
        \begin{align*}
          \wedge\quad &\text{and}&            \vee  \quad &\text{or}\\
          \rightarrow\quad &\text{implies}&    \perp\quad &\text{absurdity}\\
          \exists\ x^\rho\quad &\text{there exists}&        \forall\ x^\rho\quad &\text{for all}
          \text{.}\end{align*}
      \item We write (for any type $\rho$) $x^\rho,y^\rho,z^\rho,... $
            for variables of type $\rho$ (bound or free).
      \item We include the constants: 
        \begin{align*}
          0^0\quad &\text{natural number $0$}&  S^{00}\quad &\text{successor}\\
          \Sigma_{\delta,\rho,\tau}\quad 
                &\text{combinator of type $\tau\delta(\rho\delta)(\tau\rho\delta)$}&
          \Pi^{\rho\tau\rho}_{\rho,\tau}\quad &\text{projector}\\    
          \tup R_{\tup \rho}\quad &\text{simultaneous recursors of type 
           $\tup\rho(\tup\rho 0\tup\rho)\tup\rho0$}\text{.}&&
          \end{align*}
      \item and the binary predicate constant $=_0$ for equality between type $0$ objects.
       \end{itemize}
     \item {\em \defkeyn{Terms} of ${\mathcal{L}}(\weha)$}:
       \setcounter{equation}{0}
       \begin{enumerate}
     \item For any type $\rho$, the terms of type $\rho$ are the variables $x^{\rho},y^\rho,...$ and 
       the constants $c_1^\rho,c_2^\rho,...$ %.
     \item If $t^{\tau\rho}$ is a term of type $\tau\rho$ and $s^\rho$
       is a term of type $\rho$, then $t^{\tau\rho}(s^\rho)$ is a 
       term of type $\tau$.
     \end{enumerate}
   \item {\em \defkeyn{(Prime) Formulas} of ${\mathcal{L}}(\weha)$}:
     \begin{enumerate}
     \item Absurdity, $\perp$, is a (prime) formula.
     \item If $s^0$ and $t^0$ are terms of type $0$ 
       then $s=_0t$ is a (prime) formula. Prime formulas are also called 
       \defkey{atomic formulas}.
     \item If $\phi$ and $\psi$ are formulas and $x^\rho$ is
       a variable of type $\rho$ then
         \begin{align*}
           &\phi \wedge \psi& &\phi\vee\psi\\
           &\phi \rightarrow \psi& &\exists x^\rho\ \phi(x)\\
           &\forall x^\rho\ \phi(x)
         \end{align*}
         are formulas.
       \end{enumerate}
     \item {\em Abbreviations}:
       \begin{align*}
         &\neg\phi:\equiv\ \phi\rightarrow\perp& 
         &\phi\leftrightarrow\psi:\equiv\ (\phi\rightarrow\psi)\wedge(\psi\rightarrow\phi)
         \text{.}\end{align*}
     \end{itemize} 
   \item {\em Axioms of $\weha$}
   %  
   %  Axioms
   %  
     \begin{enumerate}
       \item Axioms of contraction:
         \begin{align*}
           &\phi\vee\phi\rightarrow\phi& 
           &\phi\rightarrow\phi\wedge\phi
           \text{.}
         \end{align*}
       \item Axioms of weakening:
         \begin{align*}
           &\phi\rightarrow\phi\vee\psi& 
           &\phi\wedge\psi\rightarrow\phi
           \text{.}
         \end{align*}
       \item Axioms of permutation:
         \begin{align*}
           &\psi\vee\phi\rightarrow\phi\vee\psi& 
           &\phi\wedge\psi\rightarrow\psi\wedge\phi
           \text{.}
         \end{align*}
       \item Ex falso quodlibet axiom:
         \[
           \perp\rightarrow\phi\text{.}
         \]
       \item Quantifier axioms:
         \begin{align*}
           &\forall x\phi(x)\rightarrow\phi(t)& 
           &\phi(t)\rightarrow\exists x\phi(x)
           \text{,}
         \end{align*}
           where $t$ is free for $x$ in $\phi$.
       \item Axioms of equality for $=_0$ ($x^0$, $z^0$ and $y^0$ are variables of type $0$):
         \begin{align*}
           x&=_0x&   x=_0y&\rightarrow y=_0x\\
           x=_0y\wedge y=_0z&\rightarrow x=_0z\text{.}&&
         \end{align*}
         The reflexivity, symmetry and transitivity of higher type equalities 
         $=_\rho$ are derivable from these axioms (see lemma \ref{l:hte}).
       \item Successor axioms:
         \begin{align*}
           &S(x^0)\neq_0 0\\
           &S(x^0)=_0S(y^0)\rightarrow x=_0y\text{.}
         \end{align*}
         We write also $x+1$ for $S(x)$.
       \item Induction schema for an arbitrary formula $\phi(x^0)$ of $\weha$:
         \[ \IA \quad:\quad \phi(0)\wedge\forall x^0(\phi(x)\rightarrow\phi(x+1))\ 
             \rightarrow\ \forall x^0\phi(x) \text{.}\]
       \item Axioms for $\Pi_{\rho, \tau}$,$\Sigma_{\delta, \rho, \tau}$ 
         and $\tup R_{\tup \rho}$:
         \begin{align*}
         (\Pi)\quad&:\quad \Pi_{\rho,\tau}x^{\rho}y^{\tau} =_{\rho}x^{\rho}\\ 
         (\Sigma)\quad&:\quad \Sigma_{\delta, \rho, \tau}
           x^{\tau \rho\delta}y^{\rho\delta}z^{\delta} =_{\tau} xz(yz)\\
         (\tup R_{\tup \rho})\quad&:\quad\text{for $i=1,...,k$:}\\
           &\quad\quad\begin{cases}
             (R_{i})_{\tup\rho}0\tup y\atup z =_{\rho_i} \tup y  \\
             (R_{i})_{\tup\rho}(Sx^{0})\tup y\atup z=_{\rho_i} 
                  z_i(\tup R_{\tup\rho}x\tup y\atup z)x 
           \end{cases}
           \text{,}
         \end{align*}
         where $\tup \rho=\rho_1,...,\rho_k$,  $\tup\rho^t:=\rho_k\ldots\rho_1$,
         $\tup y=y_1,...,y_k$ and
         $\tup z=z_1,...,z_k$ with $y_i$ of type $\rho_i$ and $z_i$ of
         type $\rho_i0\tup\rho^t$.
     \end{enumerate}
   \item {\em Rules of $\weha$}
   %
   % Rules
   %
     \begin{enumerate}
       \item Modus ponens and syllogism:
         \begin{align*}
           &\begin{prooftree}
             \phi \quad \phi\rightarrow \psi
             \justifies
             \psi
           \end{prooftree}& 
           &\begin{prooftree}
             \phi\rightarrow\psi \quad \psi\rightarrow \chi
             \justifies
             \phi\rightarrow\chi
           \end{prooftree}\text{.}
         \end{align*}
       \item Exportation and importation:
         \begin{align*}
           &\begin{prooftree}
             \phi\wedge\psi\rightarrow\chi 
             \justifies
             \phi\rightarrow(\psi\rightarrow\chi)
           \end{prooftree}& 
           &\begin{prooftree}
             \phi\rightarrow(\psi\rightarrow\chi)
             \justifies
             \phi\wedge\psi\rightarrow\chi 
           \end{prooftree}\text{.}
         \end{align*}
       \item Expansion:
         \[\begin{prooftree}
             \phi\rightarrow\psi
             \justifies
             \chi\vee\phi\rightarrow\chi\vee\psi
           \end{prooftree}\text{.}\]
       \item Quantifier rules (assuming $x$ is not free in $\psi$):
         \begin{align*}
           &\begin{prooftree}
             \psi\rightarrow\phi(x) 
             \justifies
             \psi\rightarrow\forall x\ \phi(x) 
           \end{prooftree}& 
           &\begin{prooftree}
             \phi(x) \rightarrow \psi
             \justifies
             \exists x\ \phi(x)\rightarrow\psi
           \end{prooftree}\text{.}
         \end{align*}
       \item Quantifier-free rule of extensionality:
         \[ \QFm\ER\quad:\quad
           \begin{prooftree}
             \phi_{{}_{\QF}}\rightarrow s=_\rho t
             \justifies
             \phi_{{}_{\QF}}\rightarrow r[s]=_\tau r[t]
           \end{prooftree}\text{,}\]
           where $s^\rho$, $t^\rho$ and $r[x^\rho]^\tau$ are terms of $\weha$, 
           $\phi_{{}_{\QF}}$ is a quantifier-free formula and $\rho,\tau\in\Tp$ are arbitrary.
     \end{enumerate}
     
   \end{itemize}
\end{dfn}
\begin{rmk}
A careful reader observed that the congruence axioms for $=_0$
are not explicitly included in 
the axioms of equality (6). However, it is implied by $\QFm\ER$ that:
\[
\begin{prooftree}
  x=_0y\ \rightarrow\ x=_0y
  \justifies
  x=_0y\ \rightarrow\ r[x]=_\rho r[y]
  \using \QFm\ER
\end{prooftree}
\text{.}
\]
\end{rmk}
The system obtained by the axioms 1\nbd5 together with the rules
1\nbd4 restricted to closed formulas corresponds to the one suggested by G\"odel for the 
purpose of verifying the soundness of the D-interpretation. 
Troelstra shows in \cite{Troelstra73} the equivalence
between G\"odel's and Spector's system as defined originally 
by Spector in \cite{Spector62} which, in turn, was proved to be equivalent to Kleene's 
formalization given in \cite{Kleene52} by Spector in \cite{Spector62}.\\ % as well.
Moreover, Troelstra shows the equivalence of Spector's system to 
the well known natural deduction calculus proving the deduction theorem
for Spector's system. Both, the proof and the definition, 
are given in \cite{Troelstra73}.
In that book, Troelstra also derives some additional schemata and rules, e.g. the 
contraposition rule, $(\phi\rightarrow\psi)\rightarrow(\neg\psi\rightarrow\neg\phi)$, or
the equivalence of single/triple negation, $\neg\neg\neg\phi\leftrightarrow\neg\phi$.\\
This means that as $\weha$ is an extension
of G\"odel's system it surely proves all that can be proved by the natural deduction 
system as defined in \cite{Troelstra73}.
As a facile motivation, we will derive a simple lemma in the form we will need later. To do so, we use
Troelstra's results and the axioms and rules
above. Generally, one can prove even
\[
  \neg\neg\exists x\ \phi(x)\ \leftrightarrow\ \neg\forall x\ \neg\phi(x)\text{,}
\]
again, using only Spector's system 
(or, equivalently, the selection of axioms and rules of $\weha$ as given above).
%
\begin{lemma}\label{l:exnegall}
\[\weha\quad\proves\quad
  \neg\neg\exists x\ \phi(x)\ \rightarrow\ \neg\forall x\ \neg\phi(x)\text{.}\]
\end{lemma}
\begin{proof}
First, we derive $\exists x\phi(x),\forall x\neg\phi(x)\proves\perp$:
\[
\begin{prooftree}
         \exists x\phi(x)^\text{ass.}
         \[
          \[
             \forall x\neg\phi(x)^\text{ass.}\quad
             \forall x\neg\phi(x)\rightarrow\neg\phi(x)^\text{Ax. 5}
           \justifies
           \phi(x)\rightarrow\perp
           \using \text{Rule 1}
          \]
           \justifies
           \exists x\phi(x)\rightarrow\perp
           \using \text{Rule 4}
        \]
        \justifies
        \perp
        \using \text{Rule 1}
\end{prooftree}
\text{,}
\]
which implies $\exists x\phi(x)\rightarrow\neg\forall x\neg\phi(x)$ by the deduction theorem
proved by Troelstra in \cite{Troelstra73}. 
By contraposing this implication twice, again using \cite{Troelstra73}, we obtain:
\[
\neg\neg\exists x\phi(x)\rightarrow\neg\neg\neg\forall x\neg\phi(x)
\text{,}
\]
what is equivalent to
\[
\neg\neg\exists x\phi(x)\rightarrow\neg\forall x\neg\phi(x)
\text{,}
\]
by \cite{Troelstra73} yet again.
\end{proof}
%
%

We have that $\weha$ allows $\lambda$-abstraction in the
following sense.
\begin{lemma}\emph{(see e.g. \cite{Kohlenbach08})} 
If $\,t[x^{\rho}]^{\tau}$ is a term of $\weha$, one can
construct in $\weha$ a term $\lambda x^{\rho}. t[x]$ of type 
$\tau\rho$ such that
\[ \weha \vdash (\lambda x^{\rho}. t[x])(s^{\rho}) =_{\tau} t[s], \]
and such that the free variables of $\lambda x^{\rho}. t[x]$ are the
same as the free variables of $t[x^{\rho}]^{\tau}$ only without
$x^{\rho}$.
\end{lemma}
We also note that 
prime formulas are decidable in $\weha$. In addition,
 we will be able to treat any quantifier-free formula
$\phi_{{}_{\QF}}(\tup x)$ containing only $\tup x$ free
as a prime formula, since we can find closed terms
$t$ of $\weha$
such that 
\[ \weha \vdash \phi_{0}(\underline{x}) \leftrightarrow t
\underline{x} =_{0}0\text{.}\]
We can also code a finite tuple of variables of different 
types into a single variable, whose type depends on the types of the
variables in the tuple. This can be shown in $\weha$ 
(see \cite{Troelstra73}). So, in principle, simultaneous recursion
$\tup R_{\tup\rho}$ can be reduced to $R_\tau$ for suitable type $\tau$.\\
The inequality $\leq$ between numbers is definable in $\weha$ as 
a quantifier-free predicate. 
\begin{dfn}\label{d:leq}
The inequality predicate $\leq_{\rho}$ between functionals of type
$\rho$ is defined inductively by
\setcounter{equation}{0}
\begin{align}
  x \leq_{0} y &:\equiv x \leq y\\
  x \leq_{\tau \rho} y &:\equiv \forall z^{\rho}(xz\leq_{\tau} yz)\text{.} 
\end{align}
\end{dfn}

\subsection{Fragments and Extensions of $\weha$}
%
% EXTENSIONS
%
The classical system corresponding to $\weha$ is obtained simply by
adding the law of excluded middle:
\begin{dfn} 
\label{d:wepa}
By adding the law of excluded middle 
\[ \usftext{LEM}\quad:\quad \phi \vee \neg \phi\text{,} \]
to the axioms of $\weha$ for any formula in $\weha$ we obtain \defkeyn{$\wepa$}.
\end{dfn}
%
Furthermore, we define the extensional systems:
\begin{dfn} 
\label{d:eha_epa}
By replacing the rule $\QFm\ER$ by the 
\defkey{axioms of higher type extensionality}
\[ 
\defkeyn{\usftext{E}}_\rho\quad:\quad 
%   \forall z^{\rho},x_1^{\rho_1},y_1^{\rho_1},...,x_k^{\rho_k},y_k^{\rho_k}
 \forall z^{\rho},x_1^{\rho_1},y_1^{\rho_1},\ldots,x_k^{\rho_k},y_k^{\rho_k}
   \ \Big( 
%\text{$\ORi{k}$}
%\ORi{k}
%\bigwedge_{i=1\ldots k}
\bigwedge_{i=1}^{k}
( x_i=_{\rho_i} y_i ) \rightarrow z\tup x=_0 z \tup y \Big)
\text{,} 
\]
in the systems $\weha$/$\wepa$, where $\rho=0\rho_k\ldots\rho_1$, we obtain the systems \defkey{$\eha$/$\epa$}.
\end{dfn}

%
%FRAGMENTS
%
On the other hand, sometimes it is sufficient to have weaker systems. 
This leads us to the definition of the following fragments:
\begin{dfn}
For all four systems $\weha$, $\wepa$, $\eha$ and $\epa$ let
correspondingly $\hrrweha$, $\hrrwepa$, $\hrreha$, $\hrrepa$ denote
the fragments where we only have the recursor $R_0$ for type-0-recursion and
the induction schema is restricted to the schema of
quantifier-free induction:
\[
\QF\m\IA\quad:\quad
             \left(\phi_{{}_{\QF}}(0)\wedge\forall n^0\big( \phi_{{}_{\QF}}(n)\rightarrow  \phi_{{}_{\QF}}(n+1)\big)\right)
             \rightarrow\forall n^0 \phi_{{}_{\QF}}(n)
\text{,}
\]
where $\phi_{{}_{\QF}}$ is a quantifier-free formula and may 
contain parameters of arbitrary types.
\end{dfn}

%
% REPRESENTATION
%
%
\subsection{The ordered Field of rational Numbers within $\weha$}
%
Rational numbers are represented by
codes $j(m,n)$ of pairs $(m,n)$ of natural numbers. 
Typically $j$ would be just the Cantor pairing function, but any common surjective pairing 
function $j$ with the inverse functions $j_1(j(m,n))=m$ and $j_2(j(m,n))=n$ does the job.
We need the surjectivity to be able to conceive each natural number as a code of
a uniquely determined rational number.
For more examples on paring functions see e.g. \cite{Odifreddi89}.
\begin{dfn}{\em Coding of rational numbers\\}
The rational number represented by $j(m,n)$ is 
\[
    j(m,n)\sim\begin{cases}
         \frac{\frac{n}{2}}{m+1} &\text{if $n$ is even}\\
        - \frac{\frac{n+1}{2}}{m+1} &\Telse \end{cases}\text{.}
\]
%We need also the projections $j_1$ and $j_2$:
%\begin{align*}
%j_1(j(m,n))&:=m\\
%j_2(j(m,n))&:=n\text{.}
%\end{align*}
\end{dfn}
%\begin{rmk}We can choose an arbitrary pairing function $p$. E.g. the Cantor
%pairing function or one of the common pairing functions given in \cite{Odifreddi89}.}\\
Next, we define an equality relation $=_\QQ$ on 
these representatives of the rational numbers.
\begin{dfn}{\em Equality of rational numbers\\}
We define the equality between the codes of two rational numbers
 $q_1$ and $q_2$, $=_\QQ$, as follows:
\[
    q_{1} =_\QQ q_{2} : \equiv
    \frac{\frac{j_{1}q_{1}}{2}}{j_{2}q_{1}+1} =
    \frac{\text{$\frac{j_1q_2}{2}$}}{j_{2}q_{2}+1}  
\]
if $j_{1}q_{1}$ and $j_{1}q_{2}$ are both even, and analogously
in the remaining cases. 
The equality
\[\frac{a}{b} = \frac{c}{d}\]
is defined to hold iff
   \[ad =_{0} cb \wedge bd >_0 0\text{.}\]
\end{dfn}                       
%
We write $n=_{\mathbb{Q}} \langle r \rangle$ to
express the statement that $n$ represents the rational number
$r$. By using only the representatives of natural 
numbers and $=_\QQ$
we can avoid a formal introduction of the set $\QQ$ as 
the set of the equivalence classes on $\NN$ w.r.t. $=_\QQ$.\\
For the embedding $\NN\embeded\QQ$, we define for a natural number $n$ its representation
as a rational number as follows: 
$\langle n\rangle = n_{\mathbb{Q}} : = j(2n,0)$. 
One can then define the primitive recursive operations $+_{\mathbb{Q}}, 
\cdot_{\mathbb{Q}}$ and 
the predicates $<_{\mathbb{Q}}, \leq_{\mathbb{Q}}$ on $\mathbb{N}$ in
$\weha$ such that one can prove (within $\weha$)
that \[(\mathbb{N}, +_{\mathbb{Q}}, \cdot_{\mathbb{Q}}, 0_{\mathbb{Q}},
1_{\mathbb{Q}}, <_{\mathbb{Q}})\] 
represents the field of rational numbers, $(\QQ,+,\cdot,0,1,\leq)$, in $\weha$.
%
% Reals
%
\subsection{The Archimedean ordered Field of real Numbers within $\weha$}
%
We represent the real numbers by sequences of rational numbers with
a fixed rate of convergence $2^{-n}$. 
\begin{dfn}{\em Representation of a real number\\}\label{d:real}
A function $f : \NN \mapsto \NN$ such that
\[
  \forall n\ \big(|fn -_\QQ f(n+1)|_\QQ <_\QQ \langle 2^{-(n+1)}  \rangle\big)
\text{,}
\]
which implies
\begin{align*}
  \forall n\ \forall k\!>\!m\!\geq\!n\ \Big(\ \
    |fm -_\QQ fk|_\QQ\  &\leq_\QQ\  \sum^{k-1}_{i =m}|fi-_\QQ f(i+1)|_\QQ \\
                       &\leq_\QQ\  \sum^{\infty}_{i =n}|fi -_\QQ f(i+1) |_\QQ
                       <_\QQ\ \langle 2^{-n} \rangle\ \ \Big)\text{,}
\end{align*}
represents a real number.
\end{dfn}
We can arrange for each function $f^1$ to code a unique real number in this way.
\begin{dfn} \label{d:hatReal}
Let $f\ :\ \NN\mapsto\NN$ be a function. Define $\widehat{\!f}$ by   
\[
\widehat{\!f}n := 
  \begin{cases}
    fn & \Tif\ \ \forall k<n\ \big(|fk -_\QQ f(k+1)|_\QQ <_\QQ \langle 2^{-k-1}\rangle\big) \\
    fk & \Telse
  \end{cases}\text{,}
\]
where $k$ is the least number such that $k<n$ 
and  \[|fk-_\QQ f(k+1)|_\QQ \geq_\QQ \langle 2^{-k-1} \rangle\text{.}\]
The function $\widehat{\!f}$ defines a uniquely determined real number, so
we say also that $f$ defines a uniquely determined real number, namely the one represented 
by $\widehat{\!f}$.
\end{dfn}
The functional which maps $f$ to $\widehat{\!f}$ can be defined primitive 
recursively in $\weha$. So we can reduce quantifiers ranging over
$\mathbb{R}$ to quantifiers ranging over type $1$ objects. 
The usual operations and relations on $\RR$ can be defined in $\weha$. As an
example we define $=_{\mathbb{R}}$ and $+_{\mathbb{R}}$ as follows:
\begin{dfn}\label{d:Rel}
Equality on $\RR$ is represented by the equivalence relation $=_{\mathbb{R}}$
 of the representatives as follows:
        \[
            f_{1}^{1} =_{\mathbb{R}} f_{2}^{1} : \equiv \forall n
            \big(|\widehat{f_{1}}(n+1) -_\QQ
            \widehat{f_{2}}(n+1)|_\QQ <_\QQ \langle
                                           2^{-n} \rangle \big).
        \]
We define
       \[
       f_{1} <_{\mathbb{R}} f_{2} : \equiv \exists n
            \big(\widehat{f_{2}}(n+1) -_\QQ
            \widehat{f_{1}}(n+1) \geq_\QQ \langle
                                           2^{-n} \rangle  \big),
       \]
and get $f_{1} =_{\mathbb{R}} f_{2} \in \Pi^{0}_{1}$, $f_{1}
<_{\mathbb{R}} f_{2} \in \Sigma^{0}_{1}$, and $f_{1}
\leq_{\mathbb{R}} f_{2} : \equiv \lnot( f_{2} <_{\mathbb{R}} f_{1} )
\in \Pi^{0}_{1}$.\\
The extensionality of these relations can be proved in $\weha$.
\end{dfn}
\begin{dfn}\label{d:Rpm}
We define $+_\RR$ as follows:
\[
( f_{1} +_{\mathbb{R}} f_{2}) (k): = \widehat{f_{1}}(k+1)
+_\QQ \widehat{f_{2}}(k+1)
\text{.}
\]
Analogously for  $-_\RR$.
\end{dfn}
For the embedding $\QQ\embeded\RR$, we define for the coding $n=\langle r\rangle$ of a rational number $r$
its coding $n_{\mathbb{R}} : = \lambda k. n_\QQ$ for the real
number corresponding to $r$. However, usually we omit
the intermediate encoding to a rational number and write shortly $r_\RR$ instead of $(\langle r\rangle)_\RR$.
Again, we do not introduce $\mathbb{R}$ as the set of equivalence classes of
representatives, but consider only the representatives themselves. The structure
\[              
 (\NN^\NN, +_\RR,\cdot_\RR, 0_{\mathbb{R}}, 1_{\mathbb{R}}, <_{\mathbb{R}})
\]
represents the Archimedean ordered field of real numbers $(\mathbb{R}, +, \cdot, 0, 1, <)$ in $\weha$. 
%
% INTERPRETATIONS
%
%
\subsection{Kuroda's negative Translation}
%
As mentioned earlier, we will use Kuroda's version \cite{Kuroda51} of G\"odel's
negative translation \cite{Goedel33}. We also give directly the intuitionistically
equivalent version for tuples of variables 
(compare e.g. to \cite{Kohlenbach08}).
\begin{dfn}[Kuroda '51] \label{d:NT}
To each formula $\phi$ of ${\mathcal L}(\weha)$ we associate its \defkey{negative translation}
$\phi'$, which is a formula of the form
\[
\phi':\equiv \neg\neg \phi^*
\text{,}
\]
where $\phi^*$ is defined inductively as follows:
\begin{enumerate}
\item For an atomic formula $\phi$ we have:
\[
\phi^*:\equiv\phi\text{.}
%\tag{0}
\]
\item Otherwise, suppose we already know $\phi^*$ 
and $\psi^*$. We define:
\begin{align*}
% and
(\phi\!\wedge\!\psi)^*\!\!&:\equiv
    \phi^* \wedge \psi^*\text{,}\\
% or
(\phi\!\vee\!\psi)^*\!\!&:\equiv
    \phi^* \vee \psi^*\text{,}\\
%implies
(\phi\!\rightarrow\!\psi)^*\!\!&:\equiv
    \phi^* \rightarrow \psi^*\text{,}\\
% exists
(\exists a\ \phi(a))^*\!\!&:\equiv
  \exists a (\phi(a))^*\text{,}\\
% for all
(\forall \tup a\ \phi(\tup a))^*\!\!&:\equiv
  \begin{cases}
    \forall \tup a\ \neg\neg (\phi(\tup a))^* & \text{if $\phi$ does not begin with a $\forall$-quantifier}\\
    \forall \tup a\ (\phi(\tup a))^* & \text{else}
  \end{cases}
\end{align*}\text{.}
\end{enumerate}
\end{dfn}
%
This negative translation translates a proof of 
a formula $\phi$ in $\wepa$ into a proof in
$\weha$ of $\phi'$.
%associates with each formula $\phi$
%provable in $\wepa$ the formula $\phi'$ provable in $\weha$.
\begin{thm}{\em( See e.g. \cite{Luckhardt73}.)\\}
Whenever
\[
\wepa \vdash \phi
\]
we have 
\[
\weha \vdash \phi'\text{.}
\]
\end{thm}
%
\subsection{G\"odel's Functional Interpretation}
%
%
%
As mentioned already in the introduction, the D-interpretation
was originally developed by G\"odel as a possible modification
of Hilbert's program. The D-interpretation itself was first
presented in a lecture to the Mathematics and Philosophy
Clubs at Yale University in 1941 and published in \cite{Goedel58}.
This first version was for the ordinary Heyting Arithmetic, the version presented
here is based on Luckhardt's in \cite{Luckhardt73} and Troelstra's
in \cite{Troelstra73}.
Further details on historical 
background can be found
in \cite{AF98}. 
\begin{dfn}[G\"odel '41] \label{d:FId}
To each formula $\phi(\tup{a})$ in ${\mathcal L}(\weha)$ with the tuple of
free variables $\tup{a}$ we associate its \defkey{D-interpretation}
$\phi^D(\tup{a})$, which is a formula of the form
\[
\phi^D(\tup{a}):\equiv\exists \tup x\forall \tup y\ \phi_D(\tup x, \tup y, \tup{a})
\text{,}
\]
where $\phi_D(\tup{a})$ is a quantifier-free formula.
Each of $\tup x$ and $\tup y$ is a tuple of variables whose
types, as well
as the length of each tuple, depend only on the logical structure of $\phi$. 
We also write $\phi_D(\tup x, \tup y, \tup{a})$ for $\phi_D(\tup{a})$. If some variables
$\tup z$ of $\phi$ are exhibited, as $\phi(\tup z,\tup a)$, then we write
$\phi_D(\tup x, \tup y, \tup z,\tup{a})$ for $\phi_D(\tup{a})$.\\
We define the construction of $\phi^D$ inductively as follows 
\begin{enumerate}
\item For an atomic formula $\phi(\tup{a})$, the tuples $\tup x$ and $\tup y$ are empty
and we have:
\[
\phi^D(\tup{a}):\equiv\phi_D(\tup{a}):\equiv\phi(\tup{a})\text{.}
\tag{0}
\]
\item Otherwise, suppose we already know the D-interpretation of $\phi(\tup a)$ 
and $\psi(\tup b)$ to be as follows:
\[
\phi^D(\tup{a})\equiv\exists \tup x\forall \tup y\ \phi_D(\tup{a})
\quad\text{and}\quad
\psi^D(\tup{b})\equiv\exists \tup u\forall \tup v\ \psi_D(\tup{b})
\text{.}
\]
where $\phi_D$, $\psi_D$, $\tup{a}$, $\tup x$, $\tup y$ and $\tup{b}$ satisfy the
assumptions from above and additionally the variable $z$ is not part of any of 
the tuples above (otherwise we could use
a different name for the variable e.g. $z_1$). We define:
\setcounter{equation}{0}
\begin{align}
% and
(\phi(\tup a)\!\wedge\!\psi(\tup b))^D\!\!&:\equiv\!
  \exists \tup x, \tup u\forall \tup y, \tup v
  \big(\phi_D(\tup x,\tup y,\tup{a}) \wedge \psi_D(\tup u,\tup v, \tup{b})\big)\text{,}\\
% or
(\phi(\tup a)\!\vee\!\psi(\tup b))^D\!\!&:\equiv\!
  \exists z^0\!\!,\! \tup x\!,\! \tup u\forall \tup y\!,\! \tup v
  \Big( \big(z\!\!=_0\!0\!\rightarrow\!\phi_D(\tup x,\! \tup y,\!\tup{a})\big) \wedge 
    \big(z\!\!\neq_0\!0\!\rightarrow\!\psi_D(\tup u,\!\tup v,\!\tup{b})\big)  \Big)\text{,}\\
% for all
(\forall z\ \phi(z,\tup a))^D\!\!&:\equiv\!
  \exists \tup X \forall z,\tup y\ \phi_D(\tup X z,\tup y, z, \tup{a})\text{,}\\
% exists
(\exists z\ \phi(z,\tup a))^D\!\!&:\equiv\!
  \exists z,\tup x \forall \tup y\ \phi_D(\tup x,\tup y, z, \tup{a})\text{,}\\
% implies
(\phi(\tup a)\!\rightarrow\!\psi(\tup b))^D\!\!&:\equiv\!
  \exists \tup U,\tup Y \forall \tup x, \tup v
  \big(\phi_D(\tup x,\tup Y\atup x\atup v,\tup{a}) \rightarrow 
   \psi_D(\tup U\atup x,\tup v, \tup{b})\big)\text{.}
\end{align}
\end{enumerate}
\end{dfn}
The clauses (0),(1) and (4) in the definition \ref{d:FId} are
very easy to comprehend.
The D-interpretation of (3), $(\forall z\ \phi(z,\tup a))^D$,
is basically just ``Skolemizing'' the existentially quantified variables in 
$\forall z\exists\tup x\forall\tup y \phi_D(\tup x, \tup y, z,\tup{a})$.
The most interesting part of the definition is the D-inter\-pre\-ta\-tion 
of the implication (5). A very detailed and comprehensive explanation
can be found in \cite{Kohlenbach08}.\\
% interpretation of OR
Originally, G\"odel in \cite{Goedel58} as well as Avigad and 
Feferman in \cite{AF98} defined the D-inter\-pre\-ta\-tion  of 
(2):$(\phi(\tup a)\!\vee\!\psi(\tup b))$ in the following intuitionistically
equivalent way:
\[  \exists z^0,\tup x,\tup u\forall \tup y,\tup v\ 
  \Big( \big(z\!=\!0\wedge\phi_D(\tup x, \tup y,\tup{a})\big) \vee 
  \big(z\!=\!1\wedge\psi_D(\tup u,\tup v,\tup{b})\big)\Big) \tag{+} \text{.}\]
Here we go after the version used by Troelstra (see e.g. 
\cite{Troelstra73}) and Kohlenbach (see e.g. \cite{Kohlenbach08}).
Apart from that the formula $(\phi(\tup a)\!\vee\!\psi(\tup b))_D$ gets not 
only quantifier-free but also $\vee$-free this version
assures the following idempotency property of the D-interpretation:
\begin{thm}\label{t:DDD}
For any formula $\phi$ in ${\mathcal L}(\weha)$ we have
\[
(\phi^D)^D\equiv\phi^D\text{.}
\]
\end{thm}
This would not be the case if we defined (2) in the original way of G\"odel.
Suppose $\phi$ and $\psi$ are atomic formulas. By (+) we have
\begin{align*}
\big((\phi \vee \psi)^D\big)^D
&\equiv\Big(\exists z^0 \big(  ( z\!=_00\wedge\phi ) \vee  ( z\!=_01\wedge\psi )  \big)  \Big)^D\\
&\equiv\exists z^0\bigg(\exists z_1^0 \Big(  
  \big( z_1\!=_00\wedge(z\!=_00\wedge\phi )\big) \vee  
  \big( z_1\!=_01\wedge(z\!=_01\wedge\psi )\big) \Big) \bigg)  \\
&\not\equiv \exists z^0 \big(  ( z\!=_00\wedge\phi ) \vee  ( z\!=_01\wedge\psi )  \big)  
\equiv(\phi \vee \psi)^D
\text{,}
\end{align*}
which obviously contradicts Theorem \ref{t:DDD}.\\
We should mention that as a consequence $\neg\phi$
is D-interpreted as (given that $\phi(\tup a)$ is D-interpreted as
$\exists \tup x\forall \tup y\ \phi_D(\tup{a})$) 
$\exists\tup Y\forall\tup x \neg\phi_D(\tup x,\tup Y\atup x,\tup a)$. This gives
us the following ``macro'' for double negation:
\[
(\neg\neg\phi)^D\!\!\equiv
        \exists\tup X\forall\tup Y\neg\neg\phi_D(\tup X\atup {\args Y},\tup Y\args{(\tup X\atup Y)})\text{,}\]
which is, provably in $\weha$, equivalent to:
\[   \exists\tup X\forall\tup Y\phi_D(\tup X\args{\atup Y},\tup Y\args{(\tup X\atup Y)}) \text{.}\]
We should also mention a partial (weaker) interpretation of the implication,
which can be used if the data or information of the premise are not needed
for further use of the D\nbd interpretation of a given formula. E.g. if the
premise can be proved directly. Typically, we would analyze a proof in which
the elimination of an implication is the final step in the 
proof of the conclusion $\psi$:
\[
\weha\quad\proves\quad(\phi(\tup a)\!\rightarrow\!\psi(\tup b))^D\ \rightarrow\ 
  \exists \tup U\forall \tup x, \tup v
  \big((\forall\tup y\phi_D(\tup x,\tup y,\tup{a})) \rightarrow 
   \psi_D(\tup U\atup x,\tup v, \tup{b})\big)
\text{.}\]\\

The D-interpretation is sound in the following sense.
\begin{dfn}
The \defkey{Independence-of-Premise} schema for universal premises:
\defkeyn{$\IP^\omega_\forall$}; is the union (for all types $\rho$) of
\[
\IP^\rho_\forall\quad:\quad
   \big(\forall\tup x\phi_{{}_{\QF}}(\tup x)\rightarrow \exists y^\rho\psi(y)\big)
   \rightarrow \exists y^\rho\big(\forall\tup x\phi_{{}_{\QF}}(\tup x)\rightarrow\psi(y)\big)
\text{,}
\]
where $y$ is not free in $\forall \tup x\phi_{{}_{\QF}}(x)$.
\end{dfn}

\begin{thm}{\em Soundness of D-Interpretation \cite{Kohlenbach08}\\}\label{t:sFI}
Let ${\mathcal{U}}$ be an arbitrary set of purely universal sentences of $\weha$ and
$\phi$ a formula of $\weha$ containing only $\tup a$ as free variables, then 
\[
\weha + \AC + \IP^\omega_\forall + \M + {\mathcal{U}} \quad\vdash\quad \phi(\tup a)
\]
implies that
\[
\weha + {\mathcal{U}} \quad\vdash\quad \forall \tup y\phi_D(\tup t\atup a, \tup y, \tup a)
\text{,}
\]
where $\tup t$ is a suitable tuple of closed terms of $\weha$ which can 
be extracted from a given proof.
\end{thm}
In combination with the negative translation as a pre-processing step
we get the {\em ND\nbd in\-ter\-pre\-ta\-tion}\index{ND-interpretation} with the following properties.
\begin{thm}{}
Let ${\mathcal{U}}$ be an arbitrary set of purely universal sentences of $\wepa$,
$\phi$ a formula of $\wepa$ containing only $\tup a$ as free variables 
and suppose
\[
\wepa\ \  +\ \ \QFm\AC\ \ +\ \ {\mathcal{U}}\quad\vdash\quad \phi(\tup a)\text{,}
\]
then the ND-interpretation extracts closed terms $\tup t$ of $\weha$, such that
\[
\weha\ \  +\ \ {\mathcal{U}}\quad\vdash\quad \forall 
    \tup y(\phi')_D(\tup t\atup a, \tup y,\tup a)\text{.}
\]
\end{thm}
\begin{thm}{\em Main theorem on program extraction by ND-interpretation \cite{Kohlenbach08}\\}
\/Let ${\mathcal{U}}$ be an arbitrary set of purely universal sentences of $\wepa$ 
and $\phi_{{}_{\QF}}(\tup x,\tup y)$ be a quantifier-free formula of $\wepa$ which
contains only $\tup x^{\tup\rho}$,$\tup y^{\tup\tau}$ as free variables. Suppose
\[
\wepa\ \  +\ \ \QFm\AC\ \ +\ \ {\mathcal{U}}\quad\vdash\quad 
  \forall \tup x^{\tup\rho}\exists \tup y^{\tup\tau} \phi_{{}_{\QF}}(\tup x, \tup y)
\text{,}
\]
then the ND-interpretation extracts a tuple of  closed terms $\tup t$ 
of $\weha$, such that
\[
\weha\ \  +\ \ {\mathcal{U}}\quad\vdash\quad \forall 
    \tup x^{\tup \rho}\phi_{{}_{\QF}}(\tup x, \tup t\atup x)\text{.}
\]
\end{thm}


In this thesis, it is almost always the combination of negative 
translation and functional interpretation that is used. This combination becomes 
particularly convenient to formulate if one uses a negative translation due 
to Krivine, as it then coincides with the so-called Shoenfield variant 
\cite{shoenfield01} (for the fragment $\{ \forall, \vee,\neg\}$) as 
was shown in \cite{streicherkohlenbach07}, where this interpretation is 
given for the full language (i.e. $\{ \forall, \exists, \rightarrow,\vee,
\wedge,\neg \}$ with $A\leftrightarrow B:\equiv 
(A\to B) \wedge (B\to A)$). 
\\[1mm]
We now give the definition of this Shoenfield interpretation:   
\begin{dfn}[\cite{shoenfield01,streicherkohlenbach07}] \label{d:FI}
To each formula $\phi(\tup{a})$ in ${\mathcal L}(\wepa)$ with the tuple of
free variables $\tup{a}$ we associate its {Sh-interpretation}
$\phi^{Sh}(\tup{a})$, which is a formula of the form
\[
\phi^{Sh}(\tup{a}):\equiv\forall \tup{u}\,\exists \tup{x}\, 
\phi_{Sh}(\tup{u},\tup{x}, \tup{a})
\text{,}
\]
where $\phi_{Sh}(\tup{a})$ is a quantifier-free formula.
Each of $\tup{x}$ and $\tup{y}$ is a tuple of variables whose
types, as well
as the length of each tuple, depend only on the logical structure of $\phi$. 
We also write $\phi_{Sh}(\tup{x},\tup{y},\tup{a})$ for $\phi_{Sh}(\tup{a})$. 
If some variables
$\tup z$ of $\phi$ are exhibited, as $\phi(\tup z,\tup a)$, then we write
$\phi_{Sh}(\tup x, \tup y, \tup z,\tup{a})$ for $\phi_{Sh}(\tup{a})$.\\
We define the construction of $\phi^{Sh}$ inductively as follows 
(with $\tup{x}\,\tup{y}$ denoting $x_1\tup{y},\ldots,x_n\tup{y}$ for 
$\tup{x}=x_1,\ldots,x_n$). In the inductive steps we assume that 
\[
\phi^{Sh}(\tup{a}):\equiv\forall \tup{u}\,\exists \tup{x}\, 
\phi_{Sh}(\tup{u},\tup{x}, \tup{a}) \ 
\text{and} \ \psi^{Sh}(\tup{b}):\equiv\forall \tup{v}\,\exists \tup{y}\, 
\psi_{Sh}(\tup{v},\tup{y}, \tup{b})
\] are already defined. 
\begin{enumerate}
\item[(S1)] $\phi^{Sh}(\tup{a}) 
\equiv:\phi_{Sh}(\tup{a})$ for atomic 
$\phi(\tup{a}),$
\item[(S2)] $(\neg \phi)^{Sh} \equiv \forall \tup{f} \exists \tup{u} \, 
\neg \phi_{Sh}(\tup{u},\tup{f}\,\tup{u}),$
\item[(S3)] $(\phi \vee \psi)^{Sh} \equiv 
             \forall \tup{u},\tup{v} \exists \tup{x},\tup{y} \, 
\big(\phi_{Sh}(\tup{u},\tup{x}) \vee \psi_{Sh}(\tup{v},\tup{y})\big),$
\item[(S4)] $(\forall z \, \phi)^{Sh} \equiv \forall  z,\tup{u} 
\exists \tup{x} \, \phi_{Sh}(z,\tup{u},\tup{x}),$
\item[(S5)] 
$(\phi{\to} \psi)^{Sh} 
\equiv \forall \tup{f}, \tup{v} \exists \tup{u}, \tup{y} \, \big(\phi_{Sh}
(\tup{u},\tup{f}\, \tup{u}) \to \psi_{Sh}(\tup{v},\tup{y})\big),$
\item[(S6)] 
$(\exists z \,\phi)^{Sh} \equiv \forall \tup{U} \exists z, \tup{f} \, 
\phi_{Sh}(z,\tup{U} \,z\,\tup{f},\tup{f}(\tup{U}\,z\,\tup{f})),$
\item[(S7)] $(\phi \wedge \psi)^{Sh} \leftrightarrow 
\forall \tup{u},\tup{v} \exists \tup{x}, \tup{y} \,
\big( \phi_{Sh}(\tup{u},\tup{x}) \wedge \psi_{Sh}(\tup{v},\tup{y})\big).$ 
\end{enumerate}
\end{dfn}
\begin{remark} The official definition  of (S7) in \cite{streicherkohlenbach07} 
is slightly different from the one given above (where our version is called 
(S7$^*$)) 
but is intuitionistically equivalent to that.
\end{remark}
As before, a partial (weaker) interpretation of the implication is sufficient
when the witnessing data from the premise are not needed
for further use of the Sh\nbd interpretation of a given formula. E.g. if the
premise can be proved directly. Typically, we would analyze in such cases 
an implication as:


\[ \forall \tup{f} \exists \tup{y} \, \big( \forall \tup{u}\,
\phi_{Sh}
(\tup{u},\tup{f}\,\tup{u}) \to \psi_{Sh}(\tup{v},\tup{y})\big).\]
\begin{remark} 
The Shoenfield version of the functional interpretation is often -- for 
obvious reasons -- called $\forall\exists$-form, whereas the Dialectica 
interpretation (and hence also the combination ND of some 
negative translation N with 
the Dialectica interpretation D -- see above) always is of the form $\exists\forall.$ 
If the Krivine negative translation is used (see 
\cite{streicherkohlenbach07}), 
the latter is nothing else but 
the result of a final application of the axiom schema of quantifier-free 
choice $\QFm\AC$ 
to the Shoenfield interpretation. One should stress though that this passage 
from the $\forall\exists$-form to the $\exists\forall$-version which  
also is implicitly present in the soundness theorem of the Shoenfield 
interpretation (stating the extractability of suitable terms realizing 
the $\forall\exists$-form) is necessary for the interpretation to be sound 
for the modus ponens rule.  
\end{remark}


%
\subsection{Majorizability}
%
%
We introduce an important structural property of the closed terms of all systems
used in this thesis (and more, e.g.\ all systems used in \cite{Kohlenbach08}). We need this
property of \defkey{majorizability}, which is due to W.A.Howard in \cite{Howard73}, 
to define the third interpretation we will apply. However, there are several applications
of majorizability. Mainly, to prove results on growth of the definable functionals of 
given systems or, more generally, to give 
bounds on growth or complexity of functionals belonging to a specific class.
%\begin{minipage}{\textwidth}
\begin{dfn}
\label{d:maj}
The relation \defkeyn{$x^*\maj_\rho x$} ($x^*$ \defkey{majorizes} $x$) between functionals of
type $\rho$ is defined by induction on $\rho$:
\setcounter{equation}{0}
\begin{align}
 x^*\maj_0 x &:\equiv x^*\geq_0x\text{,} \\
 x^*\maj_{\tau\rho} x &:\equiv \forall y^*,y(y^*\maj_\rho y\rightarrow 
        x^*y^*\maj_\tau xy)\text{.}
\end{align}
\end{dfn}
%\end{minipage}
%
Moreover, Howard showed in \cite{Howard73} the following very useful fact:
\begin{thm}{\em W.A.Howard \cite{Howard73}\\}
\/For each closed term $t^\rho$ of $\weha$ one can construct a closed term
$t^*$ in $\weha$ of the same type, such that:
\[
\weha\quad\proves\quad t^* \maj_\rho t \text{.}
\]
\end{thm}

\subsection{Monotone Functional Interpretation}\label{ss:mfi}
%
%
The monotone functional interpretation, also called \defkey{MD-interpretation},
 extracts the terms which majorize some functionals realizing
the usual functional interpretation directly. It was introduced by Kohlenbach in \cite{Kohlenbach96mfi}. 
However, he studied and used the combination of D-interpretation 
with majorizability earlier as well, see e.g. \cite{Kohlenbach92}.\\
We have the following soundness theorem:
\begin{thm}{\em Soundness of MD-interpretation \cite{Kohlenbach96mfi},\cite{Kohlenbach08}\\}
\label{t:mfi}
Let $\Delta$ be a set of sentences of the 
form ($\phi_{{}_{\QF}}$ is a quantifier-free formula and the tuple $\tup r$ consists of closed terms):
\[
  \forall\tup a^{\tup\delta}\exists\tup b\leq_\sigma\tup r\atup a\forall\tup c^{\tup\gamma}
  \phi_{{}_{\QF}}(\tup a,\tup b,\tup c)\text{,}
\] 
and suppose
\[
\weha\ \  +\ \ \AC\ \ +\ \ \IP_\forall^\omega\ \ +\ \ \M\ \ +\ \ \Delta
     \quad\vdash\quad \phi(\tup a)\text{,}
\]
then MD-interpretation extracts closed terms $\tup t^*$ of $\weha$ such that
\[
\weha\ \  +\ \ \tilde\Delta\quad\vdash\quad 
  \exists\tup x(t^*\maj x \wedge \forall \tup a,\tup y
    \phi_D(\tup x\atup a, \tup y,\tup a))\text{,}
\]
where $\tilde\Delta$ is the corresponding set of the Skolem normal forms of
sentences in $\Delta$:
\[ 
\tilde\Delta:=\left\{
\tilde\phi:\equiv 
  \exists\tup B\leq\tup r\forall\tup a,\tup c\phi_{{}_{\QF}}(\tup a,\tup B\atup a,\tup c)\ :\ 
\phi\equiv
  \forall\tup a^{\tup\delta}\exists\tup b\leq_\sigma\tup r\atup a\forall\tup c^{\tup\gamma}
  \phi_{{}_{\QF}}(\tup a,\tup b,\tup c)\in\Delta
\right\}\text{.}
\]
\end{thm}
\begin{dfn}{}\label{d:mfi}
Considering the theorem~\ref{t:mfi}, we say that the tuple of terms
$\tup t^*$ satisfies the \defkey{monotone functional interpretation}, or simply
\defkey{MD-interpretation}, of $\phi$.
\end{dfn} {\samepage
Kohlenbach connects in \cite{Kohlenbach08} the MD-interpretation with 
proof mining by the following theorem:
\begin{thm}{\em Main theorem on uniform bound extraction by MD-interpretation 
\cite{Kohlenbach08}\\}\label{t:ubeMD}
Let $\Delta$ be as above and $\phi(x^1,y^\rho,z^\tau)$ an arbitrary 
formula containing only $x$,$y$,$z$ free. Let $s^{\rho(1)}$ be a closed
term of $\weha$ and the type level of $\tau$ not greater than $2$.\\
Then if one can prove:
\[
\weha+\AC+\IP^\omega_\forall+\M+\Delta\quad\proves\quad
  \forall x^1\forall y\leq_\rho sx\exists z^\tau \phi(x,y,z)
\]
one can also extract a closed term $t$ of $\weha$ s.t.
\[
\weha+\AC+\IP^\omega_\forall+\M+\Delta\quad\proves\quad
  \forall x^1\forall y\leq_\rho sx\exists z^\tau\leq_\tau tx\ \phi(x,y,z)\text{.}
\]
\end{thm}
} %end samepage
As in the case of D-interpretation, we can combine the MD-interpretation with
the negative translation to obtain the \defkey{NMD-interpretation}. In this case
the main theorem, again due to Kohlenbach (see e.g. \cite{Kohlenbach08}), becomes:
\begin{thm}{\em Main theorem on uniform bound extraction by NMD-interpretation \cite{Kohlenbach08}\\}
\label{t:ubeNMD}
Let $\Delta$,$\tilde\Delta$ be as above and $\phi_{{}_{\QF}}(x^1,y^\rho,z^\tau)$ be a 
quantifier-free formula of $\wepa$ containing only $x$,$y$,$z$ free. 
Let $s^{\rho(1)}$ be a closed
term of $\weha$ and the type level of $\tau$ not greater than $2$.\\
Then if one can prove:
\[
\wepa+\QFm\AC+\Delta\quad\proves\quad
  \forall x^1\forall y\leq_\rho sx\exists z^\tau \phi_{{}_{\QF}}(x,y,z)
\]
one can also extract a closed term $t$ of $\weha$ s.t.
\[
\weha+\tilde\Delta\quad\proves\quad
  \forall x^1\forall y\leq_\rho sx\exists z^\tau\leq_\tau tx\ \phi_{{}_{\QF}}(x,y,z)\text{.}
\]
\end{thm}
Both theorems, ~\ref{t:ubeMD} and ~\ref{t:ubeNMD}, apply also to tuples of variables, 
where tuples of terms are extracted. The proofs are given e.g. in~\cite{Kohlenbach08}.
%
%
%
%
\subsection{A specific metatheorem}
%
%
%
Metatheorems which guarantee that specific uniform bounds for the realizers
can be extracted were developed in~\cite{Kohlenbach05meta} and~\cite{GK08} (see also~\cite{Kohlenbach08})
and are applicable to many theorems concerning a wide range of classes of maps and abstract spaces. 
%Many convergence theorems typically meet both conditions. 
%Apart form ergodic theory, where at least all
%the ergodic theorems mentioned in Figure~\ref{f:METtree} meet both of these conditions, such metatheorems were
%successfully applied for asymptotic regularity theorems in metric fixed point theory~\cite{Kohlenbach2010}. 
For example, they were successfully applied to 
the ergodic theorems we mention in Figure~\ref{f:METtree} in the chapter on Wittmann's strong
non-linear ergodic theorem or to 
asymptotic regularity theorems in metric fixed point theory~\cite{kohlenbachleustean10}.
In the last mentioned example, % as in this paper, 
the authors infer from the metatheorems that uniform bounds exist and derive them explicitly.\\
To apply the metatheorems, one needs the analyzed theorem to meet only two conditions:
\begin{enumerate}
\item The proof does not use axioms or rules which are too strong.
\item The analyzed theorem in its logical form is not too complex in terms of quantification.
\end{enumerate}
%spaces:
To formalize the first condition we start with a logical system for so called full classical analysis
introduced by Spector in \cite{Spector62}.\footnote{In particular this system covers full comprehension
over numbers, including also full second order arithmetic.} Kohlenbach
extended this system by an additional basic type and its defining axioms representing a given abstract space
and its properties. Kohlenbach also considers cases, where a specific subset of such a space 
(or rather its characteristic function) has to exist as a constant. For instance in~\cite{Kohlenbach08} Kohlenbach defines 
such systems for the theory of metric, hyperbolic, normed, uniformly convex or Hilbert spaces -- 
if required -- together with a (bounded) convex subset.\footnote{In any such abstract space,
its metric plays a major role as two objects are defined to be equal, if and only if their 
distance is zero.} For the purpose of this thesis (i.e. simply for a pre-Hilbert space) this extended system is denoted by $\AHilb$.
In general, the system can be extended to arbitrary Hilbert spaces, however it turns out that the completeness
is not necessary for the proof that we analyze in the corresponding chapter.\\
%functions
The second condition has to be investigated for each theorem specifically, depending on
the given theorem and the metatheorem we wish to use. Examples are metastable versions of formulas 
expressing the convergence or fixed point properties
of nonexpansive, Lipschitz, weakly quasi-nonexpansive or uniformly continuous functions even simply functions 
which are majorizable (see Corollary 6.6 in~\cite{GK08} and Theorem~\ref{t:GKmeta1} below).
%
The metatheorem applicable
in the scenario we discuss in chapter~\ref{s:Wittmann} follows from Corollary 6.6.7) in~\cite{GK08}.
In particular with 
the theory $\AHilb$ with an additional parameter for an arbitrary subset $S$ of the
abstract Hilbert space $X$.\footnote{This is analogous to the case where we add $C$ to the
theory for normed space, but this time without any additional axioms.}\\

\begin{thm}[Gerhardy-Kohlenbach~\cite{GK08} - specific case 1]\label{t:GKmeta1}
Let $\varphi_\forall$, resp. $\psi_\exists$, be $\forall$-
resp. $\exists$-formulas that contain only $x,z,f$ free, resp. $x,z,f$, $v$ free. Assume that
$\mathcal{A}^\omega[X,\langle\cdot,\cdot\rangle,S]$ proves the following sentence:
\[
\forall  x\in\NN^\NN, z\in S, f\in{S^S} 
	\big( \varphi_\forall(x, z, f)\rightarrow\exists v\in\NN\ \psi_\exists(x, z, f, v)\big).
\]
Then there is a computable functional $F : \NN^\NN\times\NN\times\NN^\NN\to\NN$ s. t. the following holds
in all non-trivial (real) inner product spaces $(X,\langle\cdot,\cdot\rangle)$ 
and for any subset $S\subseteq X$
\begin{align*}
\forall  &x\in\NN^\NN, z\in S, b\in\NN, f\in{S^S},f^*\in\NN^\NN\\
	&\big( \TMaj(f^*,f)\ \wedge\ \|z\|\leq b\ \wedge\ \varphi_\forall(x, z, f) \rightarrow 
	\exists v\leq F(x,b,f^*)\ \psi_\exists(x, z, f, v) \big),
\end{align*}
where %$0_X$ does not occur in $\varphi_\forall$ and $\psi_\exists$ and 
\[
\TMaj(f^*,f):\equiv \forall n\in\NN\forall z\in S \big( \|z\|\leq_\RR n \rightarrow \|f(z)\|\leq_\RR f^*(n)\big).
\]
The theorem holds analogously for finite tuples. % $\tup x\in \prod^n_{i=0}\NN^\NN$.
\end{thm}

%
%
%
\subsection{Models of $\epa$}
%
%
We will introduce three models of $\epa$:
\begin{itemize}
\item The full set-theoretic model: $\Set$.
\item The model of all sequentially continuous functionals: $\Cont$.
\item The model of all strongly majorizable functionals: $\Maj$.
\end{itemize}
Note that as models of $\epa$ these are also models of any other system
we presented so far.
%
\begin{dfn}The {\em full set-theoretic model}\index{model of all set-theoretic functionals} is defined inductively
as the type-structure of all set-theoretic functionals:
\setcounter{equation}{0}
\begin{align}
\SetO_0&:=\NN\text{,}\\
\SetO_{\tau\rho}&:=\{\text{ all set-theoretic functionals } F:\SetO_\rho\rightarrow\SetO_\tau\}\text{,}\\
%\defkeym{\Set}&:=\bigcup_{\rho\in\Tp}\SetO_\rho
\Set&:=\bigcup_{\rho\in\Tp}\SetO_\rho
\text{.}
\end{align}
\end{dfn}
%
%
%
To define $\Cont$ we need the following definition first:
\begin{dfn}
Let $(X,\rightarrow_X)$ and $(Y,\rightarrow_Y)$ be two L-spaces. A function $f:X\rightarrow Y$
is called continuous if $f(p_n)\rightarrow_Y f(p)$ whenever $p_n\rightarrow_X p$. We denote
the set of all continuous functions from $X$ to $Y$ by $\ContO(X,Y)$.\\
Here, by L-space we mean a `limit space' as defined in~\cite{Kuratowski52}.
\end{dfn}
%
\begin{dfn}{\bf\cite{Scarpellini71}} The \defkey{model of all sequentially continuous functionals} is defined inductively
as follows:
\setcounter{equation}{0}
\begin{align}
\ContO_0&:=\NN\text{,} & p_n\rightarrow_0 p\ &:\equiv\ \exists k\forall m>k\ p_m=p\\
\ContO_{\tau\rho}&:=\ContO(\ContO_\rho, \ContO_\tau)\text{,} &
  f_n\!\rightarrow_{\tau\rho}\! f\ &:\equiv\ \forall (p_n)\!\in\!\ContO^\NN_\rho,p\!\in\!\ContO_\rho\ 
                  \left(p_n\!\rightarrow_\rho\! p\Rightarrow f_n(p_n)\!\rightarrow_\tau\! f(p)\right)\\
%\defkeyn{\text{$\Cont$}}&:=\bigcup_{\rho\in\Tp}\ContO_\rho & &
\Cont&:=\bigcup_{\rho\in\Tp}\ContO_\rho\text{.} & &
\end{align}
\end{dfn}
To define $\Maj$ we will make use of the variant $\smaj$ of Howard's majorization relation $\maj$.
\begin{dfn}{\bf\cite{Bezem85}} The \defkey{model of strongly majorizable functionals} is the
type structure $\Maj$ defined inductively as follows:
\begin{align*}
%
\MajO_0&:=\NN,\\
&n \smaj_0 m\ :\equiv\ n\geq m\wedge\ n,m\in\NN \tag{1}\\
%
\MajO_{\tau(\rho)}&:=\left\{x\in\MajO^{\MajO_\rho}_{\tau}\ :\ 
                                 \exists x^*\in\MajO^{\MajO_\rho}_{\tau}\ (x^* \smaj_{\tau(\rho)} x)\right\},\\
&x^* \smaj_{\tau(\rho)} x\ :\equiv\ 
             \forall y^*\!,y\!\in\!\MajO_\rho\left( y^*\smaj_\rho y\ \rightarrow\ x^*y^*\smaj_\tau x^*y, xy \right)
             \wedge x^*\!,x\!\in\!\MajO^{\MajO_\rho}_{\tau} \tag{2}\\
%
%\defkeyn{\Maj}&:=\bigcup_{\rho\in\Tp}\MajO_\rho  \tag{3}
\Maj&:=\bigcup_{\rho\in\Tp}\MajO_\rho  \tag{3}
%
\text{,}
\end{align*}
%
where $\MajO^{\MajO_\rho}_{\tau}$ is the set of all total set-theoretic mappings 
from $\MajO_\rho$ into $\MajO_\tau$.
\end{dfn}
\begin{rmk} While $\Maj$ and $\Cont$ are models of $\epa$ with bar recursion (see definition~\ref{d:BR})
added, $\Set$ is obviously not. However, the models start to differ only from type $2$ on, where
we still have: $\ContO_2\subset\MajO_2\subset\SetO_2$. So, if we use bar recursion and Peano Arithmetic to
produce a functional $F$ of type level $2$, we know that it is a well defined functional in $\Cont$ therefore
also in $\Maj$ and in $\Set$ and so a total function: $\NN^\NN\mapsto\NN$.
\end{rmk}

\subsection{\defkeyn{Bar Recursion}} \label{ss:BR}
We give the definition in the form presented by Spector in 
\cite{Spector62}. Note that this form already 
uses the fact that tuples of variables of any type can be 
contracted into single variables. Alternatively, one could use a 
simultaneous form of bar recursion (see \cite{Kohlenbach08}).
\begin{dfn}
\label{d:BR}
The \defkey{bar recursor} $\B_{\rho,\tau}$ is defined by:
\[
\B_{\rho,\tau}yzunx:=_\tau\begin{cases}
   zn(\overline{x,n})&\text{if}\ y(\overline{x,n})<_0n\\
   u\big(\lambda D^\rho.\B_{\rho,\tau}yzu(n+1)(\overline{x,n}*D)\big)n(\overline{x,n})&\text{else}
   \end{cases}
\text{,}\]
where
\[
(\overline{x,n})(k^0)=_\rho\begin{cases}
  x(k)&\text{if}\ k<_0 n\\
  0^\rho&\text{else}
  \end{cases}
\]
and
\[
(\overline{x,n}*D)(k^0)=_\rho\begin{cases}
  x(k)&\text{if}\ k<_0 n\\
  D&\text{if}\ k=_0 n\\
  0^\rho&\text{else}
  \end{cases}
\text{.}\]
\end{dfn}
%
\begin{rmk}
  Note that $\overline{f,n}$, which is
  still a type 1 function, is not the same as $\bar fn$, what is a type 0 object
  (see the section Introduction, paragraph Notation and Common Expressions).
\end{rmk}
%
For better readability Spector presented in \cite{Spector62}
the recursor $\bPhi$,
\begin{align*}
& \bPhi_\rho yunxm:=_{\rho}\left\{ \begin{array}{l} 
        xm,\: \mathrm{if}\ m<_0n\\
        0^\rho,\: \mathrm{if}\ m\geq_0n \wedge y(\overline{x,n})<n\\
        \bPhi_\rho yun' (\overline{x,n}*D_0)m,\: \mathrm{otherwise}
\end{array} \right. &&  D_0=_\rho un(\lambda D^\rho.\bPhi_\rho yun'(\overline{x,n}*D))\text{,}&
\end{align*}
for a special form of bar recursion.
As in this thesis we deal only with arithmetical comprehension
over numbers we don't need the bar recursion for all types.
The recursor $\B_{0,1}$ is fully sufficient and
corresponds to $\bPhi_0$ which can be defined 
primitive recursively in $\B_{0,1}$ as follows:
% Phi in B01
%
\begin{dfn}\label{d:PhiB}
\[ \bPhi_0 yunx =_{1} \B_{ 0, 1}yz\tilde unx \text{,}\]
where
\begin{align*} 
znxm&:=_{0} \begin{cases} 
  xm &\text{if}\ m<_0n\\
  0^0 &\text{else}
\end{cases} \text{,} \\
\tilde uvnxm&:=_{0} \begin{cases} 
  xm &\text{if}\ m<_0n\\
  v(unv)m &\text{else}
\end{cases} \text{.} 
\end{align*}
\end{dfn}
%
% Phi defined directly
%
However, for any practical issue we will use the direct definition:
\[  \bPhi_0 yunxm:=_{0}\left\{ \begin{array}{l} 
        xm,\: \mathrm{if}\ m<_0n\\
        0^0,\: \mathrm{if}\ m\geq_0n \wedge y(\overline{x,n})<n\\
        \bPhi_0 yun' (\overline{x,n}*D_0)m,\: \mathrm{otherwise}
\end{array} \right. \text{,} \]
where
\[ D_0=_0 un\big(\lambda D^0.\bPhi_0 yun'(\overline{x,n}*D)\big)\text{.} \]
%
%
Furthermore, to be able to properly analyze the complexity of the witnessing functionals 
in later sections, we introduce Howard's schemas of restricted bar recursion as given
in \cite{Howard81}.
\begin{dfn}\label{d:rBR}
The \defkey{restricted bar recursor} for \defkey{Scheme 1}, $\rB_\one$, is defined by:
\begin{align*}
\rB_\one&y^2z^{(2)0}un^0x^1:=_0\\
   &\begin{cases}
   zn(\overline{x,n})&\text{if}\ y(\overline{x,n})<_0n\\
   u\big(\rB_\one yzu(n\!+\!1)(\overline{x,n}\!*\!0)\big)
    \big(\rB_\one yzu(n\!+\!1)(\overline{x,n}\!*\!1)\big)
    n(\overline{x,n})&\text{else}
   \end{cases}\text{,}
\end{align*}
and for \defkey{Scheme 2}, $\rB_\two$, by:
\[
\rB_\two y^2u^1n^0x^1:=_0
   \begin{cases}
   0&\text{if}\ y(\overline{x,n})<_0n\\
   1+\rB_\two yu(n+1)\big(\overline{x,n}*(un)\big)&\text{else}
   \end{cases}\text{.}
\]
\end{dfn}
\begin{rmk}Note that, $\rB_\one$ and $\rB_\two$ are just special
 forms of $\B_{0,0}$ since we have:
\[
\rB_\one y^2z^{(2)0}un^0x^1=\B_{0,0}yz\tilde unx\text{,}
\]
by setting
\[
 \tilde uvnx:=u(v0)(v1)nx
\]
i.e. it holds
\begin{align*}
\tilde u\big(\lambda &D^0.\B_{0,0}yz\tilde u(n+1)(\overline{x,n}*D)\big)
         n(\overline{x,n})=\\
    &u\Big(\big(\lambda D^0.\B_{0,0}yz\tilde u(n\!+\!1)(\overline{x,n}\!*\!D)\big)0\Big)
     \Big(\big(\lambda D^0.\B_{0,0}yz\tilde u(n\!+\!1)(\overline{x,n}\!*\!D)\big)1\Big)
     n(\overline{x,n})
   \text{.}
\end{align*}
And also:
\[
\rB_\two y^2u^1n^0x^1=\B_{0,0}y\zero^{(2)0}\tilde unx\text{,}
\]
by setting
\[
 \tilde uvnx:=1+v(un)
\] 
i.e. it holds
\begin{align*}
\tilde u\big(\lambda D^0.\B_{0,0}&yz\tilde u(n+1)(\overline{x,n}*D)\big)
         n(\overline{x,n})=\\
     &1+\big(\lambda D^0.\B_{0,0}yz\tilde u(n\!+\!1)(\overline{x,n}*D)\big)(un)
   \text{.}
\end{align*}
Furthermore applied only to 0,1-sequences (i.e. type one
functions $x:\NN\mapsto\{0,1\}$), as is always the case in this
thesis, is $\rB_\two$ a special form of $\rB_\one$ since we have:
\[
\rB_\two y^2u^1n^0x^1=\rB_\one y^2\zero^{(2)0}\tilde un^0x^1 \text{,}
\]
by setting
\[ 
\tilde uv_1v_2nx:=\\
   \begin{cases}
     1+v_1 &\text{if}\ un=_00\\
     1+v_2 &\text{else}
   \end{cases}\text{.}
\]
i.e. it holds
\begin{align*}
\tilde u\big(\rB_\one yz\tilde u(n+1)(\overline{x,n}*0)\big)
        &\big(\rB_\one yz\tilde u(n+1)(\overline{x,n}*1)\big)
         n(\overline{x,n})=\\
   &\begin{cases}
     1+\big(\rB_\one yz\tilde u(n+1)(\overline{x,n}*0)\big) &\text{if}\ un=_00\\
     1+\big(\rB_\one yz\tilde u(n+1)(\overline{x,n}*1)\big) &\text{else}
   \end{cases}\text{.}
\end{align*}\end{rmk}
%
%  LEM
%
%%%%%%%%%%%%%%%%%%%
%
%
\subsection{Law of excluded Middle} \label{ss:LEM}
The law of excluded middle over numbers for $\SiL$-formulas is defined as follows:
\[ 
 \defkeyn{\text{$\LEM(f)$}}\quad:\quad\forall x\exists y\forall z\  \big(f(x,y)=_00 \vee f(x,z)\neq_00\big)  
\text{.}\]
\begin{rmk}
Usually this law would be formulated in the following, under $\weha$ equivalent, way:
\[ 
 \forall x\ \big(\exists y\ f(x,y)=_00 \vee \forall z\ f(x,z)\neq_00\big)  
\text{.}\]
\end{rmk}
To obtain the ND-interpretation we first use the negative translation $(\LEM(f))'$:
\[\neg\neg\forall x^0\neg\neg\exists y^0\forall z^0\ 
          \neg\neg\big(f(x,y)=_00 \vee f(x,z)\neq_00\big)\text{,}\]
which is intuitionistically equivalent to
\[ \forall x\neg\neg\exists y\forall z\ (f(x,y)=_00 \vee f(x,z)\neq_00)  \text{.}\]
Now, following very strictly all rules, we get the functional 
interpretation of \defkey{$(\LEM)'$}:
\begin{align*}
 \forall Z^{10},&f^{10},x^0\\
       \bigg(\ &\Big(t_D\big(Z(t_YZxf)(t_DZxf)\big)Zxf=_00\wedge f(x,t_YZxf)=_00\Big)\ \vee \\
          &\Big(t_D\big(Z(t_YZxf)(t_DZxf)\big)Zxf=_01\wedge f\big(x,Z(t_DZxf)(t_YZxf)\big)\neq_00\Big)\ \bigg) \text{,} 
\end{align*}
 where
\begin{align*}
  t_D(n^0,Z,x,f)&:=_0 \begin{cases}1&\text{if }\ f(x,Z11)\neq_00 \\ 0  &\text{else}\end{cases}
      \text{,}\\
  t_Y(Z,x,f)&:=_0 \begin{cases}1&\text{if }\ f(x,Z11)\neq_00 \\ Z11&\text{else}\end{cases}
\text{.}
\end{align*}
However, we can see that this interpretation is unnecessarily blown up (e.g. the
parameter $n$ corresponding to $Z(t_YZxf)(t_DZxf)$ in the term $t_D$
can be completely ignored). Therefore, we give
a more readable, but still intuitionistically equivalent, form:
\[
 \forall Z^1,f^{10},x^0\ \Big( f(x,t_YZxf)=_00\ \vee\ f\big(x,Z(t_YZxf)\big)\neq_00 \Big)
 \text{,}
\] where \[ 
t_Y(Z,x,f):=_0 \begin{cases}1&\text{if }\ f(x,Z1)\neq_00 \\ Z1&\text{else}\end{cases}
\text{.}\]

%
\begin{rmk}From now on we will always simply prefer better readability
 whenever possible.\end{rmk}
Looking closely at the case distinction of $f(x,Z1)\neq_00$ being true or not, 
it is easy to see the correctness of the interpretation.
%
% the DNS
%
%%%%%%%%%%%%%%%%%%%%
\subsection{Double Negation Shift} \label{ss:DNS}
The Double Negation Shift
schema over numbers for $\Sigma^0_2$ formulas is defined as follows:
\[
\defkeyn{\text{$\Sigma^0_2\usftext{-}\DNS(f)$}}\quad:\quad
 \forall x^0\neg\neg\exists y^0\forall z^0\ f(x,y,z)=_00\ \ \rightarrow\ \ 
     \neg\neg\forall x^0\exists y^0\forall z^0\ f(x,y,z)=_00  
\text{.}\]
%The Double Negation Shift is intuitionistically a very strong principle. In fact, apart from
%intuitionistically very weak axioms, it suffices to prove 
This schema is interpreted under the functional interpretation using Bar Recursion essentially in the same way as full $\DNS$ as presented by Spector in \cite{Spector62}:
\[ \forall A^{2(0)},T^2,W^2\ \Big(  f\big( t_x^0, At_xt_B^1, t_B(At_xt_B)\big)=_00  
     \rightarrow f\big(Tt_U^1,t_U(Tt_U),Wt_U\big)=_00\Big)  
\]
\[  t_x:=_0 Tx_0^1\text{,}\ \ \ \ 
      t_U:=_1 x_0\text{,}\ \ \ \  
      t_B=_1 \lambda D^0.W^2(E^{1(0)}_{Tx_0}D)
  \text{,}\]
where
\begin{align*}
     x_0 &{:=_1}  \bPhi_0Tu^{1(1(0))}0^00^1 \\
     E_{n^0} &{:=_{1(0)}} \lambda D^0.\bPhi_0Tun'(\overline{x_0,n}*D) \\
     un^0v^{1(0)} &{:=_0} An\left(\lambda D^0.W(v(D))\right) \text{.}
\end{align*}
\begin{rmk}The detailed steps of the interpretation show
 that any term can depend on all three variables $A$, $T$ and $W$. 
However, for better readability we present the result in its simpler form. E.g.
by $t_x$ we mean in fact $t_xATW$. To be absolutely correct
we would have to write: $t_x:=_{((0(20))(2))(2)} \lambda A,T,W. Tx_0^{(2)2}AW$.\end{rmk}
The correctness proof as well as some more details can be 
found e.g. in \cite{Kohlenbach08}.
%\todo{Insert some steps, possibly proof?}
%
%
% the pi-01 AC
%
%
%
\subsection{Axiom of Choice} \label{ss:AC}
The very well known Axiom of Choice is used in a wide variety of forms
in Proof Theory. We follow Troelstra (see \cite{Troelstra73}) and define:
\begin{dfn}{Axiom of Choice\\}
\begin{align*}
\AC^{\rho,\tau}\quad&:\quad \forall x^\rho\exists y^\tau 
  \phi(x,y)\rightarrow\exists Y^{\tau(\rho)}\forall x^\rho\phi(x,Zx)\\
\AC\quad&:\quad \bigcup_{\rho,\tau\in\Tp}\AC^{\rho,\tau}\\
\QFm\AC^{\rho,\tau}\quad&:\quad \forall x^\rho\exists y^\tau 
  \phi_{{}_\QF}(x,y)\rightarrow\exists Y^{\tau(\rho)}\forall x^\rho\phi_{{}_\QF}(x,Zx)\\
\QFm\AC\quad&:\quad \bigcup_{\rho,\tau\in\Tp}\QFm\AC^{\rho,\tau}\text{,}
\end{align*}
where $\phi_{{}_\QF}$ is quantifier-free.
\end{dfn}
Moreover, we define an arithmetical version for universal formulas:
\begin{dfn}
Axiom of Choice over numbers for $\PiL$ formulas, \defkey{$\PiLm\AC(f)$}:
\[ \forall x^0\exists y^0\forall z^0\ 
      f(x,y,z)=_00\ \ \rightarrow\ \ \exists g^1\forall x^0, z^0\ f(x,gx,z)=_00)  
\text{.}\]
\end{dfn}
This axiom is interpreted under the functional interpretation as:
  \[ \forall x^0,z^0,Y^1\ \Big(f\big(t_XYxz,Y(t_XYxz),t_ZYxz\big)=_00\rightarrow f\big(x,(t_GY)x,z\big)=_00\Big)  \text{,}
  \]
  \[  t_X(Y^1,x^0,z^0):=_0 x\text{,}\ \ \ \
      t_Z(Y^1,x^0,z^0):=_0 z\text{,}\ \ \ \  
      t_G(Y^1):=_1 Y
  \text{.}\]
We can see the correctness immediately. For further use we will need also the 
functional interpretation of the double negation of {$\PiLm\AC$}, {$\neg\neg\PiLm\AC$}, 
which is also straightforward.\\
$\neg\neg\PiLm\AC(f)$:
\[ \neg\neg\big(\forall x^0\exists y^0\forall z^0\ f(x,y,z)=_00\ \ \rightarrow\ \ 
        \exists g^1\forall u^0, v^0\ f(u,gu,v)=_00)\big)
\text{.}\]
This double negation is interpreted under the functional interpretation as:
  \begin{align*}
    \forall Y^1,U,V\ &{\Big(}\ \ f\big(t_XYUV,Y(t_XYUV),t_ZYUV\big)=_00 \\
    {}               &{}\ \ \ \ \ \ \rightarrow f\big(U(t_GY),(t_GY)(U(t_GY)),V(T_GY)\big)=_00\ \ \Big)
     \text{,}
  \end{align*} where
  \begin{align*}
    t_X(Y,U,V)&:=_0 UY \text{,}\\
      t_Z(Y,U,V)&:=_0 VY  \text{,}\\  
      t_G(Y)&:=_1 Y
  \text{.}
  \end{align*}
To prove \defkey{$(\PiLm\AC(f))'$},
\[ 
\forall x^0\neg\neg  \exists y^0\forall z^0\ f(x,y,z)=_00\ \ \rightarrow\ \ 
\neg\neg\exists g^1\forall x^0, z^0\ f(x,gx,z)=_00 
\text{,}\]
we need only $\DNS$ and $\neg\neg(\PiLm\AC(f))$.
The double negation $\neg\neg(\PiLm\AC(f))$ is intuitionistically equivalent to:
\[ 
\neg\neg \forall x^0\exists y^0\forall z^0\ f(x,y,z)=_00\ \ \rightarrow\ \ 
\neg\neg\exists g^1\forall x^0, z^0\ f(x,gx,z)=_00 
\tag{+} 
\text{.}\]
Using $(+)$ we can prove $(\PiLm\AC(f))'$ by a single use of the rule
\[
\begin{prooftree}
\phi\rightarrow \psi \quad \psi\rightarrow \chi
\justifies
\phi\rightarrow \chi
\end{prooftree}
\text{,}\]
since by setting
\begin{align*}
\phi&:\equiv \forall x^0\neg\neg\exists y^0\forall z^0\ f(x,y,z)=_00\text{,}\\
\psi&:\equiv \neg\neg \forall x^0\exists y^0\forall z^0\ f(x,y,z)=_00\text{,}\\ 
\chi&:\equiv \neg\neg\exists g^1\forall x^0, z^0\ f(x,gx,z)=_00 \text{,}
\end{align*}
we get
\[
\begin{prooftree}
\Sigma^0_2\usftext{-}\DNS(f) \quad \neg\neg(\PiLm\AC(f))
\justifies
(\PiLm\AC(f))'
\end{prooftree}
\text{.}\]
To find the interpreting terms we follow the term construction as given e.g. 
in \cite{Kohlenbach08}.
We will use the interpretations given above (see section \ref{ss:DNS} for the
interpretation of $\DNS$) and solve the following equations:
\begin{align*}
  t_XYUV    &{=_0} Tt_U \\
  Y(t_XYUV) &{=_0} t_U(Tt_U) \\
  t_ZYUV    &{=_0} Wt_U \text{.}
\end{align*}
Now, easily by setting $Y:=t_U$ we get $T=U$ and $V=W$, which leads
us to precisely the same interpretation
as the one of $\DNS$:\\
\[
\forall A^{2(0)}, T^2, W^2\ \Big( 
  f\big(t_x,At_xt_B,t_B(At_xt_B)\big)=_00 \rightarrow f\big(Tt_U,t_U(Tt_U),Wt_U\big)=_00 \Big)
\text{,}\]
where $t_x$, $t_B$ and $t_U$ are as in section \ref{ss:DNS}.
\begin{rmk}We could conclude this already from the fact that apart from $\DNS$ the
only used principle was a double negation of intuitionistically very weak and trivially
interpretable axiom $\PiLm\AC$.\end{rmk}
%
%   CA 
%
%%%%%%%%%%%%%%%%%%%
\subsection {Arithmetical Comprehension}\label{ss:CA}
{\samepage
The \defkey{Schema of Comprehension} is known in several forms.
% Troelstra in \cite{Troelstra73} gives three variants in a very general form, using second order language. 
We give following four in $\mathcal{L}(\weha)$.
\begin{dfn}%{\em Arithmetical comprehension\\}
\label{d:CA}
The schema:
\[
\exists f^{0(\tup\tau)}\forall\tup x^{\tup \tau}\ \big(\ \phi(\tup x) \leftrightarrow f\tup x=_00\ \big)
\]
is called
\begin{itemize}
\item \defkey{Full Comprehension},
abbreviation: $\CA$; iff there are no restrictions on the formula $\phi$.
\item \defkey{Arithmetical Comprehension}, abbreviations: 
 $\Pi^0_\infty\usftext{-CA}$,  $\usftext{ACA}$ or $\CA_{ar}$; iff all quantifiers in $\phi$ 
range over numbers only. Equivalently, we may say 
$\phi$ is an arithmetical formula or $\phi$ is a $\Pi^0_\infty$ formula. 
\item (Arithmetical) \defkey{Comprehension over numbers}, abbreviations:
 $(\Pi^0_\infty\usftext{-})\CA^0$,  $(\usftext{A})\CA^0$ or $\CA^0_{(ar)}$; 
iff $x$ is a single number, i.e. $\tup\tau=(0)$. 
\item {\em Arithmetical Comprehension over numbers for purely universal formulas}, 
\defkey{$\PiLm\CA$}, iff
$\phi$ is a $\PiL$ formula and $x$ is a number. Using the fact that 
in all systems we use, we have characteristic functions for 
quantifier-free formulas, see e.g. \cite{Kohlenbach08}, we obtain the following 
equivalent formulation:
\[
\PiLm\CA\quad:\quad\forall f^{0(00)}\underbrace {\exists g^1\forall x^0
  \Big(\ \big(\forall y^0\ f(x,y)=_00\big) \leftrightarrow g x=_00\ \Big)}_{\equiv:\PiLm\CA(f^{0(00)})}\text{.}
\]
\end{itemize}
\end{dfn}
} %end samepage
\begin{rmk} Since we can derive any instance of $\CA^0_{ar}$ in $\wepa$
by iterated application of $\PiLm\CA$, the last two versions are
equivalent. However, this is only the case for full $\PiLm\CA$ and not
for concrete instances $\PiLm\CA(f)$.\end{rmk}
The schema of arithmetical comprehension is one of the core schemas we need for the
interpretation of proofs based on sequential compactness. It is also
considered to be essential for formalizing large
parts of classical analysis, see \cite{Simpson99}. 
For our purposes the suitable instances of the schema of comprehension over
numbers for $\PiL$ (or equivalently for $\SiL$) formulas are fully sufficient.
As in the case of $\PiLm\AC$ the whole complexity
of the interpretation is given by $\Sigma^0_2\m\DNS$ with a primitive recursive case distinction
from $\LEM$.\\
The even in $\epa$ stronger, but using $\QF\m\AC^{0,0}$ already in $\hrrweha$ to $\PiLm\CA$ equivalent, form called 
modified schema of comprehension over numbers for $\PiL$ formulas,
\[
\PiLm\CAhut(f)\quad:\quad\exists g^1 \forall x^0,z^0 \big(fx(gx)=_00 \vee fxz\neq_00\big)\text{,}
\]
is very practical for finding the functional interpretation as well as for finding
the functional interpretation of the negative translation.
\begin{rmk}
To see the fact that $\PiLm\CA$ is weaker than $\PiLm\CAhut$ observe that
$\PiLm\CAhut$ proves the existence of functions growing faster than any function in $\T$, which
$\PiLm\CA$ does not, as the function $f$ in $\PiLm\CA$ is always majorized by $\one^1$.\\
On the other hand, for suitable primitive recursive functional $F$, we can prove $\PiLm\CA(Ff)$ implies $\PiLm\CAhut(f)$ in $\hrrweha\ +\ \QF\m\AC^{0,0}$
in a similar way as we prove $(\PiLm\CA(f))'$ below.
First, we apply $\PiLm\CA$ to the intuitionistically valid law $\LEM(f)$:
\[
\forall x\exists y \underbrace{
                        \forall z\  \underbrace {\big(f(x,y)=_00 \vee f(x,z)\neq_00\big)}_{
                                                                                :\equiv F(f,x,y,z)=_00\ \text{for a suitable prim. rek. F} } }_{
                                                        \text{apply}\ \PiLm\CA(Ff) }
\text{,}
\]
obtaining a $g^{0(00)}$ s. t. $\forall x^0\exists y^0\ (gxy=_00\ \leftrightarrow\ \forall z^0\ Ffxyz=_00)$.
By $\QF\m\AC^{0,0}$ we get a function $h^1$ such that it holds: $\forall x\ g(x,hx)=_00$, so finally we obtain $\PiLm\CAhut(f)$.\\
On the other hand, given $\PiLm\CAhut(f)$ we can define a new $g^1$ as $\lambda x^0.f^{0(00)}(x,g^1x)$ obtaining $\PiLm\CA(f)$.
\end{rmk}
The negative translation of the modified schema of comprehension 
over numbers for $\PiL$ formulas,
\[
(\PiLm\CAhut(f))'\quad:\quad
 \neg\neg \exists g^1 \forall x^0,z^0 \big(fx(gx)=_00 \vee fxz\neq_00\big)\text{,}
\]
follows directly by applying $(\PiLm\AC(G))'$ to $(\LEM(f))'$ for a suitable $\phi$.\\
By defining
\[
G_f(x,y,z):\equiv
  \begin{cases}
    0&\text{if}\ \ f(x,z)\neq_00\\
    f(x,y)&\text{else}
  \end{cases}
\text{,}\]
$(\LEM(f))'$ becomes:
\[ \forall x^0 \neg\neg\exists y^0 \forall z^0 G_fxyz=_00 \text{.}\]
Applying $(\PiLm\AC(f))'$ is actually  a single use of a primitive rule, namely the modus ponens instance
\[
\begin{prooftree}
\phi \quad \phi\rightarrow \psi
\justifies
\psi
\end{prooftree}
\text{,}\]
since setting $\phi$ to $(\LEM(f))'$ allows us to set $\phi\rightarrow\psi$ to 
$(\PiLm\AC(G_f))'$, what leads us to the conclusion $\psi$:
\[ \neg\neg\exists g^1\forall x^0,z^0 G_fx(gx)z=_00 \text{,}\] 
what is the same, unwinding $G_f$, as $(\PiLm\CAhut(f))'$.\\
%
% FI of CA
%
To find the functional interpretation of $(\PiLm\CAhut(f))'$ first consider
the functional interpretation of $(\PiLm\AC(G_f))'$. We can even use a
partial interpretation (producing a weaker statement), since the assumption 
$\phi$ in the modus ponens is proved directly 
(satisfying the weaker statement). We obtain following representation (for $f$ free):
\begin{align*}
\forall A_f^{2(0)}, T, W\ \Big(\ \forall x^0,B^1\ G_{f}\big(x&,A_fxB,B(A_fxB)\big)=_00\rightarrow  \\
                                   & G_{f}\big(Tt_U,t_U(Tt_U),Wt_U\big)=_00\ \Big)
\text{.}\end{align*} 
\begin{rmk}This approach would be very unwise if the functional interpretation of the
implication was used to compute some further results within a larger proof.
Using this approach in such a case results in loss of information about the 
logical relation between the assumption and conclusion.\end{rmk}
From the interpretation of $\LEM'$ we know that (for $f$ free and knowing
that any term can depend on $T$ and $W$)
\[ \forall x,B\ \ G_{f}\big(x,A_fxB,B(A_fxB)\big)=_00)\]
 is satisfied by setting (for $f$ free)
\[
   A_fxB:=    
    \begin{cases}
      1&\text{if}\ \ f(x,B1)\neq_00\\
      B1&\text{else}
    \end{cases}
\text{.}\]
Obviously, the functional interpretation of $\forall f\ (\PiLm\CAhut(f))'$ just 
swaps the $\forall f$ and the exists quantification of the functional interpretation of
$(\PiLm\CAhut(f))'$ adding $f$ as a parameter to every existential variable
(see also section \ref{ss:LEM}). In particular $A$ becomes dependent on
$f$. We can actually define $t_A$, the corresponding primitive recursive
term to $A(f,x,B)$ as:
\[
   t_A:=\lambda f.\lambda x.\lambda B.
    \begin{cases}
      1&\text{if}\ \ f(x,B1)\neq_00\\
      B1&\text{else}
    \end{cases}
\text{,}\]
what satisfies
\[
   t_Af=A_f
\text{.}\]
It is important to mention the dependency of $A$ on $f$, since this is the main 
difference to the ND-interpretation of $\AC$, where the realizing terms 
were completely independent on this function. % $f$.
This means, $(\PiLm\CAhut(f))'$ is interpreted (keeping in mind that,
as before, $t_U$ always has $T$ and $W$ as parameters, and now, via $A_f$ also $f$) by:
\[
  \forall T,W\ \ \Big(G_{f}\big(Tt_U,t_U(Tt_U),Wt_U\big)=_00\Big)
\text{,}\]
which is
\[
  \forall T^2,W^2\ \ \Big(f\big(Tt_U,t_U(Tt_U)\big)=_00 \vee f(Tt_U,Wt_U)\neq_00\Big)\text{,}
\]
where $t_U:=_1 x_0$ is the same functional as in the interpretation of $\DNS$ in 
section \ref{ss:DNS}, in which $A$ is defined as above and $T$ and $W$ correspond
to the $T$ and $W$ above. We summarize:
\begin{prop}[\em{ND-interpretation of \PiLm\CAhut(f)}]\label{p:NDPiLCAhut}
The schema of arithmetical comprehension over numbers for purely universal
formulas (for a given function $f^{1(0)}$)
\[
\exists g^1 \forall x^0,z^0\ \big(fx(gx)=_00 \vee fxz\neq_00\big)\text{,}
\]
is ND-interpreted as follows:
\[
  \forall X^2,Z^2\ \ \Big(f\big(Xt_g,t_g(Xt_g)\big)=_00 \vee f(Xt_g,Zt_g)\neq_00\Big)\text{,}
\]
where
\begin{align*}
%t_g&:=_1x_0\\
%x_0&:=_1\bPhi_0Xu_f^{\!1(1(0))}0^00^1\\
%x_0&:=_1\xTWf{X}{Z}{f}\\
t_g&:=_1\xTWf{X}{Z}{f}\text{,}\\
u_{Z,f}n^0v^{1(0)}&:=_0\uWf{Z}{f}
\text{.}
\end{align*}
\end{prop}
\begin{rmk}Note that in contrast to the functional interpretation of
$\PiLm\AC$, where all realizing terms were independent on the function $f$, 
here the bar recursive term $t_g$ realizing the function $g$
has to be defined using $f$.\end{rmk}
Using the ND-interpretation of $\PiLm\CAhut(f)$ we obtain
also the ND-interpretations of  $\PiLm\CA$ and $\SiLm\CA$:
%
% cor PI01-CA
%
\begin{cor}[\em{ND-interpretation of \PiLm\CA}]\label{c:NDPiLCA}
The schema of arithmetical comprehension over numbers for purely universal
formulas 
\[
  \forall f^{1(0)}\exists h^1\leq_1\one^1 \forall x^0 (hx=_00 \leftrightarrow \forall z\ fxz\neq_00)\text{,}
\]
is ND-interpreted as follows:
\begin{align*}
  \forall X^{\typeOfXZ},Z^{\typeOfXZ}\ \big(\ (\ t_h&\args{XZ(X\args{(t_h\args{XZ})(t_z\args{XZ})})}=_00
        \rightarrow \\
        & \quad f(X \args{(t_h\args{XZ})(t_z\args{XZ})}, 
                  Z \args{(t_h\args{XZ})(t_z\args{XZ})} ) \neq_00)\ )\ \wedge\\
 (\ t_h&\args{XZ(X\args{(t_h\args{XZ})(t_z\args{XZ})})}=_00 
         \leftarrow \\
     &\quad f (
           X\args{(t_h\args{XZ})(t_z\args{XZ})}, 
           t_{z}\args{XZ (X\args{(t_h\args{XZ})(t_z\args{XZ})})
                      (Z\args{(t_h\args{XZ})(t_z\args{XZ})}) } 
      ) \neq_00\ )\ \big)
\text{.}
\end{align*}
The witnessing terms are
\begin{align*}
t_z&:=\lambda X^{\typeOfXZ},Z^{\typeOfXZ},a^0,b^0\ .\ t_g\args{(t_f\args{X})(t_f\args{Z})a}\text{,}\\
t_h&:=\lambda X^{\typeOfXZ},Z^{\typeOfXZ},n^0\ .\ 
 \overleftrightarrow{f(n,t_g\args{(t_f\args X)(t_f\args Z)n})}
\text{,}
\end{align*}
where 
\begin{align*}
\overleftrightarrow{n^0}&:=_0\begin{cases}1&\Tif\quad n=_00\\0&\Telse\end{cases}\text{,}\\
t_f&:=\lambda X^{\typeOfXZ},g^1\ .\ 
  X \args{ (\lambda n^0.\overleftrightarrow{f(n,g\args{n})})(\lambda a^0,b^0 . g\args{a}) }
\text{.}
\end{align*}
The remaining terms are defined as above in proposition \ref{p:NDPiLCAhut}, just
we give the two type $2$ arguments of $t_g$ explicitly, i.e.,
$t_g$ stands only for the term $t_g$ of type level $3$ and not for 
the type $1$ term $t_gXZ$ as above.
\end{cor}
%
% rmk
%
\begin{rmk}
We use a slightly different form of $\PiLm\CA$ than in definition~\ref{d:CA} above in two aspects:
\begin{enumerate}
\item Since we can give a realizer for $h^1$ bounded by $\one^1$
we formulate the sentence directly in the stronger version with $h^1\leq_1\one^1$. We do so also in
corollary~\ref{c:NDSiLCAnonsh}.
\item In contrast to the common formulations we want $h$ to comprehend $\forall z\ fxz\neq 0$ instead of
$\forall z\ fxz= 0$. Both formulations are equivalent, though the one used in corollary~\ref{c:NDPiLCA}
spares us some technical overhead and so improves the readability of the proof below.
\end{enumerate}
Moreover, we don't strictly follow the rules of the ND-interpretation and choose the "$\forall\exists$"
prenexiation over "$\exists\forall$" to start with. Again, this leads to an equivalent statement resulting in simpler
and more readable terms of lower type.
\end{rmk}

%
% proof PI01-CA
%
\begin{proof}[ of corollary~\ref{c:NDPiLCA}]
Rewriting proposition \ref{p:NDPiLCAhut} using the formally correct notation
of corollary \ref{c:NDPiLCA} we have that:
\[
  \forall X^2,Z^2\ \ (\ 
      f(X\args{(t_g\args{XZ})}, t_g\args{XZ (X \args{(t_gXZ)})})=_00 \vee 
      f(X\args{(t_g\args{XZ})}, Z\args{(t_g\args{XZ})} )\neq_00\ )
\text{.}\tag{+}
\]
Given any $X^{\typeOfXZ}_0$, $Z^{\typeOfXZ}_0$ set $X^2:=t_f\args{X_0}$ and $Z^2:=t_f\args{Z_0}$ to
obtain:
\begin{align*}
X\args{(t_g\args{XZ})}&= t_f\args{X_0(t_g\args{XZ})}\\
        &= X_0 \args 
         {(\lambda n . \overleftrightarrow{f(n,t_g\args{XZ}n)})
          (\lambda a,b . t_g\args{XZ}a)} \\
        &= X_0 \args
         {(\lambda n . \overleftrightarrow{f(n,t_g\args{(t_f\args{X_0})(t_f\args{Z_0})}n)}) 
          (\lambda a,b . t_g\args{(t_f\args{X_0})(t_f\args{Z_0})}a)} \\
        &= X_0\args{(t_h\args{X_0Z_0})(t_z\args{X_0Z_0})} \tag{*}
\end{align*}
and analogously $Z(t_g\args{XZ})=Z_0(t_h\args{X_0Z_0})(t_z\args{X_0Z_0})$. 
\begin{itemize}
\item Suppose we have $t_h\args{X_0Z_0(X_0\args{ (t_h\args{X_0Z_0})(t_z\args{X_0Z_0}) })}=_00$.\\
It follows by (*) that $t_h\args{X_0Z_0(X\args{(t_g\args{XZ})})}=_00$ and by 
definition of $t_h$ that
\[
f( X\args{(t_g\args{XZ})}, 
   t_g\args{ (t_f\args{X_0}) (t_f\args{Z_0}) (X\args{(t_g\args{XZ})}) } ) \neq 0
\text{.}\]
%% old notation...
By definition of $X$ and $Z$ we get $ f(X(t_g\args{XZ}),t_g\args{XZ}(X(t_g\args{XZ})))\neq 0$ which
implies 
\[f(X(t_g\args{XZ}),Z(t_g\args{XZ}))\neq0\] by (+) and 
\[
f(X_0(t_h\args{X_0Z_0})(t_z\args{X_0Z_0}),Z_0(t_h\args{X_0Z_0})(t_z\args{X_0Z_0}))\neq 0
\] by (*).
\item On the other hand, let 
\[
f( X_0(t_h\args{X_0Z_0})(t_z\args{X_0Z_0}) , t_z\args{X_0Z_0}(X_0(t_h\args{X_0Z_0})(t_z\args{X_0Z_0}))(Z_0(t_h\args{X_0Z_0})(t_z\args{X_0Z_0}))) \neq 0 \text{.}
\]
Unwinding $t_z$ we get
\[
 f( X_0\args{(t_h\args{X_0Z_0})(t_z\args{X_0Z_0})}  , 
    t_g\args{ (t_f\args{X_0}) (t_f\args{Z_0}) (X_0\args{ (t_h\args{X_0Z_0}) (t_z\args{X_0Z_0}) }) } )
         \neq 0
\text{.}
\]
Hence, we obtain $t_h\args{X_0Z_0}(X_0(t_h\args{X_0Z_0})(t_z\args{X_0Z_0})) = 0$.
\end{itemize}
\end{proof}

%
% cor SI01-CA
%
The schema $(\SiLm\CA)'$ is D-interpreted in a similar way. % as $(\PiLm\CA)'$. 
%
\begin{cor}[\em{ND-interpretation of \SiLm\CA(f)}]\label{c:NDSiLCAnonsh}
  The schema of arithmetical comprehension over numbers for purely existential
  formulas (for a given function $f^{1(0)}$)
  \[
  \exists h^1\leq_1\one^1 \forall x^0 (hx=_00 \leftrightarrow \exists z\ fxz=_00)\text{,}
  \]
is ND-interpreted as follows:
\begin{align*}
  \forall X^{\typeOfXZ},Z^{\typeOfXZ}\ \ (\ t_h&\args{XZ(X\args{(t_h\args{XZ})(t_z\args{XZ})})}=_00
        \leftarrow \\
        & \quad f(X \args{(t_h\args{XZ})(t_z\args{XZ})}, 
                  Z \args{(t_h\args{XZ})(t_z\args{XZ})} ) \neq_00)\ \ \wedge\\
 t_h&\args{XZ(X\args{(t_h\args{XZ})(t_z\args{XZ})})}=_00 
         \rightarrow \\
     &\quad f (
           X\args{(t_h\args{XZ})(t_z\args{XZ})}, 
           t_{z}\args{XZ (X\args{(t_h\args{XZ})(t_z\args{XZ})})
                      (Z\args{(t_h\args{XZ})(t_z\args{XZ})}) } 
      ) \neq_00\ )
\text{.}
\end{align*}
As above in corollary \ref{c:NDPiLCA}, the witnessing terms are:
\begin{align*}
t_z&:=\lambda X^{\typeOfXZ},Z^{\typeOfXZ},a^0,b^0\ .\ t_g\args{(t_f\args{X})(t_f\args{Z})a}\text{,}\\
t_h&:=\lambda X^{\typeOfXZ},Z^{\typeOfXZ},n^0\ .\ 
 \overleftrightarrow{f(n,t_g\args{(t_f\args X)(t_f\args Z)n})}
\text{.}
\end{align*}
The term $t_g$ corresponds to the term
defined in proposition \ref{p:NDPiLCAhut}. The only difference is that 
we give the two type $2$ arguments of $t_g$ explicitly, i.e.,
$t_g$ stands only for the term $t_g$ of type level $3$ and not for 
the type $1$ term $t_gXZ$ as above.
We redefine $\overleftrightarrow{n^0}$ as:
\[
\overleftrightarrow{n^0}:=_0\begin{cases}0&\Tif\quad n=_00\\1&\Telse\end{cases}\\
\text{,}
\]
what slightly changes the meaning of $t_f$ (and $t_h$) compared to 
corollary \ref{c:NDPiLCA}. However, we keep the exact syntactic
form for $t_f$ as before:
\begin{align*}
t_f&:=\lambda X^{\typeOfXZ},g^1\ .\ 
  X \args{ (\lambda n^0.\overleftrightarrow{f(n,g\args{n})})(\lambda a^0,b^0 . g\args{a}) }
\text{.}
\end{align*}
\end{cor}
%
% proof SI01-CA
%
\begin{proof}
The proof is essentially the same as the proof of \ref{c:NDPiLCA}. 
By (+) and (*) we refer to the statements (+) and (*) in that proof. The
notation and definitions of $X$ and $Z$ remain 
syntactically the same as above as well. 
\begin{itemize}
\item Suppose we have $t_h\args{X_0Z_0}(X_0(t_h\args{X_0Z_0})(t_z\args{X_0Z_0}))=_00$. It 
follows by (*) that \[t_h\args{X_0Z_0}(X(t_g\args{XZ}))=_00\] and by definition of $t_h$ that
\[f(X(t_g\args{XZ}), t_g(t_f\args{X_0})(t_f\args{Z_0})(X(t_g\args{XZ})))= 0\text{.}\]
By (*) and definition of $X$ and $Z$ we get 
\begin{align*}
&f(X_0(t_h\args{X_0Z_0})(t_z\args{X_0Z_0}),t_g(t_f\args{X_0})(t_f\args{Z_0})(X_0(t_h\args{X_0Z_0})(t_z\args{X_0Z_0})))=\\
&f( X_0(t_h\args{X_0Z_0})(t_z\args{X_0Z_0}) , t_z\args{X_0Z_0}(X_0(t_h\args{X_0Z_0})(t_z\args{X_0Z_0}))(Z_0(t_h\args{X_0Z_0})(t_z\args{X_0Z_0}))) ) = 0
\text{.}
\end{align*}
\item On the other hand, let 
$f( X_0(t_h\args{X_0Z_0})(t_z\args{X_0Z_0}) , Z_0(t_h\args{X_0Z_0})(t_z\args{X_0Z_0}) ) = 0$.
By (*) it means $f( X(t_g\tup{U}) ,  Z(t_g\tup{U}) ) = 0$, what implies
$f(X(t_g\args{XZ}),t_g\args{XZ}(X(t_g\args{XZ}))) = 0$ by (+). Using (*) and
the definition of $X$, $Z$, and $t_h$ we obtain
\begin{multline*}
 f(X_0(t_h\args{X_0Z_0})(t_z\args{X_0Z_0})  , 
       t_g(t_f\args{X_0})(t_f\args{Z_0})(X_0(t_h\args{X_0Z_0})(t_z\args{X_0Z_0}))) =  \\
   t_h\args{X_0Z_0}(X_0(t_h\args{X_0Z_0})(t_z\args{X_0Z_0}))) = 0
\text{.}
\end{multline*}
\end{itemize}
\end{proof}
%
%
\begin{rmk}
We mentioned earlier that we will use $\M$ and the stability of 
atomic formulas modulo double negation
silently to obtain more readable results. In addition, in this case we 
use two modifications to obtain the ND-interpretation of $\SiLm\CA$. Each modification
leads to an intuitionistically equivalent formula:
\[
\begin{prooftree}
\[
\neg\neg\exists h^1 \forall x^0 (hx=_00 \leftrightarrow \neg\neg\exists z\ fxz=_00)
\justifies
\neg\neg\exists h^1 \forall x^0 (hx=_00 \leftrightarrow \neg\forall z\ \neg(fxz=_00))
\using{\text{Lemma \ref{l:exnegall}}}
\]
\justifies
\neg\neg\exists h^1 \forall x^0 (hx\neq_00 \leftrightarrow \forall z\ fxz\neq_00)
\using{\text{Definition \ref{d:weha}}}
\end{prooftree}
\text{.}
\]\\
The ND-interpretations of $\PiLm\CA$ and $\SiLm\CA$ could be simplified even more.
Namely, we have $\forall X^{\typeOfXZ},Z^{\typeOfXZ},a^0,b^0 t_zXZab=_0t_zXZa0$, i.e. the term $t_zXZ$ does not
depend on its second type $0$ argument. Hence, we could define the terms $t_z$ and $t_f$ as:
\begin{align*}
t_z&:=\lambda X^{2(1)},Z^{2(1)},a^0\ .\ t_g\args{(t_f\args{X})(t_f\args{Z})a}\\
t_f&:=\lambda X^{2(1)},g^1\ .\ X \args{ (\lambda n^0.\overleftrightarrow{f(n,g\args{n})})g }
\end{align*}
and reformulate the corollaries~\ref{c:NDPiLCA} and~\ref{c:NDSiLCAnonsh} equivalently for all 
$X^{2(1)}$ and $Z^{2(1)}$ using 
\[
 t_{z}\args{XZ(   X \args{ (t_h\args{XZ}) (t_z\args{XZ}) }   )}
\]
instead of $ t_{z}\args{XZ(X\args{(t_h\args{XZ})(t_z\args{XZ})}) (Z\args{(t_h\args{XZ})(t_z\args{XZ})})}$.
\end{rmk}

\subsubsection*{Summary}

The main result, which we will need also in later sections is the ND-interpretation of $\CA$ (Corollary~\ref{c:NDSiLCAnonsh}), we reformulate is here using the more compact Sh-interpretation:
\begin{thm}[$Sh$-interpretation of $\SiLm\CA$]\label{c:NDSiLCA}
  The schema of arithmetical comprehension over numbers for purely existential
  formulas (for a given function $f^{1(0)}$)
  \[
  \exists h^1\forall x^0 (hx=_00 \leftrightarrow \exists z\ fxz=_00)\text{,}
  \]
is $Sh$-interpreted as follows (using the clause (S7$^*$) for the 
conjunction hidden in `$\leftrightarrow$'):
\begin{align*}
  \forall X^{\typeOfXZ},Z^{\typeOfXZ}\ \ [(\ 
t_h&\args{XZ(X\args{(t_h\args{XZ})(t_z\args{XZ})})}=_00
        \leftarrow \\
        & \quad f(X \args{(t_h\args{XZ})(t_z\args{XZ})}, 
                  Z \args{(t_h\args{XZ})(t_z\args{XZ})} ) =_00)\ \ \wedge\\
 (t_h&\args{XZ(X\args{(t_h\args{XZ})(t_z\args{XZ})})}=_00 
         \rightarrow \\
     &\quad f (
           X\args{(t_h\args{XZ})(t_z\args{XZ})}, 
           t_{z}\args{XZ (X\args{(t_h\args{XZ})(t_z\args{XZ})})
                      (Z\args{(t_h\args{XZ})(t_z\args{XZ})}) } 
      ) =_00\ )]
\text{.}
\end{align*}
The witnessing terms are:
\begin{align*}
t_z&:=\lambda X^{\typeOfXZ},Z^{\typeOfXZ},a^0,b^0\ .\ t_g\args{(t_f\args{X})
(t_f\args{Z})a}\text{,}\\
t_h&:=\lambda X^{\typeOfXZ},Z^{\typeOfXZ},n^0\ .\ 
 \overleftrightarrow{f(n,t_g\args{(t_f\args X)(t_f\args Z)n})}
\text{,}
\end{align*}
where
$
t_f:=\lambda X^{\typeOfXZ},g^1\ .\ 
  X \args{ (\lambda n^0.\overleftrightarrow{f(n,g\args{n})})
(\lambda a^0,b^0 . g\args{a}) }
\text{ and }
\overleftrightarrow{n^0}:=_0\begin{cases}0&\Tif\quad n=_00\\1&\Telse\end{cases}
\text{.}
$\\
The term $t_g$ corresponds to the term
defined in proposition \ref{p:NDPiLCAhut}. The only difference is that 
we give the two type $2$ arguments of $t_g$ explicitly, i.e.,
$t_g$ stands only for the term $t_g$ of type level $3$ and not for 
the type $1$ term $t_gXZ$ as above.
\end{thm}


%
% WKL WKL WKL WKL
%%%%%%%%%%%%%%%%%%%%%%%%%
\subsection {Weak K\"onig's Lemma}\label{s:wkl}
The K\"onig's lemma as such is mostly known in the following form:
\theQuote{Every infinite, though finitary branching, tree has an infinite path.}
However, this version is fairly strong and it can be shown 
that it is equivalent to $\CA^0$ if we allow arbitrary formulas for the
decision of belonging to the tree (see e.g. \cite{Troelstra74}).\\
If the labels for branches are bounded
by a function $\alpha$ depending only on the current node {\em and}
the tree is defined by a quantifier-free (or purely universal) formula 
we get a significantly
weaker tool, which is intuitionistically equivalent to $\WKL$, where 
we restrict the lemma to binary trees only (i.e. $\alpha(\cdot)\leq\nolinebreak 1$).
Even so, namely for quantifier-free binary trees, the lemma becomes
equivalent to $\CA^0_{ar}$ if we ask for the left-most or some other concrete
infinite path instead of just some infinite path. This means we need
all three weakenings:
\begin{itemize}
\item The decision criteria, $\phi(x^0)$, for an initial segment
of a sequence of natural numbers, $x$,
 to belong to the tree defined by the characteristic function $f$ must be
a $\PiL$ formula (we can allow one for-all\nbd quantifier since it has no 
essential influence on the structure of the lemma - see definition
\ref{l:WKL-Feferman} and proposition \ref{p:eqWKLs}). If we allow 
arbitrary formulas, we get even full %??? $\AC$. 
$\CA^0$ (regardless of any bounds on the number of branches greater than $1$).
\item For each node, the labels of its branches must be bounded by a well defined type one
function depending only on the height of the node. Otherwise, we get $\PiLm\CA$ and
by iteration $\CA^0_{ar}$ even for quantifier-free decision criteria.
\item There can't be any additional (infinite) demands on the infinite path except for its
existence. 
\end{itemize}
  We consider the following definition:
% Definition \ref{l:WKL-Feferman} follows 
%\cite{AF98} and definition \ref{l:WKL-Kohlenbach} is according to \cite{Kohlenbach08}:
%
% blab la-WKL definition
%
\begin{dfn}[\defkeyn{$\WKL(\phi)$}] \label{l:WKL-Feferman}
For a given $\phi$ we define the following theorem. 
Every infinite binary tree given by the decision criteria $\phi$
 has an infinite path,
\[
\WKL(\phi)\quad:\quad  \BTree(\phi) \wedge \forall k \UnBounded(\phi,k) 
     \rightarrow \exists b\Big(\BFunc(b)\wedge\forall k\ \phi\big(\bar{b}(k)\big)\Big)\text{,}
\]where
\begin{align*}
\BFunc(b)&:\equiv\quad \forall n^0\big(b(n)=_00 \vee b(n)=_01\big)\text{,}\\
\BTree(\phi)&:\equiv\quad \forall s \big(\ 
   \phi(s) 
      \rightarrow
 \ s\in\{0,1\}^{<\omega}
   \wedge 
   \forall t\subseteq s\ \phi(t)\ \big)\text{,}\\
%\Bounded(\phi,k^0)&:\equiv\quad 
%  \forall s\in\{0,1\}^{k}\ \neg\phi(s)\text{,}
\UnBounded(\phi,k^0)&:\equiv\quad 
  \exists s\in\{0,1\}^{k}\ \phi(s)\text{.}
\end{align*}
Furthermore, we define the schema $\Pi_n^0\m\WKL$, as the union of $\WKL(\phi)$, where
$\phi$ is a $\Pi_n^0$ formula. Also, we write $\Pi_n^0\m\WKL(\phi)$ to indicate
that we mean the concrete instance $\WKL(\phi)$ and that $\phi$ is a $\Pi_n^0$ formula.  \\
\end{dfn}
%
%
Note that, for every fixed $n \in \NN$, we can always reformulate
the schema $\Pi^0_n\m\WKL$ as a single $2^{nd}$-order axiom. 
We will use this fact implicitly.
However, in the special case for quantifier-free $\phi$ 
we define explicitly:
\begin{dfn}[\defkey{$\WKL\equiv\forall f\WKL(f)$} see also~\cite{Troelstra74}] 
%\label{l:WKL-Kohlenbach}
\label{l:WKLdelta}
Every infinite binary tree, given by the characteristic function $f$,
 has an infinite path:
\begin{align*}
\WKL(f)\quad:\quad\BTree_K(f) \wedge \forall k\exists x \big(\lh(x)=_0k \wedge f(x)=_0&0\big) 
     \rightarrow\\
 \exists b&\leq_1\one^1\ \Big(\forall k\ f\big(\bar{b}(k)\big)=_00\Big)\text{,}
\end{align*}
where
\begin{align*}
\BTree_K(f^1)&:\equiv 
  \forall x,y \big(f(x*y)=_00\rightarrow fx=_00\big)\ \wedge\ 
  \forall x,n\big(f(x*\langle n\rangle)=_00\rightarrow n\leq_0 1\big)\text{.}
\end{align*}
\end{dfn}
We mentioned earlier that the schema $\PiLm\WKL$ is essentially equivalent
to $\WKL$. this is proved by the proposition~\ref{p:eqWKLs} below, using the fact 
that in all systems used in this thesis there is
a suitable $f\tp 1$ for any quantifier-free $\phi_{\QF}(\tup n)$ with only the type $0$ parameters $\tup n$ free 
s.t. $f(\tup n)=_00\Leftrightarrow\phi_{\QF}(\tup n)$ (and vice-versa).
\begin{prop}\label{p:eqWKLs}
\[\weha\proves\PiLm\WKL\leftrightarrow\WKL\]
\end{prop}
%
% proof
%
\begin{proof}[ (see also~\cite{Simpson99})]
\begin{itemize}
\item $\PiLm\WKL\rightarrow\WKL$:\\
If we have  $\WKL(\phi)$ for $\PiL$ formulas $\phi$, it surely holds for quantifier-free formulas
especially for $\phi:\equiv f(n^0)=_00$.
\item $\PiLm\WKL\leftarrow\WKL$:\\
Since $\phi\in\PiL$, $\phi(s)$ can be written as $\forall n^0\phi_0(s,f,n)$. 
Next we involve two tricks in definition of a quantifier-free version of $\phi$:
\[
\tilde\phi(s,f):\equiv 
  \forall n^0<_0 \lh(s),t\subseteq s(\phi_0(f,t,n)\wedge s(n)\leq_01)
\text{.}
\]
\begin{enumerate}
\item Since we get $\forall n^0<_0 \lh(s) \phi_0(f,t,n)$ for-all $s$ 
in the conclusion of $\WKL$, the number $n$ is in fact unbounded.
\item We guarantee $\BTree(\tilde\phi)$ as a free property for any $\tilde\phi$. 
The tree property by forcing 
\[\forall t\subseteq s\ \phi_0(f,t,n)\]
and the binary property by 
\[\forall n_0<_0 \lh(s),t\subseteq\nolinebreak s\  s(n)\leq_0\nolinebreak 1\text{.} \]
\end{enumerate}
Now, suppose
\[ \BTree(\phi) \wedge \forall k \UnBounded(\phi,k) \] 
i.e. we have (by $\forall k \UnBounded(\phi,k)$):
\[
  \forall k\ \exists s\in\{0,1\}^{k}\ \forall n\ \ \phi_0(f,n,s)\text{.}\tag{+}
\]
By $\BTree(\phi)$ and (+) we get:
\[
  \forall k \exists s\in\{0,1\}^{k} 
     (\forall n^0<_0 \lh(s),t\subseteq s(\phi_0(f,t,n)\wedge s(n)\leq_01))\text{,}
\]
i.e. we have
\[
  \forall k \exists s\in\{0,1\}^{k} \tilde\phi(f,s)\text{,}
\]
and classically also
\[
  \forall k \UnBounded(\tilde\phi,k)\text{.}
\]
By the steps above we showed so far (using trick (2) to get $\BTree(\tilde\phi)$):
\[ 
\BTree(\phi) \wedge \forall k \UnBounded(\phi,k) 
  \quad\rightarrow\quad \BTree(\tilde\phi) \wedge \forall k \UnBounded(\tilde\phi,k) 
\text{.}
\]
We can define in $\weha$ the primitive recursive term for the 
following type one function $g^1$:
\[
g(s):=_0\begin{cases}0^0&\Tif \tilde\phi(f,s)\\1^0&\Telse \end{cases}\text{.}
\]
To get equivalently:
\[ 
\BTree(f) \wedge \forall k \UnBounded(f,k) 
  \quad\rightarrow\quad \BTree_K(g) \wedge \forall k \UnBounded(g,k) 
\text{.}
\]
Applying $\WKL(g)$ to the conclusion we get:
\[
\exists b\leq_1\one^1\forall k\ g(\bar{b}(k))=_00
\]
what is, by definition of $g$, the same as
\[
\exists b\leq_1\one^1\forall k\ \tilde\phi(\bar{b}(k),f)
\text{,}
\]
what is, by definition of $\tilde\phi$, the same as
\[
\exists b^1\leq_1\one^1\forall k^0\ 
\forall n^0<_0 k\forall t\subseteq \bar{b}(k)\ \ \phi_0(f,t,n)
\text{.}
\]
This implies (using trick (1) from the definition of $\tilde\phi$):
\[
\exists b^1\leq_1\one^1\forall k^0\forall n^0\forall t\subseteq \bar{b}(k)
 \phi_0(f,t,n)
\text{.}
\]
Observing that $\forall k\forall t\!\subseteq\!\bar{b}(k)\ \psi(t)$ is equivalent to
$\forall k\psi(\bar{b}(k))$ for arbitrary formulas $\psi$ we finally obtain:
\[
\exists b^1\leq_1\one^1\forall k^0\ (\forall n^0\ \phi_0(f,\bar{b}(k),n))
\quad\equiv\quad \exists b^1\leq_1\one^1\forall k^0\ \phi(f,\bar{b}(k))
\text{.}
\]
\end{itemize}
\end{proof}
\begin{rmk}
Where the necessity of trick (1) in the definition of $\tilde\phi$
is easy to see, one might think that the trick (2) is not needed. This
is not true. The following example shows that in general
we even {\em do not} have $\BTree(\forall n\phi_0(f,s,n))\rightarrow
\BTree(\forall n^0<_0 \lh(s)(\phi_0(f,s,n)\wedge s(n)\leq_01)$.
For some given constant number $c^0\in\NN$, define
\[
\phi_0(f,s,n):\equiv\quad 
  s\in\{0,1\}^{\lh(s)}\ \wedge\ (s(0)=_00\ \vee\ s(0)>c+1+n-2*\lh(s))\text{.}
\]
While $\forall n\phi_0(f,s,n)$ defines the tree consisting exactly of
all binary sequences starting with $0$, the
formula $\forall n^0<_0 \lh(s)\phi_0(f,s,n)$ is true for all such sequences
but also for all binary sequences of length greater than $c$. In other words,
any sequence of length e.g. $c+1$ starting with $1$ is a counterexample
for $\BTree(\forall n^0<_0 \lh(s)(\phi_0(f,s,n)\wedge s(n)\leq_01)$.\\
\end{rmk}

%\subsection{Complexity}
Kohlenbach showed in his work (see \cite{Kohlenbach92}, \cite{Kohlenbach08}), 
that $\WKL$ is intuitionistically equivalent to a sentence in $\Delta$
as defined in theorem \ref{t:mfi}. We sketch this result in \ref{ss:delta}.
This equivalence actually suggests that the lemma will not have any significant
effect on the final complexity of the realizers of sentences whose proofs 
are based on $\WKL$ in common systems.\\
There are several ways to prove this fact. Howard gives a technical
argument using restricted Bar Recursion in \cite{Howard81} which we
follow in \ref{ss:HFI}. 
%Using Kohlenbach's MD-interpretation one quickly obtains
%recursive bounds on the realizers in $\weha$ without Bar Recursion involved. 
%We connect to this approach by majorizing the realizers obtained in \ref{ss:HFI}
%in section \ref{ss:majFI}.

%\todo{Describe how can MD-interpretation simplify the proof that $\WKL$
%does not add anything to $\T_n$.}
%
% is a Delta
%%%%%%%%%%%%%%

\subsubsection*{Kohlenbach's $\WKL'$ as a Sentence $\Delta$}\label{ss:delta}
First, observe that $\WKL$ is equivalent (in $\hrrweha$) to:
\begin{align*}
\WKL_{K1}\quad:\quad
  \forall f,g\ \Big( \BTree_K(f)\wedge\forall k\big(\lh(\bar gk)=_0k\wedge f(\bar gk)=_00\big)
    &\rightarrow \\
  \exists b\leq_1 \one^1\forall &k^0\big(f(\bar bk)=_00\big)\Big)\text{.}
\end{align*}
Next, we define for any type $1$ function $f^1$ the constructions $\hat f$ and $f_g$:
\begin{dfn}\label{d:hatAndG}
\begin{align*}
\hat fn&:=\begin{cases}
  fn &\Tif fn\neq0\ \vee\ 
       \big(\forall k,l(k*l=n\rightarrow fk=0)\wedge \forall i<\lh(n)\ (n_i\leq 1)\big)\\
  1^0 &\Telse \end{cases}\\
f_gn&:=\begin{cases}
  fn &\Tif\ f\big(g(\lh(n))\big)=0\wedge \lh\big(g(\lh(n))\big)=\lh(n)\\
  0^0 &\Telse \end{cases}
\text{.}
\end{align*}
\end{dfn}
\begin{rmk}
We defined already a construction $\hat o$ for a type $1$ object in~\ref{d:hatReal}. It
should be obvious which meaning should be assigned from the context. From now on,
we will use it mostly in the sense of~\ref{d:hatAndG}.
\end{rmk}
Now, we are able to define
\begin{dfn}
\[
\WKL'\quad:\quad \forall f^1,g^1\exists b\leq_1\one^1\forall k^0
  \left(\widehat{(\hat f)_g} (\bar bk)=_00\right)\text{.}
\]
\end{dfn}
To obtain:
\begin{prop}\label{p:wkls1}
The sentence $\WKL'$ is in $\Delta$ as defined in theorem \ref{t:mfi} and:
\[ \weha\proves\WKL\leftrightarrow\WKL' 
\text{.}
\]
\end{prop}
The complete proof of an even stronger result, namely of the equivalence under $\hrrweha$, can
be found in \cite{Kohlenbach08}. In that book, Kohlenbach also proves the following
lemma:
\begin{lemma} \label{l:hatg}
  \setcounter{equation}{0}
  \begin{align}
    \hrrweha\quad&\proves\quad \forall f\ \BTree_K(\hat{f})\\
    \hrrweha\quad&\proves\quad \forall f\ \big(\BTree_K(f)\rightarrow f=_1\hat f\big)\\
    \hrrweha\quad&\proves\quad \forall f,g\forall n\exists x\ \big(\lh(x)=n\wedge f_g(x)=0\big)\\
    \hrrweha\quad&\proves\quad \forall f,g\ 
          \Big(\forall n\big(\lh(gn)=n\wedge f(gn)=0\big)\rightarrow f_g=_1 f\Big)
  \end{align}
\end{lemma}
For the purpose of this thesis mainly 
$\WKL\leftarrow\WKL'$ is interesting:\\
\begin{proof}
Assume $\BTree_K(f)$ and $\forall k \exists n (\lh(n)=k \wedge fn=0))$. Then
\[
\forall k \exists n\leq\overline{\one^1}k (\lh(n)=k \wedge fn=0))\text{.}
\]
Define primitive recursive in $f$:
\[
gk:=\begin{cases}
\min n\leq\overline{\one^1}k\ (\lh(n)=k \wedge fn=0)&\text{if such an $n$ exists}\\
0^0&\Telse
\end{cases} \text{,}
\]
to obtain $\forall k(\lh(gk)=k\wedge f(gk)=0)$.\\
Using lemma \ref{l:hatg} and other Kohlenbach's results 
from \cite{Kohlenbach08} we get $f_g=_1f$ and $f=_1\hat f$.
This proves $\widehat{(\hat f)_g}=_1 f$.\\
Finally $\WKL'$ yields $\exists b\leq_1 \one^1\forall k^0(f(\bar bk)=0)$.\\
\end{proof}\\
Using the fixed construction for $g$ from the proof above, we can define a fixed
term $s$ of $\weha$, and formulate yet another version of $\WKL$:
\begin{dfn}{\defkeyn{$\WKL_s$}$\equiv\forall f\WKL_s(f)$}, where
\[
\WKL_s(f)\quad:\equiv\quad \exists b\leq_1\one^1\forall k^0
  \left(\widehat{(\hat f)_{s}} (\bar bk)=_00\right)\text{.}
\]
\end{dfn}
By the proof of proposition \ref{p:wkls1}, we still have:
\begin{prop}\label{p:wkls2}
 \[ \weha\proves\WKL_s\rightarrow\WKL
\text{.}
\]
\end{prop}

%
% FI WKL -- Howard
%
%%%%%%%%%%%%%%%%%%
\subsubsection*{Howard's D-interpretation of $\WKL$} \label{ss:HFI}
We cannot D-interpret the lemma directly. We could apply
the negative translation to the lemma and then go for the D-interpretation, but
it turns out that the proof is simpler when we go for the ND-interpretation
of a classically equivalent formula, the so called $\FAN$ principle.
In this section, let $f^1$ be a 
characteristic function of a tree, 
$x^0$ and $y^0$ encodings of finite $\{0,1\}$-sequences, and
$b^1$ a $\{0,1\}$-sequence (i.e. $b\in\{0,1\}^\omega$). 
%We call $x$ \defkey{secured} iff $x\not\in f:\equiv f(x)\neq0$ and write \defkeyn{$\Sec(x)$}. 
We define the \defkeyn{$\FAN$} principle as follows (note that the hat construction does not
affect the general meaning - see proposition~\ref{p:wkls1} - however, it does significantly simplify
the terms as it eliminates some quantifiers):
\begin{dfn}[$\FAN\equiv\forall f \FAN(f)$] We define the $\FAN$ principle for a
given function $f$ as
\[
\FAN(f)\quad:\equiv\quad\forall b \exists j\ \Sec(\bar b j) \ \rightarrow \ 
  \exists k \forall x\ \big(\lh(x)\geq k \rightarrow \Sec(x)\big)
\text{,}\]
where $\Sec$ means \defkey{secured}:
\[
 \Sec(x)\quad:\equiv\quad \hat f(x)\neq0 \text{.}
\]
\end{dfn}
In \cite{Howard81}, Howard showed that the D-interpretation of $\WKL$
(using the classically equivalent $\FAN$)
can be obtained using only the so called restricted Bar Recursion
(defined in the same publication, see also definition~\ref{d:rBR}). 
We will present this proof filling in some details 
(the proofs of lemma \ref{l:L1} and lemma \ref{l:L2} 
as well as the witness for the binary sequence $b$).\\
%In fact, if one is interested only in the complexity of the realizing functionals
%there is an even simpler way than the one given by Howard. 
%As $\rB$ can be primitive recursively
%defined in $\nu$, which is trivially majorizable, the realizers are 
%majorized simply by $A*\one$ and we do not need .\\
The secured property is conservative over
extensions, i.e. for any two finite binary sequences $x$ and $y$ we 
have:
\[
x\subseteq y \rightarrow\big(\Sec(x)\rightarrow \Sec(y)\big)\text{.} \tag{SC}
\]
Hence, the conclusion of $\FAN(f)$ is equivalent to a 
$\SiL$ formulation (as the for-all quantifier 
can be bounded):
\[
\forall b \exists j\ \Sec(\bar b j) \ \rightarrow \ \exists k 
\underbrace{\forall x\in\{0,1\}^k\ \Sec(x)}_{\text{quantifier-free}}
\text{.}\]
Moreover, in the presence of the Markov principle,
 the negative interpretation does not change the formula. 
Already in intuitionistic logic,
$\neg\neg (A\rightarrow B)$ is equivalent to $A\rightarrow\ \neg\neg B$.
So, the negative interpretation becomes
\[
\forall b\ \neg\neg\ \exists j\ \Sec(\bar b j) \ 
       \rightarrow \ \neg\neg\big(\exists k \forall\ x\in\{0,1\}^k\ \Sec(x)\big) 
\text{.}\]
By the Markov principle we get
\[
\forall b\ \exists j\ \neg\neg \Sec(\bar b j) \ 
       \rightarrow \ \exists k\ \neg\neg\forall\ x\in\{0,1\}^k\ \Sec(x)
\]
and because of the stability of quantifier-free formulas modulo double negation, 
we get just the very same sentence we started with. 
%In other words $\FAN(f)$ retains the same form under negative translation 
%(modulo intuitionistic equivalence).
In summary we have:
\begin{align*}
\wepa &\proves\ \WKL\leftrightarrow\FAN \text{,}\\
\weha &\proves\ (\WKL)'\leftrightarrow(\FAN)' \text{,}\\
\weha+\M &\proves\ \FAN\leftrightarrow(\FAN)' \text{,}\\
\weha+\M &\proves\ (\WKL)'\leftrightarrow\FAN \text{,}\\
\end{align*}
and therefore also:
\[
\weha+\M \proves\ (\WKL)^{ND}\leftrightarrow(\FAN)^D \text{.}\\
\]
So $\FAN$ is precisely the form of $\WKL$ we want to D-interpret.\\ 
By quantifier-free Axiom of Choice $\QFm\AC^{1,0}$ we get 
from $\forall b \exists j\ \Sec(\bar b j)$:
\[ \exists A \forall b \ \Sec\big(\bar b(Ab)\big) \text{.}\]
Now, define $K_A$ as follows:
\[
K_Ax:=\KA{A}{x}
\text{.}
\]
Further define:
\begin{align*}
\BSec(k,x) &{:\equiv}
\forall y\ \Big(\big(x\subseteq y \wedge \lh(x)+k=\lh(y)\big)\ \rightarrow\ \Sec(y)\Big) \text{,}\\
\BSecA(x) &{:\equiv} \BSec(K_Ax,x) \text{.}
\end{align*}
\begin{rmk} $\BSec(k,x)$ is the predicate for: Every finite extension of $x$ with the length $\lh(x)+k$ 
is secured.\end{rmk}
We will make use of the following two lemmas (in both cases we consider some functional $A$ 
for which $\forall b \ \Sec(\bar b(Ab))$ holds, moreover, w.l.o.g. it may be assumed that $A$
is computable since it was obtained by $\QF\m\AC^{1,0}$):
\begin{lemma}
\label{l:L1}
\[ A[x] < \lh(x) \rightarrow \BSecA(x) \]
\end{lemma}
\begin{proof}
We assume:
\[ A[x] < \lh(x) \tag{*} \text{.}\]
By definition of $K_A$ we get:
\[ K_Ax=0  \tag{**} \text{.}\] 
From $\forall b \ \Sec(\bar b(Ab))$ we get
\[ \Sec(\overline{[x]}(A[x]))\text{.}\]
By (*) we know that
\[
\overline{[x]}(A[x])\subseteq x
\] hence by (SC) it follows
\[ \Sec(x) \text{.}\]
Using (**) and the definitions of $\BSec$ and $\BSec_A$ we get:
\[ \Sec(x) \equiv \BSec(0,x) \equiv \BSecA(x) \text{.}\]
\end{proof}

\begin{lemma}
\label{l:L2}
\[ \BSecA(x*0) \wedge \BSecA(x*1) \rightarrow \BSecA(x) \]
\end{lemma}
\begin{proof}
Assume:
\[ \BSecA(x*0) \wedge \BSecA(x*1)  \tag{*} \text{.}\]
If $A[x]<\lh(x)$, then by lemma \ref{l:L1} we have $\BSecA(x)$. 
So, using the definition of $K_A$ we may assume w.l.o.g:
\[ K_Ax=1+max\{K_A(x*0),K_A(x*1)\} \text{.}\]
Obviously we have:
\[ K_Ax > K_A(x*0)\ \wedge\ K_Ax>K_A(x*1)\text{.}\]
So by (*) we know that for some number $m\leq\lh(x)+K_A(x)$ all extensions 
of $(x*0)$ and $(x*1)$ of the fixed length $m$ are secured. Therefore all extensions
of $x$ with fixed length $m$ are secured. However, this is 
expressed as $\BSec(K_Ax,x)\equiv \BSecA(x)$.\\
\end{proof}

Since $\BSecA(x)$ can be written as a quantifier-free formula for any given $x$, and
hence is primitive recursive in $x$, 
we can use primitive recursion on the contrapositive of lemma \ref{l:L2}:
\[ \neg \BSecA(x)\ \rightarrow \ \neg \big(\BSecA(x*0)\wedge \BSecA(x*1)\big) \text{,}\]
to obtain a function $g\leq\one$, s.t.
\[ \neg \BSecA(\emptyset) \rightarrow \forall j \neg \BSecA(\bar gj) \text{.}\]
However, we can prove the existence of a $j$ s.t. $\BSecA(\bar gj)$, 
using only restricted Bar Recursion, and thereby proving $\BSecA(\emptyset)$.
To do so, we follow another publication from Howard, namely \cite{Howard68},
where he proves an, by lemma~\ref{l:L1}, even stronger lemma 
(for the same kind of functional $A$ as above):
%
%  The big Lemma 3C L3C L3C L3C
%
%%%%%%%%%%%%%%%%%%%%%%%%%%%%%%%%%%%%
\begin{lemma}\label{l:3c}
By restricted bar recursion plus primitive recursion, 
we can define $\theta_A$, s.t. for 
all $b$ it holds:
\[\exists k\leq \theta_Ab \emptyset\ \ \ A[\bar b k]<k\text{.}\]
\end{lemma}
%
\begin{rmk}
In fact the functional $\theta_A$ is given simply by 
$\theta_Ab\emptyset:=\mu k.A[\bar bk]\leq k$. 
Since we have to work in a model justifying bar recursion
and, as discussed e.g. by Kohlenbach in~\cite{Kohlenbach08}, we can define the $\mu$-operator
via bar recursion this functional is well defined in all such models, e.g. $\Cont$ or $\Maj$. 
Howard's proof of lemma~\ref{l:3c} is analyzed in the first appendix of \cite{Safarik08}. It is
rather technical and not essential at this point.
\end{rmk}
%

Lemma \ref{l:3c} especially implies $\forall g^1\!\leq\!\one$ $\exists k^0$  $A[\bar gk]<k$, which
implies $\forall g^1\!\leq\!\one$ $\exists k^0$ $\BSecA(\bar gk)$ by lemma \ref{l:L1}. 
So, as mentioned above, using this result we obtain:
\[ \BSecA(\emptyset)\ \leftrightarrow \BSec(K_A\emptyset, \emptyset) \text{.}\]
This is, by definition, just another form of:
\[ \forall x\ \Big(\big(\emptyset \subseteq x \wedge K_A(\emptyset)=\lh(x)\big) \rightarrow \Sec(x)\Big) \text{.}\]
So, finally we obtain:
\[  \forall x\ \big(\lh(x)\geq K_A(\emptyset)\  \rightarrow \Sec(x)\big) \text{.}\]
What completes the proof that $K_A(\emptyset)$ is the ND-realizer
of $k$ in $\FAN(f)$.\\

To complete the solution of the D-interpretation we still have to 
find the functional $B$ which satisfies:
\[
\forall A,x\ \Big(\ 
\Sec\big(\overline{B}(AB)\big)\ \rightarrow\ 
    \big(\lh(x)\geq K_A(\emptyset)\rightarrow \Sec(x)\big)
\ \Big)
\text{.}\]
Fortunately, this is easily done. Let $y^0$ encode either a binary sequence 
of length $K_A(\emptyset)$ which is in the tree or, if such a sequence doesn't exist, the empty
sequence. Now we can define the functional $B$ as the infinite extension of this sequence, $[y]$.
See also the theorem~\ref{t:FIwkl} below.\\
The same functionals interpret the standard (positive) formulation of $\WKL$, 
which is just the contrapositive of 
the outer implication:
\[
\forall A,x\ \ (\ (\lh(x)\geq K_A(\emptyset)\wedge x\in f) \rightarrow\ 
 f(\overline{B}(AB))=0\ )
\text{.}\]
{\samepage
We summarize:
\begin{thm}[{The ND-interpretation of $\WKL(f)$}]\label{t:FIwkl-nd}
The Weak K\"onig's lemma for binary trees
given by an arithmetic characteristic function $f$ 
\[
  \BTree(f) \wedge \forall k \UnBounded(f,k) 
     \rightarrow \exists b\Big(\BFunc(b)\wedge\forall k\ f\big(\bar{b}(k)\big)=0\Big)
\]
is, provably in $\weha+\QF\m\AC^{1,0}+\rB_\one$, ND-interpreted as follows:
\[
\forall A,x\ \ \Big(\ \big(\lh(x)\geq K_A(\emptyset)\wedge x\in f\big) \rightarrow\ 
 f\big(\overline{B}(AB)\big)=0\ \Big)
\text{,}\]
where
\begin{align*}
K_Ax&:=\KA{A}{x} \text{,} \\
B(A)&:=\big[F_f\big(K_A(\emptyset)\big)\big] \text{,} \\
F_f(n)&:=\Fn{f}
\text{.}
\end{align*}
In particular, these terms define total functionals in $\Maj$ and $\Cont$.
\end{thm}
} %samepage
\subsubsection*{Majorants for Howard's Solution} \label{ss:majFI}
First, note that any term $t$ majorized by a term $t^*$ has at most the complexity
of $t^*$ since it can be obtained by bounded search up to $t^*$. As the complexity
of the realizers in theorem \ref{t:FIwkl-nd} is not obvious it might be interesting
to examine suitable majorants for these realizers:%\\
%Since $F$ is majorized simply by the constant function $\one^1:=_1\lambda n^0.1^0$,
%the interesting part is to find a suitable majorant for $K_A$. 
% Furthermore
%we have:
\begin{thm}
%For any length $l\in\NN$, for all $x^0\in\{0,1\}^l$:
%For all $x$:
The solution of the ND-interpretation 
of $\WKL(f)$ is, provably in $\weha+\QF\m\AC^{1,0}+\rB_\one$, majorized as follows:
\setcounter{equation}{0}
\begin{align}
K^*:=_{(10)2}\lambda A^2,x^0.A^2\one\ &\maj_{(10)2}\ K\\
B^*:=_{1(2)}\lambda A^2. \one^1 &\maj_{1(2)}\ B
\text{.}
\end{align}
Again, these terms define total functionals in $\Maj$ and $\Cont$.
\end{thm}
\begin{proof}
\paragraph{(2)}
Follows directly from the fact:
\[
\lambda n^0,x^0.\overline{\one^1}(\lh(x)+n) \maj_{1(1)} F\text{.}
\]
\paragraph{(1)}
According to definition \ref{d:maj}-(2) suppose $A^*$ majorizes $A$. 
%Then $A[x]$ is majorized by $A^*\one$. Since $A^*$ is 
%a well defined type $2$ functional, we get $\exists n^0\ A^*\one=_0n$.
%We denote this $n$ by $n_A$. 
We get immediately from the definition of $K_A$ and
$A^* \maj_2 A$
\[
\forall s\in\{0,1\}^{A^*\one}\forall x\subseteq s\ \ K_Ax\ \leq_0\ A^*\one-\lh(x)\text{.}
\]
In other words, for any finite binary sequence $x$ we have:
\[
\lh(x)\leq_0A^*\one \rightarrow K_Ax\leq_0A^*\one\text{.}
\]
Now, suppose $\lh(x)>A^*\one$ for some $x$. 
From $A^* \maj_2 A$ we get $\lh(x)>A[x]$ 
and we have immediately from the definition of $K_A$ that $K_Ax=_00\leq_0A^*\one$.\\
So finally, we obtain: $K_Ax\leq_0A^*\one$, what is the same as
$A^*\one \maj_0 K_Ax$ for all finite binary sequences $x$.\\
\end{proof}\\
%\begin{rmk}
%Note that the majorant $A^*$ for $A$ is crucial, as $A\one$ does not necessary majorize
%$K_Ax$ in all $x$ since in general not even $A \maj_2 A$, take e.g. (for any constant number 
%$0<c^0\in\NN$):
%\begin{align*}
%A[x]&:=c*(1-[x](1))+c*(1-[x](2))+c*(1-[x](3))+\ldots+c*(1-[x](c))\\
%x&:=\langle0,0\rangle\text{,}
%\end{align*}
%and we get $\one^1\maj_1[x]\quad\wedge\quad 0=_0A\one^1 <_0 A[x]=_0c*c$. 
%Of course we get a very different functional
%$A$ as it has to satisfy $\forallb \Sec(\bar b(Ab))$, 
%but this is no reason to guarantee the monotonicity of such an $A$ in full generality.
%\end{rmk}
%So finally we can conclude:\\
%\begin{thm}
%The realizing functionals of $\WKL$ are majorized by closed terms of $\weha$ and
%\end{thm}
\subsubsection*{Summary}
Above, we discussed at length the $\WKL$ principle purposefully following the notation
as used by Kohlenbach in~\cite{Kohlenbach08}, which would unfortunately 
clash with the remainder of this thesis. Therefore, let us here briefly summarize the important
results regarding $\WKL$ using a more compatible notation. Also, we resort directly to Shoenfield
interpretation.

\begin{dfn}{{$\WKL_\Delta$}$\equiv\forall f\WKL_\Delta(f)$},
\label{d:frown_g} where
\[
\WKL_\Delta(f)\quad :\equiv\quad\exists b^1\forall k^0\ 
  \widefr{\Big(\!\text{\widefr{f}}\Big)_{\!g}} (\bar bk)=_00\text{, with}
\]
\begin{align*}
\widefr{f}n&:=\begin{cases}
  fn &\Tif \ fn\neq0\ \vee\ 
       \big(\forall k,l(k*l=n\rightarrow fk=0)\wedge \forall i<\lh(n)\ 
(n_i\leq 1)\big)\\
  1^0 &\Telse, \end{cases} \\
f_gn&:=\begin{cases}
  fn &\Tif\ f\big(g(\lh(n))\big)=0\wedge \lh\big(g(\lh(n))\big)=\lh(n)\\
  0^0 &\Telse, \end{cases} \\
gk:=g f k&:=\begin{cases}
\min n\leq\overline{\one^1}k\ (\lh(n)=k \wedge fn=0)&\text{if such an $n$ 
exists}\\
0^0&\Telse,
\end{cases}
\end{align*}
where for any given number theoretical function $f^1$, 
$\widefr{f}$ assigns a unique characteristic function of a $0/1$-tree\footnote{
If $f$ was such a characteristic function already, then it is not modified at all (i.e. we would have $\widefr{f}=f$).}
and $f_gn$ adds the full subtree if there is no path of length $n$\footnote{
Again, if $f$ defined an infinite tree already, then it is not modified at all (i.e. we would have $f_g=f$).}
(this may destroy the tree property of $f$ if present). 
The function $g$ simply looks for a path of length $n$ and retruns $0$ if none exists (otherwise the code of the path itself is returned).
\end{dfn}
We have: $\hrrweha\proves\WKL_\Delta\leftrightarrow\WKL$. \\
Howard proves in \cite{Howard81} that one can give the realizing 
functionals for the Sh-interpretation of $\WKL$ using only restricted bar 
recursion and $\T_0$. This proof is
discussed in great detail above and we use
it to obtain the Sh-interpretation of $\WKL(f)$:
\begin{thm}[{The Sh-interpretation of $\WKL_\Delta$}]\label{t:FIwkl}
The Weak K\"onig's lemma for binary trees
\[
  \forall f\exists b^1\forall k^0\ 
  \widefr{\Big(\!\text{\widefr{f}}\Big)_{\!g}} (\bar bk)=_00
\]
is, provably in $\weha+\rB_\one$, Sh-interpreted as follows:
\[ \forall f,A\ \exists b^1\  \widefr{\Big(\!\text{\widefr{f}}\Big)_{\!g}} 
\big(\overline{b}(Ab)\big)=0
\text{,}\]
where $b$ is realized by $b:=_1B^{^\WKL}Af$:
\begin{align*}
B^{^\WKL}(A,f)&:=\big[g\Big(\widefr{\Big(\!\text{\widefr{f}}\Big)_{\!gf}}
\Big) \Big(K^{^\WKL}
(A,\emptyset)\Big)\big] \text{,}  \\
K^{^\WKL}(A,x)&:=\KA{A}{x}  
\end{align*}
where $g$ is the same term as we used in the 
Definition \ref{d:WKLdelta}\footnote{We define $[\cdot]$ analogously 
for codes of sequences
as we did for sequences themselves.}.
Note that $K^{^\WKL}$ is definable by $\rB_\one$.
\end{thm}
\begin{prop}
The solution of the Sh-interpretation 
of $\WKL(f)$ is, provably in $\hrrweha+\rB_\one$, majorized as follows:
\setcounter{equation}{0}
\[
K^*:=_{1(0)(2)}\lambda A^2,x^0.A^2\one\ \maj_{1(0)(2)}\ K^{^\WKL}, \quad
B^*:=_{1(2)}\lambda A^2. \one^1 \maj_{1(2)}\ B^{^\WKL}
\text{.}
\]
\end{prop}








% Schuette60
% complexity
%%%%%%%%%%%%%%%%%%%%%%%%%
\subsection{Complexity classes and bar recursion}\label{s:compPM}

The interesting question, apart from the concrete form of the realizers,
 is how the specific bar recursors and
rules for bar recursion affect their complexity. Considerable
research was done in the general field of investigating
the effect of bar recursion on the complexity of the provably total functions
of the underlying system by 
%J. Diller, 
W. A. Howard, 
G.~Kreisel, 
H. Luckhardt,
H. Schwichtenberg
%, W. W. Tait 
and others. 
We will mainly use the results of
Howard (see \cite{Howard68}, \cite{Howard81}), who gave an 
ordinal analysis of bar recursion of type 
$0$ for the cases in which the bar recursion operator has type level $3$ or $4$ (here 
we mean the final type level of the operator, not the type level of its arguments) 
and studied the effect of recursors of specific types
on the complexity in general.
%

There are several different notions of \defkey{primitive recursive}
functionals. In connection with the D-interpretation G\"odel considered the
following class of functionals:

\begin{dfn}[G\"odel's $\T$, G\"odel 1958]\label{d:GT}
The set theoretic functionals denoted by the closed terms of $\eha/\weha$
are called \defkey{G\"odel primitive recursive functionals of finite type}.
The quantifier-free term calculus corresponding to $\weha$ is also called 
G\"odel's $\T$.
\end{dfn}

In addition we define the \defkey{$\alpha$-recursive} functionals:
\begin{dfn}[(unnested) Ordinal Recursion]\label{d:ordRec}
Let the well-ordering of natural numbers $\prec$ be as in~\cite{Schuette60}. By
$\prec_\alpha$ we mean the restriction of $\prec$ to numbers
$n\prec\alpha$, i.e. $a\prec_\alpha b :\equiv a\prec b\prec \alpha$. 
A function defined by means of a sequence of explicit definitions
and the (unnested) \defkey{ordinal recursion} 
on $\prec_\alpha$ (where $y'$ is the successor of $y$):
\setcounter{equation}{0}
\begin{align}
f\tp 1(\tup x, 0)  &= G\tp1(\tup x)\\
f\tp 1(\tup x, y') &= H\tp1\Big(\tup x,y',f\big(\tup x,\theta(\tup x, y')\big)\Big)\text{,}
\end{align}
where $\forall \tup x,y\ (\theta(\tup x, y')\prec_\alpha y'\wedge\theta(\tup x, 0)=0)$,
%is called an \defkey{$\alpha$-recursive} function. By \defkey{$<\alpha$-recursive}, we
is called an \defkey{$\alpha$-recursive} function. By \lOrdm{\alpha}re\-cur\-sive, we
mean a $\beta$-recursive function with $\beta<\alpha$.
\end{dfn}
Note that in this section $\alpha$ is used to denote an ordinal as well as to denote a variable,
depending on the context.\\
The class of  primitive recursive functionals in the sense 
of Kleene (see S1-S8 in \cite{Kleene59}) is strictly
smaller than the class of G\"odel primitive recursive functionals. In fact,
the primitive recursive functionals in the sense of Kleene are 
the functionals of pure type defined by the closed terms of $\hrrweha/\hrreha$ 
as discussed by Feferman in~\cite{Feferman77}.\\
For type $1$, the Kleene primitive recursive functionals are just the 
ordinary primitive recursive functions whereas the G\"odel primitive recursive
functionals of type $1$ are the provably total recursive functions of $\PA$, that is
%$<\epsilon_0$\nbd recursive functions.\\
\lOrdm{\epsilon_0}recursive functions.\\
To enable a more subtle description for the complexity of terms we introduce
the restricted classes of G\"odel's $\T$.
%
\begin{dfn}[$\T_n$]\label{d:GTn}
By $\T_n$ we denote the fragment of $\T$ with $R_\rho$ (see definition~\ref{d:weha})
restricted to $\rho$ of type level $\leq n$.
\end{dfn}
%
The first interesting question is, how does the Bar Recursion affect the 
primitive recursive functionals in $\T$. In other words, how does
the addition of $\B$ (and its defining axioms) to the systems
in definition~\ref{d:GT} above change the class of such functionals.\\
This was studied in different ways by several researchers. One of the first
to publish a partial answer was H. Schwichtenberg:
\begin{prop}[Schwichtenberg \cite{Schwichtenberg79}]\label{p:Schwichtenberg}
For functionals $y,z,u$ in $\T$ of proper types, also
the functionals $\B_{0,\rho}yzu$ and $\B_{1,\rho}yzu$ are in $\T$ for arbitrary $\rho\in\Tp$.
\end{prop}
In other words, the terms of $\T$ are closed under the rule of
bar recursion $\B_{1,\rho}$. 
%
This result was used by Kohlenbach in \cite{Kohlenbach99} to obtain:
\begin{prop}[Kohlenbach \cite{Kohlenbach99}]\label{p:KohBR0r}
Let $t^2[\tup x^{\tup 0}, \tup h^{\tup 1}]$ be a term of $\T$ 
containing at most the free variables $\tup x$ of type $0$
and variables $\tup h$ of type level $1$. Let $z$, $u$, $n$, and $y$
be the respective arguments of $\B_{0,\tau}$ of appropriate type for arbitrary
$\tau\in\Tp$. Then the functional
\[
\lambda \tup x,\tup h,z,u,n,y.\B_{0,\tau}(t[\tup x, \tup h],z,u,n,y)
\]
is definable in $\T$ such that $\weha$ proves its 
characterizing equations. 
\end{prop}
%
Using this result and a normalization argument Kohlenbach proves 
in \cite{Kohlenbach99} also another related result (Proposition 4.1, page 1504)
of which we state the following corollary (note that the corollary itself
can be concluded also from Howard's results in~\cite{Howard81}):
\begin{cor}\label{c:TBT}
Up to type level $2$, the terms (containing only variables of type level $\leq\!1$)
definable in $\T_1\ +\ \B_{0,1}$ are definable in 
$\T$ and vice versa.
\end{cor}
%
We should mention that the proof of~\ref{p:Schwichtenberg} was based on ordinal analysis using Tait's result
given in~\cite{Tait76}:
\begin{prop}[Tait \cite{Tait76}, p. 189--191]\label{p:Tait}
Given an ordinal $\alpha\!<\!\epsilon_0$, the $\alpha$-recursion can be reduced
to primitive recursion at higher type.
\end{prop}
%
For the purpose of this thesis an earlier published result 
of Parsons is even more interesting:
\begin{dfn}
Let $\omega_{k}(\omega)$ denote 
$\omega^{^{
  \left.
  \begin{minipage}{4.7ex} 
    $\omega^{\Ddots^\omega} $
  \end{minipage}
  \right\rbrace k
}}$.
\end{dfn}
\begin{prop}[Parsons \cite{Parsons71}, p.~361]\label{p:Parsons}
%The same functions are $<\omega_{k+1}(\omega)$-recursive 
The same functions are \lOrdm{\omega_{k+1}(\omega)}recursive
as are functions defined by the closed terms of $\T_k$.
\end{prop}
%This means there is a precise connection between $\T_n$ and the 
%$\alpha$-recursive functionals for specific ordinal $\alpha$.
The search for a finer analysis of bar recursion of type $0$
suitable to examine cases where only simple forms of bar recursion
are added to $\T_n$ only, rather than to full $\T$, was brought forward
by Howard in \cite{Howard81}.\\
For a term $t^0$, Howard defines its \defkey{computation size} as 
the length of its computation tree allowing nondeterministic contractions 
as defined in~\cite{Howard80} and~\cite{Howard81}. The corresponding deterministic 
contraction for Howard's bar recursor, $\B_H$, is defined as follows
(see~\cite{Howard81}): 
\[
\B_H AFGc\H\quad\text{contr}\quad R_0  \big(\lambda a,b.Gc\H\big)  \big(Fc( \lambda u^0.\B_H AFG(c*u) )\H\big)   
                                                               \big( \lh(c) \dot- A[c]\big)
\tag{+}
\text{,}
\]
where $c$ is the encoding of a finite sequence of natural numbers $c_0, c_1, \ldots, c_k$ for some $k$
and $\H$ stands for $H_1\cdots H_n$ for $H_i$, $i\in\{1, \ldots, n\}$, of appropriate type. % for some $n$.
This corresponds to our definition of the bar recursor $\B_{0,\tau}$, (definition~\ref{d:BR}), where
$A$, $F$, $G$, and $c$ correspond to $y$, $u$, $z$, and $\overline{x,n}$ respectively. \\
For a term $t\tp1$ or $t\tp2$, there are variables $a_1, a_2, \ldots, a_n$ s.t. $ta_1\ldots a_n$ has type
$0$. Howard then defines the computational size of $t$ as the computational size for $ta_1\ldots a_n$. 
Only terms with type level at most $2$ with free variables of type level at most $1$ are considered.\\
Howard extends $\T+\B_H$ by the terms \defkeyn{$\{\alpha, c, t\}$} of the same type as $\B_H A$, where
the forming of $\{\alpha, c, t\}$ binds all occurrences of $\alpha$ in $t$ and the proper subterms of
$\{\alpha, c, t\}$ consist of all subterms of $t$. For computation the contraction (+) is replaced by the
following four contractions (a subterm $\alpha m$ of $\{\alpha, c, t\}$ can be contracted only when
$m\!<\nolinebreak[4]\!\lh(c)$ and only to $c_m$):
\begin{enumerate}
\item
\[ \B_H AFGc\H\quad \text{contr}\quad \{\alpha, c, A\alpha\}FGc\H  
\text{,}
\]
where $\alpha$ is chosen so as not to be free in $A$.
\item If $t$ is a numeral $<\lh(c)$ then
\[
 \{\alpha, c, t\}FGc\H\quad \text{contr}\quad  Gc\H \text{.}
\]
\item For every $t$ of type $0$
\[
 \{\alpha, c, t\}FGc\H\quad \text{contr}\quad  
         R_0  \big(\lambda a,b.Gc\H\big)  \big(Fc( \lambda u^0.\{\alpha, c, t\}FG(c*u) )\H\big)   \big( \lh(c) \dot- t_c\big)
\text{,}
\]
where $t_c$ denotes the result of substituting $[c]$ for $\alpha$ in $t$.
\item
\[
 \{\alpha, d, t\}FG(d*n)\H\quad \text{contr}\quad \{\alpha, d*n, t\}FG(d*n)\H \text{.}
\]
\end{enumerate}

From section $4$, ``Constructive treatment'', of \cite{Howard81} it follows that 
the function corresponding to a term of computational size $\alpha$ is an $\alpha$-recursive 
function and vice-versa.\\
We will use mainly the following result of Howard:
\begin{prop}[Howard \cite{Howard81}, p.~23]\label{p:Howard}
Let $F$, $G$ and $t$ have computation sizes $f$,$g$ and $size(t)$, respectively,
and suppose $\{\alpha, c, t\}$ has type level $\leq 3$. Then $\{\alpha, c, t\}FGc$
has computation size $\omega^{g+f2h}$, where $h=\omega size(t)+\omega$.
\end{prop}

In other words proposition~\ref{p:Howard} states that: 
\theQuote{Any recursor, having
(as functional) type level $\leq 3$ (this includes $\B_{0,1}$), applied 
to ordinal recursive functionals on standard orderings up to $<\!\!\omega_{k}(\omega)$
results in an ordinal recursive functional on standard ordering $<$~$\!\!\omega_{k+1}$~$\!(\omega)$.}
The interested reader is encouraged to study the original paper. Here, for brevity,
we restrict ourselves just to the simplified reformulation above, using the notation of 
proposition~\ref{p:Parsons}, rather than introduce the complete set
of definitions and notation used by Howard.

%Looking closely at the the proof of proposition~\ref{p:KohBR0r} we can extract
%a less general but very useful result:
%\begin{prop}[Kohlenbach \cite{Kohlenbach99}]\label{p:PavolUlrich}
%Let $t^2[\tup x^{\tup 0}, \tup h^{\tup 1}]$ be a term of $\T_n$ 
%containing at most the free variables $\tup x$ of type $0$
%and variables $\tup h$ of type level $1$. Then the functional
%\[
%\lambda \tup x,\tup h,z,u,n,y.\B_{0,1}(t[\tup x, \tup h],z,u,n,y)
%\]
%is definable in $\T_{n+1}$ for functionals $y,z,u$ in $\T_n$ of proper types
% such that $\weha$ proves its 
%characterizing equations.
%\end{prop}





