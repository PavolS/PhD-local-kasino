% Schuette60
% complexity
%%%%%%%%%%%%%%%%%%%%%%%%%
\subsection{Complexity classes and bar recursion}\label{s:compPM}

The interesting question, apart from the concrete form of the realizers,
 is how the specific bar recursors and
rules for bar recursion affect their complexity. Considerable
research was done in the general field of investigating
the effect of bar recursion on the complexity of the provably total functions
of the underlying system by 
%J. Diller, 
W. A. Howard, 
G.~Kreisel, 
H. Luckhardt,
H. Schwichtenberg
%, W. W. Tait 
and others. 
We will mainly use the results of
Howard (see \cite{Howard68}, \cite{Howard81}), who gave an 
ordinal analysis of bar recursion of type 
$0$ for the cases in which the bar recursion operator has type level $3$ or $4$ (here 
we mean the final type level of the operator, not the type level of its arguments) 
and studied the effect of recursors of specific types
on the complexity in general.
%

There are several different notions of \defkey{primitive recursive}
functionals. In connection with the D-interpretation G\"odel considered the
following class of functionals:

\begin{dfn}[G\"odel's $\T$, G\"odel 1958]\label{d:GT}
The set theoretic functionals denoted by the closed terms of $\eha/\weha$
are called \defkey{G\"odel primitive recursive functionals of finite type}.
The quantifier-free term calculus corresponding to $\weha$ is also called 
G\"odel's $\T$.
\end{dfn}

In addition we define the \defkey{$\alpha$-recursive} functionals:
\begin{dfn}[(unnested) Ordinal Recursion]\label{d:ordRec}
Let the well-ordering of natural numbers $\prec$ be as in~\cite{Schuette60}. By
$\prec_\alpha$ we mean the restriction of $\prec$ to numbers
$n\prec\alpha$, i.e. $a\prec_\alpha b :\equiv a\prec b\prec \alpha$. 
A function defined by means of a sequence of explicit definitions
and the (unnested) \defkey{ordinal recursion} 
on $\prec_\alpha$ (where $y'$ is the successor of $y$):
\setcounter{equation}{0}
\begin{align}
f\tp 1(\tup x, 0)  &= G\tp1(\tup x)\\
f\tp 1(\tup x, y') &= H\tp1\Big(\tup x,y',f\big(\tup x,\theta(\tup x, y')\big)\Big)\text{,}
\end{align}
where $\forall \tup x,y\ (\theta(\tup x, y')\prec_\alpha y'\wedge\theta(\tup x, 0)=0)$,
%is called an \defkey{$\alpha$-recursive} function. By \defkey{$<\alpha$-recursive}, we
is called an \defkey{$\alpha$-recursive} function. By \lOrdm{\alpha}re\-cur\-sive, we
mean a $\beta$-recursive function with $\beta<\alpha$.
\end{dfn}
Note that in this section $\alpha$ is used to denote an ordinal as well as to denote a variable,
depending on the context.\\
The class of  primitive recursive functionals in the sense 
of Kleene (see S1-S8 in \cite{Kleene59}) is strictly
smaller than the class of G\"odel primitive recursive functionals. In fact,
the primitive recursive functionals in the sense of Kleene are 
the functionals of pure type defined by the closed terms of $\hrrweha/\hrreha$ 
as discussed by Feferman in~\cite{Feferman77}.\\
For type $1$, the Kleene primitive recursive functionals are just the 
ordinary primitive recursive functions whereas the G\"odel primitive recursive
functionals of type $1$ are the provably total recursive functions of $\PA$, that is
%$<\epsilon_0$\nbd recursive functions.\\
\lOrdm{\epsilon_0}recursive functions.\\
To enable a more subtle description for the complexity of terms we introduce
the restricted classes of G\"odel's $\T$.
%
\begin{dfn}[$\T_n$]\label{d:GTn}
By $\T_n$ we denote the fragment of $\T$ with $R_\rho$ (see definition~\ref{d:weha})
restricted to $\rho$ of type level $\leq n$.
\end{dfn}
%
The first interesting question is, how does the Bar Recursion affect the 
primitive recursive functionals in $\T$. In other words, how does
the addition of $\B$ (and its defining axioms) to the systems
in definition~\ref{d:GT} above change the class of such functionals.\\
This was studied in different ways by several researchers. One of the first
to publish a partial answer was H. Schwichtenberg:
\begin{prop}[Schwichtenberg \cite{Schwichtenberg79}]\label{p:Schwichtenberg}
For functionals $y,z,u$ in $\T$ of proper types, also
the functionals $\B_{0,\rho}yzu$ and $\B_{1,\rho}yzu$ are in $\T$ for arbitrary $\rho\in\Tp$.
\end{prop}
In other words, the terms of $\T$ are closed under the rule of
bar recursion $\B_{1,\rho}$. 
%
This result was used by Kohlenbach in \cite{Kohlenbach99} to obtain:
\begin{prop}[Kohlenbach \cite{Kohlenbach99}]\label{p:KohBR0r}
Let $t^2[\tup x^{\tup 0}, \tup h^{\tup 1}]$ be a term of $\T$ 
containing at most the free variables $\tup x$ of type $0$
and variables $\tup h$ of type level $1$. Let $z$, $u$, $n$, and $y$
be the respective arguments of $\B_{0,\tau}$ of appropriate type for arbitrary
$\tau\in\Tp$. Then the functional
\[
\lambda \tup x,\tup h,z,u,n,y.\B_{0,\tau}(t[\tup x, \tup h],z,u,n,y)
\]
is definable in $\T$ such that $\weha$ proves its 
characterizing equations. 
\end{prop}
%
Using this result and a normalization argument Kohlenbach proves 
in \cite{Kohlenbach99} also another related result (Proposition 4.1, page 1504)
of which we state the following corollary (note that the corollary itself
can be concluded also from Howard's results in~\cite{Howard81}):
\begin{cor}\label{c:TBT}
Up to type level $2$, the terms (containing only variables of type level $\leq\!1$)
definable in $\T_1\ +\ \B_{0,1}$ are definable in 
$\T$ and vice versa.
\end{cor}
%
We should mention that the proof of~\ref{p:Schwichtenberg} was based on ordinal analysis using Tait's result
given in~\cite{Tait76}:
\begin{prop}[Tait \cite{Tait76}, p. 189--191]\label{p:Tait}
Given an ordinal $\alpha\!<\!\epsilon_0$, the $\alpha$-recursion can be reduced
to primitive recursion at higher type.
\end{prop}
%
For the purpose of this thesis an earlier published result 
of Parsons is even more interesting:
\begin{dfn}
Let $\omega_{k}(\omega)$ denote 
$\omega^{^{
  \left.
  \begin{minipage}{4.7ex} 
    $\omega^{\Ddots^\omega} $
  \end{minipage}
  \right\rbrace k
}}$.
\end{dfn}
\begin{prop}[Parsons \cite{Parsons71}, p.~361]\label{p:Parsons}
%The same functions are $<\omega_{k+1}(\omega)$-recursive 
The same functions are \lOrdm{\omega_{k+1}(\omega)}recursive
as are functions defined by the closed terms of $\T_k$.
\end{prop}
%This means there is a precise connection between $\T_n$ and the 
%$\alpha$-recursive functionals for specific ordinal $\alpha$.
The search for a finer analysis of bar recursion of type $0$
suitable to examine cases where only simple forms of bar recursion
are added to $\T_n$ only, rather than to full $\T$, was brought forward
by Howard in \cite{Howard81}.\\
For a term $t^0$, Howard defines its \defkey{computation size} as 
the length of its computation tree allowing nondeterministic contractions 
as defined in~\cite{Howard80} and~\cite{Howard81}. The corresponding deterministic 
contraction for Howard's bar recursor, $\B_H$, is defined as follows
(see~\cite{Howard81}): 
\[
\B_H AFGc\H\quad\text{contr}\quad R_0  \big(\lambda a,b.Gc\H\big)  \big(Fc( \lambda u^0.\B_H AFG(c*u) )\H\big)   
                                                               \big( \lh(c) \dot- A[c]\big)
\tag{+}
\text{,}
\]
where $c$ is the encoding of a finite sequence of natural numbers $c_0, c_1, \ldots, c_k$ for some $k$
and $\H$ stands for $H_1\cdots H_n$ for $H_i$, $i\in\{1, \ldots, n\}$, of appropriate type. % for some $n$.
This corresponds to our definition of the bar recursor $\B_{0,\tau}$, (definition~\ref{d:BR}), where
$A$, $F$, $G$, and $c$ correspond to $y$, $u$, $z$, and $\overline{x,n}$ respectively. \\
For a term $t\tp1$ or $t\tp2$, there are variables $a_1, a_2, \ldots, a_n$ s.t. $ta_1\ldots a_n$ has type
$0$. Howard then defines the computational size of $t$ as the computational size for $ta_1\ldots a_n$. 
Only terms with type level at most $2$ with free variables of type level at most $1$ are considered.\\
Howard extends $\T+\B_H$ by the terms \defkeyn{$\{\alpha, c, t\}$} of the same type as $\B_H A$, where
the forming of $\{\alpha, c, t\}$ binds all occurrences of $\alpha$ in $t$ and the proper subterms of
$\{\alpha, c, t\}$ consist of all subterms of $t$. For computation the contraction (+) is replaced by the
following four contractions (a subterm $\alpha m$ of $\{\alpha, c, t\}$ can be contracted only when
$m\!<\nolinebreak[4]\!\lh(c)$ and only to $c_m$):
\begin{enumerate}
\item
\[ \B_H AFGc\H\quad \text{contr}\quad \{\alpha, c, A\alpha\}FGc\H  
\text{,}
\]
where $\alpha$ is chosen so as not to be free in $A$.
\item If $t$ is a numeral $<\lh(c)$ then
\[
 \{\alpha, c, t\}FGc\H\quad \text{contr}\quad  Gc\H \text{.}
\]
\item For every $t$ of type $0$
\[
 \{\alpha, c, t\}FGc\H\quad \text{contr}\quad  
         R_0  \big(\lambda a,b.Gc\H\big)  \big(Fc( \lambda u^0.\{\alpha, c, t\}FG(c*u) )\H\big)   \big( \lh(c) \dot- t_c\big)
\text{,}
\]
where $t_c$ denotes the result of substituting $[c]$ for $\alpha$ in $t$.
\item
\[
 \{\alpha, d, t\}FG(d*n)\H\quad \text{contr}\quad \{\alpha, d*n, t\}FG(d*n)\H \text{.}
\]
\end{enumerate}

From section $4$, ``Constructive treatment'', of \cite{Howard81} it follows that 
the function corresponding to a term of computational size $\alpha$ is an $\alpha$-recursive 
function and vice-versa.\\
We will use mainly the following result of Howard:
\begin{prop}[Howard \cite{Howard81}, p.~23]\label{p:Howard}
Let $F$, $G$ and $t$ have computation sizes $f$,$g$ and $size(t)$, respectively,
and suppose $\{\alpha, c, t\}$ has type level $\leq 3$. Then $\{\alpha, c, t\}FGc$
has computation size $\omega^{g+f2h}$, where $h=\omega size(t)+\omega$.
\end{prop}

In other words proposition~\ref{p:Howard} states that: 
\theQuote{Any recursor, having
(as functional) type level $\leq 3$ (this includes $\B_{0,1}$), applied 
to ordinal recursive functionals on standard orderings up to $<\!\!\omega_{k}(\omega)$
results in an ordinal recursive functional on standard ordering $<$~$\!\!\omega_{k+1}$~$\!(\omega)$.}
The interested reader is encouraged to study the original paper. Here, for brevity,
we restrict ourselves just to the simplified reformulation above, using the notation of 
proposition~\ref{p:Parsons}, rather than introduce the complete set
of definitions and notation used by Howard.

%Looking closely at the the proof of proposition~\ref{p:KohBR0r} we can extract
%a less general but very useful result:
%\begin{prop}[Kohlenbach \cite{Kohlenbach99}]\label{p:PavolUlrich}
%Let $t^2[\tup x^{\tup 0}, \tup h^{\tup 1}]$ be a term of $\T_n$ 
%containing at most the free variables $\tup x$ of type $0$
%and variables $\tup h$ of type level $1$. Then the functional
%\[
%\lambda \tup x,\tup h,z,u,n,y.\B_{0,1}(t[\tup x, \tup h],z,u,n,y)
%\]
%is definable in $\T_{n+1}$ for functionals $y,z,u$ in $\T_n$ of proper types
% such that $\weha$ proves its 
%characterizing equations.
%\end{prop}

