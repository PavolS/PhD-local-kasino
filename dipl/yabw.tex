\section{Interpreting Bolzano-Weierstra{\ss}} \label{s:bw}
In this section we will use bar recursion
 to interpret the Bolzano-Weierstra{\ss} 
\paragraph{\defkey{\BW(s)}}
\[
  \exists a\forall k\exists l\geq k\ |a-_\RR\lambda n.sl|\leq_\RR2^{-k}
\]
theorem. We will use two common principles: $\CA$ and $\WKL$. 
Their respective interpretations can be found in previous sections. Recall:
\paragraph{\defkey{$\SiLm\CA(f)$}}
\[
  \exists g \forall x\ (gx=_00 \leftrightarrow \exists y\ fxy=_00)
\]
\paragraph{\defkey{$\WKL_K(f)$}}
\[
  \BTree_K(f) \wedge \forall k\exists x (\lh(x)=_0k \wedge f(x)=_00) 
     \rightarrow \exists b\leq_1\one^1\ (\forall k\ f(\bar{b}(k))=_00)
\text{.}
\]
First, in section \ref{ss:spuWKL}, we give a simple 
proof of $\BW(s)$ using $\SiLm\CA$ and $\WKL_K$
to be able to apply $\Pi^0_2\m\WKL$ to a specific tree determined by $s$.\\
We analyze a minor modification of this proof and obtain the D-interpretation
of $\BW$ in section \ref{ss:fafi}.\\
Finally, in section \ref{ss:acr}, we discuss the complexity of the realizers 
obtained in \ref{ss:fafi} depending on the way the $\BW$ principle is used in
a given proof.

\subsection{Simple Proof based on $\Pi^0_2\m\WKL$} \label{ss:spuWKL}
Consider a tree representation of the unit interval $[0,1]$ which 
splits the unit interval at level $n$ into $2^n$ intervals of 
length $2^{-n}$. We can draw the tree for first two levels as follows:
\[
\begin{parsetree}
(.. 
   (.$[0,\frac{1}{2}]$.
      (.$[0,\frac{1}{4}]$. .. ..)
      (.$[\frac{1}{4},\frac{1}{2}]$. .. ..)
   ) 
   (.$[\frac{1}{2},1]$.
      (.$[\frac{1}{2},\frac{3}{4}]$. .. ..)
      (.$[\frac{3}{4},1]$. .. ..)
   )
)
\end{parsetree}
\text{.}\]
Note that we can define each node via the path from the 
root to this interval. This path can by represented 
by a binary sequence $b$, where the $n$-th element defines which branch 
to take (here $0$ means left and $1$ means right).\\
%
We define a predicate $I(b^0,n^0,m^0)$, which tells us, whether the 
rational number $r$ encoded by $m=\langle r \rangle$ is from an 
interval defined by such a finite binary sequence $b$
of length $n$ (we can encode any finite sequence of natural numbers 
as a natural number), i.e. in a by 
$b$ given interval of length $2^{-n}$. 
\begin{align*}
I(b^0,n^0,m^0)  :&\equiv\ \ \  
  \InIntM{b}{n}\\
&\equiv\ \ \ n\leq \lh(b) \ \wedge\  
             \sum_{i=1}^{n}\frac{b(i)}{2^i}\ \leq\ 
             r\ \leq\sum_{i=1}^{n}\frac{b(i)}{2^i}+\frac{1}{2^n} 
\end{align*}
For example, $I(<1,1>,2,m^0)$ decides whether $m$ is or is not in
the interval 
$[\sum_{i=1}^{2}\frac{(<1,1>(i)}{2^i},\lb\sum_{i=1}^{2}\frac{(<1,1>(i)}{2^i}+\frac{1}{2^2}]$
which is $[\frac{3}{4},1]$.\\
We know that for given finite binary sequence $b$ and 
an infinite sequence of encodings of rational
numbers $s$ there is a function $f_{s}^{1(0)}$ such that:
\[
 f_{s}(b,k)=_00\ \ \leftrightarrow\ \ (\ k>\lh(b)\wedge I(s,\lh(b),sk)\ )
\text{.}\]
%
Now, by $\SiLm\CA(f_s)$ we obtain a function $g_s$, s.t.:
\[
  \forall b^0\ (\ g_sb=_00 \leftrightarrow \exists z^0\ (f_{s}(b,z)=_00)\ )
\text{.}\]
%
In other words we have for all $b^0$:
\[
  g_{s}(b)=_00\ \leftrightarrow \exists k^0\geq_0 \lh(b)\ \  I(b,\lh(b),sk)
\text{.}
\tag{+}
\]
%
%
% showing bintree(f)
%
%%%%%%%%%%%%%%%%%%%%%%%%%%%%
To show $\BTree(g_s)$, consider any finite binary sequence $b$:
\begin{align*}
  g_s(b)=_00\ \wedge\ x\subseteq b 
           &{\ \rightarrow\ }  \exists k^0\geq_0\lh(b)\ \ I( b,\lh(b),sk)\ \wedge\ 
                              \lh(x)\leq_0\lh(b)\ \wedge\ 
                              x\subseteq b \\
               &{\ \rightarrow\ } \exists k^0>_0\lh(x)\ \ I( b,\lh(x),sk)\ \wedge\ 
                              x\subseteq b\  \\
               &{\ \rightarrow\ } \exists k^0>_0\lh(x)\ \ I( x,\lh(x),sk)\\
               &{\ \rightarrow\ }g_s(x)=_00
\text{.}
\end{align*}
To show 
\[
\forall k\exists x (\lh(x)=_0k \wedge g_s(x)=_00) \tag{++}
\]
just consider any given 
natural number $k$. By the definition of our tree, it splits the $[0,1]$ 
interval at any level, i.e. especially also on level $k$, completely. 
I.e. we can find also at level $k$ an interval, defined by a binary 
sequence of length $k$, such that $s(k)$ is contained in this
interval defined by some $b$ with $\lh(b)=k$  . 
So, we have: $I( b, \lh(b), s(k))$ and $ \lh(b)=_0k$. 
As we started with arbitrary $k$, this implies (++).\\
Now, we can apply $\WKL_K(g_s)$ to get:
\[
\exists b^1 (\BFunc(b) \wedge \forall k\ g_s(\overline{b}k)=_00) 
\tag{*}
\text{.}
\]
Note that in (*) (and from now on) $b^1$ is 
a binary function and $g_s$ takes
the encoding of the finite beginning, $\langle b(0),\ldots,b(k)\rangle$, 
of this infinite sequence as its type $0$ argument.
Using (+) we can conclude that (*) is equivalent to:
\[
\exists b^1\leq\one\ \forall n\ \exists l>n\ \ I(\overline{b}n,n,xl)
\text{.}\]
This means that $\BW$ is satisfied by $\hat a$ by setting:
\[
a(n^0):=_\QQ \left\langle \sum_{i=1}^n \frac{b(i)}{2^i}\ +
               \frac{1}{2^{n+1}}\right\rangle
\text{.}
\]
%
It remains to show that $a$ is a real 
number in the sense as defined in the introduction or in \cite{KO02}.
To do so, take any natural number $n$. We have:
\begin{align*}
|a(n)-a(n+1)|&=
\left|
\sum_{i=1}^{n+1} \frac{b(i)}{2^i}\ +
 \frac{1}{2^{n+2}}
-
{\left(\sum_{i=1}^n \frac{b(i)}{2^i}\ +
 \frac{1}{2^{n+1}}\right)}
\right|
\\
&=
\left|
\frac{b(n+1)}{2^{n+1}}+\frac{1}{2^{n+2}}
-
\frac{1}{2^{n+1}}
\right|
\\
&=
2^{-(n+2)} < 2^{-(n+1)}
\text{,}
\end{align*}
what concludes the proof.

%%%%%%%%%%%%%%%%%%%%%%%%%%%%%%%%%%
%
\subsection{Formal Analysis and functional Interpretation} \label{ss:fafi}
%
%%%%%%%%%%%%%%%%%%%%%%%%%%%%%%%%%%
%
%
%
Recall the standard formulation of $\BW$:\\
\[
\forall s(\ \ (\forall n\ 0\leq_\QQ sn\leq_\QQ 1) \rightarrow 
  \exists a\forall k\exists l\geq k\ |a-_\RR\lambda n.sl|\leq_\RR2^{-k}\ \ )
\text{.}
\tag{$\BW_0$}
\]
Using the $\hat f$ transformation from definition \ref{d:hat}, we can
derive $\BW_0$ from the formula:
\[
\forall s^1\leq_1\one\exists a^1\forall k^0\exists l^0\geq k_0\ 
 |\hat a(k+1)-_\QQ s(l)|\leq_\QQ2^{-(k+2)}
\text{.}
\tag{$\BW_1$}
\]
From now on we consider $s$ to be an infinite sequence of encodings of rational 
numbers within the $[0,1]$ interval, and $f_s$ to be the characteristic function of
the corresponding tree as defined in section \ref{ss:spuWKL} above.
From section \ref{s:wkl}, we know that using an 
appropriate quantifier-free formula $\phi'^{\WKL}$
we can write $\WKL$ as
\[
\forall h\exists b^1\leq\one\forall k\ \ \phi^{\WKL}(h,b,k)\equiv\forall(h)\WKL_t(h)
\text{,}
\]
where $\phi^{\WKL}$ is quantifier-free.\\
We introduce following notations for $\SiLm\CA$:
\begin{align*}
\exists g^1\forall x^0 &\phi^{\CA(f)}_{\Sigma^0_1}(x,gx)\\
&\equiv\ \exists g^1\forall x^0 (gx=_00 \leftrightarrow \exists z^0\ f(x,z)=_00)\\
\Leftrightarrow\ 
\exists g^1\forall x^0 &\exists z^0_1\forall z^0_2 \phi^{\CA(f)}(x,gx,z_1,z_2)\\
&\equiv\
\exists g^1\forall x^0\exists z^0_1\forall z^0_2
   ((gx=_00 \rightarrow f(x,z_1)=_00)\wedge(gx=_00 \leftarrow f(x,z_2)=_00) )
\text{,}
\end{align*}
where $\phi^{\CA(f)}$ is a quantifier-free formula. \\
%The index $s$ stands for the original
%sequence of rationals we are working with. The formula $\phi_{\CA}(x,y,z)$ 
%uses $f_s$ as its only free 
%variable representing the tree corresponding to $s$ as described in the previous section.\\
Now, at the first glance, the proof-tree for the proof from previous
 section is fairly simple:
\[
\begin{prooftree}
\SiLm\CA(f_s)
 \quad \quad
\WKL
%--------------------------------------------------------------------------
\justifies
\BW(s)
\end{prooftree}
\text{,}
\]
where we know that $\BW(s)$ is just $\SiLm\CA(f_s) \wedge \WKL(g_s)$, 
where $g_s$ is the realizing 
function from $\SiLm\CA(f_s)$. However, we can interpret only the
negative translations of the assumptions what leads us to:
\[
\begin{prooftree}
(\SiLm\CA(f_s))'
 \quad \quad
(\WKL)'
%--------------------------------------------------------------------------
\justifies
(\BW(s))'
\end{prooftree}
\text{.}
\]
Moreover, to be able to exploit the whole benefits of functional interpretation
we have to investigate the interpretation of this proof itself, 
namely the functional interpretation of:
\[ \ ((\SiLm\CA(f_s))' \wedge (\WKL)')\quad \rightarrow\quad (\BW(s))' \tag{+}\text{.} \]
So finally we have the following picture:
\[
\begin{prooftree}
(\SiLm\CA(f_s))' \wedge (\WKL)'
 \quad \quad
((\SiLm\CA(f_s))' \wedge (\WKL)')\ \rightarrow\ (\BW(s))'
%--------------------------------------------------------------------------
\justifies
(\BW(s))'
\end{prooftree}
\text{.}
\]
Similarly as in \ref{ss:CA} this turns down to solving the interpretation of (+). This is
not trivial, since the negative translation double negates each formula. So
 (+) becomes essentially
the finite $\DNS$ schema, what becomes even clearer using same
 notation as in the definition \ref{d:NT}:
\[ \ (\neg\neg(\SiLm\CA(f_s))^* \wedge 
\neg\neg(\WKL)^*)\quad \rightarrow\quad \neg\neg(\BW(s))^* \text{.} \]
Using the representations for $\neg\neg(\BW(s))^*$ becomes
\[
\neg\neg \exists g,b\forall x,k\ ( \phi^{\CA(f_s)}_{\SiL}(x,gx)\wedge \phi^{\WKL}(g,b,k)
\]
what is essentially $(\Pi^0_2\m\WKL(\phi))'$ for the specific 
formula $\phi^\BW(s)(\bar b(k))$ corresponding
to $\exists z\ f_s(\bar b(k),sz)$, where $f_s$ is 
defined as in section \ref{ss:spuWKL} above. So, treating (+) in the form
\begin{align*}
( \neg\neg \exists g'\forall x'\ \phi^{\CA(f)}_{\SiL}(x',g'x')\wedge
  \forall h' \neg\neg &\exists b'\forall k'\ \phi^{\WKL}(h',b',k') ) \rightarrow\\
\neg\neg \exists g,b&\forall x,k\ ( \phi^{\CA(f)}_{\SiL}(x,gx)\wedge \phi^{\WKL}(g,b,k) ) 
%\text{.}
\end{align*}
independetly of the precise structure of $f$, we obtain the 
functional interpretation of $(\Pi^0_2\m\WKL(\psi_f))'$, for which 
$\psi_f(x^0)$ $\leftrightarrow$ $\exists n^0\ f(x,n)=_00$.\\
The functional interpretation of (+) using this representation is as follows:
\[
\begin{minipage}{\textwidth}
$\exists G,Z_1,B,H',X',Z_2',K'\ \ \forall B',Z_1',G',X,Z_2,K$
\vspace{-0.3cm}
\[
\quad
\left(
\begin{minipage}{0.93\textwidth}
\vspace{-0.4cm}
\[
\left(
\phi^{\CA(f)} \left(
\begin{minipage}{0.38\textwidth} 
\vspace{-0.4cm}
   \begin{align*}
     (&G'X'Z_2')(X'(G'X'Z_2')(Z_1'X'Z_2')),\\
     &X'(G'X'Z')(Z_1'X'Z_2'),  \\
     (&Z_1'X'Z_2')(X'(G'X'Z_2')(Z_1'X'Z_2')), \\
     &Z_2'(G'X'Z')(Z_1'X'Z_2')
   \end{align*}
\end{minipage}
\right)
\wedge
\phi^{\WKL} \left(
\begin{minipage}{0.16\textwidth} 
\vspace{-0.4cm}
   \begin{align*} 
     &H', \\
     &B'H'K',\\
     &K'(B'H'K')
   \end{align*}
\end{minipage}
\right)
\right)
\rightarrow
\hspace{1cm}
\]

\vspace{-0.7cm}
\[
\hspace{4.85cm}
\left(
\phi^{\CA(f)} \left( 
\begin{minipage}{0.18\textwidth} 
\vspace{-0.4cm}
  \begin{align*}
     &G(XGZ_1B), \\
     &XGZ_1B, \\
     &Z_1(XGZ_1B), \\
     &Z_2GZ_1B
   \end{align*}
\end{minipage}
\right)
\wedge
\phi^{\WKL} \left(
\begin{minipage}{0.11\textwidth} 
\vspace{-0.4cm}
   \begin{align*} 
     &G, \\
     &B,\\
     &KGZ_1B
   \end{align*}
\end{minipage}
\right)
\right)
\]

\end{minipage}
\right)
\]
\end{minipage}
\text{,}
\]
where, again, each exists-variable (i.e. $G$, $B$, $H'$, $X'$, $Z'$, and $K'$) 
may depend on any 
for-all-variable (i.e. $B'$, $G'$, $X$, $Z$, and $K$). E.g. by $G$ we mean in 
fact $(GB'G'XZK)$.  This interpretation
yields following functional equations:
\setcounter{equation}{0}
\begin{align}
     (G'X'Z_2')(X'(G'X'Z_2')(Z_1'X'Z_2'))&=G(XGZ_1B)\\         %1
     X'(G'X'Z')(Z_1'X'Z_2')&=XGZ_1B  \\                        %2
     (Z_1'X'Z_2')(X'(G'X'Z_2')(Z_1'X'Z_2'))&=Z_1(XGZ_1B) \\    %3
     Z_2'(G'X'Z')(Z_1'X'Z_2')&=Z_2GZ_1B \\                     %4
     H'&=G \\                                                  %5
     B'H'K'&=B\\                                               %6  
     K'(B'H'K')&=KGZ_1B                                        %7
\text{.}
\end{align}
We use a very similar approach to the one used by Gerhardy in \cite{GerhardyX} to
solve such equations for finite $\DNS$. First, we conclude from (5) and (6)
that $B=B'GK'$ and from (1) and (2) that $G=G'X'Z_2'$. Using (6), we can set 
$K'$ to $\lambda b.KGZ_1b$ according to (7). This is not that trivial
for $X'$ and $Z_2'$. However, as Gerhardy presented in \cite{GerhardyX}, in the presence
of the $\lambda g$ and $\lambda z_1$, which as we know will stand for the input 
of $G$ and $Z_1$, becomes
$K'$ and thereby $B$ suddenly a well definable term:
\begin{align*} 
t_{X'}:&=\lambda g,z_1.Xgz_1(B'g(\lambda b.Kgz_1b))\\
t_{Z_2'}:&=\lambda g,z_1.Z_2gz_1(B'g(\lambda b.Kgz_1b))\text{.}
\end{align*} 
This makes the rest of our presets well defined, what is easy
to see, since for each term all dependencies are only on
the terms defined above:
\begin{align*} 
t_{Z_1}:&=Z_1't_{X'}t_{Z_2'}\\
t_{G}:&=G't_{X'}t_{Z_2'}\\
t_{H'}:&=t_{G}\\
t_{K'}:&=\lambda b.Kt_{G}t_{Z_1}b\\
t_{B}:&=B't_{G}t_{K'}\text{.}
\end{align*} 
We have found the realizing terms for ($+^D$) for any $G'$, $Z_1'$ and $B'$. To finally obtain
the functional interpretation of $(\Pi^0_2\m\WKL(\phi))'$ we just need to 
define these two functionals in such a way that the 
assumptions $\phi^{\WKL}$ and $\phi^{\CA(f)}$ are always true.\\
For $\phi^{\CA(f)}$, as we know from the functional interpretation of $\SiLm\CA$ 
(see section \ref{ss:CA}), 
i.e. nothing else as setting 
\begin{align*}
      G'^{\mathit{3}}&=t_h \\
      Z_1'^{\mathit{3}}&=\lambda T^{2(1)},\lambda W^{2(1)},a^0\ .\ t_zTWa0^0
\text{,} 
\end{align*}
where $t_h$ and $t_z$ are defined as in the ND-interpretation of $\SiLm\CA$ 
(see corollary \ref{c:NDSiLCA}).\\
For $B'$, from the interpretation of $\WKL$, we know the following
correspond:
\[ 
\underbrace{\overline {B'H'K'}(K'(B'H'K'))}_{\text{as above}}\quad = 
\underbrace{\overline{B}(AB)}_{\text{as in section \ref{ss:HFI}}} 
\text{.} 
\]
We use the same notation as we used to define $B$ in section \ref{ss:HFI} and define:
\[
        B':=\lambda h.\lambda A.[F_h(K_{A}(\emptyset),\emptyset)] \text{,}
\]
where $F_h$ and $K_A$ are defined as in the ND-interpretation of $\WKL$ 
(see theorem \ref{t:FIwkl}).\\
The terms defined above using these definitions for $G'$ and $B'$ then satisfy the following
modification of ($+^D$):
\begin{align*}
\forall X,Z_2,K\ (\ &\phi^{\CA(f)}(
 t_{G}(Xt_{G}t_{Z_1}t_{B}),
 Xt_{G}t_{Z_1}t_{B},
 t_{Z_1}( Xt_{G}t_{Z_1}t_{B} ),
 Z_2t_Gt_{B})\ \wedge\\
 &\phi^{\WKL}(t_{G},t_{B},Kt_{G}t_{Z_1}t_{B})\ ) \text{,}
\end{align*}
which is in principle the functional interpretation of $\Pi^0_2\m\WKL(\phi)$. In summary we get:
%
% ND-interpretation of \phi_BW(s)
%
%%%%%%%%%%%%%%%%%%%%%%%%%%%%%%%
\begin{prop}[The ND-interpretation of $\Pi^0_2\m\WKL(\phi)$] \label{p:ND-PI02WKL}
$ $
\\
W.l.o.g. assume:
\[
\forall k\phi(\bar bk) \Leftrightarrow \forall k^0\exists z^0\ f(\bar bk,z)=_00
\text{.}
\]
The $\Pi^0_2\m\WKL(\phi)$ principle in the classically equivalent form:
\[
\exists g,b\forall x,k\ ( \phi^{\CA(f)}_{\SiL}(x,gx)\wedge \phi^{\WKL}(g,b,k) )
\text{,}
\]
is ND-interpreted by
\begin{align*}
\forall X,Z,K\ (\ &\phi^{\CA(f)}(
 t_{G}(Xt_{G}t_{Z}t_{B}),
 Xt_{G}t_{Z}t_{B},
 t_{Z}( Xt_{G}t_{Z}t_{B} ),
 Zt_Gt_{B})\ \wedge\\
 &\phi^{\WKL}(t_{G},t_{B},Kt_{G}t_{Z}t_{B})\ ) \text{,}
\end{align*}
where
\begin{align*}
        t_B    &:= [F_{t_G}(K_{\lambda b.Kt_Gt_Zb}(\emptyset),\emptyset)]\\
        t_Z    &:= \lambda n^0\ .\ t_zt_{X}'t_{Z}'n0^0\\
        t_G    &:= t_ht_{X}'t_{Z}'\\
        t_{X}' &:=\lambda g,z.Xgz( 
                 [F_{\lambda b.Kgzb,g}(K_{\lambda b.Kgzb}(\emptyset),\emptyset)] )\\
        t_{Z}' &:=\lambda g,z.Zgz( 
                 [F_{\lambda b.Kgzb,g}(K_{\lambda b.Kgzb}(\emptyset),\emptyset)] )\text{.}
\end{align*}
The remaining terms are defined as in previous sections:
\begin{align*}
        K_Ax     &:= \KA{A}{x}        \\
        F_{A,f}(0,x)   &:= \Fnullx \\
        F_{A,f}(n+1,x) &:= \Fnx \\
        x_0TW    &:= \xTWf{T}{W}{f}   \\
        u_{f,W}n^0v^{1(0)}  
                 &:= \uWf{W}{f}\\
        t_z      &:=\lambda X^{2(1)},Z^{2(1)},a^0,b^0\ .\ t_g(t_fX)(t_fZ)a\\
        t_h      &:=\lambda X^{2(1)},Z^{2(1)},n^0\ .\ 
                        {\min}_0(f(n,t_g(t_fX)(t_fZ)n),1) \\
        t_f      &:=\lambda X^{2(1)},g^1\ .\ 
                       X(\lambda n^0 . {\min}_0(f(n,gn),1) )(\lambda a^0,b^0 . ga) 
\text{.}
\end{align*}
\end{prop}

\begin{rmk}
One might wonder, why we still need the first element of the conjunction.
The point is, the second part of our result is not as strong as it may seem. 
The formula:
\[
\forall K\ \phi^{\WKL}(t_{G},t_{B},Kt_{G}t_{Z}t_{B}) \text{,}
\]
becomes useful only if we can prove that $t_{G}$ defines the 
infinite binary tree we originally started with. 
Otherwise, all conclusions taken are about some, though primitive recursive, 
modifications of $t_G$; moreover we could not prove what does this $t_G$ stand for. 
See also the proof of the corollary \ref{c:ND-bw} below.
\end{rmk}

\begin{cor}\label{c:ND-bw}
The Bolzano-Weierstra{\ss} principle $\BW(s)$ for an infinite sequence
of rational numbers, $s$, bounded within the interval $[0,1]$:
\[
\forall s\leq_1\one
  \exists a\forall k\exists l\geq k\ |\hat a(k+1)-_\QQ sl|\leq_\QQ2^{-(k+1)}
\text{,}
\]
is ND-interpreted by
\[
\forall s\leq_1\one, K
  (\ \ t_L(Kt_At_L)\ \wedge\ 
  |\hat t_A(Kt_At_L+1)-_\QQ s(t_L(Kt_At_L))|\leq_\QQ2^{-(Kt_At_L+1)}\ \ )
\text{,}
\]
where the terms $t_L$ and $t_A$ depend on both $s$ and $K$ and are defined as follows:
\begin{align*}
t_L&:=\lambda n^0.t_Z(\overline{t_B}(n))\\
t_A&:=\lambda n^0.\sum^{n}_{i=1}\frac{t_Bi}{2^i}\ +\ \frac{1}{2^{n+1}}
\text{,}
\end{align*}
where $t_B$ and $t_Z$ are defined as above in proposition \ref{p:ND-PI02WKL} 
with fixed $X$, $Z$, and $f$:
\begin{align*}
X\tp2&:=\lambda g^1,z^1,b^1\ .\ B(Kgzb,g,z,b)\\
Z\tp2&:=\lambda g^1,z^1,b^1\ .\ N(Kgzb,g,z,b)\\
f\tp1&:=f_s\tp1
\text{,}
\end{align*}
where
\begin{align*}
B(m^0,g^1,z^1,b^1)&:=_0
\begin{cases}
  {\min}_0 \{ x^0 | \lh(x)<_0 m \wedge \neg\phi^{\CA}(x,gx,zx,N_zm) \} &\text{if it exists} \\
  X_n(m,g,b)&\Telse
\end{cases}\\
X_n(m^0,g^1,b^1)&:=_0
  \begin{cases}
    \bar bm &\Tif\ g(\bar bm)=_00\\
    {\min}_0 \{ x^0 | \lh(x)=m \wedge f_s(x,m)=_00 \} &\Telse
  \end{cases}\\
N(m^0,g^1,z^1,b^1)&:=_0\begin{cases}
\lh(m) &\Tif\ f_s(m,\lh(m))=_00\\
z(X_nmgb) &\Telse
\end{cases}\\
        f_s(b^0,z^0)
                 &:=_0\fs{b}{z}\\
        I(b^0,n^0,m^\QQ)
                 &:\equiv\ \ \  \InIntM{b}{n}
\text{.}
\end{align*}
\end{cor}
%%%%
%
% Proof
%
\begin{proof}% [ of corrolary \ref{c:ND-bw}]
Unwinding $\phi^\CA$ and $\phi^\WKL$ 
we get by proposition \ref{p:ND-PI02WKL}:
\setcounter{equation}{0}
\begin{multline}
\forall K\tp2,X\tp2,Z\tp2\ 
                     ((t_G\args{ XZK (X \args{
                                          (t_G\args{XZK})
                                          (t_Z\args{XZK})
                                          (t_B\args{XZK})
                                        } )
                           } 
                       )=_00  \rightarrow\\
                       f_s( (X \args{
                                  (t_G\args{XZK})
                                  (t_Z\args{XZK})
                                  (t_B\args{XZK})
                                } )  ,
                            t_Z \args{ XZK (X \args{
                                                (t_G\args{XZK})
                                                (t_Z\args{XZK})
                                                (t_B\args{XZK})
                                               } )
                                 }  
                       )=_00
                      )
\end{multline}
and
\begin{multline}
\forall K\tp2,X\tp2,Z\tp2\ 
                     ((t_G\args{ XZK (X \args{
                                          (t_G\args{XZK})
                                          (t_Z\args{XZK})
                                          (t_B\args{XZK})
                                        } )
                           } 
                       )=_00  \leftarrow\\
                       f_s( X \args{
                                  (t_G\args{XZK})
                                  (t_Z\args{XZK})
                                  (t_B\args{XZK})
                                }   ,
                            Z \args{
                                 (t_G\args{XZK})
                                 (t_Z\args{XZK})
                                 (t_B\args{XZK})
                               } 
                       )=_00
                      )
\end{multline}
and
\be[f:three]
\forall K\tp2,X\tp2,Z\tp2\ 
  \widehat{\ (\widehat{ t_G \args{XZK} }
             )_{ t_{ t_G\args{XZK} } }\ 
          } (\overline{
                t_B\args{XZK}
             } (K \args{
                    (t_G\args{XZK})
                    (t_Z\args{XZK})
                    (t_B\args{XZK})
                  }
               )
            )=_00
\text{.}
\ee

Note that by (1) the equality $t_G\args{XZK}(x)=_00$ implies $f_s(x,z)=_00$ and thereby
 $\BFunc([x])$.\\
We will not be able to show $\forall K\tp2\BTree(t_g\args{XZK})$. Fortunately, we need only
to show:
\begin{multline*}
\forall K\tp2\ {\big(} 
\widehat{t_G\args{XZK}}
 (\overline{t_B\args{XZK}}(K\args{ (t_G\args{XZK}) (t_Z\args{XZK}) (t_B\args{XZK}) }))=_00 
       \rightarrow\\
 t_G\args{XZK}
   (\overline{t_B\args{XZK}}(K\args{ (t_G\args{XZK}) (t_Z\args{XZK}) (t_B\args{XZK}) }))=_00\ 
{\big)} 
\end{multline*}
and
\begin{multline*}
\forall K\tp2\ {\big(}\ (t_G\args{XZK})_{t_{t_G\args{XZK}}}(\overline{t_B\args{XZK}}(K\args{ (t_G\args{XZK}) (t_Z\args{XZK}) (t_B\args{XZK}) }))=_00 \rightarrow\\
                           t_G\args{XZK}(\overline{t_B\args{XZK}}(K\args{ (t_G\args{XZK}) (t_Z\args{XZK}) (t_B\args{XZK}) }))=_00)\ {\big)}\text{.}
\end{multline*}
Due to the definition \ref{d:hatAndG} to do so it suffices to show:
\begin{multline}
\forall K\tp2,x^0 
\ \ {\Big(\ }{\big(} x\subseteq 
       \overline{t_B\args{XZK}}(K\args{ (t_G\args{XZK}) (t_Z\args{XZK}) (t_B\args{XZK}) } ) 
      \ \wedge \\
      t_G\args{XZK}
       (\overline{t_B\args{XZK}}(K\args{ (t_G\args{XZK}) (t_Z\args{XZK}) (t_B\args{XZK}) } ))=0
    {\big)}
    \rightarrow 
                           t_G\args{XZK}(x)=_00 {\ \Big)} \label{e:tree}
\end{multline}
\begin{align}
\forall K\tp2\ 
  \forall n^0 \leq_0 K\args{ (t_G\args{XZK}) (t_Z\args{XZK}) (t_B\args{XZK}) }\ \ 
\left(\ \exists x (\lh(x)=_0 n \wedge t_G\args{XZK}x=_00\ \right) \label{e:inf}
\end{align}
respectively.
\begin{itemize}
\item To prove \eqref{e:tree} suppose:
\begin{multline}\label{e:trAss}
x\subseteq 
       \overline{t_B\args{XZK}}(K\args{ (t_G\args{XZK}) (t_Z\args{XZK}) (t_B\args{XZK}) } ) 
      \ \wedge \\
      t_G\args{XZK}
       (\overline{t_B\args{XZK}}(K\args{ (t_G\args{XZK}) (t_Z\args{XZK}) (t_B\args{XZK}) } ))=0
\end{multline}
holds for some $K$ and $x$. By (1), (2), and definitions of $X$ and $B$ we have
for all binary sequences $b$ with $\lh(b)
  \leq l:=K\args{
    (t_G\args{XZK})
    (t_Z\args{XZK})
    (t_B\args{XZK})
   }$
\begin{multline}
      ((t_G\args{ XZK} (b) )=_00 \rightarrow f_s( b, t_Z \args{ XZK}(b) )=_00 )\ \ \wedge\\
      (t_G\args{ XZK } (b) )=_00 \leftarrow f_s( b, Z\args{
        (t_G\args{XZK})
        (t_Z\args{XZK})
        (t_B\args{XZK})
       }  )=_00 ))
\label{e:caFin}
\text{.}
\end{multline}
Suppose namely \eqref{e:caFin} would not hold for some $b_0\leq l$, then by definition of
$X$, is $X\args{(t_G\args{XZK})(t_Z\args{XZK})(t_B\args{XZK})}$ equal to $b_0$ 
and we get a contradiction to (1)$\wedge$(2), for our specific $X$.\\
So, together with \eqref{e:trAss} we obtain from \eqref{e:caFin}:
\[
f_s( \overline{t_B\args{XZK}}(l), t_Z \args{ XZK}(\overline{t_B\args{XZK}}(l)) )=_00
\text{,}
\]
what by definition of $Z$ implies
\[
f_s( \overline{t_B\args{XZK}}(l), Z\args{(t_G\args{XZK})(t_Z\args{XZK})(t_B\args{XZK})} )=_00
\text{.}
\]
By definition of $f_s$, see also section \ref{ss:spuWKL}, and \eqref{e:trAss} this implies
\[
f_s( x, Z\args{(t_G\args{XZK})(t_Z\args{XZK})(t_B\args{XZK})} )=_00
\text{.}
\]
Again, using both: \eqref{e:trAss} and  \eqref{e:caFin}, we obtain
\[
t_G\args{XZK}(x)=0
\text{.}
\]
\item To prove \eqref{e:inf} take any $n^0\leq l:=K\args{
    (t_G\args{XZK})
    (t_Z\args{XZK})
    (t_B\args{XZK})
   }$ for some fixed $K$.\\
If $t_G\args{XZK}( \overline{t_B\args{XZK}}(l) )=0$ we are done, otherwise 
we have:
\[
X\args{(t_G\args{XZK})(t_Z\args{XZK})(t_B\args{XZK})}=
   X_n\args{l(t_G\args{XZK})(t_B\args{XZK})}=
{\min}_0 \{ x^0 | \lh(x)=l \wedge f_s(x,l)=_00 \}
\text{.}
\]
Let us denote this binary sequence by $b_n$. Using definition of $Z$ we obtain that:
\[
f_s(
  X\args{(t_G\args{XZK})(t_Z\args{XZK})(t_B\args{XZK})} , 
  Z\args{(t_G\args{XZK})(t_Z\args{XZK})(t_B\args{XZK})}
 ) = 0
\text{.}
\]
This implies 
\[
f_s(
  \overline{[X\args{(t_G\args{XZK})(t_Z\args{XZK})(t_B\args{XZK})}]}(n) , 
  Z\args{(t_G\args{XZK})(t_Z\args{XZK})(t_B\args{XZK})}
 ) = 0
\text{.}
\]


\[
t_G\args{XZK}(X\args{(t_G\args{XZK})(t_Z\args{XZK})(t_B\args{XZK})})
\ \wedge\ \lh (X\args{(t_G\args{XZK})(t_Z\args{XZK})(t_B\args{XZK})})=n
\]
by (1).

\end{itemize}
hmky du
\[
t_G(k_0)=_00 \wedge k_1\subseteq k_0
\text{,}
\]
by (1) this implies
\[
f_s(k_0,t_Z(k_0))=_00 \wedge k_1\subseteq k_0
\text{}
\]
%
%
Hence, by lemma \ref{l:Nz2} we get:
\[
f_s(k_1,N_{t_Z\args{XZK}}(k_1))=_00
\text{,}
\]
and using (2) also $t_G(k_1)=_00$. In summary, we have:
\[
\BTree(t_G)
\text{.}
\]
By definition of $f_s$ we have (see also section \ref{ss:spuWKL}):
\[
\forall l^0 \exists k\ (\ \lh(k)=_00 \wedge f_s(k,l)\ )
\text{.}
\]
By lemma \ref{l:Nz3} this implies
\[
\forall l^0 \exists k\ (\ \lh(k)=_00 \wedge f_s(k,N_{t_Z}(k))\ )
\text{.}
\]
So, by (3) and lemma \ref{l:hatg} we get:
\[
 \forall K\ t_G(\overline{t_B}(K\tup t))=_00
\]
It follows by (1):
\[
 \forall K f_s(\overline{t_B}(Kt_Gt_Zt_B), t_Z(\overline{t_B}(Kt_Gt_Zt_B)))=_00
\text{.}
\]
Now, as the last sentence holds for any $K$, it especially holds, given any $K''$, for:
$Kt_Gt_Zt_B:=K''t_At_L$. In other words we can also write:
\[
\forall K\ \ f_s(\overline{t_B}(Kt_At_L), t_Z(\overline{t_B}(Kt_At_L)))=_00
\text{.}
\]
This implies 
\[
\forall K\ \ 
(\ t_L(Kt_At_L) \geq  Kt_At_L \wedge I(\overline{t_B}(Kt_At_L), Kt_At_L, s(t_L(Kt_At_L)))\ )
\text{}
\]
and finally, using the definition of $I$, we obtain (note that $t_A=\widehat {t_A}$; 
see end of section \ref{ss:spuWKL}):
\[
\forall K\ 
(\ t_L(Kt_At_L) \geq  Kt_At_L\  \wedge\ 
|t_A(Kt_At_L)-_\QQ s(t_L(Kt_At_L))|\leq2^{-(Kt_At_L+1)}\ )
\text{.}
\]
\end{proof}
%
%%%%


%%%%%%%%%%%%%%%%%%%%%%%%%%%%%%%%%%%%%%%%%%%%%%%%%%%%%%%%%%%
\subsection {Analysis of the Complexity of the Realizers} \label{ss:acr}
%%%%%%%%%%%%%%%%%%%%%%%%%%%%%%%%%%%%%%%%%%%%%%%%%%%%%%%%%%%

First, we will majorize the realizers similarly as we have done for $\WKL$ 
in section \ref{s:wkl} using there obtained results.\\
\begin{prop}\label{p:majBW}
\begin{align*}
\one&\maj_1 t_B\\
\Phi^* \ldots &\maj t_G\\
\end{align*}
\end{prop}
\begin{proof}
...
\end{proof}\\
In other words, proposition \ref{p:majBW} states that we can use only $\B_{0,1}$
and bounded search to obtain the realizing terms for $\BW$.\\
Depending on the way the $\BW$-principle is used in a given proof of a given
theorem, we get the following results:

\begin{thm}{\em Program extraction for proofs based on an instance of $\BW$.\\} Given a proof 
using $\BW$ on a known sequence of rational numbers given by a well defined term $t_x$
\[
\hrrepa\ +\ \QFm\AC\ +\ \Sigma^0_n\usftext{-IA}\ \ \vdash\ \ 
    \forall x^0(\BW(t_x)\rightarrow\exists y^0 \phi(x,y))
\]
we can extract by ND-interpretation a functional $F\in\T_n$ s.t.
\[
\epa\ \ \vdash\ \ \forall x^0 \phi(x,F(x)) \text{.}
\]
\end{thm}
\begin{proof}
At least for $n=\infty$, i.e. for $\epa$, follows from section \ref{sr} and 
Schwichtenberg \ref{t:Schwichtenberg}.
\end{proof}

\begin{thm}{\em Program extraction for proofs based on the full $\BW$ schema.\\} Given a proof 
using $\BW$ on any given sequence of rational numbers:
\[
\hrrepa\ +\ \QFm\AC\ +\ \Sigma^0_n\usftext{-IA}\ \ \vdash\ \ 
    \forall x^0\ \BW(x)\ \rightarrow\ \forall x\exists y^0 \phi(x,y))
\]
we can extract by ND-interpretation a functional $F\in\T$ s.t.
\[
\epa\ \ \vdash\ \ \forall x^0 \phi(x,F(x)) \text{.}
\]
\end{thm}
\begin{proof}
For $\epa$, same as above.
\end{proof}
