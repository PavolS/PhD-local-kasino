%
% WKL WKL WKL WKL
%%%%%%%%%%%%%%%%%%%%%%%%%
\subsection {Weak K\"onig's Lemma}\label{s:wkl}
The K\"onig's lemma as such is mostly known in the following form:
\theQuote{Every infinite, though finitary branching, tree has an infinite path.}
However, this version is fairly strong and it can be shown 
that it is equivalent to $\CA^0$ if we allow arbitrary formulas for the
decision of belonging to the tree (see e.g. \cite{Troelstra74}).\\
If the labels for branches are bounded
by a function $\alpha$ depending only on the current node {\em and}
the tree is defined by a quantifier-free (or purely universal) formula 
we get a significantly
weaker tool, which is intuitionistically equivalent to $\WKL$, where 
we restrict the lemma to binary trees only (i.e. $\alpha(\cdot)\leq\nolinebreak 1$).
Even so, namely for quantifier-free binary trees, the lemma becomes
equivalent to $\CA^0_{ar}$ if we ask for the left-most or some other concrete
infinite path instead of just some infinite path. This means we need
all three weakenings:
\begin{itemize}
\item The decision criteria, $\phi(x^0)$, for an initial segment
of a sequence of natural numbers, $x$,
 to belong to the tree defined by the characteristic function $f$ must be
a $\PiL$ formula (we can allow one for-all\nbd quantifier since it has no 
essential influence on the structure of the lemma - see definition
\ref{l:WKL-Feferman} and proposition \ref{p:eqWKLs}). If we allow 
arbitrary formulas, we get even full %??? $\AC$. 
$\CA^0$ (regardless of any bounds on the number of branches greater than $1$).
\item For each node, the labels of its branches must be bounded by a well defined type one
function depending only on the height of the node. Otherwise, we get $\PiLm\CA$ and
by iteration $\CA^0_{ar}$ even for quantifier-free decision criteria.
\item There can't be any additional (infinite) demands on the infinite path except for its
existence. 
\end{itemize}
  We consider the following definition:
% Definition \ref{l:WKL-Feferman} follows 
%\cite{AF98} and definition \ref{l:WKL-Kohlenbach} is according to \cite{Kohlenbach08}:
%
% blab la-WKL definition
%
\begin{dfn}[\defkeyn{$\WKL(\phi)$}] \label{l:WKL-Feferman}
For a given $\phi$ we define the following theorem. 
Every infinite binary tree given by the decision criteria $\phi$
 has an infinite path,
\[
\WKL(\phi)\quad:\quad  \BTree(\phi) \wedge \forall k \UnBounded(\phi,k) 
     \rightarrow \exists b\Big(\BFunc(b)\wedge\forall k\ \phi\big(\bar{b}(k)\big)\Big)\text{,}
\]where
\begin{align*}
\BFunc(b)&:\equiv\quad \forall n^0\big(b(n)=_00 \vee b(n)=_01\big)\text{,}\\
\BTree(\phi)&:\equiv\quad \forall s \big(\ 
   \phi(s) 
      \rightarrow
 \ s\in\{0,1\}^{<\omega}
   \wedge 
   \forall t\subseteq s\ \phi(t)\ \big)\text{,}\\
%\Bounded(\phi,k^0)&:\equiv\quad 
%  \forall s\in\{0,1\}^{k}\ \neg\phi(s)\text{,}
\UnBounded(\phi,k^0)&:\equiv\quad 
  \exists s\in\{0,1\}^{k}\ \phi(s)\text{.}
\end{align*}
Furthermore, we define the schema $\Pi_n^0\m\WKL$, as the union of $\WKL(\phi)$, where
$\phi$ is a $\Pi_n^0$ formula. Also, we write $\Pi_n^0\m\WKL(\phi)$ to indicate
that we mean the concrete instance $\WKL(\phi)$ and that $\phi$ is a $\Pi_n^0$ formula.  \\
\end{dfn}
%
%
Note that, for every fixed $n \in \NN$, we can always reformulate
the schema $\Pi^0_n\m\WKL$ as a single $2^{nd}$-order axiom. 
We will use this fact implicitly.
However, in the special case for quantifier-free $\phi$ 
we define explicitly:
\begin{dfn}[\defkey{$\WKL\equiv\forall f\WKL(f)$} see also~\cite{Troelstra74}] 
%\label{l:WKL-Kohlenbach}
\label{l:WKLdelta}
Every infinite binary tree, given by the characteristic function $f$,
 has an infinite path:
\begin{align*}
\WKL(f)\quad:\quad\BTree_K(f) \wedge \forall k\exists x \big(\lh(x)=_0k \wedge f(x)=_0&0\big) 
     \rightarrow\\
 \exists b&\leq_1\one^1\ \Big(\forall k\ f\big(\bar{b}(k)\big)=_00\Big)\text{,}
\end{align*}
where
\begin{align*}
\BTree_K(f^1)&:\equiv 
  \forall x,y \big(f(x*y)=_00\rightarrow fx=_00\big)\ \wedge\ 
  \forall x,n\big(f(x*\langle n\rangle)=_00\rightarrow n\leq_0 1\big)\text{.}
\end{align*}
\end{dfn}
We mentioned earlier that the schema $\PiLm\WKL$ is essentially equivalent
to $\WKL$. this is proved by the proposition~\ref{p:eqWKLs} below, using the fact 
that in all systems used in this thesis there is
a suitable $f\tp 1$ for any quantifier-free $\phi_{\QF}(\tup n)$ with only the type $0$ parameters $\tup n$ free 
s.t. $f(\tup n)=_00\Leftrightarrow\phi_{\QF}(\tup n)$ (and vice-versa).
\begin{prop}\label{p:eqWKLs}
\[\weha\proves\PiLm\WKL\leftrightarrow\WKL\]
\end{prop}
%
% proof
%
\begin{proof}[ (see also~\cite{Simpson99})]
\begin{itemize}
\item $\PiLm\WKL\rightarrow\WKL$:\\
If we have  $\WKL(\phi)$ for $\PiL$ formulas $\phi$, it surely holds for quantifier-free formulas
especially for $\phi:\equiv f(n^0)=_00$.
\item $\PiLm\WKL\leftarrow\WKL$:\\
Since $\phi\in\PiL$, $\phi(s)$ can be written as $\forall n^0\phi_0(s,f,n)$. 
Next we involve two tricks in definition of a quantifier-free version of $\phi$:
\[
\tilde\phi(s,f):\equiv 
  \forall n^0<_0 \lh(s),t\subseteq s(\phi_0(f,t,n)\wedge s(n)\leq_01)
\text{.}
\]
\begin{enumerate}
\item Since we get $\forall n^0<_0 \lh(s) \phi_0(f,t,n)$ for-all $s$ 
in the conclusion of $\WKL$, the number $n$ is in fact unbounded.
\item We guarantee $\BTree(\tilde\phi)$ as a free property for any $\tilde\phi$. 
The tree property by forcing 
\[\forall t\subseteq s\ \phi_0(f,t,n)\]
and the binary property by 
\[\forall n_0<_0 \lh(s),t\subseteq\nolinebreak s\  s(n)\leq_0\nolinebreak 1\text{.} \]
\end{enumerate}
Now, suppose
\[ \BTree(\phi) \wedge \forall k \UnBounded(\phi,k) \] 
i.e. we have (by $\forall k \UnBounded(\phi,k)$):
\[
  \forall k\ \exists s\in\{0,1\}^{k}\ \forall n\ \ \phi_0(f,n,s)\text{.}\tag{+}
\]
By $\BTree(\phi)$ and (+) we get:
\[
  \forall k \exists s\in\{0,1\}^{k} 
     (\forall n^0<_0 \lh(s),t\subseteq s(\phi_0(f,t,n)\wedge s(n)\leq_01))\text{,}
\]
i.e. we have
\[
  \forall k \exists s\in\{0,1\}^{k} \tilde\phi(f,s)\text{,}
\]
and classically also
\[
  \forall k \UnBounded(\tilde\phi,k)\text{.}
\]
By the steps above we showed so far (using trick (2) to get $\BTree(\tilde\phi)$):
\[ 
\BTree(\phi) \wedge \forall k \UnBounded(\phi,k) 
  \quad\rightarrow\quad \BTree(\tilde\phi) \wedge \forall k \UnBounded(\tilde\phi,k) 
\text{.}
\]
We can define in $\weha$ the primitive recursive term for the 
following type one function $g^1$:
\[
g(s):=_0\begin{cases}0^0&\Tif \tilde\phi(f,s)\\1^0&\Telse \end{cases}\text{.}
\]
To get equivalently:
\[ 
\BTree(f) \wedge \forall k \UnBounded(f,k) 
  \quad\rightarrow\quad \BTree_K(g) \wedge \forall k \UnBounded(g,k) 
\text{.}
\]
Applying $\WKL(g)$ to the conclusion we get:
\[
\exists b\leq_1\one^1\forall k\ g(\bar{b}(k))=_00
\]
what is, by definition of $g$, the same as
\[
\exists b\leq_1\one^1\forall k\ \tilde\phi(\bar{b}(k),f)
\text{,}
\]
what is, by definition of $\tilde\phi$, the same as
\[
\exists b^1\leq_1\one^1\forall k^0\ 
\forall n^0<_0 k\forall t\subseteq \bar{b}(k)\ \ \phi_0(f,t,n)
\text{.}
\]
This implies (using trick (1) from the definition of $\tilde\phi$):
\[
\exists b^1\leq_1\one^1\forall k^0\forall n^0\forall t\subseteq \bar{b}(k)
 \phi_0(f,t,n)
\text{.}
\]
Observing that $\forall k\forall t\!\subseteq\!\bar{b}(k)\ \psi(t)$ is equivalent to
$\forall k\psi(\bar{b}(k))$ for arbitrary formulas $\psi$ we finally obtain:
\[
\exists b^1\leq_1\one^1\forall k^0\ (\forall n^0\ \phi_0(f,\bar{b}(k),n))
\quad\equiv\quad \exists b^1\leq_1\one^1\forall k^0\ \phi(f,\bar{b}(k))
\text{.}
\]
\end{itemize}
\end{proof}
\begin{rmk}
Where the necessity of trick (1) in the definition of $\tilde\phi$
is easy to see, one might think that the trick (2) is not needed. This
is not true. The following example shows that in general
we even {\em do not} have $\BTree(\forall n\phi_0(f,s,n))\rightarrow
\BTree(\forall n^0<_0 \lh(s)(\phi_0(f,s,n)\wedge s(n)\leq_01)$.
For some given constant number $c^0\in\NN$, define
\[
\phi_0(f,s,n):\equiv\quad 
  s\in\{0,1\}^{\lh(s)}\ \wedge\ (s(0)=_00\ \vee\ s(0)>c+1+n-2*\lh(s))\text{.}
\]
While $\forall n\phi_0(f,s,n)$ defines the tree consisting exactly of
all binary sequences starting with $0$, the
formula $\forall n^0<_0 \lh(s)\phi_0(f,s,n)$ is true for all such sequences
but also for all binary sequences of length greater than $c$. In other words,
any sequence of length e.g. $c+1$ starting with $1$ is a counterexample
for $\BTree(\forall n^0<_0 \lh(s)(\phi_0(f,s,n)\wedge s(n)\leq_01)$.\\
\end{rmk}

%\subsection{Complexity}
Kohlenbach showed in his work (see \cite{Kohlenbach92}, \cite{Kohlenbach08}), 
that $\WKL$ is intuitionistically equivalent to a sentence in $\Delta$
as defined in theorem \ref{t:mfi}. We sketch this result in \ref{ss:delta}.
This equivalence actually suggests that the lemma will not have any significant
effect on the final complexity of the realizers of sentences whose proofs 
are based on $\WKL$ in common systems.\\
There are several ways to prove this fact. Howard gives a technical
argument using restricted Bar Recursion in \cite{Howard81} which we
follow in \ref{ss:HFI}. 
%Using Kohlenbach's MD-interpretation one quickly obtains
%recursive bounds on the realizers in $\weha$ without Bar Recursion involved. 
%We connect to this approach by majorizing the realizers obtained in \ref{ss:HFI}
%in section \ref{ss:majFI}.

%\todo{Describe how can MD-interpretation simplify the proof that $\WKL$
%does not add anything to $\T_n$.}
%
% is a Delta
%%%%%%%%%%%%%%

\subsubsection*{Kohlenbach's $\WKL'$ as a Sentence $\Delta$}\label{ss:delta}
First, observe that $\WKL$ is equivalent (in $\hrrweha$) to:
\begin{align*}
\WKL_{K1}\quad:\quad
  \forall f,g\ \Big( \BTree_K(f)\wedge\forall k\big(\lh(\bar gk)=_0k\wedge f(\bar gk)=_00\big)
    &\rightarrow \\
  \exists b\leq_1 \one^1\forall &k^0\big(f(\bar bk)=_00\big)\Big)\text{.}
\end{align*}
Next, we define for any type $1$ function $f^1$ the constructions $\hat f$ and $f_g$:
\begin{dfn}\label{d:hatAndG}
\begin{align*}
\hat fn&:=\begin{cases}
  fn &\Tif fn\neq0\ \vee\ 
       \big(\forall k,l(k*l=n\rightarrow fk=0)\wedge \forall i<\lh(n)\ (n_i\leq 1)\big)\\
  1^0 &\Telse \end{cases}\\
f_gn&:=\begin{cases}
  fn &\Tif\ f\big(g(\lh(n))\big)=0\wedge \lh\big(g(\lh(n))\big)=\lh(n)\\
  0^0 &\Telse \end{cases}
\text{.}
\end{align*}
\end{dfn}
\begin{rmk}
We defined already a construction $\hat o$ for a type $1$ object in~\ref{d:hatReal}. It
should be obvious which meaning should be assigned from the context. From now on,
we will use it mostly in the sense of~\ref{d:hatAndG}.
\end{rmk}
Now, we are able to define
\begin{dfn}
\[
\WKL'\quad:\quad \forall f^1,g^1\exists b\leq_1\one^1\forall k^0
  \left(\widehat{(\hat f)_g} (\bar bk)=_00\right)\text{.}
\]
\end{dfn}
To obtain:
\begin{prop}\label{p:wkls1}
The sentence $\WKL'$ is in $\Delta$ as defined in theorem \ref{t:mfi} and:
\[ \weha\proves\WKL\leftrightarrow\WKL' 
\text{.}
\]
\end{prop}
The complete proof of an even stronger result, namely of the equivalence under $\hrrweha$, can
be found in \cite{Kohlenbach08}. In that book, Kohlenbach also proves the following
lemma:
\begin{lemma} \label{l:hatg}
  \setcounter{equation}{0}
  \begin{align}
    \hrrweha\quad&\proves\quad \forall f\ \BTree_K(\hat{f})\\
    \hrrweha\quad&\proves\quad \forall f\ \big(\BTree_K(f)\rightarrow f=_1\hat f\big)\\
    \hrrweha\quad&\proves\quad \forall f,g\forall n\exists x\ \big(\lh(x)=n\wedge f_g(x)=0\big)\\
    \hrrweha\quad&\proves\quad \forall f,g\ 
          \Big(\forall n\big(\lh(gn)=n\wedge f(gn)=0\big)\rightarrow f_g=_1 f\Big)
  \end{align}
\end{lemma}
For the purpose of this thesis mainly 
$\WKL\leftarrow\WKL'$ is interesting:\\
\begin{proof}
Assume $\BTree_K(f)$ and $\forall k \exists n (\lh(n)=k \wedge fn=0))$. Then
\[
\forall k \exists n\leq\overline{\one^1}k (\lh(n)=k \wedge fn=0))\text{.}
\]
Define primitive recursive in $f$:
\[
gk:=\begin{cases}
\min n\leq\overline{\one^1}k\ (\lh(n)=k \wedge fn=0)&\text{if such an $n$ exists}\\
0^0&\Telse
\end{cases} \text{,}
\]
to obtain $\forall k(\lh(gk)=k\wedge f(gk)=0)$.\\
Using lemma \ref{l:hatg} and other Kohlenbach's results 
from \cite{Kohlenbach08} we get $f_g=_1f$ and $f=_1\hat f$.
This proves $\widehat{(\hat f)_g}=_1 f$.\\
Finally $\WKL'$ yields $\exists b\leq_1 \one^1\forall k^0(f(\bar bk)=0)$.\\
\end{proof}\\
Using the fixed construction for $g$ from the proof above, we can define a fixed
term $s$ of $\weha$, and formulate yet another version of $\WKL$:
\begin{dfn}{\defkeyn{$\WKL_s$}$\equiv\forall f\WKL_s(f)$}, where
\[
\WKL_s(f)\quad:\equiv\quad \exists b\leq_1\one^1\forall k^0
  \left(\widehat{(\hat f)_{s}} (\bar bk)=_00\right)\text{.}
\]
\end{dfn}
By the proof of proposition \ref{p:wkls1}, we still have:
\begin{prop}\label{p:wkls2}
 \[ \weha\proves\WKL_s\rightarrow\WKL
\text{.}
\]
\end{prop}

%
% FI WKL -- Howard
%
%%%%%%%%%%%%%%%%%%
\subsubsection*{Howard's D-interpretation of $\WKL$} \label{ss:HFI}
We cannot D-interpret the lemma directly. We could apply
the negative translation to the lemma and then go for the D-interpretation, but
it turns out that the proof is simpler when we go for the ND-interpretation
of a classically equivalent formula, the so called $\FAN$ principle.
In this section, let $f^1$ be a 
characteristic function of a tree, 
$x^0$ and $y^0$ encodings of finite $\{0,1\}$-sequences, and
$b^1$ a $\{0,1\}$-sequence (i.e. $b\in\{0,1\}^\omega$). 
%We call $x$ \defkey{secured} iff $x\not\in f:\equiv f(x)\neq0$ and write \defkeyn{$\Sec(x)$}. 
We define the \defkeyn{$\FAN$} principle as follows (note that the hat construction does not
affect the general meaning - see proposition~\ref{p:wkls1} - however, it does significantly simplify
the terms as it eliminates some quantifiers):
\begin{dfn}[$\FAN\equiv\forall f \FAN(f)$] We define the $\FAN$ principle for a
given function $f$ as
\[
\FAN(f)\quad:\equiv\quad\forall b \exists j\ \Sec(\bar b j) \ \rightarrow \ 
  \exists k \forall x\ \big(\lh(x)\geq k \rightarrow \Sec(x)\big)
\text{,}\]
where $\Sec$ means \defkey{secured}:
\[
 \Sec(x)\quad:\equiv\quad \hat f(x)\neq0 \text{.}
\]
\end{dfn}
In \cite{Howard81}, Howard showed that the D-interpretation of $\WKL$
(using the classically equivalent $\FAN$)
can be obtained using only the so called restricted Bar Recursion
(defined in the same publication, see also definition~\ref{d:rBR}). 
We will present this proof filling in some details 
(the proofs of lemma \ref{l:L1} and lemma \ref{l:L2} 
as well as the witness for the binary sequence $b$).\\
%In fact, if one is interested only in the complexity of the realizing functionals
%there is an even simpler way than the one given by Howard. 
%As $\rB$ can be primitive recursively
%defined in $\nu$, which is trivially majorizable, the realizers are 
%majorized simply by $A*\one$ and we do not need .\\
The secured property is conservative over
extensions, i.e. for any two finite binary sequences $x$ and $y$ we 
have:
\[
x\subseteq y \rightarrow\big(\Sec(x)\rightarrow \Sec(y)\big)\text{.} \tag{SC}
\]
Hence, the conclusion of $\FAN(f)$ is equivalent to a 
$\SiL$ formulation (as the for-all quantifier 
can be bounded):
\[
\forall b \exists j\ \Sec(\bar b j) \ \rightarrow \ \exists k 
\underbrace{\forall x\in\{0,1\}^k\ \Sec(x)}_{\text{quantifier-free}}
\text{.}\]
Moreover, in the presence of the Markov principle,
 the negative interpretation does not change the formula. 
Already in intuitionistic logic,
$\neg\neg (A\rightarrow B)$ is equivalent to $A\rightarrow\ \neg\neg B$.
So, the negative interpretation becomes
\[
\forall b\ \neg\neg\ \exists j\ \Sec(\bar b j) \ 
       \rightarrow \ \neg\neg\big(\exists k \forall\ x\in\{0,1\}^k\ \Sec(x)\big) 
\text{.}\]
By the Markov principle we get
\[
\forall b\ \exists j\ \neg\neg \Sec(\bar b j) \ 
       \rightarrow \ \exists k\ \neg\neg\forall\ x\in\{0,1\}^k\ \Sec(x)
\]
and because of the stability of quantifier-free formulas modulo double negation, 
we get just the very same sentence we started with. 
%In other words $\FAN(f)$ retains the same form under negative translation 
%(modulo intuitionistic equivalence).
In summary we have:
\begin{align*}
\wepa &\proves\ \WKL\leftrightarrow\FAN \text{,}\\
\weha &\proves\ (\WKL)'\leftrightarrow(\FAN)' \text{,}\\
\weha+\M &\proves\ \FAN\leftrightarrow(\FAN)' \text{,}\\
\weha+\M &\proves\ (\WKL)'\leftrightarrow\FAN \text{,}\\
\end{align*}
and therefore also:
\[
\weha+\M \proves\ (\WKL)^{ND}\leftrightarrow(\FAN)^D \text{.}\\
\]
So $\FAN$ is precisely the form of $\WKL$ we want to D-interpret.\\ 
By quantifier-free Axiom of Choice $\QFm\AC^{1,0}$ we get 
from $\forall b \exists j\ \Sec(\bar b j)$:
\[ \exists A \forall b \ \Sec\big(\bar b(Ab)\big) \text{.}\]
Now, define $K_A$ as follows:
\[
K_Ax:=\KA{A}{x}
\text{.}
\]
Further define:
\begin{align*}
\BSec(k,x) &{:\equiv}
\forall y\ \Big(\big(x\subseteq y \wedge \lh(x)+k=\lh(y)\big)\ \rightarrow\ \Sec(y)\Big) \text{,}\\
\BSecA(x) &{:\equiv} \BSec(K_Ax,x) \text{.}
\end{align*}
\begin{rmk} $\BSec(k,x)$ is the predicate for: Every finite extension of $x$ with the length $\lh(x)+k$ 
is secured.\end{rmk}
We will make use of the following two lemmas (in both cases we consider some functional $A$ 
for which $\forall b \ \Sec(\bar b(Ab))$ holds, moreover, w.l.o.g. it may be assumed that $A$
is computable since it was obtained by $\QF\m\AC^{1,0}$):
\begin{lemma}
\label{l:L1}
\[ A[x] < \lh(x) \rightarrow \BSecA(x) \]
\end{lemma}
\begin{proof}
We assume:
\[ A[x] < \lh(x) \tag{*} \text{.}\]
By definition of $K_A$ we get:
\[ K_Ax=0  \tag{**} \text{.}\] 
From $\forall b \ \Sec(\bar b(Ab))$ we get
\[ \Sec(\overline{[x]}(A[x]))\text{.}\]
By (*) we know that
\[
\overline{[x]}(A[x])\subseteq x
\] hence by (SC) it follows
\[ \Sec(x) \text{.}\]
Using (**) and the definitions of $\BSec$ and $\BSec_A$ we get:
\[ \Sec(x) \equiv \BSec(0,x) \equiv \BSecA(x) \text{.}\]
\end{proof}

\begin{lemma}
\label{l:L2}
\[ \BSecA(x*0) \wedge \BSecA(x*1) \rightarrow \BSecA(x) \]
\end{lemma}
\begin{proof}
Assume:
\[ \BSecA(x*0) \wedge \BSecA(x*1)  \tag{*} \text{.}\]
If $A[x]<\lh(x)$, then by lemma \ref{l:L1} we have $\BSecA(x)$. 
So, using the definition of $K_A$ we may assume w.l.o.g:
\[ K_Ax=1+max\{K_A(x*0),K_A(x*1)\} \text{.}\]
Obviously we have:
\[ K_Ax > K_A(x*0)\ \wedge\ K_Ax>K_A(x*1)\text{.}\]
So by (*) we know that for some number $m\leq\lh(x)+K_A(x)$ all extensions 
of $(x*0)$ and $(x*1)$ of the fixed length $m$ are secured. Therefore all extensions
of $x$ with fixed length $m$ are secured. However, this is 
expressed as $\BSec(K_Ax,x)\equiv \BSecA(x)$.\\
\end{proof}

Since $\BSecA(x)$ can be written as a quantifier-free formula for any given $x$, and
hence is primitive recursive in $x$, 
we can use primitive recursion on the contrapositive of lemma \ref{l:L2}:
\[ \neg \BSecA(x)\ \rightarrow \ \neg \big(\BSecA(x*0)\wedge \BSecA(x*1)\big) \text{,}\]
to obtain a function $g\leq\one$, s.t.
\[ \neg \BSecA(\emptyset) \rightarrow \forall j \neg \BSecA(\bar gj) \text{.}\]
However, we can prove the existence of a $j$ s.t. $\BSecA(\bar gj)$, 
using only restricted Bar Recursion, and thereby proving $\BSecA(\emptyset)$.
To do so, we follow another publication from Howard, namely \cite{Howard68},
where he proves an, by lemma~\ref{l:L1}, even stronger lemma 
(for the same kind of functional $A$ as above):
%
%  The big Lemma 3C L3C L3C L3C
%
%%%%%%%%%%%%%%%%%%%%%%%%%%%%%%%%%%%%
\begin{lemma}\label{l:3c}
By restricted bar recursion plus primitive recursion, 
we can define $\theta_A$, s.t. for 
all $b$ it holds:
\[\exists k\leq \theta_Ab \emptyset\ \ \ A[\bar b k]<k\text{.}\]
\end{lemma}
%
\begin{rmk}
In fact the functional $\theta_A$ is given simply by 
$\theta_Ab\emptyset:=\mu k.A[\bar bk]\leq k$. 
Since we have to work in a model justifying bar recursion
and, as discussed e.g. by Kohlenbach in~\cite{Kohlenbach08}, we can define the $\mu$-operator
via bar recursion this functional is well defined in all such models, e.g. $\Cont$ or $\Maj$. 
Howard's proof of lemma~\ref{l:3c} is analyzed in the first appendix of \cite{Safarik08}. It is
rather technical and not essential at this point.
\end{rmk}
%

Lemma \ref{l:3c} especially implies $\forall g^1\!\leq\!\one$ $\exists k^0$  $A[\bar gk]<k$, which
implies $\forall g^1\!\leq\!\one$ $\exists k^0$ $\BSecA(\bar gk)$ by lemma \ref{l:L1}. 
So, as mentioned above, using this result we obtain:
\[ \BSecA(\emptyset)\ \leftrightarrow \BSec(K_A\emptyset, \emptyset) \text{.}\]
This is, by definition, just another form of:
\[ \forall x\ \Big(\big(\emptyset \subseteq x \wedge K_A(\emptyset)=\lh(x)\big) \rightarrow \Sec(x)\Big) \text{.}\]
So, finally we obtain:
\[  \forall x\ \big(\lh(x)\geq K_A(\emptyset)\  \rightarrow \Sec(x)\big) \text{.}\]
What completes the proof that $K_A(\emptyset)$ is the ND-realizer
of $k$ in $\FAN(f)$.\\

To complete the solution of the D-interpretation we still have to 
find the functional $B$ which satisfies:
\[
\forall A,x\ \Big(\ 
\Sec\big(\overline{B}(AB)\big)\ \rightarrow\ 
    \big(\lh(x)\geq K_A(\emptyset)\rightarrow \Sec(x)\big)
\ \Big)
\text{.}\]
Fortunately, this is easily done. Let $y^0$ encode either a binary sequence 
of length $K_A(\emptyset)$ which is in the tree or, if such a sequence doesn't exist, the empty
sequence. Now we can define the functional $B$ as the infinite extension of this sequence, $[y]$.
See also the theorem~\ref{t:FIwkl} below.\\
The same functionals interpret the standard (positive) formulation of $\WKL$, 
which is just the contrapositive of 
the outer implication:
\[
\forall A,x\ \ (\ (\lh(x)\geq K_A(\emptyset)\wedge x\in f) \rightarrow\ 
 f(\overline{B}(AB))=0\ )
\text{.}\]
{\samepage
We summarize:
\begin{thm}[{The ND-interpretation of $\WKL(f)$}]\label{t:FIwkl-nd}
The Weak K\"onig's lemma for binary trees
given by an arithmetic characteristic function $f$ 
\[
  \BTree(f) \wedge \forall k \UnBounded(f,k) 
     \rightarrow \exists b\Big(\BFunc(b)\wedge\forall k\ f\big(\bar{b}(k)\big)=0\Big)
\]
is, provably in $\weha+\QF\m\AC^{1,0}+\rB_\one$, ND-interpreted as follows:
\[
\forall A,x\ \ \Big(\ \big(\lh(x)\geq K_A(\emptyset)\wedge x\in f\big) \rightarrow\ 
 f\big(\overline{B}(AB)\big)=0\ \Big)
\text{,}\]
where
\begin{align*}
K_Ax&:=\KA{A}{x} \text{,} \\
B(A)&:=\big[F_f\big(K_A(\emptyset)\big)\big] \text{,} \\
F_f(n)&:=\Fn{f}
\text{.}
\end{align*}
In particular, these terms define total functionals in $\Maj$ and $\Cont$.
\end{thm}
} %samepage
\subsubsection*{Majorants for Howard's Solution} \label{ss:majFI}
First, note that any term $t$ majorized by a term $t^*$ has at most the complexity
of $t^*$ since it can be obtained by bounded search up to $t^*$. As the complexity
of the realizers in theorem \ref{t:FIwkl-nd} is not obvious it might be interesting
to examine suitable majorants for these realizers:%\\
%Since $F$ is majorized simply by the constant function $\one^1:=_1\lambda n^0.1^0$,
%the interesting part is to find a suitable majorant for $K_A$. 
% Furthermore
%we have:
\begin{thm}
%For any length $l\in\NN$, for all $x^0\in\{0,1\}^l$:
%For all $x$:
The solution of the ND-interpretation 
of $\WKL(f)$ is, provably in $\weha+\QF\m\AC^{1,0}+\rB_\one$, majorized as follows:
\setcounter{equation}{0}
\begin{align}
K^*:=_{(10)2}\lambda A^2,x^0.A^2\one\ &\maj_{(10)2}\ K\\
B^*:=_{1(2)}\lambda A^2. \one^1 &\maj_{1(2)}\ B
\text{.}
\end{align}
Again, these terms define total functionals in $\Maj$ and $\Cont$.
\end{thm}
\begin{proof}
\paragraph{(2)}
Follows directly from the fact:
\[
\lambda n^0,x^0.\overline{\one^1}(\lh(x)+n) \maj_{1(1)} F\text{.}
\]
\paragraph{(1)}
According to definition \ref{d:maj}-(2) suppose $A^*$ majorizes $A$. 
%Then $A[x]$ is majorized by $A^*\one$. Since $A^*$ is 
%a well defined type $2$ functional, we get $\exists n^0\ A^*\one=_0n$.
%We denote this $n$ by $n_A$. 
We get immediately from the definition of $K_A$ and
$A^* \maj_2 A$
\[
\forall s\in\{0,1\}^{A^*\one}\forall x\subseteq s\ \ K_Ax\ \leq_0\ A^*\one-\lh(x)\text{.}
\]
In other words, for any finite binary sequence $x$ we have:
\[
\lh(x)\leq_0A^*\one \rightarrow K_Ax\leq_0A^*\one\text{.}
\]
Now, suppose $\lh(x)>A^*\one$ for some $x$. 
From $A^* \maj_2 A$ we get $\lh(x)>A[x]$ 
and we have immediately from the definition of $K_A$ that $K_Ax=_00\leq_0A^*\one$.\\
So finally, we obtain: $K_Ax\leq_0A^*\one$, what is the same as
$A^*\one \maj_0 K_Ax$ for all finite binary sequences $x$.\\
\end{proof}\\
%\begin{rmk}
%Note that the majorant $A^*$ for $A$ is crucial, as $A\one$ does not necessary majorize
%$K_Ax$ in all $x$ since in general not even $A \maj_2 A$, take e.g. (for any constant number 
%$0<c^0\in\NN$):
%\begin{align*}
%A[x]&:=c*(1-[x](1))+c*(1-[x](2))+c*(1-[x](3))+\ldots+c*(1-[x](c))\\
%x&:=\langle0,0\rangle\text{,}
%\end{align*}
%and we get $\one^1\maj_1[x]\quad\wedge\quad 0=_0A\one^1 <_0 A[x]=_0c*c$. 
%Of course we get a very different functional
%$A$ as it has to satisfy $\forallb \Sec(\bar b(Ab))$, 
%but this is no reason to guarantee the monotonicity of such an $A$ in full generality.
%\end{rmk}
%So finally we can conclude:\\
%\begin{thm}
%The realizing functionals of $\WKL$ are majorized by closed terms of $\weha$ and
%\end{thm}
\subsubsection*{Summary}
Above, we discussed at length the $\WKL$ principle purposefully following the notation
as used by Kohlenbach in~\cite{Kohlenbach08}, which would unfortunately 
clash with the remainder of this thesis. Therefore, let us here briefly summarize the important
results regarding $\WKL$ using a more compatible notation. Also, we resort directly to Shoenfield
interpretation.

\begin{dfn}{{$\WKL_\Delta$}$\equiv\forall f\WKL_\Delta(f)$},
\label{d:frown_g} where
\[
\WKL_\Delta(f)\quad :\equiv\quad\exists b^1\forall k^0\ 
  \widefr{\Big(\!\text{\widefr{f}}\Big)_{\!g}} (\bar bk)=_00\text{, with}
\]
\begin{align*}
\widefr{f}n&:=\begin{cases}
  fn &\Tif \ fn\neq0\ \vee\ 
       \big(\forall k,l(k*l=n\rightarrow fk=0)\wedge \forall i<\lh(n)\ 
(n_i\leq 1)\big)\\
  1^0 &\Telse, \end{cases} \\
f_gn&:=\begin{cases}
  fn &\Tif\ f\big(g(\lh(n))\big)=0\wedge \lh\big(g(\lh(n))\big)=\lh(n)\\
  0^0 &\Telse, \end{cases} \\
gk:=g f k&:=\begin{cases}
\min n\leq\overline{\one^1}k\ (\lh(n)=k \wedge fn=0)&\text{if such an $n$ 
exists}\\
0^0&\Telse,
\end{cases}
\end{align*}
where for any given number theoretical function $f^1$, 
$\widefr{f}$ assigns a unique characteristic function of a $0/1$-tree\footnote{
If $f$ was such a characteristic function already, then it is not modified at all (i.e. we would have $\widefr{f}=f$).}
and $f_gn$ adds the full subtree if there is no path of length $n$\footnote{
Again, if $f$ defined an infinite tree already, then it is not modified at all (i.e. we would have $f_g=f$).}
(this may destroy the tree property of $f$ if present). 
The function $g$ simply looks for a path of length $n$ and retruns $0$ if none exists (otherwise the code of the path itself is returned).
\end{dfn}
We have: $\hrrweha\proves\WKL_\Delta\leftrightarrow\WKL$. \\
Howard proves in \cite{Howard81} that one can give the realizing 
functionals for the Sh-interpretation of $\WKL$ using only restricted bar 
recursion and $\T_0$. This proof is
discussed in great detail above and we use
it to obtain the Sh-interpretation of $\WKL(f)$:
\begin{thm}[{The Sh-interpretation of $\WKL_\Delta$}]\label{t:FIwkl}
The Weak K\"onig's lemma for binary trees
\[
  \forall f\exists b^1\forall k^0\ 
  \widefr{\Big(\!\text{\widefr{f}}\Big)_{\!g}} (\bar bk)=_00
\]
is, provably in $\weha+\rB_\one$, Sh-interpreted as follows:
\[ \forall f,A\ \exists b^1\  \widefr{\Big(\!\text{\widefr{f}}\Big)_{\!g}} 
\big(\overline{b}(Ab)\big)=0
\text{,}\]
where $b$ is realized by $b:=_1B^{^\WKL}Af$:
\begin{align*}
B^{^\WKL}(A,f)&:=\big[g\Big(\widefr{\Big(\!\text{\widefr{f}}\Big)_{\!gf}}
\Big) \Big(K^{^\WKL}
(A,\emptyset)\Big)\big] \text{,}  \\
K^{^\WKL}(A,x)&:=\KA{A}{x}  
\end{align*}
where $g$ is the same term as we used in the 
Definition \ref{d:WKLdelta}\footnote{We define $[\cdot]$ analogously 
for codes of sequences
as we did for sequences themselves.}.
Note that $K^{^\WKL}$ is definable by $\rB_\one$.
\end{thm}
\begin{prop}
The solution of the Sh-interpretation 
of $\WKL(f)$ is, provably in $\hrrweha+\rB_\one$, majorized as follows:
\setcounter{equation}{0}
\[
K^*:=_{1(0)(2)}\lambda A^2,x^0.A^2\one\ \maj_{1(0)(2)}\ K^{^\WKL}, \quad
B^*:=_{1(2)}\lambda A^2. \one^1 \maj_{1(2)}\ B^{^\WKL}
\text{.}
\]
\end{prop}







