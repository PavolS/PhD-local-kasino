\section*{Conclusion}\label{s:conclusion}
\markright{\textsc{Conclusion}}
\addcontentsline{toc}{section}{Conclusion}

%We presented an in-depth analysis of
%the ND-interpretation of various versions of the arithmetical comprehension,
%giving the optimal realizers based on Spector's interpretation of a
%special case of $\DNS$ sufficient to interpret $\PiLm\CA$.\\
%We expanded W. A. Howard's investigation of the bar-recursive interpretation
%of $\WKL$ by interpreting even $\Pi^0_2\m\WKL$. Also, we filled some minor steps in 
%Howard's proof for $\WKL$.\\
%We combined these principles in a simple proof of $\BW$ based on $\Pi^0_2\m\WKL$
%obtaining the realizing terms, again, in optimal computational
%complexity and full detail. Moreover, by using the NMD-interpretation
%we give the very managable majorants for these rather complicated explicit terms.\\
%%
%Another main result, which shows that the function(al) realizing a
%$\Pi^0_2$-proposition proved in one of the systems $\hrrwepa\ +\ \QFm\AC\ +\ \Sigma^0_n\usftext{-IA}$ ($n\geq1$)
%using a concrete instance of $\BW$ can be defined by a closed term in $\T_n$ (while 
%$T_{n-1}$ would be sufficient if the propostion was proved directly, i.e. without $\BW$),
%is stated in theorem~\ref{t:PEfBW}.
%% The main result is given in theorem~\ref{t:PEfBW}, where we state that using
%% a concrete instance of $\BW$ in a proof under the weak 
%% systems $\hrrwepa\ +\ \QFm\AC\ +\ \Sigma^0_n\usftext{-IA}$ one can extract
%% function(al)s in $\T_n$ realizing the statement being proved.
%Note that this is optimal as a suitable instance of $\BW$
%is sufficient to obtain any given concrete instance of $\Pi^0_1\m\CA$ 
%(or, equivalently, $\Sigma^0_1\m\CA$)
%and thereby increases the number of quantifiers allowed in 
%the induction of the system used to prove the 
%original statement by one so forcing us to provide the appropriate recursor of higher
%type.\\
%Finally, we put this result in contrast to the case where $\BW$ is used in
%a proof in its full $2^{nd}$ order closure, theorem~\ref{t:PEBW}.

%\subsection*{Achievements}
%\subsection*{Applications}
%\subsection*{Open Questions}

