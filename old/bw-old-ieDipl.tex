\section{Interpreting Bolzano-Weierstra{\ss}} \label{s:bw}
\reviewQuote{bwint2010/review_bw}{KS10}{Nov 23, 2009}
\paragraph{Motivation}
In \cite{Tao07}, Tao discusses the \defkey{Infinite Pigeonhole Principle}:
\[
\IPP\quad:\quad\forall n\in\NN\forall f:\NN\mapsto\{1, \ldots, n\}\exists i\leq n
   \forall k\in\NN\exists m\geq k\ \ f(m)=i
\text{,}
\]
%is ``{\em basically non-finitisable}''. 
and gives a complicated finitary version without supplying bounds. \todo{Cite Jaime}
Similarly as in the case of the finite convergence principle, which is also treated by Tao,
this finitary version can be viewed simply as a specific
instance of the (monotone) G\"odel functional interpretation of $\IPP$ (see 
below or the definitions \ref{d:FI} and \ref{d:mfi} for the G\"odel functional interpretation and
its monotone version). However, in~\cite{tao-2007}, Tao
claims in remark A.4:
\theQuote{ In principle, the quantity ... can be explicitly computed ... In practice, 
though, it seems remarkably
hard to do; the proof of the Lebesgue dominated convergence theorem, if
inspected carefully, relies implicitly on the infinite pigeonhole principle, which is
notoriously hard to finitise. Indeed the situation here is somewhat reminiscent of
that of the Paris-Harrington theorem. ...  it was established ... that the Lebesgue 
dominated convergence theorem is equivalent in the
reverse mathematics sense to the arithmetic comprehension axiom ($\ACA$), which
does strongly suggest that the dependence ... on the above parameters is
likely to be fantastically poor.}
While P. Oliva derives effective
bounds for the functionals realizing the $\IPP$ in 
\cite{Oliva06}, we apply the G\"odel functional interpretation, mentioned by Tao, 
to finitise the Bolzano-Weierstra{\ss} theorem:{\samepage
\theQuote{ Any bounded sequence $(a_n)_{n\in\NN}$ in $\RR^d$ has a limit point. } }
Note that $\IPP$ is just a very special case of this theorem for the space $\{0, \ldots, n\}$.\\
 
The general task of this chapter is to investigate upper bounds for the computational complexity of the
functionals realizing the proofs based on (instances of) the Bolzano-Weierstra{\ss} theorem.
This theorem is one of the principles based on the so called \defkey{sequential compactness}
in the sense mentioned e.g. in \cite{KO02} or \cite{Kohlenbach98}.
This means principles like:
\begin{itemize}
\item  \defkey{\PCM}\ \ \emph{Principle of monotone convergence} $:\equiv$\\
        A non-increasing sequence of reals $(a_n)_{n\in\NN}$ bounded
        from below converges.
%      
\item  \defkey{\BW}\ \ \emph{Bolzano-Weierstra{\ss} principle} $:\equiv$\\
  Any bounded sequence $(a_n)_{n\in\NN}$ in $\RR^d$ has a limit point and 
therefore a convergent subsequence.
%
\item \defkey{\AA}\ \ \emph{The Arzel\`{a}-Ascoli lemma} $:\equiv$\\
  Any sequence $(f_n)_{n\in\NN}\in C[0,1]$ of equicontinuous and uniformly bounded functions 
  has a limit point and therefore a convergent subsequence (with respect to $\|\cdot\|_\infty$).
% 
\item \defkey{\Limsup}\ \ \emph{The existence of limes superior} $:\equiv$\\
For any sequence $(a_n)_{n\in\NN}\in[0,1]$, there exists a maximal limit point $x\in[0,1]$.
%
\end{itemize}
%
%
\paragraph{Simpson's $\ACA_0$.}
%All these principles, when applied in their full generality to arbitrary sequences, 
%imply arithmetical comprehension. 
Simpson showed in his 
program of Reverse Mathematics that large parts of Analysis, in particular each of these principles,
 are equivalent to arithmetical comprehension over $\hrrwepa+\QF\m\AC$, see \cite{Simpson99}.
The system  $\hrrwepa+\QF\m\AC+\CA^0_{ar}$ corresponds to the finite type extension 
of the system $\ACA_0$ and, similarly, the system $\wepa+\QF\m\AC+\CA^0_{ar}$
corresponds to the finite type extension 
of the system $\ACA$ as defined in \cite{Simpson99}.
%
\paragraph{Spector's interpretation of classical Analysis.}
Spector showed also that the use 
of arithmetical comprehension can be reduced
to arithmetical comprehension for purely universal sentences, $\PiLm\CA$, see \cite{Spector62}. 
It seems only natural to use his
approach and apply the
bar recursion to $\PiLm\CA$.
%
\paragraph{Kohlenbach's (instantiated) program extraction.}
Kohlenbach and Oliva presented bounds for the bar-recursive realizers of the 
first principle, $\PCM$, in the last section of \cite{KO02}.
Moreover, the authors analyze concrete instances of $\PCM$, thereby showing that
bar-recursion is needed only if we truly need the full power of this principle. 
However, in general to prove a statement one would apply $\PCM$ to a specific
sequence, i.e. given by a closed term.
Also, in spirit of Kohlenbach's main theorems on program extraction 
(see \cite{Kohlenbach08}),  they show that even if we use the principle in its full generality
to prove a theorem which has the form 
$\forall x^1\forall y^1\leq\tilde sx\exists z^0 \phi_{_\QF}(x,y,z)$, 
where again $\phi_{_\QF}$ is a quantifier-free formula,
the bar-recursive realizers are still bounded by some primitive recursive
function in G\"odel's $\T$ (see definition \ref{d:GT}). We will derive
similar  results for $\BW$ in this thesis.
%
\paragraph{Fixed instances in weaker systems.}
The contribution of instances of all these principles 
have been analyzed by Kohlenbach in \cite{Kohlenbach98} for proofs in somewhat 
weaker base systems: the Grzegorczyk arithmetic of level $n$, \defkeyn{$\GA$}. Again,
for fixed instances, Kohlenbach
proved the extractability of primitive recursive bounds from the proofs of
$\forall\exists$-theorems based on instances of
 any of the first three principles: $\PCM$, $\BW$, and $\AA$. However, 
even an instance of $\Limsup$ may create the growth of the Ackermann rate.
%

\paragraph{The main question.}
Even though it may be interesting to see the actual terms realizing
the ND-interpretation of $\BW$ it is hardly a contribution to
general proof-mining knowledge. The interesting question in analogy to
Kreisel's formulation, though put in a simpler way, could be stated as follows:
\theQuote{ What are the costs of the use of $\BW$ in a proof? }
%More scientifically formulated the main question should sound rather like this:
or, formulated more elaborately:
\theQuote{ What is the calibration of the contribution of the uses of sequential compactness in proofs? }
We use Kohlenbach's NMD-interpretation to obtain more readable terms and
take following conclusions with the help of
Howard's, Parsons's, and Tait's results (see section~\ref{s:compPM}):
\begin{itemize}
\item The ND-interpretation of $\forall\exists$-theorems based on fixed instance 
of $\BW$ relative to 
$\hrrwepa\ +\ \Sigma^0_n\usftext{-IA}$, for $n\geq1$, yields a $\T_n$ definable
bound, see~\ref{t:PEfBW}.
\item The ND-interpretation of $\forall\exists$-theorems based on full 
 $\BW$ relative to 
$\hrrwepa\ +\ \Sigma^0_\infty\usftext{-IA}$, yields
a $\T$ definable bound, see~\ref{t:PEBW}.
\end{itemize}

\todo{get some nice bridge here into the technical stuff}}
 
First, we define a simple primitive recursive operation, which restricts
rational and real numbers to rational numbers and real numbers in the unit interval.
%
\begin{dfn}
We define:
\begin{align*}
\widetilde{q^0}:&=_\QQ{\min}_\QQ\big(1_\QQ,{\max}_\QQ(0_\QQ,q)\big) \text{,}\\
\widetilde{x^1}:&=_\RR{\min}_\RR\big(1_\RR,{\max}_\RR(0_\RR,x)\big) \text{,}
%\widetilde{f^1(0)}:&=_\RR\min _\RR\left(1_\RR,\max _\RR(0_\RR,r)\right) \text{,}
\end{align*}
where $(\max_\RR(x^1, y^1))n^0$ is defined via $\max_\QQ(\hat x(n),\hat y(n))$, 
with $\hat x$, $\hat y$ defined as in~\ref{d:hatReal}. 
\end{dfn}
Note that $\tilde x$ defines 
a unique real number within the real unit interval even for those $x^1$,
which do not represent a real number.
%
Using this operation, we can formulate the Bolzano-Weierstra{\ss} principle 
formally in ${\mathcal L}(\weha)$ as follows (the on the first glance unintuitive
formulation of $\BW$ is motivated by proposition \ref{p:BWimpBWR} below):
\begin{dfn}\label{d:BWfinal}
\defkey{The Bolzano-Weierstra{\ss} Principle}: $\BW_{(\RR)}$\\
%Let  $s^1$ be a sequence of encodings of rational numbers.
%within the rational unit interval $[0,1]_\QQ$ given by
%the type $1$ function $s:\NN\mapsto\QQ$ s.t. $\forall n^0\ 0_\QQ \leq_\QQ s(n)\leq_\QQ 1_\QQ$.
For a sequence of encodings of rational numbers we define:
\[
 \defkeyn{\BW}\quad:\quad
  \forall s^1  
   \underbrace{\exists a^1\forall k^0\exists l^0\geq_0 k
     \ |\hat a(k+1)-_\QQ \widetilde {\ sl\ }|\leq_\QQ\langle2^{-(k+2)}\rangle }_{ \equiv:\BW(s^1)}
\text{,}
\]
where $\hat a$ is defined as in~\ref{d:hatReal}.
We give also a more general formulation (where we consider a sequence of real numbers instead of
a sequence of rational numbers):
\[
% \defkeym{\BW_\RR}\quad:\quad
 \BW_\RR\quad:\quad
  \forall f^{1(0)}
      \underbrace{ \exists a^1\forall k^0\exists l^0\geq_0 k
           \ |\hat a-_\RR\widetilde {fl\ }|\leq_\RR (\lambda n.\langle 2^{-k}\rangle) }_{\equiv:\BW_\RR(f^{1(0)})}
\text{.}
\]
And, finally, we give also a version for sequences in $[0,1]^d\subset\RR^d$ (see \cite{Kohlenbach98} for
representation of $\RR^d$ and the corresponding unit cube):
\[
% \defkeyn{\text{$\BW^d_\RR$}}\quad:\quad
 \BW^d_\RR\quad:\quad
  \forall x^{1(0)}_1,\ldots,x^{1(0)}_d
      \underbrace{ 
         \exists a^1_1,\ldots,a^1_d\forall k^0\exists l^0\geq_0 k
           \ \bigwedge^d_{i=1}|\hat a-_\RR\widetilde {x_il\ }|\leq_\RR (\lambda n.\langle 2^{-k}\rangle) 
      }_{\equiv:\BW^d_\RR(\tup x^{1(0)})}
\text{.}
\]
%where \defkeyn{$\tilde{f^1}$} is defined
%for any type $1$ function $f$ as $\tilde fn^0:=_\QQ\min_\QQ\left(1_\QQ,\max_\QQ(0_\QQ,fn)\right)$ 
%(note that for any $s$ fulfilling the conditions 
%above we have $\forall n^0 s(n)=_\QQ\tilde s(n)$).
\end{dfn}

First, note that a modification of the exponent in $\leq 2^{-(k+c)}$ by a constant $c^0$ does
not significantly change the statement. The result is merely a slightly
 different, but still valid, formalization of the well known Bolzano-Weierstra{\ss} theorem.\\
Obviously, if Bolzano-Weierstra{\ss} theorem holds for sequences in $[0,1]^d\subset\RR$ it holds 
for sequences of real numbers and for sequences of rational numbers as well. Moreover, we have:
\begin{prop}\label{p:BWimpBWR}
\[
\weha\ \proves\ \forall f^{1(0)}\ \left(\ \BW(\lambda n^0.\widehat{fn\ }(n+2))\rightarrow\BW_\RR(f) \right)
\text{.}
\]
\end{prop}
\begin{proof}
Consider any given $f^{1(0)}$. By definition \ref{d:BWfinal} the assumption $\BW(\lambda n^0.\widehat{fn\ }(n+2))$ means
\[
\exists a^1\forall k^0\exists l^0\geq_0 k
     \ |\hat a(k+1)-_\QQ \widetilde {\widehat{f\args{l\ }}\args{\!(l\!\args{+\!2})}}|\leq_\QQ\langle2^{-(k+2)}\rangle
\text{.}
\]
From the definitions \ref{d:hatReal}, \ref{d:Rel}, and \ref{d:Rpm}
one can conclude that for any type $1$ object $x$ it holds (for brevity, for a rational number
encoded by $q$  and a type one function $x$, 
we will write $q-_\RR x$ instead of $\lambda m^0.q-_\RR x$): 
\[
\forall n^0\ |\lambda m^0.\hat x(n)-_\RR x|\leq_\RR (2^{-n})_\RR
\text{, }
\]
see also \cite{Kohlenbach08}. 
Using this fact twice, to get $|\hat a(k+1)-_\RR a|\leq_\RR (2^{-k-1})_\RR$ 
and $|\widehat {fl\ }(l+2)-_\RR f|\leq_\RR (2^{-l-2})_\RR$, we
obtain:
\[
\exists a^1\forall k^0\exists l^0\geq_0 k\left(
     \ |\hat a-_\RR \widetilde {fl\ }|\leq_\RR (2^{-k-2}+2^{-k-1}+2^{-l-2})_\RR
     \leq_\RR(2^{-k})_\RR\ \right)
\text{,}
\]
where the last inequality is due to $l\geq k$.\\
\end{proof}
%

% space needed

%
%Similarly one obtains for a sequence in $\RR^d$ (with $\tup x^\rho=x^\rho_1,\ldots,x^\rho_d$):
%\[
%\forall\tup x^{1(0)}\left(               
%   \BW(F\tup x)\rightarrow\BW^d_\RR(\tup x)
%\right)
%\]
%for a primitive recursive functional $F\tp2$ (the precise type of $F$ obviously depends on $d$).
%A direct proof of $\forall \tup x^{1(0)}(\PiLm\CA(G\tp2\tup x)\rightarrow \BW^d_\RR(\tup x))$ for
%a suitable primitive recursive $G\tp2$ was
%carried out by Kohlenbach in \cite{Kohlenbach98}, moreover any instance $\PiLm\CA(f^{1(0)}$ is implied
%by the instance $\BW(F_\forall f)$ of $\BW$ (see proof of proposition \ref{p:optPEfBW}).
%This means that form now on we can consider $\BW(s)$ only.\\
To prove $\BW(s)$, we will use two common principles: $\CA$ and $\WKL$. 
Their respective interpretations can be found in previous sections. Recall:
\begin{align*}
 \SiLm\CA(f)\quad&:\quad
    \exists g \forall x\ (gx=_00 \leftrightarrow \exists y\ fxy=_00)\\
 \WKL(f)\quad&:\quad
  \BTree_K(f) \wedge \forall k\exists x (\lh(x)\!\!=_0\!k \wedge f(x)\!\!=_0\!0) 
     \rightarrow \exists b\!\leq_1\!\!\one^1\ (\forall k\ f(\bar{b}(k))\!\!=_0\!0)
\text{.}
\end{align*}
First, in section \ref{ss:spuWKL}, we give a simple 
proof of $\BW(s)$ using $\SiLm\CA(f)$ and $\WKL$
to be able to apply $\Pi^0_2\m\WKL$ to a specific tree determined by $s$. Also, we conclude that
$\BW^d_\RR$ could be proved analogously.\\
We analyze a minor modification of this proof and obtain the D-interpretation
of $\BW$ in section \ref{ss:fafi}.\\
Finally, in section \ref{ss:acr}, we discuss the complexity of the realizers 
obtained in \ref{ss:fafi} depending on the way the $\BW$ principle is used in
a given proof.

\subsection{Simple Proof based on $\Pi^0_2\m\WKL$} \label{ss:spuWKL}
In proof mining, it is very important to choose the
``right'' proof. It is essential to understand that if
we want to keep precise track of the growth of the complexity of
the realizing terms even in special scenarios, such as using a concrete
instance of $\BW$ to prove a $\forall\exists$-statement, we have 
to investigate the specific occurrence of every ineffective principle we
use. As we will see, it makes a difference whether we use
$\PiLm\CA$ independently on fixed terms, or whether we
 apply $\PiLm\CA$ to a function realizing another
use of $\PiLm\CA$ already.\\
In our case, instead of using arithmetical comprehension
for $\forall\exists$-formulas (as in the usual proof of $\BW$) or, equivalently, 
using nested $\PiLm\CA$, we go for a simple proof of $\BW$ based on applying
Weak K\"onig's Lemma (see definition~\ref{l:WKL-Feferman}) 
to a $\forall\exists$-predicate $\phi(s)\equiv\forall x\exists  y\ \phi_{_\QF}(x, y, s)$:
\[
\Pi^0_2\m\WKL\quad:\quad
    \BTree(\phi) \wedge \forall k \UnBounded(\phi,k) 
        \rightarrow \exists b\ (\BFunc(b)\wedge\forall k\ \phi(\bar{b}(k)))\text{.}
\] 
Using $\SiLm\CA^0$ (i.e. arithmetical comprehension
for purely existential formulas $\phi$) only once, namely on $\exists y\ \phi_{_\QF}(x, y, s)$ of
the $\forall\exists$-predicate, we obtain an $f$ s.t. 
$\forall x($ $\exists  y\ \phi_{_\QF}(x, y, s)$ $\leftrightarrow$ $f(x,y)=0\ )$
and thereby reduce $\Pi^0_2\m\WKL$ to $\Pi^0_1\m\WKL$ (same as above, only with 
purely universal $\phi$).
%\[
%    \BTree(\phi) \wedge \forall k \UnBounded(\phi,k) 
%        \rightarrow \exists b\ (\BFunc(b)\wedge\forall k\forall x\ f(\bar{b}(k),x)=_00)\text{.}
%\] 
In turn, we can reduce the use of $\PiLm\WKL$ to its regular
form $\WKL$ of Weak K\"onig's Lemma (same as above, 
only with quantifier-free $\phi$ - see definition~\ref{l:WKL-Kohlenbach}), 
e.g. by encoding $k$ and $x$ as one variable and using a primitive recursive modification
of $f$ which decodes the new variable and uses the results in the original way. 
However, in a more general view, this is due to the structure of the
standard $\WKL$:
We can take advantage of the fact that the conclusion in $\WKL$ is
an $\exists\forall$-statement for a quantifier-free predicate $\phi$ as well as for an
universal predicate $\phi$.
\paragraph{ND-interpretation of $\BW$.}
Howard showed in \cite{Howard81} that only a very weak
binary version of bar recursion (see definition~\ref{d:rBR}) is needed 
to interpret the standard $\WKL$, see theorem~\ref{t:FIwkl}. Moreover,
using Kohlenbach's NMD-interpretation these realizers are majorized in an almost
 trivial way. 
We derive the ND-interpretation of $\SiLm\CA$ from Spector's results in \cite{Spector62} 
using bar recursion of the lowest type, see corollary~\ref{c:NDPiLCA}.
Still, it is not straightforward to interpret the proof step which combines these 
interpretations. This step is in fact a finite $\DNS$ (see section~\ref{ss:fafi}). 
However, we give the full solution of the ND-interpretation of $\BW$ for
sequences in the rational interval $[0,1]$ in all details, see \ref{c:ND-bw}, and 
conclude that their complexity is the same and their structure similar for the
$d$\nbd dimensional cube $[0,1]^d\subset\RR^d$.

Consider a tree representation of the unit interval $[0,1]$ which 
splits the unit interval at level $n$ into $2^n$ intervals of 
length $2^{-n}$. We can draw the tree for first two levels as follows:
\[
\begin{parsetree}
(.. 
   (.$[0,\frac{1}{2}]$.
      (.$[0,\frac{1}{4}]$. .. ..)
      (.$[\frac{1}{4},\frac{1}{2}]$. .. ..)
   ) 
   (.$[\frac{1}{2},1]$.
      (.$[\frac{1}{2},\frac{3}{4}]$. .. ..)
      (.$[\frac{3}{4},1]$. .. ..)
   )
)
\end{parsetree}
\text{.}\]
Note that we can define each node via the path from the 
root to this interval. This path can by represented 
by a binary sequence $b$, where the $n$-th element defines which branch 
to take (here $0$ means left and $1$ means right).\\
%
We define a predicate $I(b^0,n^0,m^0)$, which tells us, whether the 
rational number $r$ encoded by $m=\langle r \rangle$ belongs to an 
interval defined by such a finite binary sequence $b$
of length $n$ (we can encode any finite sequence of natural numbers 
as a natural number), i.e. in an interval of length $2^{-n}$ given by 
$b$: 
\begin{align*}
I(b^0,n^0,m^0)  :&\equiv\ \ \  
  \InIntM{b}{n}\\
&\Leftrightarrow\ \ \ n\leq \lh(b) \ \wedge\  
             \sum_{i=1}^{n}\frac{b(i)}{2^i}\ \leq\ 
             r\ \leq\sum_{i=1}^{n}\frac{b(i)}{2^i}+\frac{1}{2^n} 
\text{.}
\end{align*}
For example, $I(<1,1>,2,m^0)$ expresses whether $m$ is in
the interval 
$[\sum_{i=1}^{2}\frac{<1,1>(i)}{2^i},\lb\sum_{i=1}^{2}\frac{<1,1>(i)}{2^i}+\frac{1}{2^2}]$,
which is $[\frac{3}{4},1]$.\\
We know that for a given finite binary sequence $b$ and 
an infinite sequence of encodings of rational
numbers $s$, there is a function $f_{s}^{1(0)}$, primitive recursive in $b$ and $s$, such that:
\[
 f_{s}(b,k)=_00\ \ \leftrightarrow\ \ \Big(\ k>\lh(b)\wedge I\big(b,\lh(b),\widetilde{ sk }\big)\ \Big)
\text{.}\]
%
Now, by $\SiLm\CA(f_s)$ we obtain a function $g_s$, s.t.:
\[
  \forall b^0\ \Big(\ g_sb=_00 \leftrightarrow \exists k^0\ \big(f_{s}(b,k)=_00\big)\ \Big)
\text{.}\]
%
In other words we have for all $b^0$:
\[
  g_{s}(b)=_00\ \leftrightarrow \exists k^0\geq_0 \lh(b)\ \  I\big(b,\lh(b),\widetilde{ sk }\big)
\text{.}
\tag{+}
\]
%
%
% showing bintree(f)
%
%%%%%%%%%%%%%%%%%%%%%%%%%%%%
To show $\BTree(g_s)$, consider any finite binary sequence $b$:
\begin{align*}
  g_s(b)=_00\ \wedge\ x\subseteq b 
           &{\ \rightarrow\ }  \exists k^0\geq_0\lh(b)\ \ I\big( b,\lh(b),\widetilde{ sk }\big)\ \wedge\ 
                              \lh(x)\leq_0\lh(b)\ \wedge\ 
                              x\subseteq b \\
               &{\ \rightarrow\ } \exists k^0\geq_0\lh(x)\ \ I\big( b,\lh(x),\widetilde{ sk }\big)\ \wedge\ 
                              x\subseteq b\  \\
               &{\ \rightarrow\ } \exists k^0\geq_0\lh(x)\ \ I\big( x,\lh(x),\widetilde{ sk }\big)\\
               &{\ \rightarrow\ }g_s(x)=_00
\text{.}
\end{align*}
To show 
\[
\forall k\exists x \big(\lh(x)=_0k \wedge g_s(x)=_00\big) \tag{++}
\]
just consider any given 
natural number $k$. By the definition of our tree, it splits the $[0,1]$ 
interval at any level, i.e. especially also on level $k$, completely. 
I.e. we can find also at level $k$ an interval, defined by a binary 
sequence of length $k$, such that $\widetilde{ sk }$ is contained in this
interval defined by some $b$ with $\lh(b)=k$  . 
So, we have: $I( b, \lh(b), \widetilde{ sk })$ and $ \lh(b)=_0k$. 
As we started with arbitrary $k$, this implies (++).\\
Now, we can apply $\WKL(g_s)$ to get:
\[
\exists b^1 \big(\BFunc(b) \wedge \forall k\ g_s(\overline{b}k)=_00\big) 
\tag{*}
\text{.}
\]
Note that in (*) (and from now on) $b^1$ is 
a binary function and $g_s$ takes
the encoding of the initial segment, $\langle b(0),\ldots,b(k)\rangle$, 
of this infinite sequence as its type $0$ argument.
Using (+) we can conclude that (*) is equivalent to:
\[
\exists b^1\leq\one\ \forall n\ \exists l>n\ \ I(\overline{b}n,n,\widetilde{\ sl\ })
\text{.}\]
This means that $\BW(s)$ is satisfied by $\hat a$ defined as:
\[
a(n^0):=_\QQ \left\langle \sum_{i=1}^{n+1} \frac{b(i)}{2^i}\ +
               \frac{1}{2^{n+2}}\right\rangle
\text{.}
\]
%
It remains to show that $a$ represents a real 
number in the sense of definition~\ref{d:real}.
W.l.o.g, at this point, we use $r$, $=$, $|\cdot|$, ... directly
instead of the proper syntactic form $\langle r\rangle$, $=_\QQ$, $|\cdot|_\QQ$, ...
to achieve better readability.
To prove $a=_1\hat a\in\RR$ (here $\hat a$ is used in the sense
of definition~\ref{d:hatReal} {\em not} in the sense of~\ref{d:hatAndG}), 
take any natural number $n$. We have:
\begin{align*}
|a(n)-a(n+1)|&=
\left|
\sum_{i=1}^{n+2} \frac{b(i)}{2^i}\ +
 \frac{1}{2^{n+3}}
-
\left(\sum_{i=1}^{n+1} \frac{b(i)}{2^i}\ +
 \frac{1}{2^{n+2}}\right)
\right|
\\
&=
\left|
\frac{b(n+2)}{2^{n+2}}+\frac{1}{2^{n+3}}
-
\frac{1}{2^{n+2}}
\right|
\\
&=
2^{-(n+3)} < 2^{-(n+1)}
\text{,}
\end{align*}
%what
which concludes the proof.\\
%
%

%space
Using the maximum norm, note that all norms in $\RR^d$ are equivalent,
the proof of the $d$-dimensional case differs from the proof of $\BW$, given above,
only in the quantifier-free parts. In fact, one could establish
syntactical ``macros'' for the operations on (relations between) 
tuples representing the elements of  the $d$-dimensional space
in such a way, that the proof could remain almost exactly in its current form. 
The difference is in the
definition of the tree dividing the unit interval. We have to ensure that with each level
in the tree we achieve the required ``amount of convergence'' in each dimension. 
Therefore, in contrast to
dividing each interval into two halves we need to split each cube uniformly
into $2^n$ disjoint parts of equal size. Note that,
this is no significant limitation to the use of the $\WKL$ since the number of
ancestors of each node is still bounded by this simple constant (see section \ref{s:wkl}).



%%%%%%%%%%%%%%%%%%%%%%%%%%%%%%%%%%
%
\subsection{Formal Analysis and functional Interpretation} \label{ss:fafi}
%
%%%%%%%%%%%%%%%%%%%%%%%%%%%%%%%%%%
%
%
%
%Recall the standard formulation of $\BW$:\\
%\[
%\forall s^1(\ \ (\forall n^0\ 0\leq_\QQ sn\leq_\QQ 1) \rightarrow 
%  \exists a^1\forall k^0\exists l^0\geq k\ |a-_\RR\lambda n^0.sl|\leq_\RR2^{-k}\ \ )
%\text{.}
%\tag{$\BW_0$}
%\]
%Using the $\hat f$ transformation from definition \ref{d:hatReal}, we can
%derive $\BW_0$ from the formula:
%\[
%\forall s^1\exists a^1\forall k^0\exists l^0\geq_0 k\ 
% |\hat a(k+1)-_\QQ \tilde s(l)|\leq_\QQ2^{-(k+2)}
%\text{.}
%\tag{$\BW_1$}
%\]
From now on we consider $s$ to be an infinite sequence of encodings of rational 
numbers within the $[0,1]$ interval, and $f_s$ to be the characteristic function of
the corresponding tree as defined in section \ref{ss:spuWKL} above.
From section \ref{s:wkl}, we know that using an 
appropriate quantifier-free formula $\phi^{\WKL}$
we can write $\WKL$ as
\[
\forall h^1\exists b^1\leq\one\forall k\ \ \phi^{\WKL}(h,b,k)\equiv\forall(h^1)\WKL_t(h)
\text{,}
\]
where $\phi^{\WKL}$ is quantifier-free.\\
We introduce following notations for $\SiLm\CA$:
\begin{align*}
\exists g^1\forall x^0 &\phi^{\CA(f)}_{\Sigma^0_1}(x,gx)\\
&\equiv\ \exists g^1\forall x^0 \big(gx=_00 \leftrightarrow \exists z^0\ f(x,z)=_00\big)\\
\Leftrightarrow\ 
\exists g^1\forall x^0 &\forall z^0_2\exists z^0_1 \phi^{\CA(f)}(x,gx,z_1,z_2)\\
&\equiv\
\exists g^1\forall x^0\forall z^0_2\exists z^0_1
   \big((gx=_00 \rightarrow f(x,z_1)=_00)\wedge(gx=_00 \leftarrow f(x,z_2)=_00) \big)
\text{,}
\end{align*}
where $\phi^{\CA(f)}$ is a quantifier-free formula. \\
%The index $s$ stands for the original
%sequence of rationals we are working with. The formula $\phi_{\CA}(x,y,z)$ 
%uses $f_s$ as its only free 
%variable representing the tree corresponding to $s$ as described in the previous section.\\
Now, at the first glance, the proof-tree for the proof from previous
 section is fairly simple:
\[
\begin{prooftree}
\SiLm\CA(f_s)
 \quad \quad
\WKL
%--------------------------------------------------------------------------
\justifies
\BW(s)
\end{prooftree}
\text{,}
\]
where we know that $\BW(s)$ is just $\SiLm\CA(f_s) \wedge \WKL(g_s)$, 
where $g_s$ is the realizing 
function from $\SiLm\CA(f_s)$. However, we can interpret only the
negative translations of the assumptions, what leads us to:
\[
\begin{prooftree}
(\SiLm\CA(f_s))'
 \quad \quad
(\WKL)'
%--------------------------------------------------------------------------
\justifies
(\BW(s))'
\end{prooftree}
\text{.}
\]
Moreover, to be able to exploit the whole benefits of functional interpretation
we have to investigate the interpretation of this proof itself, 
namely the functional interpretation of:
\[ \big(\SiLm\CA(f_s))' \wedge (\WKL)'\big)\quad \rightarrow\quad \big(\BW(s)\big)'\  \tag{+}\text{.} \]
So finally we have the following picture:
\[
\begin{prooftree}
(\SiLm\CA(f_s))' \wedge (\WKL)'
 \quad \quad
((\SiLm\CA(f_s))' \wedge (\WKL)')\ \rightarrow\ (\BW(s))'
%--------------------------------------------------------------------------
\justifies
(\BW(s))'
\end{prooftree}
\text{.}
\]
Similarly as in \ref{ss:CA} this turns down to solving the interpretation of (+). This is
not trivial, since the negative translation double negates each formula. So,
 (+) becomes essentially an instance of a finite version of the $\DNS$ principle. 
This becomes even clearer using the same
 notation as in the definition \ref{d:NT}:
\[ \ \big(\neg\neg(\SiLm\CA(f_s))^* \wedge 
\neg\neg(\WKL)^*\big)\quad \rightarrow\quad \neg\neg\big(\BW(s)\big)^* \text{.} \]
Using the representations from above, $\neg\neg(\BW(s))^*$ becomes
\[
\neg\neg \exists g,b\forall x,k\ \big(\ \phi^{\CA(f_s)}_{\SiL}(x,gx)\wedge \phi^{\WKL}(g,b,k)\ \big)
\]
what is essentially $(\Pi^0_2\m\WKL(\phi))'$ for the specific 
formula $\phi^\BW(s)(\bar b(k))$ corresponding
to $\exists z\ f_s(\bar b(k),sz)$, where $f_s$ is 
defined as in section \ref{ss:spuWKL} above. So, treating (+) in the form
\begin{align*}
\big( \neg\neg \exists g'\forall x'\ \phi^{\CA(f)}_{\SiL}(x',g'x')\wedge
  \forall h' \neg\neg &\exists b'\forall k'\ \phi^{\WKL}(h',b',k') \big) \rightarrow\\
\neg\neg \exists g,b&\forall x,k\ \big( \phi^{\CA(f)}_{\SiL}(x,gx)\wedge \phi^{\WKL}(g,b,k) \big) 
%\text{.}
\end{align*}
independently on the precise structure of $f$, we obtain the
functional interpretation of $(\Pi^0_2\m\WKL(\psi_f))'$, where
$\psi_f(x^0)$ $\leftrightarrow$ $\exists n^0\ f(x,n)=_00$.\\
The functional interpretation of (+) using this representation is as follows:
\[
\begin{minipage}{\textwidth}
$\exists G,Z_1,B,H',X',Z_2',K'\ \ \forall B',Z_1',G',X,Z_2,K$
\vspace{-0.3cm}
\[
\quad
\left(
\begin{minipage}{0.98\textwidth}
\vspace{-0.4cm}
\[
\left(
\phi^{\CA(f)} \left(
\begin{minipage}{0.4\textwidth} 
\vspace{-0.4cm}
   \begin{align*}
     &X'\args{(G'\args{X'Z'})(Z_1'\args{X'Z_2'})},  \\
     (&G'\args{X'Z_2'})\args{(X'(G'\args{X'Z_2'})(Z_1'\args{X'Z_2'}))},\\
     (&Z_1'\args{X'Z_2'})
       \args{(X'\args{(G'\args{X'Z_2'})(Z_1'\args{X'Z_2'})})(Z_2'\args{(G'\args{X'Z'})(Z_1'\args{X'Z_2'})})}, \\
     &Z_2'\args{(G'\args{X'Z'})(Z_1'\args{X'Z_2'})}
   \end{align*}
\end{minipage}
\right)
\wedge
\phi^{\WKL} \left(\!\!
\begin{minipage}{0.15\textwidth} 
\vspace{-0.4cm}
   \begin{align*} 
     &H', \\
     &B'\args{H'K'},\\
     &K'\args{(B'\args{H'K'})}
   \end{align*}
\end{minipage}
\right)
\right)
\rightarrow
\hspace{1cm}
\]

\vspace{-0.7cm}
\[
\hspace{4.85cm}
\left(
\phi^{\CA(f)} \left( 
\begin{minipage}{0.18\textwidth} 
\vspace{-0.4cm}
  \begin{align*}
     &X\args{GZ_1B}, \\
     &G\args{(X\args{GZ_1B})}, \\
     &Z_1\args{(X\args{GZ_1B})(Z_2\args{GZ_1B})}, \\
     &Z_2\args{GZ_1B}
   \end{align*}
\end{minipage}
\right)
\wedge
\phi^{\WKL} \left(
\begin{minipage}{0.11\textwidth} 
\vspace{-0.4cm}
   \begin{align*} 
     &G, \\
     &B,\\
     &K\args{GZ_1B}
   \end{align*}
\end{minipage}
\right)\!\!
\right)
\]
\end{minipage}
\right)
\text{,}
\]
\end{minipage}
\]
where, again, each exists-variable (i.e. $G$, $B$, $H'$, $X'$, $Z'$, and $K'$) 
may depend on any 
for-all-variable (i.e. $B'$, $G'$, $X$, $Z$, and $K$). E.g. by $G$ we mean in 
fact $(GB'G'XZK)$.  This interpretation
yields following functional equations:
\setcounter{equation}{0}
\begin{align}
     (G'X'Z_2')(X'(G'X'Z_2')(Z_1'X'Z_2'))&=G(XGZ_1B)\text{,}\\         %1
     X'(G'X'Z')(Z_1'X'Z_2')&=XGZ_1B  \text{,}\\                        %2
     (Z_1'X'Z_2')(X'(G'X'Z_2')(Z_1'X'Z_2'))(Z_2'(G'X'Z')(Z_1'X'Z_2'))&=Z_1(XGZ_1B)(Z_2GZ_1B) \text{,}\\    %3
     Z_2'(G'X'Z')(Z_1'X'Z_2')&=Z_2GZ_1B \text{,}\\                     %4
     H'&=G \text{,}\\                                                  %5
     B'H'K'&=B \text{,}\\                                               %6  
     K'(B'H'K')&=KGZ_1B                                        %7
\text{.}
\end{align}
We use a very similar approach to the one used by Gerhardy in \cite{GerhardyX} to
solve such equations for finite $\DNS$. First, we conclude from (5) and (6)
that $B=B'GK'$ and from (1) and (2) that $G=G'X'Z_2'$. Using (6), we can set 
$K'$ to $\lambda b.KGZ_1b$ according to (7). This is not that trivial
for $X'$ and $Z_2'$. However, as pointed out by Gerhardy in \cite{GerhardyX}, in the presence
of the $\lambda g$ and $\lambda z_1$, which as we know will stand for the input 
of $G$ and $Z_1$, $K'$ becomes, and thereby $B$ as well, suddenly a well definable term:
\begin{align*} 
t_{X'}:&=\lambda g,z_1.Xgz_1(B'g(\lambda b.Kgz_1b))\text{,}\\
t_{Z_2'}:&=\lambda g,z_1.Z_2gz_1(B'g(\lambda b.Kgz_1b))\text{.}
\end{align*} 
This makes the rest of our presets well defined, what is easy
to see, since for each term all dependencies are only on
the terms defined above:
\begin{align*} 
t_{Z_1}:&=Z_1't_{X'}t_{Z_2'}\text{,}\\
t_{G}:&=G't_{X'}t_{Z_2'}\text{,}\\
t_{H'}:&=t_{G}\text{,}\\
t_{K'}:&=\lambda b.Kt_{G}t_{Z_1}b\text{,}\\
t_{B}:&=B't_{G}t_{K'}\text{.}
\end{align*} 
We have found the realizing terms for ($+^D$) for any $G'$, $Z_1'$ and $B'$. To finally obtain
the functional interpretation of $(\Pi^0_2\m\WKL(\phi))'$ we just need to 
define these two functionals in such a way that the 
assumptions $\phi^{\WKL}$ and $\phi^{\CA(f)}$ are always true.\\
For $\phi^{\CA(f)}$, as we know from the functional interpretation of $\SiLm\CA$ 
(see section \ref{ss:CA}), we get:
\begin{align*}
      {G'}\tp3&=t_h \text{,}\\
      {Z_1'}\tp3&=t_z
%      Z_1'^{\mathit{3}}&=\lambda T^{2(1)},\lambda W^{2(1)},a^0\ .\ t_zTWa0^0
\text{,} 
\end{align*}
where $t_h$ and $t_z$ are defined as in the ND-interpretation of $\SiLm\CA$ 
(see corollary \ref{c:NDSiLCA}).\\
For $B'$, from the interpretation of $\WKL$, we know the following
equality holds:
\[ 
\underbrace{\overline {B'H'K'}\big(K'(B'H'K')\big)}_{\text{as above}}\quad = 
\underbrace{\overline{B}(AB)}_{\text{as in section \ref{ss:HFI}}} 
\text{.} 
\]
We use the same notation as we used to define $B$ in section \ref{ss:HFI} and define:
\[
        B':=\lambda h.\lambda A.\big[F_h\big(K_{A}(\emptyset)\big)\big] \text{,}
\]
where $F_h$ and $K_A$ are defined as in the ND-interpretation of $\WKL$
(see theorem \ref{t:FIwkl}).\\
The terms defined above, using these definitions for $G'$ and $B'$, then satisfy the following
modification of ($+^D$):
\begin{align*}
\forall X,Z_2,K\ (\ &\phi^{\CA(f)}(
 Xt_{G}t_{Z_1}t_{B},
 t_{G}(Xt_{G}t_{Z_1}t_{B}),
 t_{Z_1}( Xt_{G}t_{Z_1}t_{B} )( Z_2t_Gt_{Z_1}t_{B} ),
 Z_2t_Gt_{Z_1}t_{B})\ \wedge\\
 &\phi^{\WKL}(t_{G},t_{B},Kt_{G}t_{Z_1}t_{B})\ ) \text{,}
\end{align*}
which is in principle the D-interpretation of (the negative translation of)\linebreak[2] $\Pi^0_2\m\WKL(\phi)$.
Using that $\forall X^{\typeOfXZ},Z^{\typeOfXZ},a^0,b^0 t_zXZab=_0t_zXZa0$ (see also the remark
to corollaries~\ref{c:NDPiLCA} and~\ref{c:NDSiLCA}) we conclude:
%
% ND-interpretation of \phi_BW(s)
%
%%%%%%%%%%%%%%%%%%%%%%%%%%%%%%%
\begin{prop}[The ND-interpretation of $\Pi^0_2\m\WKL(\phi)$] \label{p:ND-PI02WKL}
$ $
\\
W.l.o.g. assume:
\[
\forall k\phi(\bar bk) \Leftrightarrow \forall k^0\exists z^0\ f(\bar bk,z)=_00
\text{.}
\]
Let $\tau$ denote the type $2(1(1))$. The $\Pi^0_2\m\WKL(\phi)$ principle in the classically equivalent form:
\[
\exists g^1,b^1\forall x^0,k^0\ \big( \phi^{\CA(f)}_{\SiL}(x,gx)\wedge \phi^{\WKL}(g,b,k) \big)
\text{,}
\]
is ND-interpreted by
\begin{align*}
\forall X^\tau,Z^\tau,K^\tau\ \Big(\ &\phi^{\CA(f)}\big(
 Xt_{G}t_{Z}t_{B},
 t_{G}( Xt_{G}t_{Z}t_{B} ),
 t_{Z}( Xt_{G}t_{Z}t_{B} ),
% t_{Z}( Xt_{G}t_{Z}t_{B} )( Zt_Gt_{B} ),
 Zt_Gt_Zt_{B}\big)\ \wedge\\
 &\phi^{\WKL}(t_{G},t_{B},Kt_{G}t_{Z}t_{B})\ \Big) \text{,}
\end{align*}
where
\begin{align*}
        t_B    &:=_1 \big[F_{t_G}\big(K_{\lambda b.Kt_Gt_Zb}(\emptyset)\big)\big]\text{,}\\
        t_Z    &:=_1 \lambda n^0\ .\ t_zt_{X}'t_{Z}'n0^0\text{,}\\
        t_G    &:=_1 t_ht_{X}'t_{Z}'\text{,}\\
        t_{X}' &:=_{2(1)}\lambda g^1,z^{1} . Xgz 
                 \big[F_{g}\big(K_{\lambda b^1.Kgzb}(\emptyset)\big)\big] \text{,}\\
        t_{Z}' &:=_{2(1)}\lambda g^1,z^{1} . Zgz 
                 \big[F_{g}\big(K_{\lambda b^1.Kgzb}(\emptyset)\big)\big] \text{,}\\
%        t_{Z}' &:=_{2(1)}\lambda g^1,z^{1} . Zgz( 
%                 [F_{\lambda b^1.Kgzb,g}(K_{\lambda b^1.Kgzb}(\emptyset),\emptyset)] )\text{.}
\end{align*}
The remaining terms are defined as in previous sections:
\begin{align*}
        K_Ax     &:= \KA{A}{x}     \text{,}   \\
%        F_{A,f}(0,x)   &:= \Fnullx \text{,}\\
        F_{f}(n) &:= \Fn{f}    \text{,}\\
        t_gTW    &:= \xTWf{T}{W}{f}\text{,}   \\
        u_{W,f}n^0v^{1(0)}  
                 &:= \uWf{W}{f}    \text{,}\\
        t_z      &:=\lambda X^{\typeOfXZ},Z^{\typeOfXZ},a^0,b^0\ .\ t_g(t_fX)(t_fZ)a\text{,}\\
        t_h      &:=\lambda X^{\typeOfXZ},Z^{\typeOfXZ},n^0\ .\ 
                        {\min}_0(f(n,t_g(t_fX)(t_fZ)n),1) \text{,}\\
        t_f      &:=\lambda X^{\typeOfXZ},g^1\ .\ 
                       X(\lambda n^0 . {\min}_0(f(n,gn),1) )(\lambda a^0,b^0 . ga) 
\text{.}
\end{align*}
\end{prop}

\begin{rmk}
In proposition~\ref{p:ND-PI02WKL}, we used the somewhat evasive formulation 
``in the classically equivalent form''. This is due to the following:
\begin{enumerate}
\item In fact, we ND-interpreted only $\SiLm\WKL$, which is equivalent to $\Pi^0_2\m\WKL$ in
$\weha$. The equivalence is proved 
analogously to 
%the proof of 
proposition~\ref{p:eqWKLs}.
\item We do not consider $\SiLm\WKL$ in its form as defined in~\ref{l:WKL-Feferman}, 
but combine the ND-interpretations of the
corresponding parts of $\SiLm\WKL$: $\SiLm\CA$ and $\WKL$.
\end{enumerate}
As to the resulting interpretation itself, one might wonder why we still need the first element of the conjunction.
The point is, the second part of our result is not as strong as it may seem. 
The formula:
\[
\forall K\ \phi^{\WKL}(t_{G},t_{B},Kt_{G}t_{Z}t_{B}) \text{,}
\]
becomes useful only if we can prove that $t_{G}$ defines the 
infinite binary tree we originally started with. 
Otherwise, all conclusions taken are about some, though primitive recursive, 
modifications of $t_G$; moreover we could not prove what does this $t_G$ stand for. 
See also the proof of the corollary \ref{c:ND-bw} below.
\end{rmk}
%
%%%%%%%%%%%%%%%%%%%
%
%  ND of BW
%
%%%%%%%%%%%%%%%%%%%
%
\begin{cor}\label{c:ND-bw}
The Bolzano-Weierstra{\ss} principle $\BW$ for an infinite sequence
of rational numbers, $s$, bounded within the interval $[0,1]$:
\[
\forall s
  \exists a\forall k\exists l\geq k\ |\hat a(k+1)-_\QQ \widetilde{\ sl\ }|\leq_\QQ2^{-(k+2)}
\text{,}
\]
is ND-interpreted by
\[
\forall s, K^{2(1)}
  \big(\ \ t_L(Kt_At_L)\geq_0 Kt_At_L\ \wedge\ 
  |\hat t_A(Kt_At_L+1)-_\QQ \widetilde{s(t_L(Kt_At_L))}|\leq_\QQ2^{-(Kt_At_L+2)}\ \ \big)
\text{,}
\]
where, given any $K\tp2$, the terms $t_L\tp3$ and $t_A\tp3$ (for brevity, above
$t_L$ and $t_A$ are shortcuts for $t_L\args{sK}$ and $t_A\args{sK}$) depend on 
both $s$ and $K$ and are defined as follows:
\begin{align*}
t_L(s,K)&:=_1\ \lambda n^0.t_Z\big(\overline{t_B}(n)\big) \text{,}\\
t_A(s,K)&:=_1\ \lambda n^0.\ \bigg\langle\sum^{n+1}_{i=1}\frac{t_Bi}{2^i}\ +\ \frac{1}{2^{n+2}}\bigg\rangle
\text{,}
\end{align*}
with fixed $X$, $Z$, and $f$:
\begin{align*}
X&:=_\tau\lambda g^1,z^{1},b^1\ .\ B(K'gzb,g,z,b)\text{,}\\
Z&:=_\tau\lambda g^1,z^{1},b^1\ .\ N(K'gzb,g,z,b)\text{,}\\
f\tp1&:=_{1(0)}f_s\tp1
\text{,}
\end{align*}
where $t_B$ and $t_Z$ are shortcuts for $t_B\args{XZK'}$ and $t_Z\args{XZK'}$, where
$K'$ is defined for any given $K$ similarly as $X$ and $Z$ as follows:
\[
{K'}\tp2:=\lambda g^1,z^1,b^1\ .\ K\args{
 \underbrace{  
   \left( \lambda n^0.\bigg\langle \sum^{n+1}_{i=1}\frac{b(i)}{2^i}\ +\ \frac{1}{2^{n+2}} \bigg\rangle \right)
 }_{\text{\normalsize $\thicksim t_A$}}
 \underbrace{  
   \left( \lambda n^0.z(\overline{b}(n)) \right)
 }_{\text{\normalsize $\thicksim t_L$}}
}
\text{.}
\]
The terms $B$ and $N$ are primitive recursive, though not trivial, case distinctions:
\begin{align*}
B(m^0,g^1,z^1,b^1)&:=_0
\begin{cases}
  {\min}_0 \{ x^0 | \lh(x)<_0 m \wedge \neg\phi^{\CA}(x,gx,zx,Nmgzb) \} &\text{if it exists} \\
  X_n(m,g,b)&\Telse
\end{cases}  \text{,} \\
X_n(m^0,g^1,b^1)&:=_0
  \begin{cases}
    \bar bm &\Tif\ g(\bar bm)=_00\\
    {\min}_0 \{ x^0 | \lh(x)=m \wedge f_s(x,m)=_00 \} &\Telse
  \end{cases}  \text{,} \\
N(m^0,g^1,z^1,b^1)&:=_0\begin{cases}
\lh(m) &\Tif\ f_s(m,\lh(m))=_00\\
z(X_nmgb) &\Telse
\end{cases}  \text{,} \\
        f_s(b^0,z^0)
                 &:=_0\fs{b}{z}  \text{,} \\
        I(b^0,n^0,m^\QQ)
                 &:\equiv\ \ \  \InIntM{b}{n}
\text{,}
\end{align*}
where remaining terms are defined as before except that $t_g$ is applied to the
concrete function $f_s$ and not to an arbitrary type $1$ function:
\begin{align*}
        t_B    &:= \big[F_{t_G}\big(K_{\lambda b.Kt_Gt_Zb}(\emptyset)\big)\big] \text{,} \\
        t_Z    &:= \lambda n^0\ .\ t_zt_{X}'t_{Z}'n0^0 \text{,} \\
        t_G    &:= t_ht_{X}'t_{Z}' \text{,} \\
        t_{X}' &:=\lambda g,z.Xgz
                 \big[F_{g}\big(K_{\lambda b.Kgzb}(\emptyset)\big)\big]  \text{,} \\
        t_{Z}' &:=\lambda g,z.Zgz
%                 [F_{g}(K_{\lambda b.Kgzb}(\emptyset),\emptyset)] )\text{,}\\
                 \big[F_{g}\big(K_{\lambda b.Kgzb}(\emptyset)\big)\big]  \text{,} \\
        K_Ax     &:= \KA{A}{x}         \text{,} \\
%        F_{A,f}(0,x)   &:= \Fnullx  \text{,} \\
        F_f(n) &:= \Fn{f}  \text{,} \\
%        x_0TW    &:= \xTWf{T}{W}{f}   \\
        t_z      &:=\lambda X^{\typeOfXZ},Z^{\typeOfXZ},a^0,b^0\ .\ t_g(t_fX)(t_fZ)a \text{,} \\
        t_h      &:=\lambda X^{\typeOfXZ},Z^{\typeOfXZ},n^0\ .\ 
                        {\min}_0(f(n,t_g(t_fX)(t_fZ)n),1)  \text{,} \\
        t_f      &:=\lambda X^{\typeOfXZ},g^1\ .\ 
                       X(\lambda n^0 . {\min}_0(f(n,gn),1) )(\lambda a^0,b^0 . ga)  \text{,} \\
        t_g      &:=\lambda X,Z\ .\ \xTWf{X}{Z}{f_s} \text{,} \\
        u_{Z,f}n^0v^{1(0)}  
                 &:= \uWf{Z}{f} \text{.} \\
\end{align*}
\end{cor}
%%%%
%
% Proof
%
\begin{proof}% [ of corollary \ref{c:ND-bw}]
Unwinding $\phi^\CA$ and $\phi^\WKL$ 
we get by proposition \ref{p:ND-PI02WKL}:
\setcounter{equation}{0}
\begin{multline}
\forall K\tp2,X\tp2,Z\tp2\ 
                     ((t_G\args{ XZK (X \args{
                                          (t_G\args{XZK})
                                          (t_Z\args{XZK})
                                          (t_B\args{XZK})
                                        } )
                           } 
                       )=_00  \rightarrow\\
                       f_s( (X \args{
                                  (t_G\args{XZK})
                                  (t_Z\args{XZK})
                                  (t_B\args{XZK})
                                } )  ,
                            t_Z \args{ XZK (X \args{
                                                (t_G\args{XZK})
                                                (t_Z\args{XZK})
                                                (t_B\args{XZK})
                                               } )
                                 }  
                       )=_00
                      )
\end{multline}
and
\begin{multline}
\forall K\tp2,X\tp2,Z\tp2\ 
                     ((t_G\args{ XZK (X \args{
                                          (t_G\args{XZK})
                                          (t_Z\args{XZK})
                                          (t_B\args{XZK})
                                        } )
                           } 
                       )=_00  \leftarrow\\
                       f_s( X \args{
                                  (t_G\args{XZK})
                                  (t_Z\args{XZK})
                                  (t_B\args{XZK})
                                }   ,
                            Z \args{
                                 (t_G\args{XZK})
                                 (t_Z\args{XZK})
                                 (t_B\args{XZK})
                               } 
                       )=_00
                      )
\end{multline}
and
\be[f:three]
\forall K\tp2,X\tp2,Z\tp2\ 
  \widehat{\ (\widehat{ t_G \args{XZK} }
             )_{ t_{ t_G\args{XZK} } }\ 
          } (\overline{
                t_B\args{XZK}
             } (K \args{
                    (t_G\args{XZK})
                    (t_Z\args{XZK})
                    (t_B\args{XZK})
                  }
               )
            )=_00
\text{.}
\ee

Note that by (1) for $x:=X \args{
                                          (t_G\args{XZK})
                                          (t_Z\args{XZK})
                                          (t_B\args{XZK})
                                        }$
the equality $t_G\args{XZK}(x)=_00$ implies $f_s(x,z)=_00$ for some $z^0$ and thereby
 $\BFunc([x])$.\\
We will not be able to show $\forall K\tp2\BTree(t_G\args{XZK})$. Fortunately, we need only
to show:
\begin{multline*}
\forall K\tp2\ {\big(} 
\widehat{t_G\args{XZK}}
 (\overline{t_B\args{XZK}}(K\args{ (t_G\args{XZK}) (t_Z\args{XZK}) (t_B\args{XZK}) }))=_00 
       \leftarrow\\
 t_G\args{XZK}
   (\overline{t_B\args{XZK}}(K\args{ (t_G\args{XZK}) (t_Z\args{XZK}) (t_B\args{XZK}) }))=_00\ 
{\big)} 
\end{multline*}
and
\begin{multline*}
\forall K\tp2\ {\big(}\ (t_G\args{XZK})_{t_{t_G\args{XZK}}}(\overline{t_B\args{XZK}}(K\args{ (t_G\args{XZK}) (t_Z\args{XZK}) (t_B\args{XZK}) }))=_00 \rightarrow\\
                           t_G\args{XZK}(\overline{t_B\args{XZK}}(K\args{ (t_G\args{XZK}) (t_Z\args{XZK}) (t_B\args{XZK}) }))=_00)\ {\big)}\text{.}
\end{multline*}
Due to the definition \ref{d:hatAndG} to do so it suffices to show:
\begin{multline}
\forall K\tp2,x^0 
\ \ {\Big(\ }{\big(} x\subseteq 
       \overline{t_B\args{XZK}}(K\args{ (t_G\args{XZK}) (t_Z\args{XZK}) (t_B\args{XZK}) } ) 
      \ \wedge \\
      t_G\args{XZK}
       (\overline{t_B\args{XZK}}(K\args{ (t_G\args{XZK}) (t_Z\args{XZK}) (t_B\args{XZK}) } ))=0
    {\big)}
    \rightarrow 
                           t_G\args{XZK}(x)=_00 {\ \Big)} \label{e:tree}
\end{multline}
and
\begin{align}
%\forall K\tp2\ 
%  \forall n^0 \leq_0 K\args{ (t_G\args{XZK}) (t_Z\args{XZK}) (t_B\args{XZK}) }\ \ 
%\left(\ \exists x (\lh(x)=_0 n \wedge t_G\args{XZK}x=_00\ \right) \label{e:inf}
\forall K\tp2\ \exists x\ \left(
    \lh(x)=_0 K\args{ (t_G\args{XZK}) (t_Z\args{XZK}) (t_B\args{XZK}) }
    \wedge t_G\args{XZK}(x)=_00\ \right) \label{e:inf}
\end{align}
respectively, where $x$ is witnessed by $X\args{(t_G\args{XZK})(t_Z\args{XZK})(t_B\args{XZK})}$.
\begin{itemize}
\item To prove \eqref{e:tree} suppose:
\addtocounter{equation}{1}
\begin{align*}
&x\subseteq 
       \overline{t_B\args{XZK}}(K\args{ (t_G\args{XZK}) (t_Z\args{XZK}) (t_B\args{XZK}) } ) 
      \quad \wedge \tag{\arabic{equation}a}\label{e:trAss-A}\\
&t_G\args{XZK}
       (\overline{t_B\args{XZK}}(K\args{ (t_G\args{XZK}) (t_Z\args{XZK}) (t_B\args{XZK}) } ))=0
                   \tag{\arabic{equation}b}\label{e:trAss-B}
\end{align*}
holds for some $K$ and $x$. By (1), (2), and definitions of $X$ and $B$ we have
for all binary sequences $b$ with $\lh(b)
  \leq l:=K\args{
    (t_G\args{XZK})
    (t_Z\args{XZK})
    (t_B\args{XZK})
   }$
\begin{multline}
      ((t_G\args{ XZK} (b) )=_00 \rightarrow f_s( b, t_Z \args{ XZK}(b) )=_00 )\ \ \wedge\\
      (t_G\args{ XZK } (b) )=_00 \leftarrow f_s( b, Z\args{
        (t_G\args{XZK})
        (t_Z\args{XZK})
        (t_B\args{XZK})
       }  )=_00 ))
\label{e:caFin}
\text{.}
\end{multline}
Suppose namely \eqref{e:caFin} would not hold for some $b_0$ with $\lh(b_0)\leq l$, then by definition of
$X$, is $X\args{(t_G\args{XZK})(t_Z\args{XZK})(t_B\args{XZK})}$ equal to such a $b_0$ 
and we get a contradiction to (1)$\wedge$(2), for this specific $X$.\\
So, together with \eqref{e:trAss-B} we obtain from \eqref{e:caFin}:
\[
f_s( \overline{t_B\args{XZK}}(l), t_Z \args{ XZK}(\overline{t_B\args{XZK}}(l)) )=_00
\text{,}
\]
which by \eqref{e:trAss-B} and definition of $Z$ implies
\[
f_s( \overline{t_B\args{XZK}}(l), Z\args{(t_G\args{XZK})(t_Z\args{XZK})(t_B\args{XZK})} )=_00
\text{.}
\]
By definition of $f_s$, see also section \ref{ss:spuWKL}, and \eqref{e:trAss-A} this implies
\[
f_s( x, Z\args{(t_G\args{XZK})(t_Z\args{XZK})(t_B\args{XZK})} )=_00
\text{.}
\]
From \eqref{e:trAss-A} we get $\lh(x)\leq l$ and by  \eqref{e:caFin} we obtain
\[
t_G\args{XZK}(x)=0
\text{.}
\]
\item To prove \eqref{e:inf} consider again a fixed $K$ and $l:=K\args{
    (t_G\args{XZK})
    (t_Z\args{XZK})
    (t_B\args{XZK})
   }$.\\
Using the same argument as we used for \eqref{e:caFin} we have:
\[
f_s(
  X\args{(t_G\args{XZK})(t_Z\args{XZK})(t_B\args{XZK})} , 
  l
 ) = 0
\ \vee \ 
t_G\args{XZK}( \overline{t_B\args{XZK}}(l) ) = 0
\text{,}
\]
by definition of $X$. In both cases, this implies 
\[
f_s(
  X\args{(t_G\args{XZK})(t_Z\args{XZK})(t_B\args{XZK})} , 
  Z\args{(t_G\args{XZK})(t_Z\args{XZK})(t_B\args{XZK})}
 ) = 0
\text{,}
\]
by definition of $Z$ and, in the second case, (1).\\
Using $\lh(X\args{(t_G\args{XZK})(t_Z\args{XZK})(t_B\args{XZK})})=l$ 
and \eqref{e:caFin} this implies
%and the definition of $f_s$ we get
%\[
%f_s(
%  \overline{[X\args{(t_G\args{XZK})(t_Z\args{XZK})(t_B\args{XZK})}]}(n) , 
%  Z\args{(t_G\args{XZK})(t_Z\args{XZK})(t_B\args{XZK})}
% ) = 0
%\text{.}
%\]
\[
t_G\args{XZK}(
 X\args{(t_G\args{XZK})(t_Z\args{XZK})(t_B\args{XZK})}
)=_00
\text{.}
\]
\end{itemize}
This concludes the proof of (4) and (5).\\
Now, by (3), (4), (5), and definition \ref{d:hatAndG} we get:
\[
 \forall K\ t_G\args{XZK}(\overline{t_B\args{XZK}}(
    K\args{(t_G\args{XZK})(t_Z\args{XZK})(t_B\args{XZK})}  )
   )=_00
\text{.}
\]
It follows by (1):
\begin{align*}
 \forall K \ \ f_s\Bigl(\ 
    &\overline{t_B\args{XZK}}(K\args{(t_G\args{XZK})(t_Z\args{XZK})(t_B\args{XZK})}), \\
    &t_Z\args{XZK}(\overline{t_B\args{XZK}}
                   (K\args{(t_G\args{XZK})(t_Z\args{XZK})(t_B\args{XZK})}))
 \ \bigr)=_00
\text{.}
\end{align*}
As the last sentence holds for any $K$, it especially holds, given any ${K'}{\tp2}$, for:
\[
K:=\lambda g^1,z^1,b^1\ .\ K'\args{
   \left( \lambda n^0. \bigg\langle \sum^{n+1}_{i=1}\frac{b(i)}{2^i}\ +\ \frac{1}{2^{n+2}} \bigg\rangle \right)
   \left( \lambda n^0.z(\overline{b}(n)) \right)
}
\text{.}
\] 
In other words we can also write:
\begin{align*}
 \forall K' \ \ f_s\bigl(\  
     \overline{t_B\args{XZK}} (K'\args{(t_A\args{XZK})(t_L\args{XZK})}),\ 
     t_Z\args{XZK}(\overline{t_B\args{XZK}}
                   (K'\args{(t_A\args{XZK})(t_L\args{XZK})}))
 \ \bigr)=_00
\text{.}
\end{align*}
This implies 
\begin{align*}
\forall K'\ \ \Bigl(\ 
  &t_L\args{XZK}(K'(t_A\args{XZK})(t_L\args{XZK})) 
         \geq  K'\args{(t_A\args{XZK})(t_L\args{XZK})}\ \ \wedge\\
  &I \bigl(\overline{t_B\args{XZK}}(K'\args{(t_A\args{XZK})(t_L\args{XZK})}), 
             K'\args{(t_A\args{XZK})(t_L\args{XZK})}, 
             s(t_L\args{XZK}(K'\args{(t_A\args{XZK})(t_L\args{XZK})})) \bigr)
\ \Bigr)
\text{}
\end{align*}
and finally, using the definition of $I$, we obtain (note that $t_A=\widehat {t_A}$; 
see end of section \ref{ss:spuWKL}):
\[
\forall K\ 
(\ t_L(Kt_At_L) \geq  Kt_At_L\  \wedge\ 
|t_A(Kt_At_L)-_\QQ s(t_L(Kt_At_L))|\leq2^{-(Kt_At_L+1)}\ )
\text{.}
\]
\end{proof}


%%%%
%
%
% NMD
%
%%%%

Applying the majorization results from section~\ref{ss:majFI} we obtain a
more readable version:
%
\begin{cor}\label{c:NMD-bw}
The Bolzano-Weierstra{\ss} principle $\BW$ for an infinite sequence
of rational numbers, $s$, bounded within the interval $[0,1]$:
\[
\forall s
  \exists a\forall k\exists l\geq k\ |\hat a(k+1)-_\QQ \widetilde{\ sl\ }|\leq_\QQ2^{-(k+2)}
\text{,}
\]
is NMD-interpreted by (again, we write shortly $A$ and $L$ for $A\args{K}$ and $L\args{K}$):
\begin{align*}
\exists L\tp3,A\tp3\Big(
 &(t_L^*,t_A^*)\maj_{\tup{\mathbf 3}}(L,A) \wedge \\
 &\forall s, K\tp2
   \big(  L(K\args{AL})\geq_0 K\args{AL} \wedge 
          |\hat A(K\args{AL}+1)-_\QQ \widetilde{\ s(L\args{(K\args{AL})})\ }|\leq_\QQ2^{-({K\args{AL}}+2)}  \big)
\Big)
\text{,}
\end{align*}
where, given any $K\tp2$, the terms $t_L\tp3$ and $t_A\tp3$ depend only
on $K$ and are defined as follows:
\begin{align*}
t_L^*(K)&:=_1
   \lambda n^0.\xTWfM{\left(\lambda g.X^*K\one g\right)}{(\lambda g.Z^*K\one g)}\big(\bar\one(n)\big)\text{,} \\
t_A^*(K)&:=_1\lambda n^0.\left\langle\sum^{n+2}_{i=1}\frac{1}{2^i}\right\rangle 
\text{,}
\end{align*}
with $X^*K$ and $Z^*K$ defined primitive recursively in $K$ as follows:
\begin{align*}
{X^*K}&:=\lambda g^1,z^1\ .\ \bar\one(K'gz)  \text{,} \\
{Z^*K}&:=\lambda g^1,z^1\ .\ {\max}_0\big(\ \lh(K'gz),\ z(\bar\one(K'gz))\ \big)
\text{,}
\end{align*}
where $K'$ is a primitive recursive modification of $K$:
\[
{K'}\tp2:=\lambda g^1,z^1\ .\ K\args{
   \left( \lambda n^0.\ \bigg\langle\sum^{n+2}_{i=1}\frac{1}{2^i}\bigg\rangle \right)
   \left( \lambda n^0.z(\overline{\one}(n)) \right)
}
\text{.}
\]
%and $g'$ a primitive recursive modification of $g$:
%\[
%g':=_1\lambda a^0,b^0.ga
%\text{.}
%\]
The functional $\bPhi_0^*$ can be constructed easily by using any standard majorant for
the bar recursor in the definition~\ref{d:PhiB}. See 
\cite{Kohlenbach08} or \cite{Bezem85} for a majorant of $\B_{\rho,\tau}$.\\
The majorized version of $u$, is given by:
\[
u^*_{Z}n^0v^{1(0)}  := \uWfM{Z}
\text{.}
\]
\end{cor}
Though it is not entirely obvious how we obtain this corollary, the
actual steps are purely technical. See appendix~\ref{a:2} for more details.


%%%%%%%%%%%%%%%%%%%%%%%%%%%%%%%%%%%%%%%%%%%%%%%%%%%%%%%%%%%
\subsection {Analysis of the Complexity of the Realizers} \label{ss:acr}
%%%%%%%%%%%%%%%%%%%%%%%%%%%%%%%%%%%%%%%%%%%%%%%%%%%%%%%%%%%
\newcommand{\tn}{$t\in\T_n$}
\newcommand{\tsn}{$t\in\T_0$}

Proposition \ref{p:ND-PI02WKL} implies that we can use only a single
application of $\B_{0,1}$ on primitive recursive functionals
and primitive recursion to obtain the realizing terms for $\BW$. This might
be a trivial conclusion, however it becomes obvious first when we have the
simple majorants from corollary~\ref{c:NMD-bw}.\\
Now, we can investigate how does the $\BW$ principle affect the
complexity of the realizers of a given theorem proved using this principle.
Depending on the way the $\BW$-principle is used in such a proof, 
we get the results stated in theorems \ref{t:PEfBW} and \ref{t:PEBW}.\\
%
\begin{thm}{\em Program extraction for proofs based on an instance of $\BW$.\\} 
\label{t:PEfBW}
For $n\geq1$, given a proof 
using $\BW$ on a known sequence of real numbers in the unit interval $[0,1]$ 
specified by a well defined term~\tsn:
\[
\hrrwepa\ +\ \QFm\AC\ +\ \Sigma^0_n\usftext{-IA}\ \ \vdash\ \ 
    \forall x^0\big(\BW_\RR(tx)\rightarrow\exists y^0 \phi(x,y)\big)\text{,}
\]
we can extract a function $f\in\T_n$ by ND-interpretation s.t.
\[
\Som\ \ \models\ \ \forall x^0 \phi\big(x,f(x)\big) \text{.}
\]
\end{thm}
%
% rmk
%
\begin{rmk}
This means that for $n=1$ we obtain a \lOrdm{\omega^{\omega^\omega}}recursive realizer.
Moreover, for the special case $n=0$, Kohlenbach showed in~\cite{Kohlenbach98} that when some weaker systems
than $\hrrwepa\ +\ \QFm\AC$ are used, namely $\usftext{G}_m\usftext{A}^\omega$,
one obtains even \lOrdm{\omega^\omega}recursive realizers.\\
\end{rmk}
%
% proof --------------------------------------------------------------------------------------------------
%
\begin{proof}[ of theorem \ref{t:PEfBW}]
W.l.o.g we can replace $\BW_\RR$ by $\BW$ due to proposition~\ref{p:BWimpBWR}.
By soundness of $\hrrwepa$ (see e.g.~\cite{Kohlenbach08}) we know there
are closed terms $t_K$ and $t_Y$ in $T_{n-1}$ realizing the ND-interpretation 
of the analyzed proof:
\[
\forall a,L,x (\BW_D(a,t_KaLx,L(t_KaLx),tx)\rightarrow\phi_D(x,t_YaLx))
\text{.}
\]
Now, using \ref{c:ND-bw}, we know which terms satisfy $\BW_D(a,t_KaLx,L(t_KaLx),tx)$
and we can conclude:
\[
F:= t_Y\args{
     (t_A\args{(
            tx,
            \lambda a,L.t_KaLx)
           })
     (t_L\args{(
            tx,
            \lambda a,L.t_KaLx)
           })
    }
\]
satisfies $\forall x^0 \phi(x,F(x))$. Using Parson's result
stated in proposition \ref{p:Parsons} we know that
$t_Y$, $tx$, and $t_K$ have computational sizes $<\omega_{n}(\omega)$.
This fact allows us to apply Howard's proposition, see~\ref{p:Howard},
from which it follows that $t_L\args{(tx,\lambda a,L.t_KaLx)}$ has 
computational size $<\omega_{n+1}(\omega)$ what implies that $F$
has computational size $<\omega_{n+1}(\omega)$. Finally,
by proposition \ref{p:Parsons} this leads to the desired conclusion.\\
\end{proof}

%space necessary



The theorem~\ref{t:PEfBW} is optimal in the following sense:
%
\begin{prop}\label{p:optPEfBW}
Any function $h$ given by a closed term in $\T_n$ can be proved to be total
in $\hrrwepa\ +\ \QFm\AC\ +\ \Sigma^0_n\usftext{-IA}$ using a concrete instance of
$\BW(s)$ for a suitable closed term $s^1$ in $\T_0$.
\end{prop}
%
{\samepage
\begin{proof}
%
Kohlenbach gave a construction for a
functional $F\tp2$ s.t. $\forall f^{1(0)}(\PCM(Ff)\rightarrow\PiLm\CAhut(f))$ 
 in~\cite{Kohlenbach00} (proposition 5.5, page 19), 
where $\PCM(f^{1(0)})$ is defined as follows:
\[
\forall n\big(0\leq_\RR f(n+1)\leq_\RR f(n)\big)\ \rightarrow 
       \exists g\forall k\forall m_1,m_2\geq gk\left(|fm_1-_\RR fm_2|\leq\frac{1}{k+1}\right)
\text{.}
\]
 Moreover, in the presence of $\QFm\AC$
we have $\forall f^{1(0)}( \BW_\RR(f) \rightarrow\PCM(f) )$. Together
with proposition~\ref{p:BWimpBWR} this means that we
 know that there are effective functionals $F_\exists$, and $F_\forall$, s.t. for any
function $f^{1(0)}$ the instance $\BW(F_\exists f)$, resp. $\BW(F_\forall f)$, 
of $\BW$ implies the instance $\SiLm\CA(f)$ of $\SiLm\CA$, resp. $\PiLm\CA(f)$ of $\PiLm\CA$.\\
% comprehending $\forall b\ t(a,b)$ by $g(a)$.\\
%
%The totality of $f$ is a $\Pi^0_2$ sentence which we shall call $\phi_{\Pi^0_2}$.\\
%
By Parsons's results in \cite{Parsons72} we know that
$\hrrwepa\ +\ \QFm\AC$ proves the $\Pi^0_2$ sentence stating
the totality of $h$ using additionally just finitely many, say $m$, instances
of $\Sigma^0_{n+1}\usftext{-IA}$ with number parameters only.\\
%
Assume that $n$ is even. Now, each of these $m$ instances
can be reformulated 
as an instance of $\Sigma^0_{n}\usftext{-IA}$ with number and function parameters
plus the formula $\forall \tup a^{\tup0} \PiLm\CA(t_i'(\tup a))$, 
where each term $t_i'\in\T_0$ ($i\in\{1\ldots m\}$) corresponds to a specific
lambda abstraction of the term used in the $i^{th}$ instance of $\Sigma^0_{n+1}\usftext{-IA}$. 
Furthermore, for suitable closed terms $t_i\in\T_0$
 each formula $\forall \tup a^{\tup0} \PiLm\CA(t_i'(\tup a))$ is
equivalent to an instance $\PiLm\CA(t_i)$ of $\PiLm\CA$.
Moreover, $\bigwedge^{m}_{i=0} \PiLm\CA(t_i)$ is 
equivalent to $\PiLm\CA(t)$ for a suitable \tsn.\\
Analogously, if $n$ is odd we obtain the equivalence to $\SiLm\CA(t)$.\\
Finally, $\PiLm\CA(t)$ and $\SiLm\CA(t)$ are
derivable from the instances $\BW(F_\forall t)$ resp. $\BW(F_\exists t)$ of $\BW$.\\
\end{proof}
} %samepgae
%

%space needed

%
The case where $\BW$ is used in its full generality, theorem~\ref{t:PEBW} below,
is not so surprising. It is well known that the terms of
$\T$ are closed under $\B_{0,1}$ (see section~\ref{s:compPM}). We present this result
mainly to stress how great may be the impact of the seemingly subtle difference in the way
this principle is used.\\
 While using a concrete instance of $\BW$ under 
$\hrrwepa\ +\ \QFm\AC\ +\ \Sigma^0_n\usftext{-IA}$ we get bounds in $\T_n$, the best bound for
realizing terms for $\Pi^0_2$-statements
 provable in such a weak system as $\hrrwepa\ +\ \QFm\AC\ +\ \Sigma^0_1\usftext{-IA}\ +\ \BW$ 
is only full G\"odel's $\T$. This is due to the fact that even under this weak system $\BW$
implies $\PiL\m\CA$ and this, via iteration, $\CA^0_{ar}$ (see also proof of theorem~\ref{t:PEBW} below).
% and thereby, in combination with $\Sigma^0_1\usftext{-IA}$, implies $\epa$.
\begin{thm}{\em Program extraction for proofs based on the full $\BW_\RR$ schema.\\} 
\label{t:PEBW}
Given a proof 
using $\BW_\RR$ on any given sequence of rational numbers:
\[
\hrrwepa\ +\ \QFm\AC\ +\ \Sigma^0_\infty\usftext{-IA}\ +\ \BW_\RR\ \ \vdash\ \ 
    \forall x^1\exists y^0 \phi(x,y)
\text{,}
\]
we can extract by ND-interpretation a function $f^1\in\T$ s.t.
\[
\Som\ \ \models\ \ \forall x^0 \phi\big(x,f(x)\big) \text{.}
\]
\end{thm}
\begin{proof}
As above, w.l.o.g we can replace $\BW_\RR$ by $\BW$ due to proposition~\ref{p:BWimpBWR}.
By the proof of proposition~\ref{p:optPEfBW} we know that $\BW$ implies $\PCM$ and that by Kohlenbach's result
in~\cite{Kohlenbach00} it therefore implies also $\PiLm\CA$ and $\SiLm\CA$. (Actually, in that proof and in~\cite{Kohlenbach00}
the finer formulation for instances is used, however here the weaker version is sufficient.) Since we can reduce any use of $\Sigma^0_n\usftext{-IA}$
to a quantifier-free induction by iterative application of $\PiLm\CA$/ $\SiLm\CA$ (note that this implies the
optimality of this theorem), we can infer:
\[
\hrrwepa\ +\ \QFm\AC\ +\ \BW\ \ \vdash\ \ 
    \forall x^1\exists y^0 \phi(x,y)
\text{.}
\]
Hence, by soundness of the ND-interpretation and by corollary~\ref{c:ND-bw} 
we know that there is a closed term $t_Y\tp2$
%which can be written 
in $\T_0\ +\ \B_{0,1}$ realizing the functional $Y\tp2$ in the 
ND-interpretation of the analyzed conclusion, i.e. satisfying
 $\forall x^1\phi_D(x,t_Yx)$. Finally, we can conclude that $t_Y\tp2$ can be
rewritten in $\T$ by corollary~\ref{c:TBT}.\\
\end{proof}
%

% space needed
%
Note that both main results, theorems \ref{t:PEBW} and \ref{t:PEfBW}, remain valid when we replace $\BW_\RR$ by
$\BW^d_\RR$ since we can treat $\BW^d_\RR$ analogously to $\BW$. See the discussion at the end of section \ref{ss:spuWKL}.

