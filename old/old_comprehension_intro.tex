\subsection{The Axiom of Comprehension}\label{s:basicInterpretations}

 G\"odel and Kreisel, and we should mention P. Bernays to complete the list,
 were also involved in another result, obtained by C. Spector, 
 which is essential for this thesis.
Namely, in \cite{Spector62} Spector showed that the consistency of
classical analysis with the schema of countable choice, 
\[
\AC^0\ :\quad \forall x^0\exists y\ \ \phi(x,y)\rightarrow\ \ \exists Y\forall x \phi(x,Yx)
\text{,}
\]
which contains the schema of full comprehension over numbers,
\[
\CA^0\ :\quad \exists f^1\forall x^0\ \big(f(x)=_0 0 \leftrightarrow \phi(x)\big)\text{,}
\]
and, in particular, the schema of arithmetical comprehension over numbers $\CA^0_{ar}$, where
$\phi$ has to be an arithmetic formula, can be reduced to a suitable intuitionistic system 
plus the double negation shift schema: 
\[
\DNS\ :\quad \forall x^0\neg\neg\phi(x)\ \rightarrow \neg\neg\forall x^0 \phi(x)\text{.}
\]
It is very important for our work that Spector also gives the 
realizing functionals for the functional interpretation of this axiom.
These realizers are defined via the so called bar recursion (see section \ref{ss:BR}) 
which is used to define functionals based on their behavior after
reaching a given ``bar''. Simply because of this result we 
% http://www.askoxford.com/concise_oed/apriori
% http://www.merriam-webster.com/dictionary/apriorihttp://www.merriam-webster.com/dictionary/apriori
% http://dict.leo.org/ende?lp=ende&p=/gQPU.&search=priori
% always a _ priori, as adjective or adverb 
know that the realizers of the bar-recursive functional interpretation of the
Bolzano-Weierstra{\ss} theorem exist in advance.\\

\todo{integrate this with the stuff below}
%
% definitions
\newcommand{\typeOfXZ}{0(10)(1)}
%
To interpret principles based on sequential compactness, 
we can use the ND-in\-ter\-pre\-ta\-tion 
of the schema of 
comprehension over numbers for $\PiL$-formulas:
\[
\PiLm\CA(f)\ :\ \exists g^1 \forall x^0\ (\ gx=_00 \leftrightarrow \forall y^0\ fxy=_00\ )
\text{.}
\]
We will use the modified schema of 
comprehension over numbers for $\PiL$-formulas, 
\[
\PiLm\CAhut(f)\ :\ \exists g^1 \forall x^0,z^0\ \big(fx(gx)=_00 \vee fxz\neq_00\big)
\text{,}
\]
since its realizers obtained by ND-interpretation are significantly shorter.
This principle and several of its variants are discussed
in section \ref{ss:CA}.
The negative translation of arithmetical comprehension (see
definition \ref{d:CA} below) itself can 
in turn be intuitionistically proved from 
very weak principles: The negative translation of
the law of excluded middle over numbers for $\SiL$ formulas, $(\LEM)'$, 
and the negative translation of the Axiom of Choice for 
$\PiL$-formulas $(\PiLm\AC)'$ (see sections \ref{ss:LEM} and \ref{ss:AC} below).
The negative translation of $\PiLm\AC$, on the other hand, is not trivial at all.
In fact, to prove $(\PiLm\AC)'$ we need a special form of the
 Double Negation Shift schema, $\DNS$, where $\phi$ is a $\Sigma^0_2$-formula:
\[
\Sigma^0_2\m\DNS\ :\ \forall x^0\neg\neg\phi(x)\rightarrow\neg\neg\forall x^0\phi(x) \text{.}
\]
To find the ND-interpretation of the intuitionistically underivable $\DNS$
was a major problem until a solution was obtained by Spector in \cite{Spector62}. 
In fact, Spector's interpretation leads to the ND-interpretation
of the so called schema of countable choice, which implies full
comprehension over numbers - $\CA^0$. And, in particular, gives the functional interpretation 
of full analysis (see \cite{Spector62}, \cite{Kohlenbach08} or the Introduction section).
In summary we get the following picture:\\
\[
\begin{prooftree}
(\LEM)'
\[
\neg\neg(\PiLm\AC)  \quad  \Sigma^0_2\usftext{-}\DNS
\justifies
(\PiLm\AC)'
\]
\justifies 
(\PiLm\CAhut)'
\end{prooftree}
\text{.}
\] {\samepage
Whereas the first two principles, $(\LEM)'$ and $\neg\neg(\PiLm\AC)$, are interpreted 
quite easily, the schema $\DNS$ requires the
use of bar recursion.\\
Below, we first introduce this recursion in \ref{ss:BR} and then
give the D- and ND-interpretations of the principles 
as they appear in the proof-tree. We 
start with $\LEM$ in \ref{ss:LEM}, followed by $\DNS$ in \ref{ss:DNS} and $\AC$ in \ref{ss:AC}.
Finally we combine the interpretations to obtain the ND-interpretation of $\PiLm\CAhut$ in~\ref{ss:CA}.}\\

We distinguish between full schemas and principles and their restrictions
to a concrete instance. For a given principle $\phi$ we write  $\phi(f)$ for the
concrete instance of  $\phi $ applied to $f$. We will define the principles
in this form in most cases. It is also the preferable form to be analyzed, since  most
proofs in mathematics use only concrete instances of the common principles. Moreover,
proofs of $\forall\exists$-theorems (i.e. sentences of the form 
$\forall\tup x\exists\tup y\ \psi_0(x,y)$, where $\psi_0$ is a quantifier-free formula) based
on a concrete instance of such a principle $\phi(t)$, $t$ being a well 
defined closed term (or a term depending only on the parameters of
the problem), tend to produce simpler realizers than proofs based
on their full counterparts $\forall f \phi(f)$ (see section \ref{s:sr}).\\
We will write shortly $\phi$ for the general version $\forall f \phi(f)$.\\
\begin{rmk}
In some cases, in particular for $\phi\equiv\PiLm\CA$, there is a 
primitive recursive ${F\tp2}_\phi$ s.t.:
\[
 \weha\proves \phi(F_\phi t)\rightarrow\forall l^0\phi(tl)\text{.}
\]
Here, $t\tp1$ is a closed term of type $1(0)$.
\end{rmk}
%
%
