\section{Effective learnability}

\reviewQuote{mind2013/review}{KS13}{Nov 22, 2012}
%
% use Ulrich
%
%For some time now, it is well known that from a proof of a convergence statement, i.e. a statement like:
%\[ \ \forall k\in\NN\,\exists n\in\NN\,\forall m,\tilde{m}\ge n
%\ \big( d(x_m,
%x_{\tilde{m}})\le 2^{-k}\big),\]
%one can extract a computable rate of convergence if it is provable
%in
%\[ \mbox{HA$^{\omega} +$AC$+$LEM}_{\neg} \]
%and a computable rate of metastability if it is provable in
%\[\].
%Moreover, already the addition of $\LEM$...\\

\subsection*{Motivation}

As mentioned earlier, in this chapter we investigate 
various levels of (effective) quantitative 
information on convergence theorems or, more precisely,
theorems about Cauchy property of a sequence 
in a metric space $(X,d)$
\[
\forall k\in\NN\,\exists n\in\NN\,\forall m,\tilde{m}\ge n
\ \big( d(x_m,
x_{\tilde{m}})\le 2^{-k}\big) 
\]
and also more general $\Pi^0_3$-theorems.  
The probably weakest information is 
a rate of metastability, namely a functional 
$\Phi$ such that 
\[\forall k\in\NN\,\forall f:\NN\to\NN\, \exists n\le \Phi(k,f) 
\,\forall i,j \in [n;n+f(n)]\ \big( d(x_i,x_j)\le 2^{-k}\big). \]
In next chapter we will discuss some general logical metatheorems for 
strong systems of analysis based 
on full classical logic, which guarantee the extractability of (sub-)recursive 
(and highly uniform) rates of metastability and give
a case study on such an extraction.

Another such result is the already discussed extraction of a 
uniform rate of metastability for the strong convergence in the mean 
ergodic theorem for uniformly convex Banach spaces $X$ carried out in 
\cite{kohlenbachleustean09}. The also aforementioned recent
observation by Avigad and Rute \cite{Avigad/Rute} that the analysis 
in \cite{kohlenbachleustean09} can be used to obtain a simple effective (and 
also highly uniform) 
bound on the number of fluctuations, was one of the
main inspirations for the article~\cite{KS13}. We base
large parts of this chapter on~\cite{KS13}. With our effective learnability, we will try 
to give answer to the question whether there are general logical conditions on 
convergence proofs to guarantee the extractability of effective bounds 
on fluctuations, which arises naturally.\\
Of course, the extractability of 
a computable rate of convergence is a sufficient condition for this, however this
topic is well known (see e.g. \cite{Kohlenbach08} or \cite{GerKoh06}).
Let us note though, that all such criteria are based on intuitionistic
systems, with the use of ``classicality'' limited to $\mbox{LEM$_\neg$} $.
Moreover, adding an important weak principle of classical logic,
the so-called Markov principle (extended to all finite types)
\[ \mbox{M}^{\omega}\ :\ 
\neg\neg\exists \underline{x}^{\underline{\rho}} \ \varphi_0(\underline{x})\to 
\exists \underline{x}^{\underline{\rho}}\,\varphi_0(\underline{x}), \] 
is permissible only if LEM$_{\neg}$ is weakened to the so-called lesser-limited-omniscience-principle 
LLPO. This is particularly interesting, since this is one of the standard
examples for how much ``classicality'' comes so to say automatically in non-standard systems,
which we discuss in the last chapter.

The so to say smallest step towards classical logic not covered by either of the 
above (but provable in their union) is $\LEM$.
Note that already HA$+\Sigma^0_1$-LEM 
allows to prove the Cauchyness of the Specker sequence, primitive recursive 
monotone decreasing sequences of rational numbers in $[0,1]$ which do not 
have a computable rate of convergence, while the scenarios above
typically include $\AC$ (which would make $\LEM$ significantly stronger).
Much related work to this was done in 
\cite{Toftdal, Akama, Hayashi/Nakata, Coquand,Berardi/Coquand/Hayashi, Aschieri/Berardi,Aschieri1,Aschieri2}.
An important observation is that $\Sigma^0_1$-LEM is already the general case. 
%let $(x_n)$ be a sequence of real numbers definable by a term $t$ 
%in HA$^{\omega}$ 
%(which may have variables of arbitrary type as parameters). Suppose that 
%\[ \mbox{PA}^{\omega}\vdash \forall k\,\exists n\,\forall m,\tilde{m}\ge n\,
%(|x_m-x_{\tilde{m}}|\le_{\RR} 2^{-k}). \]
%Then by negative translation (see \cite{Kohlenbach08}) 
%\[ \mbox{HA}^{\omega}\vdash  \forall k\,\neg\neg 
%\exists n\,\forall m,\tilde{m}\ge n\,
%(|x_m-x_{\tilde{m}}|\le_{\RR} 2^{-k}). \]
%Adapting Friedman's proof for the closure of HA$^{\omega}$ under the Markov 
%rule one can show (this is stated for HA without proof in 
%\cite{Hayashi/Nakata} and we include a proof -- also for 
%HA$^{\omega}[X,\|\cdot\|]$ -- below) that HA$^{\omega}+
%\Sigma^0_1$-LEM  
%is closed under the rule version of $\Sigma^0_2$-DNE so that we get 
%\[ \mbox{HA$^{\omega}+\Sigma^0_1$-LEM} \ 
%\vdash \forall k\,\exists n\,\forall m,\tilde{m}\ge n\,
%(|x_m-x_{\tilde{m}}|\le_{\RR} 2^{-k}). \] 
%Moreover, if $t$ contains at most number parameters it also suffices to 
%use the restriction $\Sigma^0_1$-LEM$^-$ of $\Sigma^0_1$-LEM to 
%$\Sigma^0_1$-formulas with number parameters only. All this also holds for 
%the systems HA$^{\omega}[X,\|\cdot\|]$ and PA$^{\omega}[X,\|\cdot\|]$ 
%and sequences $(x_n)$ in $X$ defined by terms of these systems.
%\\[2mm] 
Namely, as far as $\Pi^0_3$-theorems go\footnote{even
with parameters in $\NN,\NN^{\NN}$}, 
there is no difference in proofs based 
on full classical logic and proofs using only $\LEM$. 
%\footnote{for proofs which can be formalized in systems to which 
%negative translation and Friedman's $A$-translation apply}. 
We discuss this in detail prior to Proposition~\ref{p:limComp}, where 
we formalize and prove this claim.\\
So in order to have a stronger computational content than simply metastability, but
also cover more than the well studied case when we have a full and computable
rate of convergence we have to restrict the use of ``classicality'' somewhere
between $\LEM$ and $\mbox{LEM}_\neg$.  \\  
It turns out, that the right amount is to allow $\LEM$ to be used
only fixed many times as an instance. See Theorem~\ref{t:bdLem}, where
we show that from such proofs one can always extract effective (and in 
fact primitive recursive in the sense of G\"odel's $T$) bounds $B,L$ on the 
effective learnability of a rate of convergence of the sequence in question.
 Here 
$B,L$ are effective functionals (in the parameters of the proved statement). We
call $L$ the learner or learning procedure and $B$ the bound on the number of 
steps along this procedure.  
We will show that this is a strictly stronger information than a 
rate of metastability as 
the latter can be obtained from the lernability
by a simple and uniform primitive 
recursive procedure (in the ordinary sense of Kleene -- which we will use in this motivation as default).
However, there are primitive recursive 
Cauchy sequences with a primitive recursive rate of metastability which 
do not admit any computable bound for the 
learnability (of a rate of convergence).\\ 
We will further show that the $B,L$ learnability of a Cauchy statement
demonstrates itself in a particular
simple structure of the rate of metastability of the underlyinf sequence:
\[ (L^*(\underline{a}^*)\circ g)^{(B^*(\underline{a}^*))}(0), \] 
where $f^{(x)}(0)$ denotes the $x$-times iteration of the function $f$ 
starting from $0$ (note that the majornts $B^*,L^*$ do not 
involve the counterfunction). It is precisely this form of a rate 
of metastability that has been observed many times in concrete unwindings 
in ergodic theory and fixed point theory (see e.g.~\cite{AGT10,kohlenbachleustean09,kohlenbachleustean10,Kohlenbach2011,Kohlenbach/Leustean6,Kohlenbach/Leustean7}). For the first time this, until the knowledge of this result rather dazzling, circumstance can be explained
in natural terms. This is because we not only connect the underlying logical structure but also give a rather natural condition,
which is to be expected to occur. \\
On the other hand, this makes our results from the next chapter even more interesting.
Namely, a notable exceptions to this restricted format of metastability are the rates 
of metastability for 
the ergodic theorem for odd (and even more general) operators in 
\cite{Safarik(11)} where Wittmann's proof maps to a nested use of the iteration procedure and 
for the (weak convergence in the) Baillon nonlinear ergodic theorem 
in \cite{Kohlenbach(Baillon)} where the proof maps nested use of a bar recursive 
functional. However, both underlying proofs violate our criterion of a bounded use of $\LEM$
as defined in~\ref{t:bdLem}.\\

Comming back to the original motivation, a bound on the number of fluctuations is a   
strictly stronger quantitative information still (we give a separating example).
Together with the already discussed Specker sequences 
%(which have 
%trivial fluctuation bounds but no effective rates of convergence) 
we get the (w.r.t. effectivity) strict hierarchy of 
quantitative data for the convergence of Cauchy sequences 
(with -- as quoted at various places above -- a plenty of real examples from analysis/ergodic theory as
they arise in praxis of proof mining for each level, in particular for the newly defined level 3):
\begin{enumerate}
\item 
A rate of convergence.
\item 
A bound on the number of approximate fluctuations.
\item 
The $B,L$\nbd learnability of a rate of convergence by 
$B$-many mind changes by a learning procedure $L$ (see below for a precise definition). 
\item 
A rate of metastability.
\end{enumerate}   
%While -- as discussed above -- the extractability of effective data 
%for the levels 1, 3 and 4 of this hierarchy is guaranteed by 
%relatively easy to check logical {\bf a-priori}  
%conditions on the framework in which a Cauchy statement is proven, 
%this seems to be different for level 2: we give a kind of gap condition 
%to be satisfied by $L$ in `3.' which suffices to convert the 
%information provided by $B,L$ into a bound $F$ on the fluctuations. 
%Since to check this condition requires the inspection of the extracted 
%data $B,L$ this is an {\bf a-postiori} condition.
%\\[2mm] 
%The above hierarchy apparently fits well to distinguish the 
%computational content that recently has been obtained from proofs 
%in the context of ergodic theory: \\[1mm] 
%As discussed already, Avigad and Rute \cite{Avigad/Rute} observed 
%that the extraction of a rate of metastability from Birkhoff's 
%proof of the mean ergodic theorem in uniformly convex Banach 
%spaces carried out in \cite{kohlenbachleustean09} can in fact been 
%used to even obtain a uniform effective fluctuation bound 
%This corresponds to second level of our hierarchy 
%which by \cite{Avigad/Gerhardy/Towsner} 
%cannot be further improved effectively to the first level.
%\\[1mm] The logical condition needed to assure the extractability of a 
%primitive recursive 
%(in the ordinary sense of Kleene) data $B,L$ for the third 
%level is e.g. satisfied in 
%the proofs of the strong convergence of so-called Halpern iterations 
%in Hilbert spaces (due to Wittmann \cite{Wittmann(92)}) and -- more 
%generally --  in CAT(0) spaces (due to Saejung in 
%\cite{Saejung}). This follows from the 
%analysis of Saejung's proof 
%given in \cite{Kohlenbach/Leustean6} where a primitive 
%recursive (in the ordinary sense) rate of metastability is extracted
%(the analysis of Wittmann's proof in \cite{Kohlenbach2011} also 
%shows this in the Hilbert case).  
%A special form of the Halpern iteration (covered by 
%\cite{Wittmann(92),Saejung}) is given by  
%\[ x_{n+1} :=\frac{1}{n+2}x_0+\left(1-\frac{1}{n+2}\right) Tx_n \]
%can be viewed as a nonlinear generalization of 
%the ergodic average 
%\[ \frac{1}{n+1}\sum^n_{i=0}T^ix_0 \] 
%in the mean ergodic theorem with which it coincides 
%for {\bf linear} nonexpansive  maps $T.$ 
%As a corollary we obtain the extractability even 
%of primitive recursive learnability data $B,L$ in this case. 
%However, the aforementioned gap condition does not seem to be 
%satisfied and so, as it stands, a fluctuation bound does not seem 
%to follow.
%\\[1mm]  
%While the strong convergence of the ergodic average in general is known 
%to fail for nonlinear nonexpansive maps (whereas weak convergence 
%holds by a deep theorem of Baillon \cite{Baillon(75)}), 
%it does hold e.g. 
%for odd operators as shown again by Baillon \cite{Baillon(76)} 
%and -- for a much more 
%general class of mappings -- by Wittmann \cite{Wittmann(90)}. 
%Recently, the second author 
%\cite{Safarik(11)} extracted 
%a (primitive recursive in the sense of Kleene) rate of metastability 
%(i.e. level 4) from Wittmann's proof. However, Wittmann's proof does 
%not satisfy the condition sufficient to guarantee level-3 learnability 
%data and, in fact, the extracted bound has the same structure as the 
%one in our example of a primitive recursive sequence separating the 
%levels 3 and 4 given section 4 below. 



\subsection{Fluctuations versus effective learnability}
To be specific, let us use in the following the language of (intuitionistic) 
arithmetic in all finite types HA$^{\omega}$ (more precisely the system 
WE-HA$^{\omega}$, see \cite{Kohlenbach08}) as well as its extension HA$^{\omega}[X,\|\cdot\|]$ 
by an abstract normed space $X$ in the sense of \cite{Kohlenbach05meta,GerKoh06,Kohlenbach08} in order to be able to cover also the aforementioned 
recent applications of proof mining to ergodic and fixed point theory which 
need this enriched language. Everything we say extends mutatis mutandis also 
to theories where more conditions on $X$ are prescribed (e.g. $X$ being 
a uniformly convex or a Hilbert space) and convex subsets $C$ of $X$ being 
added as well as to metric structures $X$ 
such as metric, $W$-hyperbolic and CAT(0) spaces (see \cite{Kohlenbach08} 
for all this). The type of natural numbers $\NN$ is usually denoted by $0$ 
while $1$ denotes the type of functions $\NN\to\NN.$
\\[1mm] ${\cal S}^{\omega,X}$ denotes the full set-theoretic model of these 
theories over the base types $\NN$ and $X.$ Occasionally, we will need 
the relation `$x^*$ majorizes $x$' (short: $x^* \ maj \ x$) due to W.A. Howard 
(for the finite types over $\NN$)  
which is defined in the usual hereditary 
way by induction on the type of $x$ starting from 
\[ x^* \ maj_0 \ x\ :\equiv x^*\ge x \] for $x^*,x$ of type $0$ and 
\[ x^* \ maj_X \ x\ :\equiv x^* \ge \| x\|\] for $x$ of type $X$ and 
$x^*$ of type $0.$ 



\begin{dfn}[the number of fluctuations]
\label{d:nfluc}
For a sequence $x_{(\cdot)}$ in some metric space $(X,d)$ and an $\epsilon>0$
let $\Fluc(n,i,j)$ denote that there are $n$ fluctuations whose indexes are encoded into $i$ and $j$.
\begin{align*}
\Fluc(n,i,j):\equiv\Fluc_{x_{(\cdot)},\epsilon}(n,i,j):\equiv\quad &\lh(i)=\lh(j)=n\quad \wedge\\ 
&\forall k<n\ (i_k<j_k) \quad \wedge\\
&\forall k<n-1\ (j_k\leq i_{k+1}) \quad \wedge\\
&\forall k<n\ (d(x_{i_k},x_{j_k})>\epsilon).
\end{align*}
We call $b$ a bound on the number of $\epsilon$-fluctuations of $x_{(\cdot)}$, iff 
\[
\forall n>b \forall i,j \neg\Fluc(n,i,j).
\]
We call $b$ effective if it is computable in $\epsilon\in\QQ^*_+$ and $x_{(\cdot)}$.
\end{dfn}



In the Language identification in the limit model for inductive inference, the 
notion of learnable with an existence of a mind change bound was introduced in 
the sixties (see e.g.~\cite{Gold(67)}). We define a similar concept in the context of 
general formal theories like $\pa$. Since we require both the learning 
procedure and the bound on mind changes to be recursive (effective) we call 
this property {\em effective learnability}. 
\begin{rmk}
The proof-theoretic study of 
learnability by finitely 
many (though not necessarily effectively bounded) mind changes in analysis 
has been initiated by Hayashi (see e.g. \cite{Hayashi02,Hayashi06}) 
who (with Nakata) 
established the close relation of this 
concept to limit computability (see e.g. 
\cite{Hayashi/Nakata}). The concept of mind change for Cauchy statements is 
also implicit in section 5.1 of \cite{Ziegler(07)} (Proof of Lemma 31.c). 
Effective learnability concepts for 
functionals $F:D\to\NN^{\NN}$ (with $D\subseteq\NN^{\NN}$) have recently 
been investigated in \cite{Higuchi/Kihara}.
\end{rmk}
Before we consider our most general definition, let us consider learnability of monotone formulas (which is a rather rich class of statements including, in 
particular, Cauchy statements), since in this case, the definition is not 
only extremely intuitive, but also fairly definite.

% 
%	fmcMon
%
\begin{dfn}[$(B,L)$-learnable monotone formulas]\label{d:fmcMon}
Consider a $\Sigma^0_2$ formula $\phi$ with the only parameters $\tup a^{\tup \sigma}$, i.e.
\[\phi\ \equiv\ \exists n^0 \forall x^0\ \phi_0(x,n,\tup a),\]
which is monotone in $n$, i.e.
\[ 
\forall n^0\ \forall n'\ge n\ \forall x^0\ \big(  \phi_0\ (x,n,\tup a) \rightarrow  \phi_0(x,n',\tup a) \big).
\]
We call such a formula $\phi$ 
{\em (B,L)-learnable},
if there are function(al)s $B$ and $L$ such that the following holds 
(in the full set-theoretic model ${\cal S}^{\omega,X}$):
\[ 
\exists i\leq B(\tup a)\ \forall x\ \phi_0(x,c_i,\tup a),\] where
\begin{align*}
c_0&:=0,\\
c_{i+1}&:=
\begin{cases}
L(x, \tup a),&\text{for the $x$ with } \neg\phi_0(x,c_i,\tup a)\ \wedge\ \forall y<x\ \phi_0(y,c_i,\tup a) \text{ if it exists}\\
c_i,&\Telse.
\end{cases}
\end{align*}
We call such a $\phi$ 
{\em effectively learnable (with effectively bounded many mind changes)} 
if it is $(B,L)$-learnable with computable functionals $B$ and $L.$ 
\end{dfn}


This definition is very intuitive in the sense that it formalizes the concept of an (effective) learning process $L$ which learns the witness in an effectively bounded number of attempts in a very straightforward way.\\
Moreover, this definition allows the learner, i.e. the function $L,$ to use the least amount of non-computable information possible, namely only the smallest counterexample to the learners last candidate for the witness. 
Nevertheless, we will show that this amount of information is, in a sense, 
already exhaustive. More precisely, we have the following (we use in the 
rest of this section a primitive recursive surjective sequence coding 
denoted by $\langle\cdots\rangle$)\footnote{I.e. the projection, length and 
concatenation functions are primitive recursive.}:
\begin{prop}\label{p:allx}
Consider a monotone formula $\phi$ as above. Suppose there are $B$ and $L'$ s.t.
\[ \exists i\leq B(\tup a)\ \forall x\ \phi_0(x,c'_i,\tup a),\] where this 
time $L'$ can access all reasonable information, i.e.
\begin{align*}
c'_0&:=0,\\
c'_{i+1}&:=
\begin{cases}
L'(\langle x_0,\ldots,x_i\rangle, \langle c'_0,\ldots,c'_i\rangle,\tup a), &\text{for those $x_j,c'_j$, $j\leq i$ with }\\
 &\neg\phi_0(x_j,c'_j,\tup a)\ \wedge\ \forall y<x_j\ \phi_0(y,c'_j,\tup a)\\
  &\text{ if each exists},\\
c'_i, &\Telse.
\end{cases}
\end{align*}
Then $\phi$ is $(B,L)$-learnable in the original sense (as defined in Definition~\ref{d:fmcMon}) and $L$ is primitive recursively definable in $L'$.
\end{prop}
\begin{proof}
W.l.o.g we can assume that there is a $j\leq B'(\tup a)$ s.t. $\forall i<j\ (c_{i+1}>c_i)$ and $\forall i\geq j\ (c_{i+1}=c_i)$ (we can actually primitive recursively define a learner which satisfies this property whenever we have $B'$ and $L'$ as above). We set
\begin{align*}
L(x, \tup a):=L(\langle\rangle,\langle\rangle,x, \tup a)&:=\begin{cases}
0,&\Tif \phi_0(x,0,\tup a),\\
L( \langle X(x,0,\tup a)\rangle,\langle0\rangle,x,\tup a),&\Telse,\end{cases}\\
L(\langle \underbrace{x_0,\ldots,x_i}_{\tup x}\rangle, \langle \underbrace{c_0,\ldots,c_i}_{\tup c}\rangle,x,\tup a)&:=\begin{cases}
c_i,&\Tif \bigvee_{j\leq i}\phi_0(x_j,c_j,\tup a),\\
l':=L'(\langle {\tup x}\rangle, \langle {\tup c}\rangle,\tup a),&\Tif x=x_i\vee \phi_0(x,l',\tup a),\\
&\phantom{\Tif} \bigwedge_{j\leq i}\neg\phi_0(x_j,c_j,\tup a),\\
L( \langle {\tup x},X(x,l',\tup a)\rangle,
\langle \tup c,l'\rangle,x,\tup a) &\Telse,
\end{cases}\\
\end{align*}
where $X(x,c,\tup a):=\min\{x'\leq x\ :\ \neg \phi_0(x',c,\tup a)\}$ (or $x$ if there is no such $x'$).\\
{\em We show by induction on $i$ that $\forall i\ (c'_i = c_i)$.} This is obvious for $i=0$, moreover 
if $\forall x\ \phi_0(x,c'_{i},\tup a)$ then also $\forall x\ \phi_0(x,c_{i},\tup a)$ and so $c'_{(\cdot)}=c_{(\cdot)}$ (both) by the induction hypothesis.\\ Otherwise, we have the smallest counterexample $x'_i$ to $c'_i$, since by our hypothesis $c'_i=c_i$ we have also $x_i=x'_i$ for the smallest counterexample to $c_i$. 
So, by monotonicity of $\phi$ we obtain for all $j<i$ that $x'_j\leq x_i$, so $X(x_i,c'_j,\tup a)=x'_j$ and by definition of $L$ we get in total that 
\begin{align*}
c_{i+1}=L(x_i,\tup a)&=L(\langle X(x_i,0,\tup a)\rangle,\langle0\rangle,x_i,\tup a)\\
&=L(\langle x'_0\rangle,\langle0\rangle,x_i,\tup a)\\
&=L(\langle x'_0,  X(x_i,   L'(\langle x'_0\rangle,\langle0\rangle,\tup a)    ,\tup a)\rangle,\langle0, 
     L'(\langle x'_0\rangle,\langle0\rangle,\tup a)  \rangle,x_i,\tup a)\\
&=L(\langle x'_0,  X(x_i,   c'_1    ,\tup a)\rangle,\langle0, 
     c'_1  \rangle,x_i,\tup a)\\
&=L(\langle x'_0,  x'_1\rangle,\langle0, 
     c'_1  \rangle,x_i,\tup a)\\
&=\ldots\\
&=L'(\langle\tup x'\rangle,\langle\tup c'\rangle, \tup a)=c'_{i+1}.
\end{align*}


\end{proof}

So from now on we will simply use $L$ in the form which suits us best.
\\[2mm] 
Speaking of a Cauchy sequence $a_{(\cdot)}$, we would say that it has an effectively learnable rate of convergence, if there is a recursive computation for a bound $b$ from the sequence $a_{(\cdot)}$ and a positive $\epsilon$, such that there is a procedure to learn an $\epsilon$-Cauchy point with at most $b$ computable corrections (computable in a counterexample $x$, which in turn may not be computable itself!).
%
%  some error
%
\begin{rmk}\label{r:smallestCE}
In Definition~\ref{d:fmcMon}, even the condition that $x$ is the smallest counterexample, i.e. $\forall y<x\ \phi_0(y,c_i,\tup a)$, isn't really necessary. Of course, then we need $B$ to be a bound for any such sequence $c_{(\cdot)}$,\footnote{The alternative, letting $B$ be a bound for just some sequence makes little sense, since if there was any bound and learner at all, then $B(\tup a)=1$ would be a correct bound for the same learner as well.} since it is not unique anymore (not in general). However, such a definition would be equivalent:
\begin{itemize} 
\item Any given bound $B$ for all such sequences is obviously in particular a bound for the one we used in the original definition.
\item On the other hand, given $B$ and $L$ satisfying our original definition, we could modify $L$ in such a way, that it looks for the smallest counterexample and uses that for its computation, assuring that we actually generate the same sequence $c_{(\cdot)}$ after all (e.g. set 
$L'(x,c,\tup a):=L(\min \{x'\leq x\ :\  \neg\phi_0(x',c,\tup a)\ \wedge\ \forall y<x'\ \phi_0(y,c,\tup a)\},c,\tup a)$ if such an $x'$ exists and 
$L'(x,c,\tup a):=L(x,c,\tup a)$ otherwise).
\end{itemize}
\end{rmk}

As far as monotone formulas are concerned we have yet another nice property.
\begin{prop}\label{p:majBL}
A monotone $\Sigma^0_2$-formula $\phi$ (see also Definition~\ref{d:fmcMon}) that is $(B,L)$-learnable (uniformly in the parameters $\underline{a})$ 
is also $(B^*,L^*)$-learnable (uniformly in majorants $\tup a^*$ of 
$\tup a$) for any majorants $B^*$,$L^*$ of $B,L,$ i.e. 
(in ${\cal S}^{\omega,X}$)  
\[ \forall 
\underline{a}^*,\underline{a} \big( 
\underline{a}^* \ \maj\ \underline{a}\to 
\exists i\leq B^*(\tup a^*)\ \forall x^0\ \phi_0(x,c^*_i,\tup a)\big),\] where
\begin{align*}
c^*_0&:=0,\\
c^*_{i+1}&:=
\begin{cases}
L^*(x, \tup a^*),&\text{for the $x$ with } \neg\phi_0(x,c^*_i,\tup a)\ \wedge\ \forall y<x\ \phi_0(y,c^*_i,\tup a) \text{ if it exists}\\
c^*_i,&\Telse.
\end{cases}
\end{align*}
\end{prop}
{\bf Proof.} Note that $B^*,L^*,\tup a^* \ maj \ B,L,\tup a$ implies that 
\[ B^*(\underline{a}^*) \ge B(\underline{a}) \,\wedge\,\forall x^0,y^0 
(x\ge y\to L^*(x,\underline{a}^*)\ge L(y,\underline{a}))).\] 
Now assume that $c^*_i\ge c_i.$ Then by the monotonicity of $\varphi$ 
we have for all $x$ 
\[ \neg \varphi_0(x,c^*_i,\underline{a})\rightarrow \neg \varphi_0
(x,c_i,\underline{a}) \] 
and so the smallest counterexample $x_i$ to $c_i$ is smaller (or equal) 
than the smallest counterexample $x^*_i$ to $c^*_i$ and so 
$c^*_{i+1}=L^*(x^*_i,\underline{a}^*) \ge L(x_i,\underline{a})=c_{i+1}.$ 
Inductively, we get $c^*_i\ge c_i$ for all $i\le B(\underline{a}).$ 
\hfill $\Box$  


\begin{rmk} If the parameters $\underline{a}$ have all types of degree 
$\le 1$, then $\varphi$ in the above proposition is learnable in $B^*,L^*$ 
uniformly in $\underline{a}$ since $a^M \ maj \ a^1,$ where 
$a^M(n):=\max\{ a(i):i\le n.\}.$
\\ In this sense, we can extend the term effectively learnable as follows. A monotone $\Sigma^0_2$-formula $\phi(\tup a)$ is 
{\em effectively learnable with finitely mind changes uniformly in 
majorants $\underline{a}^*$ of the parameters $\underline{a}$} if it is 
$(B^*,L^*)$-learnable (uniformly in majorants $\underline{a}^*$ of the 
parameters $\underline{a}$ by computable functionals $B^*,L^*$ and all elements of $\tup a^*$ are of type level at most one). 
Note that this means that in the System $\ha[X,\|\cdot\|]$, $\tup a$ could include parameters of types like $X,\NN\to X,X\to X$.
\end{rmk}



There are several ways to generalize our learnability definition. 
Of course one can drop the monotonicity condition, but we can also allow higher or abstract types for $n$ and $x$ (for $x$ we need to consider all sequences $c_{(\cdot)}$ since we can provide only a counterexample $x$, not the smallest counterexample $x$ -- see also Remark~\ref{r:smallestCE}). The question here is, what kind of information we do allow the learning function(al) $L$ to access. At the moment, it seems that there is not such a nice and definitive answer as in the monotone case. However, we will stick with a (not necessarily unique) definition, which
\begin{itemize} 
\item fits very nicely into the hierarchy of different concepts for computational information (see Proposition~\ref{p:bg2meta}),
\item makes effective learnability guaranteed by very clear logical conditions on the provability of the learned formula (see Theorem~\ref{t:bdLem}),
\end{itemize} 

% 
%	fmcNum
%
Although we do not treat the case of learnability for higher type objects 
in this paper, the following definition easily applies to this case as well 
and, therefore, is written in this generality:
\begin{dfn}[(B,L)-learnabilty for general (not necessarily monotone)formulas]
\label{d:fmcNum}
Consider an $\exists\forall$ formula $\phi$ with the only parameters $\tup a^{\tup \sigma}$, i.e.
\[\phi\ \equiv\ \exists n^\rho \forall x^0\ \phi_0(x,n,\tup a).\]
We call such a formula $\phi$ 
{\em $(B,L)$-learnable},
if there are function(al)s $B$ and $L$ such that the following holds:
\[ \exists i\leq B(\tup a)\ \forall x\ \phi_0(x,c_i,\tup a),\] where
\begin{align*}
c_0&:=0,\\
c_{i+1}&:=
\begin{cases}
L(\langle x_0,\ldots,x_i\rangle, \tup a), &\text{for those $x_j$, $j\leq i$ with }\\
 &\neg\phi_0(x_j,c_j,\tup a)\ \wedge\ \forall y<x_j\ \phi_0(y,c_j,\tup a)\\
  &\text{ if each exists},\\
c_i, &\Telse.
\end{cases}
\end{align*}
We call such a $\phi$ {\em effectively learnable}, if it is $(B,L)$-learnable,
$\sigma_i$ and $\rho$ have type level at most one, and $B$ and $L$ are computable.
\end{dfn}
\newcommand{\seq}{_{(\cdot)}}
\begin{rmk}
Again, this definition already captures (so to say in a primitive recursive way) the case where the learner could access the previous values of $c\seq$ as well. Simply consider
\begin{align*}
&L(\overbrace{\langle x_0,\ldots,x_i}^{\tup x:=}\rangle, \tup a):=\\&\quad \quad
L'\big(\langle \tup x\rangle, \big\langle
 \underbrace{L'(\langle x_0\rangle, \langle 0\rangle,\tup a)}_{c'_1:=},
 \underbrace{L'(\langle x_0,x_1\rangle, \langle 0,c'_1\rangle,\tup a)}_{c'_2:=}, 
 \ldots,
 L'(\langle \tup x\rangle, \langle 0,c'_1,\ldots,c'_i\rangle,\tup a)
 \big\rangle, \tup a\big). 
\end{align*}
\end{rmk}

Of course, one could consider weaker concepts, like a learner which can access only the last counterexample (as in the monotone case). We considered also a learner of the kind $L'(x,c,\tup a)$ (i.e. a learner who is allowed to use in addition only the lastly learned solution candidate), which doesn't seem to be equivalent to any of the other two concepts.\\[2mm]
Let us make the properties of our learnability definition discussed above 
more transparent by proving the following results:
\\[1mm] 
In the following IP$^{\omega}_{\forall}$ denotes 
the independence-of-premise principle for universal 
premises in all finite types:
\[\mbox{IP}^{\omega}_{\forall}: \ \ (\forall \underline{x}\,A_0(\underline{x})
\to\exists y \,
B(y))\rightarrow \exists y\,( \forall \underline{x}\,
A_0(\underline{x})\to B(y)), \]
where $A_0$ is a quantifier-free formula and $\underline{x},y$ are 
variables of 
arbitrary types.
\begin{thm}[Bounded $\LEM$ guarantees learnability]\label{t:bdLem}
Given that
\[ \ba{l} 
\ha[X,\|\cdot\|]+\, \mbox{\rm AC$+$M$^{\omega}+$IP}^{\omega}_{\forall} 
\vdash\\[1mm] \hspace*{2cm} \forall \underline{a}\,\exists l^0\ \big( 
\forall m\leq_0 l \exists u^0 \forall v^0\ 
\big(\psi_0(u,m,\tup a) \vee \neg\psi_0(v,m,\tup a)\big)\rightarrow \exists 
n^0 \forall x^0 \phi_0(x,n,\tup a)\big),\ea 
\]
where $\phi_0,\psi_0$ are quantifier-free formulas (containing at most the 
parameters $\underline{a}$ free) and $t$ is a closed term in $\T$, 
then $\exists n\forall x \phi_0(x,n,\tup a)$ 
is (valid in ${\cal S}^{\omega,X}$) $(B,L)$-learnable (in the sense 
of Definition \ref{d:fmcNum} and for monotone formulas in the 
sense of Definition \ref{d:fmcMon}) by functionals given by 
closed terms of $\ha[X,\|\cdot\|].$ \\
To $B,L$ one can construct majorants 
$B^*,L^*$ given by closed terms of $\ha$ such that 
if $\exists n\forall x \phi_0(x,n,\tup a)$ is monotone (in the sense of Definition~\ref{d:fmcMon}), then it is even learnable in $B^*,L^*$ uniformly in majorants $\underline{a}^*$ of the parameters $\underline{a}.$
\end{thm}

\begin{proof}
Suppose that 
\[ \ba{l} 
\ha[X,\|\cdot\|]+\mbox{AC$+$M$^{\omega}+$IP}^{\omega}_{\forall} 
\vdash \\[1mm] \hspace*{2cm} 
\forall \underline{a}\exists l^0\ \big( 
\forall m\leq l \exists u \forall v\ \big(\psi_0(u,m,\tup a) \vee \neg\psi_0(v,m,\tup a)\big)\rightarrow \exists n \forall x \phi_0(x,n,\tup a)\big).
\ea \]
Then by the soundness of the G\"odel functional (`Dialectica') 
interpretation for $\ha[X,\|\cdot\|]+\mbox{AC$+$M$^{\omega}+$IP}^{\omega}_{\forall}$ (see \cite{Kohlenbach08}) we obtain that (note that since we do not need 
bar recursion to interpret $\ha[X,\|\cdot\|]$ we do not have to go through 
the model of strongly majorizable functionals and so do not need to assume 
any smallness condition on the types of $\underline{a}$ to pass to 
${\cal S}^{\omega,X}$)   
\[ {\cal S}^{\omega,X} \models \exists l,V,N \forall U,x\ \Big(\forall m\leq l 
\big (\psi_0(Um,m,\tup a) \vee \neg\psi_0(VxU,m,\tup a)\big)
\rightarrow \phi_0(x,N(U),\tup a)\Big).
\]
where `$\exists l,V,N$' is witnessed (uniformly in $\underline{a}$) 
by closed terms $t,s_V,s_N$ of 
$\ha[X,\|\cdot\|]$. 
The result when the terms $s_V,s_N$ are applied to 
$\tup a$, we conveniently name $V$ and $N$.\\
To show the learnability, let $U_{\tup v}$ (where $\tup v$ is a $t(\tup a)$-tuple) denote the function \[
U_{\tup v}(i):=\begin{cases}
v_i&\Tif\ i<t(\tup a),\\
0&\Telse,
\end{cases}\] set $B(\tup a):=t(\tup a)$ and define $L$ in $N$ and $V$ via a sequence of $t(\tup a)$-tuples $\tup v^{(\cdot)}$.
More precisely to compute $L(\langle \underbrace {x_0,x_1,\ldots,x_i}_{\tup x:=}\rangle,\tup a)$ for some $i$ we need to
define the tuples $\tup v^{(0)},\ldots,\tup v^{(i)}$ as follows.
\begin{enumerate}
\item[$\tup v^0$] 
Set $\tup v^0:=0,\ldots,0$ and $c_1:=L(\langle x_0\rangle,\tup a):=
N(U_{\tup v^0})$.
\item[$\tup v^1$] If $\forall x \phi_0(x,c_1,\tup a)$ holds, then there is nothing to be 
done\footnote{
Of course this is undecidable, however the conclusion discussed next is. In this sense if the conclusion is wrong for the $x_1$ give as input to $L$, 
we can simply set $\tup v^1=\tup v^0$ and $L(\langle x_0,x_1\rangle,\tup a):=c_1$ (or even $0$ for all that it matters).
}. 
Otherwise, we have in particular (provided that $x_1$ is the minimal 
counterexample) 
\[
\exists m\leq t(\tup a) \big (\neg\psi_0(U_{\tup v^0}m,m,\tup a) \wedge \psi_0(Vx_1(U_{\tup v^0}),m,\tup a)\big)
\]
so we can denote the least such an $m$ by $m_0$ (put $m_0:=0$ in case such an 
$m$ does not exist) and define $\tup v^1$ as $\tup v^0$ except that we set $v^1_{m_0}:=Vx_1(U_{\tup v^0})$.
Furthermore, we set 
$c_2=L(\langle x_0,x_1\rangle,\tup a):=N(U_{\tup v^1})$. 
Note that we have \[ \neg\psi_0(U_{\tup v^0}m_0,m_0,\tup a) \wedge \psi_0(v^1_{m_0},m_0,\tup a). \tag{v0}\label{e:v0}\]
\item[$\tup v^2$] Now, if $\forall x \phi_0(x,c_2,\tup a)$ then we are finished. Otherwise, similarly as before we have 
\[
\exists m\leq t(\tup a) \big (\neg\psi_0(U_{\tup v^1}m,m,\tup a) \wedge \psi_0(Vx_2(U_{\tup v^1}),m,\tup a)\big)
\]
and we can denote the least such an $m$ by $m_1$ and define $\tup v^2$ as $\tup v^1$ except that we set $v^2_{m_1}:=Vx_2(U_{\tup v^1})$. As before this means that
\[ \neg\psi_0(U_{\tup v^1}m_1,m_1,\tup a) \wedge \psi_0(v^2_{m_1},m_1,\tup a), \tag{v1}\label{e:v1}\]
so in particular we obtain that $m_1\neq m_0$ by~\eqref{e:v0} as $U_{\tup v^1}m_1=v^1_{m_1}$. We set $c_3=L(\langle x_0,x_1,x_2\rangle,\tup a):=
N(U_{\tup v^2})$ and continue.
\item[$\tup v^3$] Again, if $\forall x \phi_0(x,c_3,\tup a)$ then we are finished. Otherwise, as before, we have that
\[
\exists m\leq t(\tup a) \big (\neg\psi_0(U_{\tup v^2}m,m,\tup a) \wedge \psi_0(Vx_3(U_{\tup v^2}),m,\tup a)\big)
\]
and we can denote the least such an $m$ by $m_2$ and define $\tup v^3$ as $\tup v^2$ except that we set $v^3_{m_2}:=Vx_3(U_{\tup v^2})$. As before this means that
\[ \neg\psi_0(U_{\tup v^2}m_2,m_2,\tup a) \wedge \psi_0(v^3_{m_2},m_2,\tup a), \tag{v2}\label{e:v2}\]
so in particular we obtain that $m_2\neq m_1$ by~\eqref{e:v1}. Moreover, by~\eqref{e:v0} and~\eqref{e:v2} we have $m_2\neq m_0$, since from $m_1\neq m_0$ follows that $v^2_{m_0}=v^1_{m_0}$.
 We set $c_4=L(\langle x_0,x_1,x_2,x_3\rangle,\tup a):=N(U_{\tup v^3})$ and continue.\\
\item[$\tup v^{n+1}$] Finally, in general assume that for some $n$ we have that $\forall i< n\forall j<i\ m_i\neq m_j$
 and $\forall i\leq n+1 \neg \forall x \phi_0(x,c_i,\tup a)$.
Then we have also that
\[ 
\forall i<n \big (\neg\psi_0(U_{\tup v^i}m_i,m_i,\tup a) \wedge \psi_0(Vx_{i+2}(U_{\tup v^{i+1}}),m_i,\tup a)\big). \tag{vi}\label{e:vi}
\]
As usual we have in particular that
\[ 
\exists m\leq t(\tup a) \big (\neg\psi_0(U_{\tup v^n}m,m,\tup a) \wedge \psi_0(Vx_{n+1}(U_{\tup v^n}),m,\tup a)\big)
\]
and we can denote the least such an $m$ by $m_n$ and define $\tup v^{n+1}$ as $\tup v^n$ except that we set $v^{n+1}_{m_n}:=Vx_{n+1}(U_{\tup v^n})$. As before this means that
\[ \neg\psi_0(U_{\tup v^n}m_n,m_n,\tup a) \wedge \psi_0(v^{n+1}_{m_n},m_n,\tup a). \tag{vn}\label{e:vn}\]
From $\forall 0<i<n\ ( m_0\neq m_i )$ it follows that $\forall0<i<n\ ( v^n_{m_0}=v^i_{m_0})$. Assume that $m_n = m_0$, then
\[U_{\tup v^n}m_n=v^n_{m_n}=v^n_{m_0}=v^1_{m_0}\]
and we obtain a contradiction as $\neg\psi_0(v^1_{m_0},m_0,\tup a)$ follows from~\eqref{e:vi} and $\psi_0(v^1_{m_0},m_0,\tup a)$ follows from~\eqref{e:vn}. This shows that 
$m_n \neq m_0$, similarly one shows that \[ \forall i<n\ (m_n \neq m_i).\]
As usual, we set $c_{n+2}=L(\langle x_0,\ldots,x_{n+1}\rangle,\tup a):=
N(U_{\tup v^{n+1}})$.
\end{enumerate}
This leads to the following definition of $L$:
\begin{align*}
 L(x,\tup a):= 
 L(\langle \underbrace {x_0,x_1,\ldots,x_i}_{\tup x:=}\rangle,\tup a):=
N(U_{\tup v^{i}}). 
\end{align*}
Note that since $N$ and $U$ are total, so is $L$. 
Moreover, if the values of $i$, $\tup x$, and $c$ satisfy the conditions from Proposition~\ref{p:allx}, then $L$ 
behaves as described above.\\
Finally, since there can be only $t(\tup a)$ many different $m_i$'s, it can happen at most $t(\tup a)$ many 
times that $\forall x \phi_0(x,c_i,\tup a)$ does not hold, where
$c_i$ is defined as in Proposition~\ref{p:allx} with $L$ as above. \\ The 
second claim follows from the fact that $t,s_V,s_v$ have majorants $t^*,
s^*_V,s^*_N$ 
given by closed terms of $\ha$ (see \cite{Kohlenbach08}) which then 
yield majorants $B^*,L^*$ of $B,L.$ Now apply  
Proposition \ref{p:majBL}.
\end{proof} 
\begin{rmk} \label{simple-L}
Assume that $\varphi_0$  in the theorem comes from a Cauchy statement 
\[ j_1(x),j_2(x)\ge n\to \widehat{\| 
a_{j_1(x)}-a_{j_2(x)}\|}(k+1)\le_{\QQ} 
2^{-k}, \] where -- referring to the representation of real numbers 
by number theoretic functions $f$ representing fast Cauchy sequences of 
rationals -- $\widehat{f}(k+1)$ is a $2^{-k-1}$-rational approximation to $f$ 
(see \cite{Kohlenbach08} for details).  Then 
$\varphi$ is monotone and a counterexample $x$ to $n$ satisfies 
$x\ge n$ (using that for the Cantor pairing function $x\ge j_i(x)$). 
Assume also that $\psi_0$ is 
monotone in $u$ (which always can be arranged by taking 
$\psi'_0(u,m,\underline{a}):\equiv \exists \tilde{u}\le u\,\psi_0
(u,m,\underline{a})$). \\ 
Then the complicated iteration used in defining 
$L$ can be avoided by taking simply 
\[ L^*(\langle x_0,\ldots,x_i\rangle,\underline{a}^*):= 
N^*_{\underline{a}^*}(\lambda k.V^*_{\underline{a}^*}(x_i,
\lambda n.x_i)), \] 
where $\lambda \underline{a}^*.N^*_{\underline{a}^*},\lambda \underline{a}^*.
V^*_{\underline{a}^*}$ are majorants of 
\[ \tilde{N}_{\underline{a}}(f):=\max\left\{ \max\left\{ 
N(v^0*0):lh(v)=t\underline{a}\wedge \forall l\le t\underline{a}(v_l\le f(l))
\right\},f(l):l\le t\underline{a}\right\}\] and 
\[ \tilde{V}_{\underline{a}}(x,f):=\max\left\{ \max\left\{ 
V(x,v^0*0):lh(v)=t\underline{a}\wedge \forall l\le t\underline{a}(v_l\le f(l))
\right\},f(l):l\le t\underline{a}\right\},\] 
where $N,V,t$ are as in Theorem \ref{t:bdLem}. Note that with $N,V$ also 
$\tilde{N},\tilde{V}$ satisfy the claim in the proof and that $L^*$ 
(for counterexamples $x_0,\ldots,x_i$) is 
an upper bound for the $L$ defined in terms of $\tilde{N},\tilde{V}$ as 
an elementary calculation shows (using that -- by the form of 
$\varphi_0$ -- a counterexample $x$ to $n$ has to satisfy $x\ge n$). 
By monotonicity, $\varphi$ is then also $(L^*,B^*)$-learnable (uniformly 
in majorants $\underline{a}^*$ of $\underline{a}$) where $B^*$ is some 
majorant of $B.$ 
\end{rmk}  
\begin{rmk} The theorem remains valid if arbitrary ${\cal S}^{\omega,X }$-true 
purely universal sentences are added as axioms to 
{\rm HA$^{\omega}[X,\|\cdot\|]$}. The part about the majorizing terms $B^*,L^*$ 
even remains valid -- using monotone functional interpretation -- 
if one adds sentences of the form $\Delta:\equiv 
\forall a\exists b\le sa\forall c F_0(a,b,c)$ with quantifier-free 
$F_0$ and closed $s$ as axioms which covers the case of the binary (`weak') K\"onig's lemma 
WKL (which together with AC even implies K\"onig's lemma KL); see 
\cite{Kohlenbach08} for extensive details on all this.
\end{rmk}
Using the representation of real numbers from 
\cite{Kohlenbach08}, each sequence of type $0\to 1$ can be viewed 
as a name of a sequence $(a_n)$ of reals. Now define 
$\tilde{a}_n:=\max_{\RR}\big(0,\min\nolimits_{i\le n}a_i\big).$ 
Let PCM$_{ar}(a_n)$ denote the statement that the monotone decreasing 
sequence $(\tilde{a}_n)$ in $[0,1]$ is Cauchy (see \cite{Kohlenbach08} 
for details) 
\[\mbox{PCM}_{ar}(a_n): \ \forall k\in\NN\,\underbrace{\exists n\in\NN\, 
\forall m\ge n\ (|\tilde{a}_m- \tilde{a}_n|\le 
2^{-k})}_{\mbox{PCM}_{ar}((a_n),k)
:\equiv} \]
(if $(a_n)$ is already 
a decreasing sequence in $[0,1]$, then $(\tilde{a}_n)=(a_n)).$ 
The usual classical proof of PCM$(a_n)$ uses $\Sigma^0_2$-DNE, but it can be 
converted into a proof that only needs the weaker $\Sigma^0_1$-LEM (see 
Proposition \ref{p:limComp} below). 
In \cite{Toftdal}, an explicit such proof is constructed 
exhibiting a concrete sequence of instances of $\Sigma^0_1$-LEM sufficient 
for this. From this proof one can read off the following even more detailed 
fact:
\begin{prop} There is a primitive recursive functional (in the 
ordinary sense) $\Phi$ such that (using the Cantor pairing function)
\[ \mbox{\rm HA}^{\omega}\vdash \forall a^{0\to 1}_{(\cdot)}, k^0\ 
\big( \forall m \le j(2^k-1,2^k) \,\Sigma^0_1\mbox{\rm -LEM}(\Phi(a_{(\cdot)},m)
\rightarrow 
\ \mbox{\rm PCM}_{ar}(a_{(\cdot)},k)\big). \] 
\end{prop} 
{\bf Proof.} The crucial step in Toftdal's proof in \cite{Toftdal} is 
to show by induction on $k$ that 
\[ \forall k\exists i\in \{ 1,\ldots,2^k\} \exists n \forall m \, 
\left( \frac{i-1}{2^k}\le \tilde{a}_{n+m}\le \frac{i}{2^k}\right), \]
where in the induction step $\Sigma^0_1$-LEM is used in the form 
(note that, based on our representation of real numbers, $<_{\RR}\in\Sigma^0_1$) 
\[\exists n\ \left( \tilde{a}_n <\frac{2i-1}{2^{k+1}}\right) \vee 
\neg \exists n\ \left( \tilde{a}_n <\frac{2i-1}{2^{k+1}}\right). \]
So to establish PCM$_{ar}(a_{(\cdot)},k)$ one needs all the instances 
\[\exists n\ \left( \tilde{a}_n <\frac{i}{2^{j}}\right) \vee 
\neg \exists n\ \left( \tilde{a}_n <\frac{i}{2^{j}}\right) \] 
for $i\le l-1$ and $l\le 2^k,$ i.e. the codes $j(i,l)$ of the instances 
used can be bounded by $t(k):=j(2^k-1,2^k).$  The construction of $\Phi$ is 
clear. 
 \hfill $\Box$ 

\mbox{ } 

While the usual classical proof of PCM$_{ar}$ only needs $\Sigma^0_1$-induction 
(but $\Sigma^0_2$-DNE), the above proof due to Toftdal needs an instance of 
the $\Sigma^0_2$-induction rule ($\Sigma^0_2$-IR) which, apparently, is the 
price to be paid for using only $\Sigma^0_1$-LEM (instead of 
$\Sigma^0_2$-DNE). Classically, $\Sigma^0_2$-IR is quite strong and proves 
(relative to PRA) the same $\Pi^0_3$-sentences as $\Pi^0_2$-IA 
(see e.g. \cite{Sieg}[Theorem 3.11]) and so, 
in particular, the totality of the Ackermann function. In our intuitionistic 
context, however, it is weak and the functional interpretation used 
(without negative translation!) in the 
proof of Theorem~\ref{t:bdLem} (and the corollary below) to extract $B,L$ solves 
$\Sigma^0_2$-IR using only ordinary primitive recursion in the form of 
$R_0.$ 
\\[2mm]
As a corollary we obtain that Theorem~\ref{t:bdLem} also holds with 
the original assumption being replaced by 
PCM$_{ar}(s(\underline{a},l),t(\underline{a})),$ where 
$(s(\underline{a},l)^{0\to 1})_l$ represents some sequence of reals defined by 
a closed term $s$ in $\underline{a}:$
\begin{cor} \label{cor.2.11}
Given that 
\[\ba{l} \mbox{\rm HA$^{\omega}[X,\|\cdot\|]+$AC$+$M$^{\omega}+
$IP}^{\omega}_{\forall}\vdash \\[1mm] \hspace*{2cm}  
\forall \underline{a}\,\exists k^0 ,l^0\ \big( 
\ \forall m \le l\ \mbox{\rm PCM}_{ar}(s(\underline{a},m),k)
 \rightarrow \exists n^0\forall x^0\,\varphi_0(x,n,\underline{a})\big), 
\ea \] 
where $s$ is a closed term and $\varphi_0$ as in Theorem \ref{t:bdLem}, then 
$\exists n^0\forall x\,\varphi_0(x,n,\underline{a})$ 
is (valid in ${\cal S}^{\omega,X}$) 
$(B,L)$-learnable (uniformly in $\underline{a}$) by functionals given by 
closed terms of the system $\ha[X,\|\cdot\|]$.\\
To $B,L$ one can construct majorants 
$B^*,L^*$ given by closed terms of $\ha$ such that 
if $\exists n\forall x \phi_0(x,n,\tup a)$ is monotone (see Definition~\ref{d:fmcMon}) then it is even learnable in $B^*,L^*$ uniformly in
majorants $\underline{a}^*$ of the parameters $\underline{a}$.
\end{cor}   
\begin{remark}
Of course, instead of sequences in $[0,1]$ one can also consider sequences 
in any compact interval $[-C,C],$ where then the functionals $B,L$ will 
additionally depend on $C.$ \\[1mm] Likewise, instead of decreasing 
sequences we may also have increasing ones or, if the Cauchy property 
is changed into the existence of an approximate infimum 
\[ \exists n\,\forall m (a_n\le a_m+2^{-k}), \] 
also arbitrary sequences in $[-C,C].$
\end{remark}

\begin{prop}\label{p:bg2meta}
Let $\exists m^0\forall k^0\,\varphi_0(n,m,k,\underline{a})$ be a formula 
in the language of $\ha$ (or $\ha[X,\|\cdot\|]$) with $\varphi_0$ being 
quantifier-free that 
is $(B,L)$-learnable in the sense of Definition \ref{d:fmcNum}  
(uniformly in $n$ and  $\underline{a}$) and let $B^*,L^*$ be majorants of  
$B,L$. \\ 
Then a rate of metastability $\Omega$ 
(valid in ${\cal S}^{\omega,X}$) 
\[ \forall n^0\,\forall g^1\, \exists m\le_0\Omega(g,n) \ 
\phi_0(n,m,g(m),\underline{a}) \]
for\footnote{Note that in order to talk about metastability, we need  one of the parameters, to have type 0 and we treat it separately.}  
\[
\forall n^0 \exists m^0 \forall k^0\ \phi_0(n,m,k,\underline{a})\tag{$\phi$}\label{e:phi}
\]
is given by $\Omega(B^*,L^*,\underline{a}^*)$ 
(uniformly in majorants $\underline{a}^*$ of 
the parameters $\underline{a}$), where 
$\tilde g(c):=\max\big(c,\max_{c'\leq c}(g(c'))\big)$ and
\begin{align*} 
  \Omega&:=\lambda B^*,L^*,\underline{a}^*,g,n\ .\ C(L^*,g,n,
B^*(n,\underline{a}^*),\underline{a}^*),\\
C(i):=C(L^*,g,n,i,\underline{a}^*)&:=
\begin{cases}0,&\Tif\ i=0,\\ L^*(\langle\overbrace{ \tilde g(C(i-1)+1),\ldots,\tilde g(C(i-1)+1) }^{i\times}\rangle,n,\underline{a}^*),&\Telse.\end{cases}
\end{align*}
Note that $\Omega$ is defined using only recursion $R_0$ of type $0$ and 
hence is primitive recursive in the usual sense of Kleene.
\end{prop}
\begin{proof} We reason in ${\cal S}^{\omega,X}.$
Since $\phi_0$ is quantifier-free, there is a closed term $f$ with 
$f(n,m,k,\underline{a})=0\leftrightarrow \phi_0(n,m,k,\underline{a})$. 
Hence, we have that (for $\underline{a}^*$ majorizing $\underline{a}$) 
\[ \forall n\ \exists i\leq B^*(n,\underline{a}^*)\ \forall k\ \ \big( 
f(n,c_i,k,\underline{a})=0 \big), \]
by the assumptions of the proposition, and we need to show that
\[ \forall g,n\ \exists m\leq \Omega(B^*,L^*,\underline{a}^*,g,n) 
\ \ \big( f(n,m,g(m),\underline{a})=0 \big). \]
Now, fix any $g$, $n$ and assume towards contradiction that 
$\Omega(B^*,L^*,\underline{a}^*,g,n)$ is not a rate of metastability, i.e. that
\be[e:NE] 
\forall m\leq \Omega(B^*,L^*,\underline{a}^*,g,n)  \ \ \big( f(n,m,g(m),\underline{a})\neq0 \big).
\ee
By induction on $i$ we obtain that
\be[e:IC] 
\forall i\leq B^*(n,\underline{a}^*) \,\big(c_i\leq 
C(L^*,g,n,i,\underline{a}^*)
\big).
\ee
The case $i=0$ is trivial as $c_0=C(L^*,g,n,0,\underline{a}^*)=0$. 
Next, suppose that for some 
$1\leq i\leq B^*(n,\underline{a}^*)$ the following holds
\be[e:IH] 
\forall j<i\ \big(c_j\leq C(L^*,g,n,j,\underline{a}^*)\big). 
\ee
Denoting the smallest $k$ s.t. $f(n,m,k,\underline{a})\neq0$ by $x_m$ 
(if this does not exist, we have $c_i=c_{i-1}$ and are done), 
we obtain by~\eqref{e:IH} that\footnote{Here
we simply assume that our encoding is monotone in its components. If for some reason it was not, we could use a $L'$ which returns
the maximal value among all codes coordinatewise bounded by the elements of the encoded input of $L^*$.}
\[
   c_i\leq L^*(\langle x_{c_0},\ldots,x_{c_{i-1}}\rangle, n,\underline{a}^*)
   \leq L^*(\langle \tilde g({c_0}),\ldots,\tilde g({c_{i-1}})\rangle, n,\underline{a}^*)\le C(L^*,g,n,i,\underline{a}^*),
\]
since by~\eqref{e:NE} we have (using that $C(i)$ is nondecreasing in $i$) 
that \[ \forall i\leq B^*(n,\underline{a}^*)\ 
\big(m\leq C(L^*,g,n,i,\underline{a}^*)\rightarrow f(n,m,g(m),\underline{a})\neq0 \big)\] and 
so, in particular, that
\[ \forall i\leq B^*(n,\underline{a}^*)\ \big(m\leq C(L^*,g,n,i,\underline{a}^*)
\rightarrow x_m\leq g(m)\leq \tilde g(m) \big). \]
Finally, we can infer from~\eqref{e:IC} that
\[ \forall i\leq B^*(n,\underline{a}^*)\ 
\big(c_i\leq \Omega(B^*,L^*,\underline{a}^*,g,n)\big),\]
and therefore and by~\eqref{e:NE} also
\[ \forall i\leq B^*(n,\underline{a}^*)\ \neg\forall k^0\big( 
f(n,c_i,k,\underline{a})=0\big) \]
for all majorizable $\underline{a}$ contradicting the $(B,L)$-learnability 
(uniformly in $n$ and $\underline{a}$) of \\ 
$\exists m\forall k\,\varphi_0(n,m,k,\underline{a})$ and the fact that $B^*$ 
majorizes $B$ (so that $B^*(n,\underline{a}^*)\ge B(n,\underline{a})).$

\end{proof}

\begin{rmk}\label{r:metastr}
Note that $\Omega$ has essentially the following form\footnote{Note that 
the additional dependency on the number $i$ of iterates in 
Proposition \ref{p:bg2meta} via the length of the sequence 
$\langle \ldots\rangle$ can also 
be covered by 
this normal form, since -- by $\tilde{g}(C(i-1)+1)\ge\tilde{g}(i)\ge i$ -- 
the length of the sequence $\langle\ldots\rangle$  can be majorized by 
$\tilde{g}(C(i-1)+1)$ itself.}
\begin{align*}
&(L_{n,\tup a^*}\circ \tilde{g})^{B^*(n,\tup a^*)}(0),\\
&L_{n,\tup a^*}:=\lambda x\ .\ L^*(x,n,\tup a^*).
\end{align*}
Moreover, if we have such a rate of metastability for some Cauchy statement 
$\varphi$ as considered in Remark \ref{simple-L} so that 
$\varphi$ is monotone and a counterexample $x$ is always greater than the 
witness candidate, and given any $n,\tup a^*$ we have an 
$f^1$ and a $b^0$ such that for all $\underline{a}$ 
that are majorized by $\underline{a}^*$ 
\be[e:nfbA]
\forall g \exists m \leq (f\circ \tilde{g})^b(0) \phi_0(n,m,g(m),\tup a),
\ee
then $\phi$ is $B,L$-learnable (uniformly in $n$ and majorants 
$\underline{a}^*$ for $\underline{a}$) with
\[
B(n,\tup a^*):=b,\quad L(x,n,\tup a^*):=f(x).
\]
To prove this fact, we argue as follows. Fix arbitrary $n,\tup a^*$ and
consider corresponding $f$ and $b$. Let
\[
g(m)=\min\{x\ : \neg\phi_0(n,m,x,\tup a) \},
\]
if such an $x$ exists and $0$ otherwise. Note that due to the monotonicity
of $\phi$ we have $g(m)\neq0\rightarrow \tilde g(m)=g(m)$. This implies that
as long as there is a (the smallest) counterexample $x_i$ to $c_i$, it holds that
\be[e:nfb0]
c_{i+1}=L(x_i,n,\tup a^*)=f(x_i)
=(f\circ \tilde g)c_i=(f\circ \tilde g)^{i+1}(0).
\ee
Given all this, assume towards contradiction that
\be[e:nfb1]
\forall i\leq B(n,\tup a^*) \exists x \neg\phi_0(n,c_i,x,\tup a),
\ee
and consider any $m\leq (f\circ \tilde{g})^b(0)$. Due to~\eqref{e:nfb0} and~\eqref{e:nfb1},
we get that $m\leq c_b$ and due to the monotonicity of $\phi$ this means that
there is a counterexample to $m$ (since there is one for $c_b$ by~\eqref{e:nfb1}, as $B(n,\tup a)=b$), which 
means that $\neg\phi_0(n,m,g(m),\tup a)$ by definition of $g$. This is a contradiction
to~\eqref{e:nfbA}.
\end{rmk}

{\bf Discussion:} What the main results in this section (Theorem 
\ref{t:bdLem} and Proposition \ref{p:bg2meta}), taken together, show 
is that if the proof of a (monotone) $\Pi^0_3$-statement (e.g. a 
Cauchy statement) uses only a bounded (in the 
parameters) number of unnested 
$\Sigma^0_1$-LEM$^-$-instances but may use induction 
of unrestricted complexity, then we get a rate of metastability which 
-- as a functional in the counterfunction $g$ -- has a very simple structure 
(namely only a single use of iteration of g). This is remarkable as e.g. 
HA$^{\omega}$ does, of course, 
prove $\forall g \exists x\,\psi_0(g,x)$-sentences 
which need 
much more complicated functionals in $g$ (namely every type-2 functional
definable in G\"odel's calculus $T$ arrives in this way). What we have shown, 
however, is that this cannot happen if $\forall g \exists x\,\psi_0(g,x)$
results as the Herbrand normal form of a $\Pi^0_3$-statement 
$\varphi\equiv \forall n^0\exists m^0\forall k^0\,\varphi_0(n,m,k)$ 
that is provable 
in HA$^{\omega}$ from the aforementioned restricted uses of 
$\Sigma^0_1$-LEM$^-.$ To see the difference, 
let us consider the simple but already illuminating 
case where the proof of $\varphi$ does not use classical logic 
at all. Then one can use modified realizability or functional interpretation 
to extract a (definable in $T$) witnessing term $t$ for $\varphi$ 
(and hence a bound $t^*$ which is uniform in majorants of the parameters) 
which then a fortiori is also a rate of metastability for $\varphi$ 
which does not use the argument $g$ at all.
The `$g$-involvement' displayed by a rate of metastability reflects the 
amount of $\Sigma^0_1$-LEM$^-$ used in the proof and the former 
is simple if the latter is low. Moreover the extraction of the rate of 
metastability via the extraction of the learning procedure $L^*$ proceeds 
without any use of negative translation but with direct functional 
interpretation (while modified realizability would not be sufficient as 
we need the functional witnessing $V$ in the proof of Theorem 
\ref{t:bdLem}).
\\[2mm] 
A sort of complementary scenario would be to allow full classical logic 
in a proof but to restrict the use of induction to a bounded number of 
unnested instances of $\Sigma^0_1$-IA, the latter being used e.g. in 
the form of PCM$_{ar}.$ So consider a proof 
\[\mbox{\rm G$_3$A}^{\omega}[X,\|\cdot\|]\,
\vdash   
\forall \underline{a}\, \big( 
\ \mbox{\rm PCM}_{ar}(t_1(\underline{a}),
t_2(\underline{a}))
 \rightarrow \exists n^0\forall x^0\,\varphi_0(x,n,\underline{a})\big).  \] 
Then by negative translation and functional interpretation one can 
extract a rate of metastability for the conclusion 
making a single use of the rate of 
metastability of PCM$_{ar}$ given by a single application of the iteration 
$\tilde{g}^{k}(0)$ (see \cite{Kohlenbach08} prop.2.26) 
and terms $t[g]$ of G$_3$A$^{\omega}$ which can be 
majorized by terms using only a fixed number of $g$-nestings (reasoning 
as in the proof of Proposition \ref{p:nonLearnablePhi} below). 
This again leads to a rate of 
metastability which can be put into the form in Remark \ref{r:metastr}.\\[1mm] 
So to get a more complicated rate of metastability (e.g. of the form 
$\tilde{g}^{(\tilde{g}^x(0))}(0)$) requires a nested use of a {\bf combination} of 
$\Sigma^0_1$-LEM$^-$ and (at least) 
$\Sigma^0_1$-IA$^-$ as provided in our example 
of a sentence $\varphi$ in Proposition 
\ref{p:nonLearnablePhi} that is not effectively learnable.
 

\subsection{Cauchy statements and unrestricted use of $\Sigma^0_1$-LEM}
In the following, we refer to Friedman's so-called $A$-translation from 
\cite{Friedman(78)} (see e.g. \cite{Kohlenbach08}). Since we 
work in the context of weakly extensional systems and the quantifier-free 
rule of extensionality QF-ER is not sound under the $A$-translation we simply 
add for the reminder of this section 
all ${\cal S}^{\omega}$-true (resp. ${\cal S}^{\omega,X}$-true) purely 
universal sentences ${\cal P}$ in the language of the respective system 
as axiom (making the use of QF-ER in proofs superfluous as it only proves 
universal consequences). This, anyhow, is a common device in proof mining 
as universal axioms do not contribute to the computational content of a 
proof (this has been stressed by G. Kreisel since the 50's). We denote 
the extension of the theory ${\cal T}$ by the axioms ${\cal P}$ by 
${\cal T}_*.$ 

\begin{lemma}\label{l:Atrans+}
Friedmann's $A$-translation is sound also for $\ha_* + \LEM$.\\ Similarly for 
{\rm HA$_*^{\omega}[X,\|\cdot\|]$} (and related extensions).
\end{lemma}
\begin{proof}
Consider the following instance of $\LEM$
\[\forall y \phi_0(\tup a,y)  \vee \exists y \neg\phi_0(\tup a,y).\] 
W.l.o.g assume $\phi_0$ is atomic. 
It suffices to show that
\[\ha_* + \LEM\vdash (\LEM)^A.\]
This means we need to prove
\[ \forall y \big( \phi_0(\tup a,y) \vee A\big) \vee \exists y \big(  (\phi_0(\tup a,y) \vee A) \rightarrow A\big),\tag{1}\]
in $\ha_* + \LEM$.

Suppose that
\begin{enumerate}
\item $\forall y \phi_0(\tup a,y)$ holds. Then also $\forall y \big( \phi_0(\tup a,y) \vee A\big)$ holds and therefore also (1).
\item $\exists y \neg\phi_0(\tup a,y)$ holds. Then fix such a $y$. For this $y$ we get 
\[(\phi_0(\tup a,y) \vee A) \rightarrow A\]
and so $ \exists y \big(  (\phi_0(\tup a,y) \vee A) \rightarrow A\big)$ holds and therefore also (1). 
\end{enumerate}
For HA$_*^{\omega}[X,\|\cdot\|]$ one just has to observe that still every 
quantifier-free formula can be written as an atomic formula of the form 
$t\underline{a}=_00$ and that the additional axioms are all purely universal 
and so easily imply their own $A$-translation.  
\end{proof}


\begin{prop}\label{p:Atrans}
The theory $\ha_* + \LEM$ is closed under the $\Sigma^0_2\m\DNE$ rule. \\ 
Similarly for HA$_*^{\omega}[X,\|\cdot\|].$
\end{prop}
\begin{proof} 
Suppose \[\ha_* + \LEM \vdash \neg\neg \exists x\forall y\ \phi_0(\tup a,x,y),\]
where $\phi_0$ is quantifier free and contains only $\tup a$ as free variables. Moreover, w.l.o.g we assume that $\phi_0$ is atomic.  \\
Rewriting the negations in terms of ``$\rightarrow$'' and ``$\perp$'' we obtain that
\[\ha_* + \LEM \vdash \big(\ \exists x\forall y\ \phi_0(\tup a,x,y)\rightarrow\perp\big)\rightarrow \perp,\]
and using Friedman's A-translation (with Lemma~\ref{l:Atrans+}) that
\[\ha_* + \LEM \vdash \Big(\ \exists x\forall y\ \big(\phi_0(\tup a,x,y) \vee A\big)\rightarrow A\Big)\rightarrow A,\]
for any formula $A$ (not containing $x,y$ free). By setting \[A:\equiv\ \exists x'\forall y' \phi_0(\tup a,x',y'),\]
(we consider only this $A$ throughout the remainder of the proof) we obtain that $\ha_* + \LEM$ proves
\[\big(\ \exists x\forall y\ (\phi_0(\tup a,x,y) \vee \exists x'\forall y' \phi_0(\tup a,x',y'))\rightarrow \exists x'\forall y' \phi_0(\tup a,x',y')\big)\rightarrow \exists x'\forall y' \phi_0(\tup a,x',y').\tag{1}\]
Now the claim follows from
\[
\LEM\vdash \forall y\big(\phi_0(\tup a,x,y) \vee \exists x'\forall y' \phi_0(\tup a,x',y') \big) \rightarrow \big(\forall y\ \phi_0(\tup a,x,y) \vee \exists x'\forall y' \phi_0(\tup a,x',y')\big),
\tag{2}
\]
since using (2) the statement (1) is equivalent to
\[  \big((\exists x \forall y \phi_0(\tup a,x,y) \vee \exists x'\forall y' \phi_0(\tup a,x',y'))\rightarrow \exists x'\forall y' \phi_0(\tup a,x',y')\big)\rightarrow \exists x'\forall y' \phi_0(\tup a,x',y'),\]
which is equivalent to $\exists x\forall y \phi_0(\tup a,x,y)$.\\
To show (2) consider the following instance of $\LEM$
\[
\forall y\ \phi_0(\tup a,x,y) \vee \exists y\ \neg\phi_0(\tup a,x,y).
\]
Now suppose that
\begin{enumerate}
\item $\forall y\ \phi_0(\tup a,x,y)$ holds, then (2) is trivially true.
\item $\exists y\ \neg \phi_0(\tup a,x,y)$ holds, then for such a $y$ we have
\[
\big(\phi_0(\tup a,x,y) \vee \exists x'\forall y' \phi_0(\tup a,x',y') \big) \rightarrow \exists x'\forall y' \phi_0(\tup a,x',y')
\]
and so certainly we have also that
\[
\forall y\big(\phi_0(\tup a,x,y) \vee \exists x'\forall y' \phi_0(\tup a,x',y') \big) \rightarrow \exists x'\forall y' \phi_0(\tup a,x',y').
\]
Finally $\big(\forall y\ \phi_0(\tup a,x,y) \vee \exists x'\forall y' \phi_0(\tup a,x',y')\big)$ follows from $\exists x'\forall y' \phi_0(\tup a,x',y')$ so (2) holds as well.
\end{enumerate}
\end{proof}

It is known, that a Cauchy rate is limit computable (which corresponds 
to $\Sigma^0_2$-DNE which -- as mentioned in the introduction -- is 
strictly stronger than $\Sigma^0_1$-LEM). However, for every provable   
Cauchy sequence we have that $\Sigma^0_1$-LEM is sufficient:
\begin{prop}\label{p:limComp}
If a sequence of real numbers $(a_n)$ (or in some $\pa_*$-definable Polish 
space) defined by a closed term of 
$\pa_*$, can be proved to be Cauchy in $\pa$, then the proof can be carried 
out already in $\ha_*+\LEM$. Similarly for $\pa_*[X,\|\cdot\|]$ and 
sequences in 
$X.$
\end{prop}
\begin{proof}
Consider a sequence $x_{(\cdot)}$ and suppose
\[ \pa_*\vdash \forall k\exists n\forall i,j>n\ 
\big(| x_i - x_j| \leq 2^{-k}\big). \]
Then by the Kuroda negative translation (see e.g. \cite{Kohlenbach08}) 
we obtain that
\[ \ha_*\vdash \forall k\neg\neg\exists n\forall i,j>n\ \big(| 
x_i - x_j| \leq 2^{-k}\big). \]
By Proposition~\ref{p:Atrans} this implies that
\[ \ha_*+\LEM\vdash \forall k\exists n\forall i,j>n\ \big(| x_i - x_j| 
\leq 2^{-k}\big) \] (recall that $\le_{\RR}\in\Pi^0_1$).
\end{proof}

\subsection{Which Cauchy statements are effectively learnable and which are not}

\begin{prop}[Implications between different bounding information 
for Cauchy statements] \label{prop.hierarchy}
Let $(x_n)$ be a Cauchy sequence in a metric space $(X,d).$ 
\begin{enumerate}
\item A rate of convergence is a bound on the number of fluctuations.
\item A bound for the number of fluctuations is a bound $B$ 
on the number of mind changes to learn a rate of convergence (with a very 
simple learning function $L$, which is basically the successor function).
\item 
Primitive recursively (in the ordinary sense of Kleene) in majorants 
$B^*,L^*$ of functionals $B,L$ such that the Cauchy rate is $(B,L)$-learnable 
one can obtain a rate of metastability. 
\end{enumerate}
\end{prop}
\begin{proof}
Consider a Cauchy sequence $x_{(\cdot)}$.
\begin{enumerate}
\item Let $b$ be a rate of convergence, i.e.
\[ \forall k \forall n,m\geq b(k)\ \big( d(x_n,x_m)\leq 2^{-k}\big). \]
Then $b(k)$ is also a bound on the number of $2^{-k}$ fluctuations, since any fluctuation has to occur before $b(k)$ (i.e. that one of the indexes of the 
fluctuation has to be smaller than $b(k)$) and there can be at most $b(k)$ many fluctuations indexed within $[0;b(k)]$.
\item Let $b$ be a bound on the number of $2^{-k}$ fluctuations, i.e.
\[ \forall k \forall n>b(k) \forall i,j \neg\Fluc_{2^{-k}}(n,i,j). \]
Then $b(k)$ is also a bound on the number of mind changes to learn a rate of convergence, since for 
$L(n):=n+1$ we have that
\[ \forall k\ \exists l\leq b(k)\ \forall n,m>c_l\ \ \big(  d(x_n,x_m)\leq 2^{-k}\big). \tag{BE}\label{e:be}\]
%

Formally, $L(i,x_{(\cdot)},k):=i$, and (where -- again -- to have 
$\varphi_0$ quantifier-free we officially have to use the $2^{-k-1}$-rational 
approximation $\widehat{d(x_n,x_m)}(k+1)$ of $d(x_n,x_m)$) 
\[
\phi_0(j(n,m),c_i,x_{(\cdot)},k):\equiv\ \big((n>c_i \wedge m>c_i) \rightarrow d(x_n,x_m)\leq 2^{-k}\big),
\]
where $j(n,m)$ is the Cantor pairing function.
%
The statement~\eqref{e:be} can be inferred from the fact that each mind change ($c_i$) corresponds to a (different) fluctuation (as it is based on a counterexample for $d(x_n,x_m)\leq 2^{-k}$, whose both indexes are greater than the last $c_i$).
\item Follows directly from Proposition~\ref{p:bg2meta}.
\end{enumerate}
\end{proof}



In the rest of this section we show that the hierarchy in Proposition \ref{prop.hierarchy} 
between the four different 
quantitative notations for Cauchy sequences discussed in the introduction 
is strict. That an effective bound on the number of fluctuations does not imply an 
effective rate of convergence, follows already from the existence of Specker sequences. We can also use the following
very simple example with a $2^{-k}$-fluctuation bound $k$ and no effective rate of convergence, since such a rate would decide the halting problem:\\ 
\begin{dfn}[$\alpha_{(\cdot)}$] 
We take the Cantor pairing function $j$ and set
\[
\alpha_{j(k,n)}:=
\begin{cases}
2^{-k}, &\Tif T(k,0,n),\\
0, &\Telse,
\end{cases}
\]
where $T$ is the primitive recursive Kleene $T$-predicate.
\end{dfn}
We next construct primitive recursive sequence $\beta_{(\cdot)}$ of rational 
numbers in $[0,1]$ with an effectively (even primitive recursively) learnable Cauchy rate (so in particular with a primitive recursive rate of metastability), which has no computable bound on fluctuations (this example is not captured by the rough sketch of Avigad and Rute as here the number of the oscillations is determined by the length of the computation, not by the index of the machine as suggested in \cite{Avigad/Rute}). 
Furthermore, we also give an example of a primitive recursive (in the ordinary 
sense) sequence $\gamma_{(\cdot)}$ of rational numbers in $[0,1]$ 
which (provably in the fragment 
of PA based on $\Sigma^0_1$-IA only) is Cauchy (and so 
has a primitive recursive in the sense of Kleene rate of metastability) 
which does not have an effectively learnable Cauchy rate. 

\subsubsection*{An effectively learnable sequence with no computable bound on fluctuations}

\begin{dfn}[$\beta_{(\cdot)}$]\label{d:beta}
We fix a primitive recursive 
surjective encoding of triples which is monotone in the third component 
satisfying $\langle k,n,m\rangle\ge k,n,m$ and set
\[
\beta_{\langle k,n,l\rangle}:=
\begin{cases}
2^{-k}, &\Tif T(k,0,n) \wedge l\leq n \wedge l\text{ is even},\\
0, &\Telse.
\end{cases}
\]
\end{dfn}

In the next propositions we will show that the sequence $\beta_{(\cdot)}$
\begin{itemize}
\item is Cauchy (in fact, it converges to zero) -- 
Proposition~\ref{p:alphaIsCauchy},
\item its Cauchy rate is effectively learnable -- Proposition~\ref{p:alphaIsLearnable},
\item there is no computable (in $\epsilon$ and $\beta$) bound on the number of $\epsilon$-fluctuations -- Proposition~\ref{p:alphaHasNoFlucBd}.
\end{itemize}

\begin{prop}\label{p:alphaIsCauchy}
The sequence $\beta_{(\cdot)}$ is Cauchy, provably in 
{\rm $\ha+\Sigma^0_1$-LEM$^-$}. More precisely we show that
\[ 
\ha
\vdash 
\forall k \Big( 
\forall m\leq k \,\big(\exists u\,T(m,0,u) \vee \forall v\,\neg T(m,0,v)\big)\rightarrow \exists n \forall x\geq n\big(\beta_x\leq2^{-k}\big)\Big).
 \]
\end{prop}
\begin{proof}
Consider the terminating computations on input $0$ of the Turing machines encoded by $0,\ldots,k$. Then for every $k$ there is an $n$ corresponding to the
code of the longest such computation. W.l.o.g. we can assume that $n\geq k$ (otherwise set $n:=k$). This means we have that
\begin{align}
n\geq k\ \wedge\ \forall n'\forall k'\leq k\ \big( T(k',0,n')\rightarrow n'\leq n\big).\label{e:ac-n}
\end{align}
Now, set \[ c(k):=\max\{ \langle k',n',l' \rangle\ :\ n'\leq n,\ k'\leq k,\ l'\leq n'\}.\]
Then $c$ is even a rate of convergence, since
\[\ba{l} 
\langle k',n',l'\rangle\ >\ c(k)\ \rightarrow\ k'>k \vee (k'\le k \wedge n'>n)
\vee (k'\le k\wedge n'\le n\wedge l'>n')\\  
\rightarrow\ k'>k \vee \beta_{\langle k',n',l'\rangle}=0 \ \rightarrow\ \beta_{\langle k',n',l'\rangle} < 2^{-k}. \ea 
\]
These arguments are constructive, except for the existence of the longest computation $n$.
This existence is a consequence of $\LEM^-$ and $\Pi^0_1\m\CP^-,$ where 
$\Pi^0_1$-CP is the bounded collection principle for $\Pi^0_1$-formulas 
(also called $B\Sigma^0_2$ in the literature) which is easily provable by 
induction in HA$^{\omega}$.
Consider the following $k+1$ instances of $\LEM^-$:
\[ \forall j\leq k\ \big( \exists n T(j,0,n) \vee \forall m \neg T(j,0,m)\big) \] which over HA$^{\omega}$ implies 
\[ \forall j\leq k \exists n_j \,\forall m\,\big( T(j,0,n_j) \vee \neg T(j,0,m)\big). \] 
By an application of $\Pi^0_1$-CP$^-$, this in turn
implies
\begin{align*}
\exists n\forall j\leq k ( \exists n'\leq n T(j,0,n') \vee \forall m \neg T(j,0,m) ),
\end{align*}
 (consider $n=\max \{n_j : j\leq k\}$).\\
This shows that the convergence is provable in $\ha$+$\LEM^-$ and the convergence up to an error $2^{-k}$ in $\ha$ uses
only $k+1$ instances of $\LEM^-$.
\end{proof}



\begin{lemma}\label{l:G3ACP} 
For a quantifier-free formula $\phi_0$ with parameters only of type $0$, we have that
\[
\mbox{\rm G$_3$A}^\omega + 
\SiLm\IA^- \vdash 
\forall x^0\ \exists u^0 
\forall\tilde x\leq x \big(  
\forall y^0 \phi_0(\tilde x,y)
\vee\exists\tilde u\leq u\neg\phi_0(\tilde x,\tilde u) \big),
\]
where {\rm G$_3$A$^{\omega}$} is the finite type extension of Kalmar-elementary 
arithmetic (based on quantifier-free induction only but with classical logic) 
from 
{\rm \cite{Kohlenbach(lowrate)}} (see also {\rm \cite{Kohlenbach08}}).
\end{lemma}
\begin{proof}
See \cite{Kohlenbach08} Lemma 3.18.
\end{proof}

\begin{rmk}
$\Sigma^0_1$-IA$^-$ is (over G$_3$A$^{\omega}$) strictly weaker than 
$\Pi^0_1$-CP$^-$ but the proof in Lemma \ref{l:G3ACP} needs $\Sigma^0_2$-DNE 
and so more of classical logic than necessary in the proof based on 
$\Pi^0_1$-CP$^-$.
In general, it seems that considering the extraction of computational content 
from proofs, often some amount of classical logic can be reduced on the cost of more recursion.\\
If one is interested (only) in a classical proof, we obtain due to Lemma~\ref{l:G3ACP}
(simply consider $\varphi_0(x,y):\equiv \neg T(x,0,y)$) a proof without the use of $\PiLm\CP$, which can be formalized in G$_3$A$^{\omega}+\SiLm\IA^-$.
\end{rmk}


\begin{prop}\label{p:alphaIsLearnable}
The rate of convergence is effectively learnable in $k$, i.e.
there are total (elementary) 
recursive functions $B$ and $L$, s.t. for any $\beta_{(\cdot)}$, and any $k$ we have that
\[ \forall k\ \exists n\leq B(k)\ \forall m > c_n\quad \big(\beta_m\leq 2^{-k}\big),\]
where $c_{(\cdot)}$ is defined as in Definition~\ref{d:fmcMon} with
\[
\phi_0(x,n,k):\equiv\ x> n\rightarrow \beta_x\le 2^{-k}.
\]
\end{prop}
\begin{proof}
Obviously, this follows already from Proposition~\ref{p:alphaIsCauchy}. Also, it is easy to see that the 
rate is $(B,L)$-learnable with the following $B$ and $L$:
\begin{align*}
B(k)&:=k+1  \\
L(n,k)&:= \langle k,n,n \rangle+1.
\end{align*}
Let $x\ge L(n,k)>\langle k,n,n\rangle$ be a counterexample. Then 
\[ j_1(x)\le k\,\wedge\,(j_2(x)>n\vee j_3(x)>n)\,\wedge \,j_3(x)\le j_2(x) \] 
and so 
\[ j_1(x)\le k\,\wedge\,j_2(x)>n\,\wedge\,j_3(x)\le j_2(x). \]
The 2nd conjunct implies $j_2(x)>j_2(n)$ and hence $j_1(x)\not= j_1(n)$ 
if $n$ is a preceding counterexample. However, for numbers $\le k$ this can 
happen at most $k$-many times. Hence $B(k):=k+1$ and $L$ do the job.
\end{proof}


\begin{cor} \label{metastability-alpha}
$\beta_{(\cdot)}$ has a primitive recursive (in the ordinary sense 
of Kleene) rate of metastability.
\end{cor}
\begin{proof} One can apply Proposition \ref{p:bg2meta} to convert 
the bounds $(B,L)$ from Proposition \ref{p:alphaIsLearnable} (which are 
trivially self-majorizing using standard monotonicity properties of the 
triple coding) into a primitive 
recursive rate of metastability. Alternatively, one can use that by 
Lemma \ref{l:G3ACP} the Cauchy property of $\beta_{(\cdot)}$ is provably in 
G$_3$A$^{\omega}+\Sigma^0_1$-IA$^-$ and so a fortiori  in 
$\widehat{\rm PA}^{\omega}\res+$QF-AC (see \cite{Kohlenbach08}, Prop.3.31).
Then proposition 10.54 in \cite{Kohlenbach08} implies the extractability 
of a primitive recursive rate of metastability (the latter being essentially 
the rate of metastability for the statement in Lemma \ref{l:G3ACP} 
which is computed in \cite{Kohlenbach08}[Prop.3.19]).
\end{proof} 

\begin{rmk}\label{r:metaB}
One can obtain such a rate of metastability for $\beta_{(\cdot)}$ directly, using previous results
of the first author.\\
By~\cite{Kohlenbach08} (Prop.13.19) we have that
\be[e:maU1]
\forall x,f \forall \tilde x\leq x\big(\exists y\leq\Phi xf\ T(\tilde x,0,y)\vee \forall z\leq f(\Phi xf)\neg T(\tilde x,0,z)\big),
\ee
where $\Phi xf\leq \max\{f^{i}(0)\ 
:\ i\leq x+1\}=:\Phi^*xf$. \eqref{e:maU1} (here $f^i(0)$ again denotes the 
$i$-times iteration of $f$) implies
\be[e:maU2]
\forall z \big(\Phi xf< z\leq f(\Phi xf)\rightarrow \forall \tilde x \leq x \neg T(\tilde x,0,z)\big):
\ee
Define $\tilde f(n):=\max\{f(n),n\}$ and $f_k(n):=\tilde f(\langle k,n,n\rangle+1)$ and let
\[
\Psi(k,f):=\langle k,\Phi^*kf_k,\Phi^*kf_k\rangle + 1.
\]
Then $\Psi$ is a rate of metastability for $\beta_{(\cdot)}$, i.e.:
\be[e:rmB]
\forall k,f\ \exists n\leq \Psi(k,f)\ \forall z\in[n,\tilde f(n)]\ \ 
(|\beta_z|< 2^{-k}).
\ee
To prove~\eqref{e:rmB}, define $n:=\langle k, \Phi kf_k, \Phi kf_k\rangle+1\leq\Psi(k,f)$ and let $z\in[n,\tilde f(n)]$. Then 
$z\geq n > \langle k, \Phi kf_k, \Phi kf_k\rangle$ and so one of the following cases holds:
\hspace{35mm}\begin{enumerate}
\item $j_1(z)>k$. Then $|\beta_z|\leq2^{-j_1(z)}<2^{-k}$.
\item $j_2(z)>\Phi kf_k\wedge j_1(z)\leq k$. Then 
$
\Phi kf_k < j_2(z)\leq z \leq \tilde f(n)=f_k(\Phi kf_k).
$
Hence, by~\eqref{e:maU2} (applied to $k,j_1(z),f_k,j_2(z)$ 
for $x, \tilde{x}, f, z$), we get 
$\neg T(j_1(z),0,j_2(z))$ and so $\beta_z=0$.
\item $j_3(z)>\Phi kf_k\wedge j_2(z)\leq\Phi kf_k\wedge j_1(z)\leq k$. Then $j_3(z)>j_2(z)$ and so $\beta_z=0$.
\end{enumerate}
\end{rmk}


\begin{lemma}[Termination causes at least $n$ fluctuations]\label{l:2n}
Suppose the $k^\text{th}$-machine terminates on $0$ with computation encoded by $n$ (i.e. $T(k,0,n)$ holds). Then the sequence $\beta_{(\cdot)}$ contains at 
least $n$ many $2^{-k}$\nbd fluctuations.
\end{lemma}
\begin{proof}
Consider the tuples of indexes $\tup i$, $\tup j$, 
s.t. $i_l:=\langle k,n,l\rangle$, $j_l:=\langle k,n,l+1\rangle$ and $l+1\leq n$. Then we have by definition of $\beta_{(\cdot)}$ (using the monotonicity of the encoding in $l$) that
$
\Fluc_{\beta_{(\cdot)},2^{-k}}(n,\langle \tup i \rangle, \langle \tup j\rangle).
$
\end{proof}

\begin{prop}\label{p:alphaHasNoFlucBd}
There is no computable bound on the fluctuations of $\beta_{(\cdot)}$.
\end{prop}
\begin{proof}
Suppose $b_k$ is a bound on the number of fluctuations by $2^{-k}$, 
then $b_k$ can be used to effectively compute whether the $k^{\text{th}}$ Turing machine terminates on input $0$ as follows.\\
Let the machine run until the code of the computation reaches $b_k$ (or until it stops). If it terminated, we are done.\\
Now suppose it terminates with a computation encoded by some $n>b_k$. Then $\beta_{(\cdot)}$ would have at least $n$ many $2^{-k}$\nbd fluctuations by Lemma~\ref{l:2n}, which is a contradiction.\\
Therefore if the machine does not terminate with a code of computation at most $b_k$ it does not terminate at all.
\end{proof}



\subsubsection*{Metastability does not imply effective learnability}

In the next propositions we define a primitive recursive sequence 
$\gamma_{(\cdot)}$ of rational numbers in $[0,1]$ 
(defined in Corollary~\ref{c:gammaee} using Definition~\ref{d:gammaf}) that
\begin{itemize}
\item is Cauchy (in fact, it converges to zero) -- by 
Proposition~\ref{p:gammaf},
\item has a primitive recursive rate of metastability -- by Proposition~\ref{p:gammaf},
\item has no effectively learnable Cauchy rate -- by Corollary~\ref{c:gammaee}.
\end{itemize}

We use the upper index as a name extension (like $\tup k^K$, meaning a tuple $\tup k$ corresponding to a particular $K$) and as iteration of functions (like $f^n(x)$, meaning we iterate the function $f$ $n$-many times with the starting point $x$). When unclear, we use the notation $(f)^n$ to make explicit, that we mean the iteration.


\begin{dfn}
For any function $f$ we define $\widehat{f}(k,n):=0$, 
if $f(k,n)=0\wedge\forall m<n\ (f(k,m)\neq 0)$ and $\widehat{f}(k,n):=1,$ 
otherwise.
\end{dfn}

\begin{prop}\label{p:nonLearnablePhi} The following statement 
(which is easily provable using {\rm $\Sigma^0_1$-LEM$^-$} combined with 
{\rm $\Pi^0_1$-CP$^-$}) 
\[
\phi:\equiv\quad\forall f^1\leq_1 1 \forall x^0 \exists p  \forall z \phi_0 
(\widehat{f},x,p,z)
\]
where, interpreting $p=j(y,u)$ 
as a surjective code of some pair $y,u$,
\begin{align*}
\phi_0(f,x,p,z):\equiv\quad &\forall \tilde x \leq x \exists \tilde y\leq y \forall \tilde z\leq z 
\big(f(\tilde x,\tilde y)=0\vee f(\tilde x,\tilde z)\neq0\big) 
\wedge \\
&\forall \tilde y \leq y_{f,x} \exists \tilde u\leq u \forall \tilde z\leq z 
\big(f(\tilde y,\tilde u)=0\vee f(\tilde y,\tilde z)\neq0\big),
\end{align*}
with
\[
y_{f,x}:=\begin{cases}
\max\big\{y'\leq y\ :\ y'=\min\{y''\leq y\ :\ \exists x'\le x\,(f(x',y'')=0)
\ \}\big\},&\text{if such $y'$ exists}\\
x,&\Telse.
\end{cases}
\]
is not effectively learnable.\end{prop}
\begin{proof}
We prove the claim in three steps.
\paragraph{Step 1} We will show that (informally speaking, since 
formally we cannot express function iteration and the conclusion would have to use $\psi_0$ and $\psi^y_0$ defined below)
\[
\mbox{G$_3$A}^\omega\vdash \forall g \forall x\ \big(\exists p\forall z\phi_0(f_g,x,p,z)\rightarrow \exists y'(y'=g^{g^{x}(0)}(0))\big),
\tag{GA}\label{e:GA}
\]
for \[
f_g(b,d):=\begin{cases}
0,&\Tif \ \lh(d) = b+1 \wedge d_0=0\wedge \forall i< b\ (d_{i+1}=g(d_i),\\
1,&\Telse.
\end{cases}
\]
Formally, $\exists y' (y'=g^{g^x(0)}(0))$ is to be read as 
\[ \exists y,u \, (f_g(x,y)=0\wedge f_g(y_x,u)=0),
\tag{GA$^*$}\label{e:GA$^*$} \] 
where then $y':=u_{y_x}.$  
Note that we have $f_g=_1\widehat{f_g}$. To show~\eqref{e:GA} fix arbitrary $g^1$ and $x^0$.
Now assume $\exists p\forall z\phi_0(f_g,x,p,z)$ and let us fix such 
a $p=j(y,u)$ to obtain:
\begin{align}
&\forall z\forall\tilde x \leq x \exists \tilde y\leq y \forall \tilde z\leq z 
\big(f_g(\tilde x,\tilde y)=0\vee f_g(\tilde x,\tilde z)\neq0\big) \label{e:A}
\wedge \\
&\forall z\forall\tilde y \leq y_{f_g,x} \exists \tilde u\leq u \forall \tilde z\leq z 
\big(f_g(\tilde y,\tilde u)=0\vee f_g(\tilde y,\tilde z)\neq0\big). \label {e:B}
\end{align}
We now show by quantifier-free induction that 
\be[e:IC1]
\forall x'\leq x \exists y'\leq y\ f_g(x',y')=0.
\ee
Note that $\exists y'\,f_g(x',y')=0$ implies that $y'_{x'}=g^{x'}(0).$ 
The case $x'=0$ is trivially satisfied by $y'=\langle 0 \rangle$. Then 
$y'\le y$ by ~\eqref{e:A}. So suppose for some $x'<x$ we have
$\exists y'\leq y\,f_g(x',y')=0.$ Then we can set 
\[
z:=y'*\langle g(y'_{x'}) \rangle =\langle y'_0,y'_1,\ldots,g(y'_{x'})\rangle
\] to get $f_g(x'+1,z)=0$
which concludes the proof of~\eqref{e:IC1} since -- again by 
~\eqref{e:A} -- $z\le y.$ Note, furthermore,
that for $y'\leq y$ (and $x'\leq x$),
\[ f_g(x',y')=0\ \rightarrow\ y'\leq y_{f,x}.\] 
So we have even that \\
\be[e:IC1+]
\forall x'\leq x \exists y'\leq y_{f_g,x}\, f_g(x',y')=0.
\ee
Now, let $y^*$ denote a $y'$ which satisfies~\eqref{e:IC1+} for $x'=x$ and 
note that $y^*_x\le y^*\le y_{f,g}.$ 
By quantifier-free induction we show that 
\be[e:IC2]\forall x'\leq y_{f_g,x} \exists u'\leq u \, f_g(x',u')=0.\ee
The case $x'=0$ is again trivially satisfied by $u'=\langle 0 \rangle \le u$ 
(using ~\eqref{e:B}) 
So suppose for some $x'< y_{f_g,x}$ that
$\exists u'\leq u \,f_g(x',u')=0$. Then we can set
\[z:=u'*\langle g(u'_{x'})\rangle= \langle u'_0,u'_1,\ldots,g(u'_{x'})\] to get $f_g(x'+1,z)=0$,
which by~\eqref{e:B} implies 
\[\exists \tilde u\leq u \,f_g(x'+1, \tilde u)=0\]
and so concludes the proof of ~\eqref{e:IC2}. Applying ~\eqref{e:IC2} 
we obtain (for $x'=y^*_x$)
\[
\exists u'\leq u\, (f_g(x,y^*)=0\wedge f_g(y^*_x,u')=0),
\]
which concludes the proof of~\eqref{e:GA$^*$} and, therefore, 
also the proof of~\eqref{e:GA}.\\
\paragraph{Step 2} We investigate the terms witnessing the implication~\eqref{e:GA}. By prenexation we obtain
\[
\mbox{G$_3$A}^\omega\vdash \forall g,x,p\exists y',z\ \big(\phi_0(f_g,x,p,z)\rightarrow y'=g^{g^{x}(0)}(0)\big),
\]
and, therefore, by program extraction theorems (see Corollary 3.1.3 in~\cite{Kohlenbach(lowrate)}), we get closed terms $s$ and $t$ in G$_3$A$^\omega$, s.t.
\be[e:st]
\forall g,x,p\ \big(\phi_0(f_g,x,p,sgxp)\rightarrow tgxp=g^{g^{x}(0)}(0)\big).
\ee
\paragraph{Step 3} Finally, we show that with sufficiently large (in the sense of growth) $g$, this contradicts the effective learnability of $\phi$. Suppose namely that $\phi$ were learnable by computable functionals $B(f,x),L(y,f,x).$ 
Then also 
\[ B^*(x):=\sup \{ B(f,\tilde{x}): f\le_1 1, \tilde{x}\le x\} \ 
\mbox{and} \ 
L^*_x(y):=\sup \{ L(\tilde{y},f,\tilde{x}): f\le_1 1,\tilde{y}\le y,
\tilde{x}\le x\} \] 
are computable (in $x$ resp. in $x,y$) and majorize $B,L.$ 
Now by Proposition~\ref{p:bg2meta} and the fact that $f_g$ is trivially 
majorized by $1$ we get, in particular, that 
\[
\exists p\leq \Omega(B^*,L^*_x,h_x,x)\ \phi_0(f_g,x,p, sgxp),
\]
for any $g$ and $x$, by setting \[h^1_x:=\lambda p\ .\ sgxp.\]
So, we obtain together with~\eqref{e:st} that
\[
\forall g,x \exists p_x\leq \Omega(B^*,L^*_x,h_x,x)\ 
\big( tgxp_x =g^{g^{x}(0)}(0)\big).
\]
Since $s$ and $t$ are closed terms of G$_3$A$^\omega$ and the variables 
$g,x,p_x$ have types $\le 1$ by normalization arguments
(see Corollary 2.2.24 and Remark 2.2.25 in~\cite{Kohlenbach(lowrate)}) we know that there is 
a constant $D$, s.t. (for any $g$ that majorizes $\lambda n.2^n$) 
\[ \forall x,v\, \big( \tilde{g}^D(x+v)\ge sgxv,tgxv\big).\] 
Since we may assume that $\tilde{g}(n)>n$ and, therefore, 
$\tilde{g}^x(v)\ge x+v,$ this yields 
\[ \forall x,v\,\big( \tilde{g}^{D+x}(v)\ge sgxv,tgxv\big). \]
As a consequence, we get (using the $\Omega$-definition 
and that that $B^*,L^*_x$ are selfmajorizing and that we may 
assume $L^*_x(y)\ge y$) 
that for all $x$  
\[ \tilde{g}^{D+x}\left(\Omega(B^*,L^*_x,\tilde{g}^{D+x},x)\right) \ge 
\tilde{g}^{D+x}\left(\Omega(B^*,L^*_x,h_x,x)\right) \ge 
tgxp_x=g^{g^x(0)}(0). \]
By the definition of $\Omega$ (see also Remark~\ref{r:metastr}) 
\[   \ba{l} \tilde{g}^{D+x}\big(\Omega(B^*,L^*_x,\tilde{g}^{D+x},x\big)\le 
\tilde{g}^{D+x}
\left((L^*_x\circ\tilde{g}^{D+x})^{B^*(x)}(0)\right) \le \\ 
\tilde{g}^{D+x}\left((L^*_x\circ\tilde{g})^{(D+x)B^*(x)}(0)\right)\le 
(L^*_x\circ \tilde{g})^{\widehat{B}^*(x)}(0)\ea \] 
and so (for all $x$)
\[  g^{g^x(0)}(0)\le (L^*_x\circ \tilde{g})^{\widehat{B^*}(x)}(0) \]
where $\lambda x,y.L^*_x(y)$ and $\widehat{B^*}(x):=(D+x)(B^*(x)+1)$ 
are fixed total recursive functions that do not depend 
on $g$ which is not possible for sufficiently fast growing $g$.
\end{proof}

\begin{cor}\label{c:anyStrongerPhi}
Let $\phi_0$ be as in the previous Proposition. If for a quantifier free formula $\psi_0$ (with no hidden parameters)
\[
\mbox{\rm G$_3$A}^\omega+\QF\m\AC 
\vdash \forall f\leq 1 \forall x^0\big( \exists y^0\forall z^0 \psi_0(\xi(f),\chi(x),y,z)
 \rightarrow \exists p^0\forall z^0 \phi_0(\widehat{f},x,p,z) \big),
\]
where $\xi$ and $\chi$ are closed terms of {\rm G$_3$A$^\omega$,}
then $\forall f\leq 1\forall x^0\exists y^0\forall z^0 \psi_0(f,x,y,z)$ is also not effectively learnable. Here 
\[ \mbox{\rm QF-AC}: \ \forall x\,\exists y\,F_0(x,y)\rightarrow 
\exists f\,\forall x\,F_0(x,f(x)), \] 
with quantifier-free $F_0$ and $x,y$ of arbitrary types.
\end{cor}
\begin{proof}
This follows analogously from~\cite{Kohlenbach(lowrate)} (Corollary 3.1.3.) 
and our Proposition~\ref{p:bg2meta} as in the proof
of Proposition~\ref{p:nonLearnablePhi}.
\end{proof}

\begin{rmk}
We can prove in {\rm G$_3$A$^\omega+\SiLm\CP$} (which is included in 
{\rm G$_3$A$^\omega+\QF\m\AC^{0,0}$}) that $\phi$ in Proposition
~\ref{p:nonLearnablePhi} is actually equivalent to its monotone version, 
\[
\tilde\phi\equiv \forall f^1\leq 1\forall x^0\exists q\forall z \exists 
p\leq q\forall \tilde z\leq z \phi_0(\widehat{f},x,p,z).
\]
Since this equivalence holds also pointwise (in $f,x$), 
we can use Corollary \ref{c:anyStrongerPhi} to 
infer that there is actually a monotone formula, which is not effectively 
learnable.
\end{rmk}

\begin{dfn}\label{d:gammaf} Define (using a surjective quadruple coding) a 
primitive recursive sequence of rational numbers in $[0,1]$ by
\[
\gamma(f)_{\langle k,n,i,m\rangle}:=\begin{cases}
2^{-k},&\Tif\ \widehat{f}(k,n)=0\wedge i\leq n\wedge \widehat{f}(i,m)=0,\\
0,&\Telse.\end{cases}
\]
\end{dfn}

\begin{prop}\label{p:gammaf}
The formula stating the existence of a Cauchy point for any $f$ 
\[
\psi:=\forall f^1\leq_1 1,x^0\exists z \forall k\geq z\ \big( \gamma(f)_z<2^{-x} \big).
\]
though being true, is not effectively learnable. However, there is
a primitive recursive (in the ordinary sense of Kleene) rate 
of metastability for $\gamma(f)\seq$, which does not 
depend on $f$.
\end{prop}
\begin{proof}
The existence of a metastability rate follows from the fact
that $\gamma(f)\seq$ is Cauchy for any $f\leq_1 1$. Moreover, since the
proof can be formalized in G$_3$A$^\omega+\SiLm\IA$ there is a primitive recursive rate and since $f$ is trivially majorizable it is also
clear that there is even a primitive recursive rate which does not 
depend on $f$. (In Remark \ref{metastable} below, we actually give such a 
rate explicitly.)\\
To show the unlearnability, due to Corollary~\ref{c:anyStrongerPhi} it 
suffices to show that
\[ \mbox{G$_3$A}^\omega+\QF\m\AC^{0,0}\vdash 
\forall f\leq 1 \forall x^0\big( \exists z \forall k\geq z\ \big( \gamma(f)_z<2^{-x} \big)
 \rightarrow \exists p\forall z' \phi_0(\widehat{f},x,p,z') \big).
 \]
To prove $\phi,$ fix arbitrary $f^1$ and $x^0$ and suppose that
\[\exists z\forall k\geq z \big( \gamma(f)_z<2^{-x} \big).\]
Moreover, assume towards contradiction
\be[e:CD]
\exists \tilde x\leq x \exists a\geq\max(z,x) \widehat{f}(\tilde x,a)=0. 
\ee
Since $a\ge \tilde{x},z,$  
this implies that $k:=\langle \tilde x, a, \tilde x, a\rangle \ge z$ 
and $\gamma(f)_k=2^{-\tilde x}$, which is a contradiction.\\
Hence we can conclude that
\[
\forall \tilde x\leq x\big( \exists\tilde y<\max(x,z) \widehat{f}(\tilde x,\tilde y)=0\ \vee\ \forall a \widehat{f}(\tilde x,a)\neq 0\big),
\] 
which is equivalent to
\be[e:A1]
\forall \tilde x\leq x \forall a\exists\tilde y<\max(x,z) \ \big(  
\widehat{f}(\tilde x,\tilde y)=0\ \vee\   \forall \tilde{a}\le a 
\,\widehat{f}(\tilde x,\tilde{a})\neq 0\big).
\ee 
Next, set $y:=\max(x,z)$ and assume towards contradiction that
\be[e:CD2]
 \exists \tilde y\leq y_{\widehat{f}, x} \exists a\geq z \widehat{f}(\tilde y,a)=0. 
\ee
Recall that
\be
y_{\widehat{f}, x}:=\begin{cases}
\max\big\{y'\leq y\ :\ y'=\min\{y''\leq y\ :\ \exists x'\le x\,(
\widehat{f}(x',y'')=0\, \}\big\},&\text{if such $y'$ exists}\\
x,&\Telse.
\end{cases}
\ee
Note that if $y_{\widehat{f},x}=x$, then $\phi$ follows already from \eqref{e:A1}. Otherwise, denote the smallest $x'\leq x$ for which
$\widehat{f} (x',y_{\widehat{f},x})=0$ by $\tilde x$. Then $k:=\langle 
\tilde x, y_{\widehat{f}, x}, \tilde y, a\rangle \ge z$ and $\gamma(f)_k=2^{-\tilde x}$, which is a 
contradiction.\\
Finally, $\exists p\forall z' \phi_0(\widehat{f},x,p,z')$ follows from not-\eqref{e:CD2} and~\eqref{e:A1} (with $y:=\max(x,z)$, $p:=\langle y, z\rangle$).
\\
\end{proof}

\begin{rmk}\label{metastable} 
Similarly as before, we can give an explicit such rate of metastability. As in Remark~\ref{r:metaB}, there is a $\Phi_fxg\le\Phi^*xg:=
\max\{ g^i(0):i\le x+1\}$ such that 
\[ \forall x,f,g\forall\tilde{x}\le x\,\big(\exists y\le\Phi_fxg \,(
\widehat{f}(\tilde{x},y)=0)\vee \forall z\le g(\Phi_fxg)\,
(\widehat{f}(\tilde{x},z)\not= 0)\big). \]
Define 
\begin{align*}
\Phi_1 xg&:= \Phi_f\Big(x,\lambda y.g_y\big(\Phi_f(y,g_x)\big)\Big),\\
\Phi_2 xg&:= \Phi_f\big(\Phi_1 xg,g_{\Phi_1 xg}\big),
\end{align*}
where $g_y(n):=g(y,n)$. Then
\[ \ba{l}
\forall x,f,g\ \forall\tilde x\leq x\,\forall \tilde{y}\le \Phi_1xg \\[1mm] 
\hspace*{1cm} \left\{ \ba{l} \Big( 
(\exists y\leq \Phi_1 xg \big(\widehat{f}(\tilde x, y)=0\big) \vee 
\forall z\leq g(\Phi_1 xg,\Phi_2xg)\ \big(\widehat{f}(\tilde x,z)\neq0)) \ 
\wedge \\ \hspace*{2mm}
 (\exists u\le\Phi_2xg\,(\widehat{f}(\tilde{y},u)=0)\vee \forall z\le 
g(\Phi_1xg,\Phi_2xg)\,(\widehat{f}(\tilde{y},z)\not= 0)\big) \Big).
\ea \right. \ea \]
This implies
\be[e:rmGU2]
\forall z \big(\Phi_1 xg< z\leq g(\Phi_1 xg,\Phi_2 xg)\rightarrow \forall \tilde x \leq x\ (\widehat{f}(\tilde x,z)\neq0)\big)
\ee
and
\be[e:rmGU3]
\forall z \big(\Phi_2 xg< z\leq g(\Phi_1 xg,\Phi_2 xg)\rightarrow \forall \tilde x \leq \Phi_1 xg\ (\widehat{f}(\tilde x,z)\neq0)\big).
\ee
Define $\tilde g(n):=\max\{g(n),n\}$ and $g_k(n,m):=\tilde g(\langle k,n,n,m\rangle+1)$ and 
\[
\Psi(k,g):=\langle k,\Phi^*_1 kg_k,\Phi^*_1 kg_k,\Phi^*_2 kg_k\rangle+1,
\]
where $\Phi^*_i$ is defined as $\Phi_i$ but with $\Phi^*$ and $(g_k)^{M}$ instead of $\Phi_f$ and $g_k$.
To show
\[
\forall k,g,f\ \exists n\leq \Psi(k,g)\ \forall z\in[n,\tilde g(n)]\ \ (|\gamma(f)_z| < 2^{-k}), \] 
define $n:=\langle k,\Phi_1 kg_k,\Phi_1 kg_k,\Phi_2 kg_k\rangle+1\leq \Psi(k,g)$ and let $z\in[n,\tilde g(n)]$. Then 
$z\geq n>\langle k,\Phi_1 kg_k,\Phi_1 kg_k,\Phi_2 kg_k\rangle$. Hence one of the following holds:
\begin{enumerate}
\item $j_1(z)>k$. Then $|\gamma(f)_z|\leq2^{-j_1(z)}<2^{-k}$.
\item $j_2(z)>\Phi_1 kg_k\wedge j_1(z)\leq k$. Then 
$
\Phi_1 kg_k < j_2(z)\leq z \leq \tilde g(n)=g_k(\Phi_1 kg_k,\Phi_2 kg_k).
$
Hence, by~\eqref{e:rmGU2} (applied to $j_1(z),k,g_k,j_2(z)$ for 
$\tilde{x},x,g,z$), $\gamma(f)_z=0$.
\item $j_3(z)>\Phi_1 kg_k\wedge j_2(z)\leq\Phi_1 kg_k\wedge j_1(z)\leq k$. Then $j_3(z)>j_2(z)$ and so $\gamma(f)_z=0$.
\item $j_4(z)>\Phi_2 kg_k\wedge j_3(z)\leq\Phi_1 kg_k\wedge j_2(z)\leq\Phi_1 kg_k\wedge j_1(z)\leq k$. 
Then 
$
\Phi_2 kg_k < j_4(z)\leq z \leq \tilde g(n)=g_k(\Phi_1 kg_k,\Phi_2 kg_k).
$
Hence, by~\eqref{e:rmGU3} (applied to $j_3(z),k,g_k,j_4(z)$ for 
$\tilde{y},x,g,z$), $\gamma(f)_z=0$.
\end{enumerate}
Note that the 2-nested use of primitive recursive iteration hidden in the 
2-nested application of 
$\Phi_f$ in the definition of $\Phi_i$ very much resembles the basic 
structure of the rate of metastability extracted from a concrete proof 
in ergodic theory in \cite{Safarik(11)} (see the discussion in the 
introduction).
\end{rmk}

\begin{cor}\label{c:gammae}
Let $\gamma(e)_{(\cdot)}$ be defined as $\gamma(f)_{(\cdot)}$ but 
with $\widehat{f}(x,y)=0$ being replaced by 
\[ P(e,x,y):\equiv \mbox{\rm lh} (y)=x+1\wedge y_0=0\wedge\forall i<x(T(e,y_i,y_{i+1})).\]  
Then
the formula stating the existence of a Cauchy point of 
$\gamma(e)\seq$ for any number $e$  
\[
\psi:=\forall e^0,x^0\exists z \forall k\geq z\ \big( \gamma(e)_k<2^{-x} \big).
\]
is not effectively learnable.
\end{cor}

\begin{proof}
First note that the Cauchy property of $\gamma(e)_n$ follows from the 
Cauchy property of $\gamma(f)_n$ since taking $f(x,y):=0,$ if $P(e,x,y)$, 
and $f(x,y):=1,$ otherwise, both sequences coincide (note that 
$f=\widehat{f}$).\\ 
Let $e$ by a code of a total recursive function and define $g(x):=\mu 
y\,.\,T(e,x,y).$
The arguments of both Proposition~\ref{p:nonLearnablePhi} and Proposition~\ref{p:gammaf} then remain valid,
except the fact that the set of 
G\"odel numbers $e$ -- in contrast to $f_g$  -- is not majorizable. 
We also need the additional assumption $\forall x\,(T(e,x,g(x)))$ in 
(\ref{e:GA}), which 
expresses that $g(x):=\mu y\,.\,T(e,x,y)]$ defines a total function. This 
assumption, however, does not contribute in the course of the functional 
interpretation argument applied to (\ref{e:GA}) (`Step 2' in the proof of 
Proposition~\ref{p:nonLearnablePhi}) as it is purely universal. 
So as before, the learnability would lead to a constant $D$ and 
total recursive functions $\lambda e,x,y.
L^*_{e,x}(y)$, $B^*$ such that
\[ \forall e\,\forall g\,\left(\forall x^0 \,T(e,x,g(x))\rightarrow 
\forall x^0\ (L^*_{e,x}\circ \tilde{g})^{\widehat{B^*}(e,x)}(0)
\geq g^{g^{x}(0)}(0))\right)\] where 
$\widehat{B^*}(e,x):=(D+x+e)(B^*(x)+1).$  \\ 
We can argue similarly as in the proof of Proposition~\ref{p:nonLearnablePhi}, 
since for any fixed $g$ given by some $e$ as above, eventually we have $x>e$, 
so we can simply choose
a total recursive $g$ which grows much faster than $L^*_{x,x}$ and $B^*(x,x)$.
\end{proof}

\begin{cor}\label{c:gammaee}
Define the primitive recursive sequence of rational numbers in $[0,1]$  
(using the Cantor pairing function)
\[
\gamma_{j(e,n)}:= \gamma(e)_n\cdot 2^{-e}.
\]
This sequence converges to $0$ (provably in {\rm 
G$_3$A$^{\omega}+\Sigma^0_1$-IA$^-$} and hence with a primitive recursive -- 
in the sense of Kleene -- rate of metastability) but 
the formula stating the existence of a Cauchy point of $\gamma\seq$
\[
\psi:=\forall x^0\exists z \forall k\geq z\ \big( \gamma_k<2^{-x} \big).
\]
is not effectively learnable.
\end{cor}
\begin{proof} 
Let $\rho_e(x)$ be a rate of convergence for $\gamma(e)_n$ and define 
$\rho(x):=\max\{ \rho_{\tilde{x}}(x):\tilde{x}\le x\}.$ Then 
\[ \widehat{\rho}(x):=j(x,\rho(x)) \] is a rate of convergence for 
$\gamma_n$ towards $0.$
\\
Conversely, for any rate $\rho$ of convergence for $\gamma_n$ we have that 
$\rho_e(x):=\rho(e+x)$ is a rate of convergence of $\gamma(e)_n$. In 
particular, the $2^{-e-x}$-Cauchy property for $\gamma_n$ implies the 
$2^{-x}$-Cauchy property of $\gamma(e)_n.$
\end{proof}


\subsection{When learnability implies fluctuation bounds}


In some cases, the effective $(B,L$)-learnability of a convergence rate 
(meaning that the convergence rate can be learned by $L$ with 
$B(\underline{a})$-many mind 
changes) gives a bound on the number of fluctuations. This, for instance, 
is the case for bounded monotone sequences. In general, we can say that
effective learnability implies the existence of an
effective bound of fluctuations, if the learner and the sequence satisfy
certain gap conditions.

\begin{prop}[{Gap conditions on the learner}]\label{p:gap}
Let $a\seq$ be some sequence in a metric space $(X,d)$ and let 
\[
\phi:\equiv \forall k\exists n\forall i,j\ \underbrace{\big(n\leq i<j\rightarrow d(a_{i},a_{j})\leq 2^{-k}\big)}_{\phi_0(j(i,j),n,k):\equiv}.
\] 
be a $(B,L)$-learnable formula (which states simply the Cauchy property 
of the sequence) and let $B^*,L^*$ be
majorants of $B,L.$ \\
Moreover, suppose that there are functions $\Delta_*$, $J^*$, s.t. (for $i=j_1(x),j=j_2(x)$ and analogously for $x'$ using the Cantor pairing function)
\[\forall n,x,x'\ \Big( \big( \neg\phi_0(x,n,k)\wedge\neg\phi_0(x',n,k)\wedge j\leq i'\big)\rightarrow
|j'-j|\geq\Delta_*(x,k)\Big),\]
\[\forall n,x \big( \neg\phi_0(x,n,k) \rightarrow L^*(x,k)-i\leq J^*(x,k)\big),\]
and
\be[e:O]
J^*\in\mathcal{O}\big(\Delta_*\big)\equiv \forall k\exists N_k,K_k\ 
\forall x\geq N_k\ \big( 
K_k\Delta_*(x)\geq J^*(x,k)\big).
\ee
Then there is a bound on the number of $2^{-k}$-fluctuations,
which is primitive recursive 
in $B^*,\Delta_*,J^*$ and $N_k,K_k$ (which witness~\eqref{e:O}) given by
\[b(k):=B^*(k)\bigg(1+K_k+\max_{n<N_k}\Big(\frac{J^*(n,k)}{\Delta_*(n,k)}\Big)\bigg).\]
Note that in the case where $N_k=0,$ we get
\[b(k)=(1+K_k)B^*(k).\]
\end{prop}
\begin{proof}
For simplicity, assume $N_k=0$ (otherwise we could just replace every occurrence 
of $K_k$ by $(K_k+\max_{n<N_k}(\frac{J^*(n,k)}{\Delta_*(n,k)}))$).\\
Firstly, note that any $2^{-k}$-fluctuation between two indexes $i$ and $j$ corresponds to the counterexample $x=j(i,j).$ So, by definition there 
are never any fluctuations in the intervals $[c_l,i_l]$, where 
$x_l=j(i_l,j_l)$ is the smallest counterexample to the solution 
candidate $c_l$.\\
Secondly, from our assumption on $\Delta_*$ we get that there are at most
\[
\left\lceil \frac{c_{l+1}-i_l}{\Delta_*(x_l,k)} \right\rceil \leq \left\lceil \frac{J^*(x_l,k)}{\Delta_*(x_l,k)} \right\rceil \leq K_k
\]
many fluctuations within an interval $[i_l,c_{l+1}]$.\\
Finally, there are at most $B^*(k)$ such intervals, before a $2^{-k}$-Cauchy point is reached, and there can be one additional fluctuation for each 
pair of such intervals.
\end{proof}

We now consider the general form of the structure of Birkhoff's proof 
of the mean ergodic theorem as analyzed in \cite{kohlenbachleustean09} and 
the argument used in \cite{Avigad/Rute} to convert the rate of metastability 
obtained in \cite{kohlenbachleustean09} into a bound on the number of 
fluctuations: \\[1mm]
Let $x_{(\cdot)}$ be a sequence in some normed space $X$ (in the case at hand 
$X$ is a uniformly convex Banach space) and $y_{(\cdot)}$ be a sequence in $\RR_+$ definable by terms in $\ha[X,\|\cdot\|,\ldots].$ 
Suppose the Cauchyness of $x_{(\cdot)}$ 
is proved using that $y_{(\cdot)}$ has arbitrarily good approximate 
infima, i.e.
\begin{align}
\forall \delta>0\exists n &\forall k \forall \tilde k\le k\ (y_{\tilde k}\geq y_n-\delta)\\
&\rightarrow \forall \epsilon>0\exists m \forall u \forall i,j\in[m,u]\ (\|x_i-x_j\|\leq \epsilon).
\end{align} 
This implication is classically equivalent to 
\begin{align*}
\forall \epsilon>0 \exists \delta>0\ \Big( \exists n &\forall k \forall \tilde 
k\le k\ (y_{\tilde k}\geq y_n-\delta)\rightarrow \\ \tag{+}\label{e:U2}
&\exists m \forall u \forall i,j\in[m,u]\ (\|x_i-x_j\|\leq \epsilon) \Big).
\end{align*}
Suppose now that we are in the situation of Corollary \ref{cor.2.11}, i.e.  
\[
\ha[X,\|\cdot\|,\ldots]+\mbox{AC$+$M$^{\omega}+$IP}^{\omega}_{\forall} 
\vdash\text{\eqref{e:U2}}
\]
then
\begin{align*}
\ha&[X,\|\cdot\|,\ldots]+{\rm AC}+{\rm M}^{\omega} +{\rm IP}^\omega_\forall \vdash\\
&\forall\epsilon>0\exists\delta>0\forall n\exists m\geq n\forall u\exists k(\forall \tilde k\le k(y_{\tilde k}\geq_\RR y_n-\delta)
\rightarrow \forall i,j\in[m,u](\|x_i-x_j\|<_\RR\epsilon).
\end{align*}
Hence by monotone functional interpretation one extracts terms 
 $\delta_\epsilon>0$, 
$m_\epsilon$ and $k_\epsilon$ (depending additionally only 
on majorants of the parameters $\underline{a}$ used in the definition of 
our sequences) s.t. (valid in 
${\cal S}^{\omega,X}$) for all majorants $\underline{a}^*$ of $\underline{a}$ 
\[ \hspace*{-5mm} \ba{l} 
\forall \epsilon>0\ \forall n,u\\[1mm] 
\bigg( m_\epsilon(n)\ge n \wedge \Big(\forall \tilde k\le k_\epsilon(n,u)\ (y_{\tilde k}\geq y_n-\delta_\epsilon)\rightarrow 
 \forall i,j\in[m_\epsilon(n),u]\ (\|x_i-x_j\|\leq \epsilon) \Big)
\hspace*{-1mm} \bigg).
\ea  \hspace*{-1mm}\tag{$*$}\label{e:U4-me}\]
Now define $k^*_\epsilon(u):=\max\{k_\epsilon(i,u)\ :\ i\leq u\}$ and consider
\[
\forall \epsilon>0\ \forall n,u\ \Big( \forall \tilde k\le k^*_\epsilon(u)\ (y_{\tilde k}\geq y_n-\delta_\epsilon)\rightarrow 
 \forall i,j\in[m_\epsilon(n),u]\ (\|x_i-x_j\|\leq \epsilon) \Big).
\tag{$**$}\label{e:U4}\]
We can infer \eqref{e:U4} from \eqref{e:U4-me} by the following case 
distinction:\footnote{We are grateful to P. Oliva for pointing this out to us.}
Now fix $\varepsilon >0$ and $n.$ 
\begin{itemize}
\item[Case]  1: $u<m_\epsilon(n)$. Then the conclusion and hence the whole implication is trivially true.
\item[Case]  2: $u\geq m_\epsilon(n) \geq n$.Then $k^*_\epsilon(u)\geq 
k_\epsilon(n,u)$ and so $\forall \tilde k\le k^*_\epsilon(u)\ 
(y_{\tilde k}\geq y_n-\delta_\epsilon)$ implies $\forall \tilde k\le 
k_\epsilon(n,u)\ (y_{\tilde k}\geq y_n-\delta_\epsilon)$ and so the claim 
follows as well.
\end{itemize}

Now suppose w.l.o.g. that $k^*_\epsilon:\NN\to\NN$ is injective and for any given $u$ define
\[
l_u\ :=\ (k^*_\epsilon)^{-1}(u).
\]
Then \eqref{e:U4} applied to $u:=l_u$ yields
\[
\forall \epsilon>0\ \forall n,u\ \Big( \forall \tilde k\le u\ 
(y_{\tilde k}\geq y_n-\delta_\epsilon)\rightarrow 
 \forall i,j\in[m_\epsilon(n),(k^*_\epsilon)^{-1}(u)]\ 
(\|x_i-x_j\|\leq \epsilon) \Big).
\tag{-}\label{e:U5}\]
Now let $N_0$, $N_1$, ..., $N_{S_\epsilon}$ be integers s.t. $N_0=0$ and $N_{i+1}$ is the least $m>N_i$ s.t. $y_m<y_{N_i}-\delta_\epsilon$ as long as such an $m$ exists.
Assume that $b\geq y_1$ (for some $b$) and so $S_\epsilon\leq\frac{b}{\delta_\epsilon}$.\\
By \eqref{e:U5} there are no $\epsilon$-fluctuations of $x_{(\cdot)}$ on the 
$S_\epsilon$ many intervals $[m_\epsilon(N_i),(k^*_\epsilon)^{-1}
(N_{i+1})]$ for $i=0,\ldots,S_{\varepsilon}-1$.\\
In the intervals $[(k^*_\epsilon)^{-1}(N_{i}),m_\epsilon(N_i)]$ 
for $i=1,\ldots,
S_\epsilon$ and $[0,m_\epsilon(N_0)]$ we have to show that if we have for 
any $N\in\NN$ $s$ many
fluctuations indexed within $[(k^*_\epsilon)^{-1}(N),m_\epsilon(N)]$ 
(or in $[0,m_\epsilon(N_0)]$) each indexed by a pair of indexes $(i,j)$ 
then the highest index of such fluctuation ($j_s$) has to be greater than 
(or equal to) some $\phi_\epsilon(s,N)$, where $\phi_\epsilon$ is such that
\[
\exists \tilde s\forall n\ \big(\phi_\epsilon(\tilde s,n)>m_\epsilon(n)\big).
\]
Then, given such an $\tilde s$, we have at most
\[
\frac{b}{\delta_\epsilon}+\tilde s\left(\frac{b}{\delta_\epsilon}+1\right)
\]
many fluctuations.
\\[2mm]
In the case of Birkhoff's proof, the analysis in \cite{kohlenbachleustean09} 
and the discussion in \cite{Avigad/Rute} 
gives the following data used in \cite{Avigad/Rute}: 
\begin{align*}
\delta_\epsilon &:= \frac{\epsilon^2\mu}{16b},& m_\epsilon(n)&:=\big\lceil \frac{16b}{\epsilon}\big\rceil n,\\
(k^*_\epsilon)^{-1}(n)&:=\frac{n}{2},& 
\phi_\epsilon(s,n)&:=\big(1+ \frac{\epsilon}{2b}\big)^s n,
\end{align*}
and so \[ \tilde s \leq \frac{ 2 b\log \big\lceil 
\frac{16b}{\epsilon}\big\rceil }{\epsilon}.\]
The function $\varphi_{\varepsilon}$ results (see \cite{Avigad/Rute} for 
the calculation) from the fact that 
\[ \| x_ {n+k}-x_k\|\le 2n\| x\|/(n+k) \]
which is established already in Birkhoff's proof and which -- for $n=1$ -- 
shows that $(x_k)$ has a linear rate of asymptotic regularity.
\begin{rmk}
Naturally, we could use the data, which led to the bound 
of $\tilde s$ above, also simply with Proposition~\ref{p:gap} to
obtain a similar fluctuation bound (which has the same structure in $\epsilon$).
\end{rmk}

For the case of Halpern iterations (with scalar $1/(n+1)$) mentioned in 
the introduction, the analysis given in \cite{Kohlenbach/Leustean6}   
yields (roughly) the following data for Hilbert spaces $X$ 
(see \cite{Kohlenbach/Leustean6} for the 
detailed definition of $\Theta_n$):
\begin{align*}
\delta_\epsilon &:= \frac{\epsilon^2}{14(b+1)^2},& m_\epsilon(n)&\approx \Theta_n(\frac{\epsilon^2}{4})n^6\approx n^6,\\
(k^*_\epsilon)^{-1}(n)&:=\frac{3nb^2}{\epsilon}.& 
\end{align*}
Most recently, the function $m_{\varepsilon}$ could be improved in 
\cite{Koernlein} to 
$m_{\varepsilon}(n) \approx n^2$ based on the analysis of a different 
proof for the strong convergece of the Halpern iteration from 
\cite{Xu}.\\
However, now the rate of asymptotic regularity roughly is of order (see 
corollary 6.3 in \cite{Kohlenbach/Leustean6}) 
\[ \| x_{k+1}-x_k\| \le \frac{b}{\sqrt{k}}, \] 
which does not lead to a linear (in $n$) $\varphi_{\varepsilon}(s,n)$ and 
even if it would, this would not suffice to dominate $m_{\varepsilon}(n).$ 
So as it stands, the analysis does not seem to yield any fluctuation 
bound for the Halpern iteration $(x_k).$

\begin{prop}
Given a bound $B_{\varepsilon}$ on the number of fluctuations, there is an in $B_{\varepsilon}$ (and the given data 
$(k^*_{\varepsilon})^{-1}$ and $m_{\varepsilon}$ and the majorants $\underline{a}^*$ of their parameters including $\varepsilon)$) 
primitive recursive $\phi_\epsilon$
satisfying the conditions in the proof:
\begin{align}
\forall \varepsilon >0 \forall s,N,i,j \big( i,j\in[(k^*_\epsilon)^{-1}(N),
m_\epsilon(N)]\ \wedge\ \Fluc_{\varepsilon}(s,i,j)\ 
\rightarrow j_s\geq \phi_\epsilon(s,N) \big),\label{e:FU1}\\
\exists \tilde s\forall n\ \big(\phi_\epsilon(\tilde s,n)>m_\epsilon(n)\big).\label{e:FU2}
\end{align}
\end{prop}
\begin{proof}
Set
\[
\phi_\epsilon(s,n):=\begin{cases}
(k^*_\epsilon)^{-1}(n)&\Tif\ s\leq B_{\varepsilon},\\
m_\epsilon(n)+1&\Telse.
\end{cases}
\]
\end{proof}
\vspace*{-3mm}
\begin{rmk}
In particular, this means that if we know there is for computable 
(in the majorants $\underline{a}^*$ of the parameters including $\varepsilon$) 
$(k^*_{\varepsilon})^{-1}$ and $m_{\varepsilon}$ 
no computable $\phi_{\varepsilon}$ (in $\tup a^*$) satisfying these conditions,
then there cannot be a bound on the number of fluctuations, which is computable (in $\tup a^*$).
\end{rmk}


