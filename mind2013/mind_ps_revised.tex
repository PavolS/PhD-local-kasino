\documentclass[1p]{elsarticle}

\usepackage{amssymb,latexsym}

\usepackage{enumerate}



%%%% Put my macros here:

%%%%%%%%%%%%%%%%%%%%%%%%%%%   usepackage   %%%%%%%%%%%%%%%%%%%%%%%%%%%%%%%%%%%%%%%

%%%%%%%%%%%%%%%%%%%%%%%%%%%   usepackage   %%%%%%%%%%%%%%%%%%%%%%%%%%%%%%%%%%%%%%%

%%%%%%%%%%%%%%%%%%%%%%%%%%%   usepackage   %%%%%%%%%%%%%%%%%%%%%%%%%%%%%%%%%%%%%%%

%\usepackage{graphicx, color}

\usepackage{amsmath,amsthm,amssymb,amscd}

\usepackage[all]{xy}



\newcommand{\usftext}[1]{\textsf{\upshape #1}}



%%%%%%%%%%%%%%%%%%%%%%%%%%%   Kohlenbach   %%%%%%%%%%%%%%%%%%%%%%%%%%%%%%%%%%%%%%%

%%%%%%%%%%%%%%%%%%%%%%%%%%%   Kohlenbach   %%%%%%%%%%%%%%%%%%%%%%%%%%%%%%%%%%%%%%%

%%%%%%%%%%%%%%%%%%%%%%%%%%%   Kohlenbach   %%%%%%%%%%%%%%%%%%%%%%%%%%%%%%%%%%%%%%%



\parindent0pt





\newcommand{\ba}{\begin{array}} \newcommand{\ea}{\end{array}}

\newfont{\bsl}{cmbxsl10 scaled 1095}









\newcommand{\ra}{\rightarrow}

\newcommand{\Ra}{\Rightarrow}

\newcommand{\Lra}{\Leftrightarrow}

\newcommand{\se}{\subseteq}

\newcommand{\si}{\wedge}

\newcommand{\sau}{\vee}

\newcommand{\ol}{\overline}

\newcommand{\nin}{\in\!\!\!\!\!/}

%\renewcommand{\topmargin}{-1cm}

\newfont{\deu}{eufm10 scaled 1000}

\newcommand{\scripta}{\mbox{{\deu A}}}

%\newcommand{\nvdash}{\mathop{\vdash\!\!\!\!\!/}\nolimits}

\newcommand{\nmodels}{\mathop{\models\!\!\!\!\!\!/}\nolimits}

\newcommand{\res}{|\!\raisebox{1mm}{$\scriptscriptstyle\setminus$}}

\newcommand{\remin}{\mathop{-\!\!\!\!\!\hspace*{1mm}\raisebox{0.5mm}{$

\cdot$}}\nolimits}

\newcommand{\aquant}{\forall}

\newcommand{\equant}{\exists}



\newcommand{\beq}{\begin{equation}}

\newcommand{\eeq}{\end{equation}}





%%%%%%%%%%%%%%%%%%%%%%%%%%%   equations   %%%%%%%%%%%%%%%%%%%%%%%%%%%%%%%%%%%%%%

\newcommand{\be}[1][{e:\arabic{equation}}] { \begin{equation}\label{#1} }

\newcommand{\ee} { \end{equation} }







%%%%%%%%%%%%%%%%%%%%%%%%%%%%   General Maths   %%%%%%%%%%%%%%%%%%%%%%%%%%%%%%%

%%%%%%%%%%%%%%%%%%%%%%%%%%%%   General Maths   %%%%%%%%%%%%%%%%%%%%%%%%%%%%%%%

%%%%%%%%%%%%%%%%%%%%%%%%%%%%   General Maths   %%%%%%%%%%%%%%%%%%%%%%%%%%%%%%%



%%stuff - mainly KL

\DeclareMathOperator{\lh}{lh}  %length of encoding of a finite sequence

\DeclareMathOperator{\TMaj}{Maj}

\DeclareMathOperator{\TAN}{W}

\DeclareMathOperator{\Id}{id}

\DeclareMathOperator{\Fluc}{Fluc}

\DeclareMathOperator{\pair}{j} % piar encoding



%%commands

\renewcommand{\emptyset}{\varnothing}



\newcommand{\ORi}[1]{\ensuremath{\bigwedge^{#1}_{i=1}}}





\newcommand{\RR}{\ensuremath{\mathbb{R}}}

\newcommand{\NN}{\ensuremath{\mathbb{N}}}

\newcommand{\QQ}{\ensuremath{\mathbb{Q}}}

\newcommand{\II}{\ensuremath{\mathbb{I}}}



\newcommand{\zero}{\ensuremath{\mathbf0}}

\newcommand{\one}{\ensuremath{\mathbf1}}

\newcommand{\two}{\ensuremath{\mathbf2}}



\newcommand{\xor}{\ensuremath{\dot\vee}}



%%%%%%%%%%%%%%%%%%%%%%%%%   Proof Theory   %%%%%%%%%%%%%%%%%%%%%%%%%%%%%%%

%%%%%%%%%%%%%%%%%%%%%%%%%   Proof Theory   %%%%%%%%%%%%%%%%%%%%%%%%%%%%%%%

%%%%%%%%%%%%%%%%%%%%%%%%%   Proof Theory   %%%%%%%%%%%%%%%%%%%%%%%%%%%%%%%



%% stuff

%\input prooftree



\DeclareMathOperator{\maj}{maj} %``majorizes

\DeclareMathOperator{\smaj}{s-maj} %``strongly majorizes

\DeclareMathOperator{\K}{K} %K_A from Howard's WKL ND-int

\DeclareMathOperator{\I}{I} %the ``in Interval predicate from BW







%% Types

\newcommand{\Tp}{\ensuremath{\emph{\protect\textbf{T}}}} %set of finite types T

\newcommand{\PT}{\ensuremath{\emph{\protect\textbf{P}}}} %set of pure types P

\newcommand{\tp}[1]{\ensuremath{^\mathbf{#1}}}



%% Models

\newcommand {\SetO}  { \ensuremath{\mathcal{S} } }

\newcommand {\Som}  { \ensuremath{\SetO^\omega} }

\newcommand {\Set}  { \Som }

\newcommand {\ContO}  { \ensuremath{\mathcal{C} } }

\newcommand {\Cont}  { \ensuremath{\ContO^\omega} }

\newcommand {\MajO}  { \ensuremath{\mathcal{M} } }

\newcommand {\Maj}  { \ensuremath{\MajO^\omega} }



%% Special sets

\newcommand{\universal}{\ensuremath{\emph{\protect\textbf{U}}}} %set of universal axioms



%% Systems

\newcommand{\Ax}{\ensuremath{\mathcal{A}^\omega}} %GnA iaft

\newcommand{\AHilb}{\ensuremath{\mathcal{A}^\omega[X,\langle\cdot,\cdot\rangle]}} %A Hilbert space

\newcommand{\AHilbS}{\ensuremath{\mathcal{A}^\omega[X,\langle\cdot,\cdot\rangle,S]}} %A Hilbert sp. + S

\newcommand{\GA}{\ensuremath{\usftext{G}_n\usftext{A}^\omega}} %GnA iaft

\newcommand{\weha}{\ensuremath{{\usftext{WE-HA}}^{\omega}}} % WE - HA iaft

\newcommand{\wepa}{\ensuremath{{\usftext{WE-PA}}^{\omega}}} % WE - PA iaft

\newcommand{\HA}{\ensuremath{{\usftext{HA}}}} % HA 

\newcommand{\PA}{\ensuremath{\usftext{PA}}} % PA 

\newcommand{\ha}{\ensuremath{{\usftext{HA}}^\omega}} % HA iaft

\newcommand{\pa}{\ensuremath{{\usftext{PA}}^\omega}} % PA iaft

\newcommand{\epa}{\ensuremath{{\usftext{E-PA}}^\omega}} % E - PA iaft

\newcommand{\eha}{\ensuremath{{\usftext{E-HA}}^\omega}} % E - HA iaft

\newcommand{\hrrepa}{\ensuremath{\widehat{\usftext{E-PA}}^\omega\kleene}} %hrr E - PA iaft

\newcommand{\hrreha}{\ensuremath{\widehat{\usftext{E-HA}}^\omega\kleene}} %hrr E - HA iaft

\newcommand{\rreha}{\ensuremath{\usftext{E-HA}^\omega\kleene}} %rr E - HA iaft

\newcommand{\rrweha}{\ensuremath{\usftext{WE-HA}^\omega\kleene}} %rr WE - HA iaft

\newcommand{\kleene}{\ensuremath{\!\!\!\restriction}}   % upper arrow

\newcommand{\hrrwepa}{\ensuremath{\widehat{\usftext{WE-PA}}^\omega\kleene}} %hrr WE - PA iaft

\newcommand{\hrrweha}{\ensuremath{\widehat{\usftext{WE-HA}}^\omega\kleene}} %hrr WE - HA iaft



\newcommand{\HAS}{\ensuremath{\usftext{HAS}}} %second order logic

\newcommand{\HAH}{\ensuremath{\usftext{HAH}}} %higher -/-

\newcommand{\ACA}{\ensuremath{\usftext{ACA}}} %

\newcommand{\RCA}{\ensuremath{\usftext{RCA}}} %



%% Principles

\newcommand{\IA}{\ensuremath{\usftext{IA}}} %induction schema

\newcommand{\IP}{\ensuremath{\usftext{IP}}} %induction principle

\newcommand{\IR}{\ensuremath{\usftext{IR}}} %induction rule

\newcommand{\BR}{\ensuremath{\usftext{BR}}} %induction rule





\newcommand{\IPP}{\ensuremath{\usftext{IPP}}}

\newcommand{\PCM}{\ensuremath{\usftext{PCM}}}

\newcommand{\LEM}{\ensuremath{\Sigma^0_1\usftext{-LEM}}}

\newcommand{\lLEM}{\ensuremath{\usftext{LEM}}}

\newcommand{\CP}{\ensuremath{\usftext{CP}}}

\newcommand{\BW}{\ensuremath{\usftext{BW}}}

\renewcommand{\AA}{\ensuremath{\usftext{AA}}}

\newcommand{\Limsup}{\ensuremath{\usftext{Limsup}}}

\newcommand{\DNS}{\ensuremath{\usftext{DNS}}}

\newcommand{\DNE}{\ensuremath{\usftext{DNE}}}

\newcommand{\CA}{\ensuremath{\usftext{CA}}}

\newcommand{\QF}{\ensuremath{\usftext{QF}}}

\newcommand{\QFm}{\ensuremath{\usftext{QF-}}}

\newcommand{\CAhut}{\ensuremath{\widehat{\CA}}}



\newcommand{\AC}{\ensuremath{\usftext{AC}}} 

\newcommand{\ER}{\ensuremath{\usftext{ER}}} 



\newcommand{\WKL}{\ensuremath{\usftext{WKL}}}

\newcommand{\FAN}{\ensuremath{\usftext{FAN}}}



%% General abreviations

\newcommand{\PiL}{\ensuremath{\Pi^0_1}} 

\newcommand{\PiLm}{\ensuremath{\Pi^0_1\usftext{-}}} 

\newcommand{\SiL}{\ensuremath{\Sigma^0_1}} 

\newcommand{\SiLm}{\ensuremath{\Sigma^0_1\usftext{-}}} 

\newcommand{\m}{\ensuremath{\usftext{-}}}





%% for WKL



\newcommand{\BTree}{\ensuremath{\usftext{BinTree}}}

\newcommand{\BFunc}{\ensuremath{\usftext{BinFunc}}}

\newcommand{\UnBounded}{\ensuremath{\usftext{Unbounded}}}

%\newcommand{\Bounded}{\ensuremath{\usftext{Bounded}}}

\newcommand{\Sec}{\ensuremath{\usftext{Sec}}} % Boundedly secured

\newcommand{\BSec}{\ensuremath{\usftext{BarSec}}} % Boundedly secured at bar k

\newcommand{\BSecA}{\ensuremath{\usftext{BarSec}_A}} % Boundedly secured at bar K_A



\newcommand{\B}{\ensuremath{\usftext{B}}} %bar recursor

\newcommand{\rB}{\ensuremath{\usftext{B'}}} %restricted bar recursor

\newcommand{\R}{\ensuremath{\usftext{R}}} %recursor

\newcommand{\bPhi}{                       %special bar recursor

 \raisebox{-1.0pt} {

   \ensuremath{\usftext{\Large {\!$\Phi$\!}}}

 }

}



\newcommand{\T}{\ensuremath{\mathcal{T}}} %G???els T

\newcommand{\M}{\ensuremath{\usftext{M}^\omega}} %Markov principle iaft

\renewcommand{\H}{\ensuremath{\usftext{H}}} %Funny Howards argument



\newcommand{\proves}{\vdash}  %proves |-

\newcommand{\forces}{\Vdash}  %||-

\renewcommand{\models}{\vDash}  %|=







\newcommand{\tup}{\underline} %tuple

\newcommand{\atup}{\ensuremath{\,\underline}} %tuple as a parameter



\newcommand{\Tif}{\text{if}\ }

\newcommand{\Telse}{\text{otherwise}}



%%%%%%%%%%%%%%%%%%%%%%%%%   Theorems   %%%%%%%%%%%%%%%%%%%%%%%%%%%%%%%%%%%

%%%%%%%%%%%%%%%%%%%%%%%%%   Theorems   %%%%%%%%%%%%%%%%%%%%%%%%%%%%%%%%%%%

%%%%%%%%%%%%%%%%%%%%%%%%%   Theorems   %%%%%%%%%%%%%%%%%%%%%%%%%%%%%%%%%%%





\theoremstyle{plain}

\newtheorem{thm}{Theorem}[section]

\newtheorem{lemma}[thm]{Lemma}

%\newtheorem*{lemma*}{Lemma}

\newtheorem{prop}[thm]{Proposition}

\newtheorem{cor}[thm]{Corollary}

\newtheorem{con}[thm]{Conjecture}



\theoremstyle{definition}

\newtheorem{dfn}[thm]{Definition}

\newtheorem{rmk}[thm]{Remark}



\theoremstyle{remark}

\newtheorem*{remark}{Remark}

\newtheorem*{eg}{Example}



%\newenvironment{lemma*}[2][]{\noindent{\bf Recall Lemma~\ref{#2}} ({#1}).}{\\}

\newenvironment{lemma*}[2][]{\noindent{\bf Recall Lemma~\ref{#2}} ({#1}). \begin{it}}{\end{it}\\}







%  from Klaus:

 \renewenvironment{proof}[1][]{\noindent{\bf Proof{#1}. }}{\nopagebreak[4]{\hspace*{\fill}

%   \rule{1.2ex}{1.2ex} % Full big box

  $\Box$              % Empty box

 }{\vspace{2ex}}}



%%%%%%%%%%%%%%%% get it stylish - shortcuts %%%%%%%%%%%%%%%%%%%%%%

%%%%%%%%%%%%%%%% get it stylish - shortcuts %%%%%%%%%%%%%%%%%%%%%%

%%%%%%%%%%%%%%%% get it stylish - shortcuts %%%%%%%%%%%%%%%%%%%%%%



\renewcommand{\phi}{\varphi}

\renewcommand{\epsilon}{\varepsilon}

\newcommand{\lb}{\linebreak[0]}

\newcommand{\pb}{\pagebreak[0]}

\newcommand{\nbd}{\nobreakdash-}



\newcommand{\lOrd}[1]{\text{$\!\!$

\begin{smaller}<\end{smaller}\nolinebreak[4] $\!#1$\hspace{-3pt}

}}



\newcommand{\lOrdm}[1]{\text{

\lOrd{#1}\nbd 

%\hspace{-4pt}

}}



%\makeatletter

\def\Ddots{\mathinner{\mkern1mu\raise\p@

\vbox{\kern7\p@\hbox{.}}\mkern2mu

\raise4\p@\hbox{.}\mkern2mu\raise7\p@\hbox{.}\mkern1mu}}

%\makeatother



\newcommand{\embeded}{\hookrightarrow}







%% A numbered theorem with a fancy name:



\newtheorem{mainthm}[thm]{Main Theorem}



%% Numbered objects of "non-theorem" style (text roman):



\theoremstyle{definition}

\newtheorem{defin}[thm]{Definition}

\newtheorem{rem}[thm]{Remark}

\newtheorem{exa}[thm]{Example}



\begin{document}





%%%%%%%%%%%





\title{Fluctuations, effective learnability and metastability in analysis}

\author{Ulrich Kohlenbach, Pavol Safarik}

\ead{kohlenbach@mathematik.tu-darmstadt.de, pavol.safarik@gmail.com}

\address{Department of Mathematics, Technische Universit\"at Darmstadt, Schlossgartenstra{\ss}e 7, 64289 Darmstadt, Germany\\[1mm] 

Dedicated to Professor Sergei Artemov on the occasion of his 60th birthday}



\date{}



\begin{abstract}

This paper discusses what kind of quantitative information one can extract 

under which circumstances from proofs of convergence statements in analysis. 

We show that from proofs using only a limited amount of the 

law-of-excluded-middle, 

one can extract functionals $(B,L),$ where $L$ is a learning procedure for 

a rate of convergence which succeeds after at most $B(a)$-many mind changes. 

This $(B,L)$-learnability provides quantitative information strictly in 

between a full rate of convergence (obtainable in general only from 

semi-constructive proofs) and a rate of metastability in the sense of Tao 

(extractable also from classical proofs). In fact, it corresponds to rates 

of metastability of a particular simple form. Moreover, 

if a certain gap condition is satisfied, then $B$ and $L$ yield a bound on 

the number of possible fluctuations. We explain recent 

applications of proof mining to ergodic theory in terms of these results. 

\end{abstract}



\begin{keyword}

Fluctuations, effective learnability, metastability, proof mining, uniform 

bounds, functionals of finite type, nonlinear ergodic theory, hard analysis.\\

\MSC[2010]{03F10, 03F60, 47H25, 37A30}.

\end{keyword}





\maketitle





\section{Introduction}



\newtheorem{definition}{Definition}[section]

\newtheorem{proposition}[definition]{Proposition}

%\newtheorem{remark}[definition]{Remark}

\newtheorem{theorem}[definition]{Theorem}

\newtheorem{corollary}[definition]{Corollary}

%\newtheorem{lemma}[definition]{Lemma}

\newtheorem{exercise}[definition]{Exercise}

\newtheorem{clm}[definition]{Claim}

%\newtheorem{prop}[definition]{Proposition}

\newtheorem{example}[definition]{Example}

\newtheorem{notation}[definition]{Notation}

\newtheorem{application}[definition]{Application} 





In this paper we investigate different levels of effective quantitative 

information on theorems stating the Cauchy property of some sequence 

$(x_n)$ in a metric space $(X,d)$

\[ (1) \ \forall k\in\NN\,\exists n\in\NN\,\forall m,\tilde{m}\ge n

\ \big( d(x_m,

x_{\tilde{m}})\le 2^{-k}\big) \] 

and also more general $\Pi^0_3$-theorems (also with higher type parameters) 

\[ (2) \ \varphi\equiv 

\forall k\in\NN\,\exists n\in\NN\,

\forall m\in\NN\,\varphi_0(k,n,m),\] where $\varphi_0$ is quantifier-free. 

Since we refer to real numbers as fast 

converging Cauchy sequences of rational numbers we have $\le_{\RR}\,\in\Pi^0_1$ 

so that $(1)$ has the form $(2).$

\\[2mm] 

Cauchy statements $(1)$ are special forms of finiteness statements expressing 

that there are only finitely many $2^{-k}$-fluctuations $(i_l,j_l)$ 

with 

\[ (3) \ j_l>i_l\wedge d(x_{i_l},x_{j_l})>2^{-k}. \] 

As with general finiteness statements one can ask for a bound on the height 

of $2^{-k}$-fluctuations, i.e. an $\rho(k)$ above which no such fluctuation 

occurs (so $\rho$ is a rate of convergence) 

or for a weaker bound $F(k)$ on the number $l$ 

of such fluctuations $(i_0,j_0),\ldots,(i_l,j_l)$ with $i_{n+1}\ge j_{n}$ 

for $n<l.$ 

As to be expected from standard recursion theoretic facts about finiteness 

statements (see \cite{Luckhardt(89)}), even primitive recursive Cauchy sequences $(x_n)$ in $\RR$ 

in general will not admit a computable (in $k$) bound $F$ on the fluctuations 

and even in cases where they do, in general there will be no computable 

rate of convergence $\rho.$\\[2mm] 

A yet weaker information than a bound $F$ on the number of fluctuations is 

a bound on the Kreisel no-counterexample 

interpretation (called `metastability' by Tao) of $(x_n)$, namely a functional 

$\Phi(k,g)$ such that 

\[ (4) \ \forall k\in\NN\,\forall g:\NN\to\NN\, \exists n\le \Phi(k,g) 

\,\forall i,j \in [n;n+g(n)]\ \big( d(x_i,x_j)\le 2^{-k}\big). \] We call 

$\Phi$ a {\it rate of metastability} for $(x_n).$ \\  

Note that already the underlying reformulation 

\[ (5) \ \forall k\in \NN\,\forall g:\NN\to\NN\, \exists n

\,\forall i,j \in [n;n+g(n)]\ \big( d(x_i,x_j)\le 2^{-k}\big) \]

of the Cauchy property 

still expresses the full Cauchy property of $(x_n).$ However, the proof 

of the latter from the former is noneffective, corresponding to the fact 

that there is no way to 

pass (even pointwise let alone uniformly) 

from an effective $\Phi$ in $(4)$ to an 

effective bound on fluctuations $F$ or an effective rate of convergence $\rho.$

\\[2mm] Using unbounded search (over the code of the 

pair of $\exists n$ and the existential quantifier hidden in $<_{\RR}$) one 

can always obtain a rate of metastability that is computable {\bf relative} to 

$(d(x_i,x_j)_{i,j\in\NN},$ but unless  $(x_n)$ is a sequence in a metric 

space $X$ with a computability 

structure (e.g. $X:=\RR^n$ with the Euclidean norm as a computable metric), 

it makes no sense to talk about the computability of $(x_n).$ 

Moreover, such an unbounded search does not provide any complexity 

information and the bound will be highly nonuniform (being dependent 

on all the data used to define 

$(x_n)$). In all the applications to which we refer to below, $X$ is an 

abstract (completely general) Hilbert space or CAT(0) space and so the 

computability of $(x_n)$ is not even defined. So to be able to talk about 

a {\bf computable in the data used to define} {\boldmath{$(x_n)$}} 

rate of metastability $\Phi$ this rate must only depend 

on general bounding data in $\NN$ or $\NN^{\NN},$ i.e. $\Phi$ must be 

highly uniform. While this uniformity sometimes can be established by 

going to ultraproducts of $X$ (see \cite{Avigad/Iovino}) this not even 

seems to yield the existence of a {\bf computable} $\Phi$ let alone  

of some complexity information.

\\[2mm] General logical metatheorems for strong systems of analysis based 

on full classical logic guarantee that the extractability of (sub-)recursive 

(and highly uniform) rates of metastability $\Phi$ is always possible for 

large classes of convergence proofs. This has been applied extensively in 

the context of nonlinear analysis, fixed point theory and ergodic theory 

during the last 10 years. One of these results is the extraction of a 

uniform rate of metastability for the strong convergence in the mean 

ergodic theorem for uniformly convex Banach spaces $X$ from a proof 

due to Birkhoff \cite{Bir39} carried out in 

\cite{Kohlenbach/Leustean4}. This rate only depends on a norm bound 

$\NN\ni b\ge \| x\|$ of the starting point $x,$ a modulus $\eta\in\NN^{\NN}$ 

of uniform convexity of $X$ and the error $2^{-k}$ but, otherwise, is 

independent of $x,$ the operator and $X.$  

That a computable rate of convergence (even 

for an effective Hilbert space and a computable operator) in general 

is impossible has been shown in \cite{Avigad/Gerhardy/Towsner}. However, 

as recently observed by Avigad and Rute \cite{Avigad/Rute}, the analysis 

in \cite{Kohlenbach/Leustean4} can be used to obtain a simple effective (and 

also highly uniform) 

bound on the number of fluctuations (for the case of Hilbert spaces this 

was already obtained with an even better bound in \cite{Jones}). This 

raises the question whether there are general logical conditions on 

convergence proofs to guarantee the extractability of effective bounds 

on fluctuations. Obviously, any condition guaranteeing the extractability of 

a computable rate of convergence is a sufficient condition for this. Though 

not satisfied in the particular case just discussed, let us first consider 

this in order to see in what sense we might try to liberalize such conditions 

towards rates of fluctuations. To do so in somewhat more precise 

terms we fix a formal framework such as intuitionistic arithmetic 

HA$^{\omega}$ in all finite types (actually we use the so-called weakly 

extensional variant called WE-HA$^{\omega}$ in \cite{Kohlenbach(book)}) 

or its extension by an abstract (metric or) 

normed space $(X,\|\cdot\|)$ resulting 

in HA$^{\omega}[X,\|\cdot\|]$ possibly with further axioms stating that 

$X$ is uniformly convex or even a Hilbert space (see \cite{Kohlenbach(book)} 

for details). \\ Let AC be the full schema of choice and 

LEM$_{\neg}$ be the law-of-excluded-middle schema restricted to arbitrary 

negated formulas $\neg\psi$ (which, in particular, includes the case 

of existential-free formulas and so, as a very special case, $\Pi^0_1$-LEM, 

i.e. LEM restricted to $\Pi^0_1$-formulas). Then from a proof of $(1)$ (for 

some sequence $(x_n)$ definable by a term $t$ of the system having at most 

number and function parameters $a,f$) in 

\[ \mbox{HA$^{\omega} +$AC$+$LEM}_{\neg} \] 

(and in fact even stronger theories augmented with certain noneffective 

axioms $\Omega$), the extractability of a rate of convergence $\rho$ that 

is definable (in the same parameters as $t$) in  

G\"odel's calculus of primitive recursive functionals of 

finite type is guaranteed.  \\ 

This follows from the bound extraction theorem for monotone modified 

realizability from \cite{Kohlenbach(relative),Kohlenbach(book)} 

(and for theories with abstract spaces $X$ in \cite{GerKoh06}). 

In the case of 

HA$^{\omega}[X,\| \cdot\|]$ even parameters of types such as $X, \NN\to X, 

X\to X$ are allowed in the definition of the sequence $(x_n)$ in $X$ where 

then $\rho$ depends additionally on majorants for these parameters (which are 

natural numbers, in the case of the type $X,$ and number-theoretic 

functions, in the case of the types $\NN\to X$, $X\to X.$) \\[2mm] 

An important weak principle of classical logic not covered by this is 

the so-called Markov principle which, extended to all finite types, reads 

as follows 

\[ \mbox{M}^{\omega}\ :\ 

\neg\neg\exists \underline{x}^{\underline{\sigma}} \ 

\varphi_0(\underline{x})\to 

\exists \underline{x}^{\underline{\sigma}}\,\varphi_0(\underline{x}), \] 

where $\varphi_0$ is a quantifier-free formula (with arbitrary further 

parameters) and $\underline{\sigma}$ an 

arbitrary tuple of types. However, M$^{\omega}$ becomes permissible once 

LEM$_{\neg}$ is weakened to the so-called lesser-limited-omniscience-principle 

LLPO (which is the precise amount of classical logic needed to prove the 

binary (`weak') K\"onig's lemma WKL, which with AC intuitionistically implies 

K\"onig's lemma KL; see \cite{Kohlenbach(book)} for details).

So instead of HA$^{\omega}([X,\|\cdot\|])+$AC$+$LEM$_{\neg}$ we may also have 

\[ \mbox{HA$^{\omega}([X,\|\cdot \| ])+$AC$+$M$^{\omega}+$LLPO}, \] 

where then the extraction of a rate of convergence uses the so-called 

monotone functional interpretation (see \cite{Kohlenbach(book)}).

\\[2mm] The in a sense weakest principle covered by neither of these systems 

(but provable in their union!) is LEM restricted to $\Sigma^0_1$-formulas, 

which we denote by $\Sigma^0_1$-LEM: 

\[ \Sigma^0_1\mbox{-LEM}: \ \ \exists n\in\NN \,\varphi_0(n)\vee

\forall m\in\NN\,\neg \varphi_0(m), \]

where $\varphi_0$ is quantifier-free (but may contain parameters of 

arbitrary type). \\  

While $\Sigma^0_1$-LEM in the presence of 

AC (even when restricted to numbers) 

creates highly noncomputable functions (in particular when function 

parameters are allowed to occur in $\Sigma^0_1$-LEM which then makes it 

possible to climb up the entire arithmetical hierarchy) it remains fairly 

weak over HA$^{\omega}.$ Nevertheless, HA$+\Sigma^0_1$-LEM 

already allows one

to prove the Cauchyness of the Specker sequence \cite{Specker(49)}, 

a primitive recursive 

monotone decreasing sequence of rational numbers in $[0,1]$ which does not 

have a computable rate of convergence. In fact, as shown in \cite{Toftdal}, 

the principle that every bounded monotone sequence of reals is Cauchy can be 

proven in HA$^{\omega}+\Sigma^0_1$-LEM and for sequences defined by terms 

of HA$^{\omega}$ using only number parameters even with 

$\Sigma^0_1$-LEM$^-$ (where $P^-$ denotes the restriction of an axiom 

schema $P$ to number parameters only). This is not obvious and requires 

a novel 

proof as the usual argument uses the (by \cite{Akama} e.g. over HA) 

strictly stronger principle 

\[ \Sigma^0_2\mbox{-DNE}\ :\ \neg\neg\exists n\in\NN\,\forall m\in\NN\,

\varphi_0(n,m)\to\exists n\in\NN\,\forall m\in\NN\,\varphi_0(n,m) \]

(`double-negation-elimination principle' for 

$\Sigma^0_2$-formulas). While $\Sigma^0_2$-DNE is limit computable in the 

sense of Hayashi and 

Nakana \cite{Hayashi/Nakata}, 

any single instance of $\Sigma^0_1$-LEM is even learnable 

with a single mind change. Note also that bounded monotone sequences of 

real numbers (say in $[0,1]$) always have the simple fluctuation bound 

$F(k):=2^k.$ That $\Sigma^0_1$-LEM$^-$ has a strictly stronger 

computational interpretation than $\Sigma^0_2$-DNE$^-$ can be spelled 

out in terms of proof interpretations:  $\Sigma^0_1$-LEM$^-$ (when added 

e.g. to HA$^{\omega}$) admits  

a modified realizability interpretation 

by terms that are primitive recursive (in the 

sense of G\"odels $T$) relative to a Skolem function $f_{\varphi_0}$ for 

\[ \forall k\in\NN\exists n \in\NN\forall m\in\NN 

\ (\varphi_0(k,n)\vee\neg\varphi_0(k,m)),\]

i.e. 

\[ \forall k,m\in\NN 

\ (\varphi_0(k,f_{\varphi_0}(k))\vee\neg\varphi_0(k,m)),  \] 

and much work studying this interpretation in terms of learning 

theory and so-called 1-backtracking games (in the sense of 

Coquand's game semantics \cite{Coquand,Berardi/Coquand/Hayashi}) 

has been carried out in recent years e.g. by 

Aschieri and Berardi (see e.g. 

\cite{Aschieri/Berardi,Aschieri1,Aschieri2}). By contrast to this, 

$\Sigma^0_2$-DNE$^-$ does not allow such an interpretation and not 

even a (in this context) weaker monotone modified realizability 

interpretation (in the sense of the \cite{Kohlenbach(relative)}, 

see also \cite{Kohlenbach(book)}) as shown in \cite{Akama}, where this 

is used to prove that $\Sigma^0_2$-DNE$^-$ not even follows from 

$\Pi^0_2$-LEM$^-$. 

\begin{rmk} The various realizability and functional interpretations 

referred to above are different ways of giving a precise meaning to the 

informal so-called Brouwer-Heyting-Kolmogorov (`BHK') 

interpretation of intuitionistic 

logic. A rather different way of formalizing BKH -- based on the concept 

of operations on formal proofs -- has been developed since 

the mid 90's by Sergei Artemov, first for the propositional case (see 

e.g. \cite{Artemov}) and very recently (together with 

Tatiana Yavorskaya) for predicate logic (see \cite{Artemov11}).

\end{rmk}

At a first look, all this might suggest to consider 

HA$^{\omega}+\Sigma^0_1$-LEM$^{(-)}$ as a promising framework to 

guarantee computable bounds on the number of fluctuations for provable Cauchy 

sequences (while in general not computable rates of convergence). 

However, this 

turns out to be mistaken as $\Sigma^0_1$-LEM is already the general case: 

let $(x_n)$ be a sequence of real numbers definable by a term $t$ 

in HA$^{\omega}$ 

(which may have variables of arbitrary type as parameters). Suppose that 

for the extension PA$^{\omega}$ of HA$^{\omega}$ by full classical logic 

\[ \mbox{PA}^{\omega}\vdash \forall k\,\exists n\,\forall m,\tilde{m}\ge n\,

(|x_m-x_{\tilde{m}}|\le_{\RR} 2^{-k}). \]

Then by negative translation (see \cite{Kohlenbach(book)}) 

\[ \mbox{HA}^{\omega}\vdash  \forall k\,\neg\neg 

\exists n\,\forall m,\tilde{m}\ge n\,

(|x_m-x_{\tilde{m}}|\le_{\RR} 2^{-k}). \]

Adapting Friedman's proof for the closure of HA$^{\omega}$ under the Markov 

rule one can show (this is stated for HA without proof in 

\cite{Hayashi/Nakata} and we include a proof -- also for 

HA$^{\omega}[X,\|\cdot\|]$ -- below) that HA$^{\omega}+

\Sigma^0_1$-LEM  

is closed under the rule version of $\Sigma^0_2$-DNE so that we get 

\[ \mbox{HA$^{\omega}+\Sigma^0_1$-LEM} \ 

\vdash \forall k\,\exists n\,\forall m,\tilde{m}\ge n\,

(|x_m-x_{\tilde{m}}|\le_{\RR} 2^{-k}). \] 

Moreover, if $t$ contains at most number parameters it also suffices to 

use the restriction $\Sigma^0_1$-LEM$^-$ of $\Sigma^0_1$-LEM to 

$\Sigma^0_1$-formulas with number parameters only. All this also holds for 

the systems HA$^{\omega}[X,\|\cdot\|]$ and PA$^{\omega}[X,\|\cdot\|]$ 

and sequences $(x_n)$ in $X$ defined by terms of these systems.

\\[2mm] 

So, as far as $\Pi^0_3$-theorems are concerned 

(also with parameters in $\NN,\NN^{\NN}$ 

or -- in the case of theories with an abstract normed space $X$ -- 

also in $X,X\to X, \NN\to X$), there is no difference in proofs based 

on full classical logic versus proofs using only $\Sigma^0_1$-LEM 

(at least as long as the proofs can be formalized in systems to which 

negative translation and Friedman's $A$-translation apply). So in 

order to have a computational content stronger than metastability 

guaranteed, we have to look for more restricted uses of $\Sigma^0_1$-LEM$^-.$  

\\  

Looking more carefully into the $\Sigma^0_1$-LEM-based proof of the 

Cauchyness of bounded monotone sequences as given in \cite{Toftdal} reveals 

that one can define a sequence of instances $\Sigma^0_1$-LEM$(s(n))$ 

of $\Sigma^0_1$-LEM$^-$ 

\[ \exists m\in\NN \,(s(n,m)=0)\vee \forall m\in\NN\,(s(n,m)\not= 0) \] 

such that to prove the Cauchy property with error $2^{-k}$ one only needs 

the first $n=0,\ldots,s(t(k))$-many instances of this sequence where $t$ is 

a simple primitive recursive function. So a more promising approach would be

to look at proofs of Cauchy statements which can be formalized as follows:

\[ \mbox{HA}^{\omega}\ \vdash \forall k \  \big( \forall l\le t(k) \ 

\Sigma^0_1\mbox{-LEM}(s(l))\to \exists n\,\forall m,\tilde{m}\ge n\,

(|x_m-x_{\tilde{m}}|\le_{\RR} 2^{-k})\big) \] 

or 

\[ \mbox{HA}^{\omega}[X,\|\cdot\|] \ \vdash \forall k \  \big( \forall l\le 

t(k) \ 

\Sigma^0_1\mbox{-LEM}(s(l))\to \exists n\,\forall m,\tilde{m}\ge n\,

(\|x_m-x_{\tilde{m}}\|\le_{\RR} 2^{-k})\big), \]

where $t$ may contain parameters of type $\NN, \NN\to\NN$ (and $X,\NN\to X, 

X\to X$ in the extended system). \\[2mm] 

We show that from such proofs one can always extract effective (and in 

fact primitive recursive in the sense of G\"odel's $T$) bounds $B,L$ on the 

effective learnability of a rate of convergence of $(x_n).$ Here 

$B,L$ are effective functionals (in the parameters $\underline{a}$ 

of the problem) where 

$L$ is the learning procedure and $B$ a bound on the number of 

necessary steps along this procedure.  

This is, as we will show, a strictly stronger information than a 

rate of metastability as 

the latter can be obtained from (majorants $B^*,L^*$ of) the former, 

even by a uniform primitive 

recursive procedure (in the ordinary sense of Kleene). However, there are 

primitive recursive 

Cauchy sequences of rational numbers 

with a primitive recursive (again in the ordinary 

sense of Kleene) rate of metastability which 

do not admit any computable bound for the 

learnability of a rate of convergence. \\ So while (as discussed above) a 

computable Cauchy sequence in $\RR$ always has a computable rate of 

metastability (by unbounded search) it in general will not have computable 

bounds $B,L$ on the effective learnability of a rate of convergence.\\ 

In fact, the additional 

information provided by $B^*,L^*$ becomes visible by the particular 

simple structure of the rate of metastability obtained from 

$B^*,L^*$ which is guaranteed 

to be of the form (essentially) 

\[ (L^*(\underline{a}^*)\circ \tilde{g})^{B^*(\underline{a}^*)}(0), \] 

where $f^{x}(0)$ denotes the $x$-times iteration of the function $f$ 

starting from $0$ and $\tilde{g}(n):=\max\{ g(i):i\le n\}+n.$ 

The essential point here is that $B^*,L^*$ do not 

involve the counterfunction $g.$ It is precisely this form of a rate 

of metastability that has been observed many times in concrete unwindings 

in ergodic theory and fixed point theory (see e.g.\cite{Avigad/Gerhardy/Towsner,Kohlenbach/Leustean4,Kohlenbach/Leustean3,Kohlenbach(Browder),Kohlenbach/Leustean6,Kohlenbach/Leustean7}) and which we can explain 

now for the first time in terms of the logical structure of the given 

proof.\footnote{Note that minor modifications of the above format which show 

up in these bounds, e.g. $L^*_1\circ \tilde{g}\circ L^*_2$, 

can easily be reduced 

to this format.} For Cauchy statements, a metastability rate which has 

the simple form given above conversely implies that the Cauchy statement is 

$(B^*,L^*)$-learnable. 

Notable exceptions to this restricted format of metastability are the rates 

of metastability for 

the ergodic theorem for odd (and even more general) operators in 

\cite{Safarik(11)} (making a nested use of the iteration procedure) and 

for the (weak convergence in the) Baillon nonlinear ergodic theorem 

in \cite{Kohlenbach(Baillon)} (making even nested use of a bar recursive 

functional). However, both underlying proofs 

violate precisely our criterion of a bounded use of $\Sigma^0_1$-LEM.  

\\[2mm]

A bound on the number of fluctuations in general is a still  

strictly stronger quantitative information than bounds on the effective 

learnability: we construct a primitive recursive sequence of rational 

numbers in $[0,1]$ which 

has primitive recursive bounds on the learnability of its Cauchy rate 

but does not admit a computable bound on the number of its fluctuations.

Together with the already discussed Specker sequences (which have 

trivial fluctuation bounds but no effective rates of convergence) 

we get the (w.r.t. effectivity) strictly decreasing hierarchy of 

quantitative data for the convergence of Cauchy sequences $(x_n)$:

\begin{enumerate}

\item 

A rate $\rho$ of convergence of $(x_n).$

\item 

A bound $F$ on the number of approximate fluctuations of $(x_n).$

\item 

Function(al)s $B,L$ (see the next section for a precise definition) 

of the learnability of a rate of convergence for $(x_n)$ by 

$B$-many mind changes by a learning procedure $L.$

\item 

A rate $\Phi$ of metastability for $(x_n).$

\end{enumerate}   

While -- as discussed above -- the extractability of effective data 

for the levels 1, 3 and 4 of this hierarchy is guaranteed by 

relatively easy to check logical {\bf a-priori}  

conditions on the framework in which a Cauchy statement is proven, 

this seems to be different for level 2: we give a kind of gap condition 

to be satisfied by $L$ in `3.' which suffices to convert the 

information provided by $B,L$ into a bound $F$ on the fluctuations. 

Since to check this condition requires the inspection of the extracted 

data $B,L$ this is an {\bf a-postiori} condition (which is reminiscent of 

the growth conditions used in Luckhardt's \cite{Luckhardt(89)} extraction 

of bounds on the number of solutions in Roth's theorem).

\\[2mm] 

The above hierarchy apparently fits well to distinguish the 

computational content that recently has been obtained from proofs 

in the context of ergodic theory: \\[1mm] 

As discussed already, Avigad and Rute \cite{Avigad/Rute} observed 

that the extraction of a rate of metastability from Birkhoff's 

proof of the mean ergodic theorem in uniformly convex Banach 

spaces carried out in \cite{Kohlenbach/Leustean4} can in fact been 

used to even obtain a uniform effective fluctuation bound.  

This corresponds to second level of our hierarchy 

which by \cite{Avigad/Gerhardy/Towsner} 

cannot be further improved effectively to the first level. We explain 

this in terms of our gap condition satisfied by $(B,L).$

\\[1mm] The logical condition needed to assure the extractability of a 

primitive recursive 

(in the ordinary sense of Kleene) data $B,L$ for the third 

level is e.g. satisfied in 

the proofs of the strong convergence of so-called Halpern iterations 

in Hilbert spaces (due to Wittmann \cite{Wittmann(92)}) and -- more 

generally --  in CAT(0) spaces (due to Saejung in 

\cite{Saejung}). This follows from the 

analysis of Saejung's proof 

given in \cite{Kohlenbach/Leustean6} where a primitive 

recursive (in the ordinary sense) rate of metastability is extracted

(the analysis of Wittmann's proof in \cite{Kohlenbach(Browder)} also 

shows this in the Hilbert case).  

A special form of the Halpern iteration (covered by 

\cite{Wittmann(92),Saejung}) is given by  

\[ x_{n+1} :=\frac{1}{n+2}x_0+\left(1-\frac{1}{n+2}\right) Tx_n \]

which can be viewed as a nonlinear generalization of 

the ergodic average 

\[ \frac{1}{n+1}\sum^n_{i=0}T^ix_0 \] 

in the mean ergodic theorem with which it coincides 

for {\bf linear} nonexpansive  maps $T.$ 

As a corollary we obtain the extractability even 

of primitive recursive learnability data $B,L$ in this case. 

However, the aforementioned gap condition does not hold for the data 

we extracted from the proofs. So, as it stands, a fluctuation bound does 

not seem to follow. Of course, this does not rule out at all that a different 

proof might yield better data which possibly could satisfy the gap 

condition and that, consequently, effective fluctuation bounds might 

result. 

\\[1mm]  

While the strong convergence of the ergodic average in general is known 

to fail for nonlinear nonexpansive maps (whereas weak convergence 

holds by a deep theorem of Baillon \cite{Baillon(75)}), 

it does hold e.g. 

for odd operators as shown again by Baillon \cite{Baillon(76)} 

and -- for a much more 

general class of mappings -- by Wittmann \cite{Wittmann(90)}. 

Recently, the second author 

\cite{Safarik(11)} extracted 

a (primitive recursive in the sense of Kleene) rate of metastability 

(i.e. level 4) from Wittmann's proof. However, Wittmann's proof does 

not satisfy the condition sufficient to guarantee level-3 learnability 

data and, in fact, the extracted bound has a structure similar to the 

one in our example of a primitive recursive sequence of rational 

numbers in $[0,1]$ separating the 

levels 3 and 4 given in section 4 below. 







\section{Fluctuations versus effective learnability}

To be specific, let us use in the following the language of (intuitionistic) 

arithmetic in all finite types HA$^{\omega}$ (more precisely the system 

WE-HA$^{\omega}$, see \cite{Kohlenbach(book)}) as well as its extension HA$^{\omega}[X,\|\cdot\|]$ 

by an abstract normed space $X$ in the sense of \cite{Kohlenbach(metapaper),GerKoh06,Kohlenbach(book)} in order to be able to cover also the aforementioned 

recent applications of proof mining to ergodic and fixed point theory which 

need this enriched language. Everything we say extends mutatis mutandis also 

to theories where more conditions on $X$ are prescribed (e.g. $X$ being 

a uniformly convex or a Hilbert space) and convex subsets $C$ of $X$ being 

added as well as to metric structures $X$ 

such as metric, $W$-hyperbolic and CAT(0) spaces (see \cite{Kohlenbach(book)} 

for all this). The type of natural numbers $\NN$ is usually denoted by $0$ 

while $1$ denotes the type of functions $\NN\to\NN.$

\\[1mm] ${\cal S}^{\omega,X}$ denotes the full set-theoretic model of these 

theories over the base types $\NN$ and $X.$ Occasionally, we will need 

the relation `$x^*$ majorizes $x$' (short: $x^* \ maj \ x$) due to W.A. Howard 

(for the finite types over $\NN$)  

which is defined in the usual hereditary 

way by induction on the type of $x$ starting from 

\[ x^* \ maj_0 \ x\ :\equiv x^*\ge x \] for $x^*,x$ of type $0$ and 

\[ x^* \ maj_X \ x\ :\equiv x^* \ge \| x\|\] for $x$ of type $X$ and 

$x^*$ of type $0.$ 



\begin{rmk} \label{remark-coding}

Throughout this paper, we will use several encodings. We use $j$ for the Cantor pairing function and $j_1$ and $j_2$ for the corresponding

projections. Moreover, we use $\langle \tup a \rangle$ for both

\begin{itemize}

\item a surjective primitive recursive 

encoding of tuples of a given length $l$, with the corresponding projections $j_1,\ldots,j_l$ and

\item 

a surjective sequence encoding (which then includes the length of the encoded sequence) with primitive recursive functions for length $\lh$, concatenation $*$, 

and projection $(\cdot)_{(\cdot)}$ (i.e. $(n)_k$ for the $k+1$-th element 

of a finite sequence encoded by $n$ and $0$ for $k\ge lh(n)$,) 

when there is not danger of confusion we use also the simpler notation $n_k$). 

For details see~\cite{Kohlenbach(book)}.

\item 

For both the tuple and the sequence encoding we assume the coding to be 

increasing in each component and that $\langle a_0,\ldots,a_{k-1}\rangle 

\ge a_i$ for $i<k.$

\end{itemize}

For a specific encoding satisfying these requirements 

see~\cite{Kohlenbach(book)}, where the sequence

encoding is denoted by $\langle a_0,\ldots,a_{k-1}\rangle$ while the

$k$-tuple encoding is denoted by $\nu^k(a_0,\ldots,a_{k-1})$. \\[1mm] Whether we mean a tuple or a sequence 

coding should be mostly clear from the context (roughly, we mean tuple encoding whenever the length is fixed and sequence encoding otherwise), but whenever this is relevant we say also explicitly which encoding is meant.

\end{rmk}



\begin{dfn}[the number of fluctuations]

For a sequence $x_{(\cdot)}$ in some metric space $(X,d)$ and an $\epsilon>0$

let $\Fluc(n,i,j)$ denote that there are $n$ fluctuations whose indexes are encoded into $i$ and $j$.

\begin{align*}

\Fluc(n,i,j):\equiv\Fluc_{x_{(\cdot)},\epsilon}(n,i,j):\equiv\quad &\lh(i)=\lh(j)=n\quad \wedge\\ 

&\forall k<n\ (i_k<j_k) \quad \wedge\\

&\forall k<n-1\ (j_k\leq i_{k+1}) \quad \wedge\\

&\forall k<n\ (d(x_{i_k},x_{j_k})>\epsilon).

\end{align*}

We call $b$ a bound on the number of $\epsilon$-fluctuations of $x_{(\cdot)}$, iff 

\[

\forall n>b \forall i,j \neg\Fluc(n,i,j).

\]

We call $b$ effective if it is computable in $\epsilon\in\QQ^*_+$ and $x_{(\cdot)}$.

\end{dfn}



In the Language identification in the limit model for inductive inference, the 

notion of learnable with an existence of a mind change bound was introduced in 

the sixties (see e.g.~\cite{Gold(67)}). We define a similar concept in the context of 

general formal theories like $\pa$. Since we require both the learning 

procedure and the bound on mind changes to be recursive (effective) we call 

this property {\em effective learnability}. 

\begin{rmk}

The proof-theoretic study of 

learnability by finitely 

many (though not necessarily effectively bounded) mind changes in analysis 

has been initiated by Hayashi (see e.g. \cite{Hayashi02,Hayashi06}) 

who (with Nakata) 

established the close relation of this 

concept to limit computability (see e.g. 

\cite{Hayashi/Nakata}). The concept of mind change for Cauchy statements is 

also implicit in section 5.1 of \cite{Ziegler(07)} (Proof of Lemma 31.c). 

Effective learnability concepts for 

functionals $F:D\to\NN^{\NN}$ (with $D\subseteq\NN^{\NN}$) have recently 

been investigated in \cite{Higuchi/Kihara}.

\end{rmk}

On the one hand we would like the learning procedure to be as simple as possible and on the other hand we would like to formalize that

it can access as much finite information as is available (at a given learning step). In the case of monotone formulas (which is a rather rich class of statements including, in 

particular, all Cauchy statements), there is a straightforward answer (see Definition~\ref{d:fmcMon} and Proposition~\ref{p:allx}) 

allowing us to simplify the theory of learnability, if we assume monotonicity.

We give a more general definition (Definition~\ref{d:fmcNum}), which coincides with Definition~\ref{d:fmcMon} in the monotone case, a few pages later.





\begin{dfn}[$(B,L)$-learnable monotone formulas]\label{d:fmcMon}

Consider a $\Sigma^0_2$ formula $\phi$ with the only parameters $\tup a^{\tup \sigma}$, i.e.

\[\phi\ \equiv\ \exists n^0 \forall x^0\ \phi_0(x,n,\tup a),\]

which is monotone in $n$, i.e.

\[ 

\forall n^0\ \forall n'\ge n\ \forall x^0\ \big(  \phi_0\ (x,n,\tup a) \rightarrow  \phi_0(x,n',\tup a) \big).

\]

We call such a formula $\phi$ 

{\em (B,L)-learnable},

if there are function(al)s $B$ and $L$ such that the following holds 

(in the full set-theoretic model ${\cal S}^{\omega,X}$):

\[ 

\exists i\leq B(\tup a)\ \forall x\ \phi_0(x,c_i,\tup a),\] where

\begin{align*}

c_0&:=0,\\

c_{i+1}&:=

\begin{cases}

L(x, \tup a),&\text{for the $x$ with } \neg\phi_0(x,c_i,\tup a)\ \wedge\ \forall y<x\ \phi_0(y,c_i,\tup a) \text{ if it exists}\\

c_i,&\Telse.

\end{cases}

\end{align*}

We call such a $\phi$ 

{\em effectively learnable (with effectively bounded many mind changes)} 

if it is $(B,L)$-learnable with computable functionals $B$ and $L$.\footnote{Note that $c_i$ is the $i$-th attempt to produce a $c$andidate for a valid $n$, while $B$ is a $B$ound on the number of such attempts produced before a valid candidate is $L$earned by the procedure $L$ .}

\end{dfn}

In section \ref{section4.2} (Proposition \ref{p:nonLearnablePhi}) 

we will construct a $\varphi$ which is true 

for all parameters $\underline{a}\in\NN^{\NN}$ 

but which is not $(B,L)$-learnable with 

computable $B,L.$ 

\\[1mm] 

This definition is very intuitive in the sense that it formalizes the concept of an (effective) learning process $L$ which learns the witness in an effectively bounded number of attempts in a very straightforward way.\\

Moreover, this definition allows the learner, i.e. the function $L$ to use the least amount of non-computable information possible, namely only the smallest counterexample to the learners last candidate for the witness. 

Nevertheless, we will show that this amount of information is, in a sense, 

already exhaustive. More precisely, we have the following (we use in the 

rest of this section a surjective sequence coding 

denoted by $\langle\cdots\rangle$ with primitive recursive functions 

$lh,*,(n)_k$ as discussed in Remark \ref{remark-coding}):

\begin{prop}\label{p:allx}

Consider a monotone formula $\phi$ as above. Suppose there are $B$ and $L'$ s.t.

\[ \exists i\leq B(\tup a)\ \forall x\ \phi_0(x,c'_i,\tup a),\] where this 

time $L'$ can access all reasonable information, i.e.

\begin{align*}

c'_0&:=0,\\

c'_{i+1}&:=

\begin{cases}

L'(\langle x_0,\ldots,x_i\rangle, \langle c'_0,\ldots,c'_i\rangle,\tup a), &\text{for those $x_j,c'_j$, $j\leq i$ with }\\

 &\neg\phi_0(x_j,c'_j,\tup a)\ \wedge\ \forall y<x_j\ \phi_0(y,c'_j,\tup a)\\

  &\text{ if each exists},\\

c'_i, &\Telse.

\end{cases}

\end{align*}

Then $\phi$ is $(B,L)$-learnable in the original sense (as defined in Definition~\ref{d:fmcMon}), where $L$ is primitive recursively definable in $B,L'$ and 

the characteristic function of $\varphi_0$ (and so, in particular as 

$\xi(B,L')$ for a closed term $\xi$ of the system at hand).

\end{prop}

\begin{proof}

W.l.o.g we can assume that there is a $j\leq B(\tup a)$ s.t. $\forall i<j\ 

(c'_{i+1}>c'_i)$ and $\forall i\geq j\ (c'_{i+1}=c'_i)$ (we can actually primitive recursively define a learner which satisfies this property whenever we have $B$ and $L'$ as above). We set

\begin{align*}

L(x, \tup a):=L(\langle\rangle,\langle\rangle,x, \tup a)&:=\begin{cases}

0,&\Tif \phi_0(x,0,\tup a),\\

L( \langle X(x,0,\tup a)\rangle,\langle0\rangle,x,\tup a),&\Telse,\end{cases}\\ L(\langle \underbrace{x_0,\ldots,x_i}_{\tup x}\rangle, \langle \underbrace{c_0,\ldots,c_i}_{\tup c}\rangle,x,\tup a)&:=\begin{cases}

c_i,& \hspace*{-10mm}\Tif \bigvee_{j\leq i}\phi_0(x_j,c_j,\tup a)\vee i\ge B(\underline{a}),\\

l':=L'(\langle {\tup x}\rangle, \langle {\tup c}\rangle,\tup a),&

\hspace*{-10mm}\Tif x=x_i\vee \phi_0(x,l',\tup a),\\

&\hspace*{-10mm}\phantom{\Tif} \bigwedge_{j\leq i}\neg\phi_0(x_j,c_j,\tup a)\wedge i< 

B(\underline{a}),\\

L( \langle {\tup x},X(x,l',\tup a)\rangle,

\langle \tup c,l'\rangle,x,\tup a) &\Telse,

\end{cases}\\

\end{align*}

where $X(x,c,\tup a):=\min\{x'\leq x\ :\ \neg \phi_0(x',c,\tup a)\}$ (or $x$ if there is no such $x'$). \\

{\em We show by induction on $i$ that $\forall i\ (c'_i = c_i)$.} This is obvious for $i=0$, moreover 

if $\forall x\ \phi_0(x,c'_{i},\tup a)$ then also $\forall x\ \phi_0(x,c_{i},\tup a)$ and so $c'_{(\cdot)}=c_{(\cdot)}$ (both) by the induction hypothesis 

$\forall j\le i (c'_j=c_j).$\\ Otherwise, we have the smallest counterexample $x'_i$ to $c'_i$, and since by our hypothesis $c'_i=c_i$ we have also $x_i=x'_i$ for the smallest counterexample to $c_i$ (note that, by the 

$(B,L')$-learnability of $\varphi,$ we have $i<B(\underline{a})$). 

So, by the monotonicity of $\phi,$ we obtain for all $j<i$ that $x'_j\leq x_i$, so $X(x_i,c'_j,\tup a)=x'_j$ and by definition of $L$ we get in total that 

\begin{align*}

c_{i+1}=L(x_i,\tup a)&=L(\langle X(x_i,0,\tup a)\rangle,\langle0\rangle,x_i,\tup a)\\ 

&=L(\langle x'_0\rangle,\langle0\rangle,x_i,\tup a)\\

&=L(\langle x'_0,  X(x_i,   L'(\langle x'_0\rangle,\langle0\rangle,\tup a)    ,\tup a)\rangle,\langle0, 

     L'(\langle x'_0\rangle,\langle0\rangle,\tup a)  \rangle,x_i,\tup a)\\

&=L(\langle x'_0,  X(x_i,   c'_1    ,\tup a)\rangle,\langle0, 

     c'_1  \rangle,x_i,\tup a)\\

&=L(\langle x'_0,  x'_1\rangle,\langle0, 

     c'_1  \rangle,x_i,\tup a)\\

&=\ldots\\

&=L'(\langle\tup x'\rangle,\langle\tup c'\rangle, \tup a)=c'_{i+1}.

\end{align*}





\end{proof}



So from now on we will simply use $L$ in the form which suits us best.

\\[2mm] 

Speaking of a Cauchy sequence $a_{(\cdot)}$, we would say that it has an effectively learnable rate of convergence, if there is a recursive 

computation for a bound $b$ from the sequence $a_{(\cdot)}$ (resp. the 

parameters used in defining $a_{(\cdot)}$) and an  

$\epsilon >0$, such that there is a procedure to learn 

an $\epsilon$-Cauchy point with at most $b$ computable corrections (computable in a counterexample $x$, which in turn may not be computable itself!).

\begin{rmk}\label{r:smallestCE}

In Definition~\ref{d:fmcMon}, even the condition that $x$ is the smallest counterexample, i.e. $\forall y<x\ \phi_0(y,c_i,\tup a)$, 

is not really necessary. Of course, in such

case, for a given learner, the sequence $c_{(\cdot)}$ is not unique and we need to specify what actually is bounded by $B$. Fortunately, there are only two natural options.

 Either we say that $B$ is a bound on any sequence $c_{(\cdot)}$ (i.e. $B$ is independent on

the choice of the counterexamples) or we say that $B$ is a bound for some sequence $c_{(\cdot)}$ (i.e. for at least one suitable choice of counterexamples). 

It seems rather obvious that the second option makes little sense, since if there was any bound and learner at all, 

then $B(\tup a)=1$ would be a correct bound for the same learner as well (simply by choosing the right counterexample as the first input).

Moreover, a definition in the new sense that $B$ is a bound on any sequence $c_{(\cdot)}$ (any choice of $x_{(\cdot)}$) would be equivalent to Definition~\ref{d:fmcMon}.

\begin{itemize} 

\item Any given bound $B$ for all such sequences is obviously, in particular, a bound for the one we used in the original definition. 

Therefore any formula $(B,L)$-learnable in the new sense is, in particular, $(B,L)$-learnable in the old sense.

\item On the other hand, given $B$ and $L$ satisfying our original definition, we could modify $L$ to $L'$ in such a way, that it actually looks for the smallest counterexample and uses that for its computation, assuring that we in fact generate the same sequence $c_{(\cdot)}$ after all (e.g. set 

$L'(x,c,\tup a):=L(\min \{x'\leq x\ :\  \neg\phi_0(x',c,\tup a)\ \wedge\ \forall y<x'\ \phi_0(y,c,\tup a)\},c,\tup a)$ if such an $x'$ exists and 

$L'(x,c,\tup a):=L(x,c,\tup a)$ otherwise). In other words, any formula $(B,L)$-learnable in the old sense, is $(B,L')$-learnable in the new sense.

\end{itemize}

\end{rmk}



As far as monotone formulas are concerned, we have yet another nice property.

\begin{prop}\label{p:majBL}

A monotone $\Sigma^0_2$-formula $\phi$ (see also Definition~\ref{d:fmcMon}) that is $(B,L)$-learnable (uniformly in the parameters $\underline{a})$ 

is also $(B^*,L^*)$-learnable (uniformly in majorants $\tup a^*$ of 

$\tup a$) for any majorants $B^*$,$L^*$ of $B,L,$ i.e. 

(in ${\cal S}^{\omega,X}$)  

\[ \forall 

\underline{a}^*,\underline{a} \big( 

\underline{a}^* \ \maj\ \underline{a}\to 

\exists i\leq B^*(\tup a^*)\ \forall x^0\ \phi_0(x,c^*_i,\tup a)\big),\] where

\begin{align*}

c^*_0&:=0,\\

c^*_{i+1}&:=

\begin{cases}

L^*(x, \tup a^*),&\text{for the $x$ with } \neg\phi_0(x,c^*_i,\tup a)\ \wedge\ \forall y<x\ \phi_0(y,c^*_i,\tup a) \text{ if it exists}\\

c^*_i,&\Telse.

\end{cases}

\end{align*}

\end{prop}

{\bf Proof.} Note that $B^*,L^*,\tup a^* \ maj \ B,L,\tup a$ implies that 

\[ B^*(\underline{a}^*) \ge B(\underline{a}) \,\wedge\,\forall x^0,y^0 

(x\ge y\to L^*(x,\underline{a}^*)\ge L(y,\underline{a}))).\] 

Now assume that $c^*_i\ge c_i.$ Then by the monotonicity of $\varphi$ 

we have for all $x$ 

\[ \neg \varphi_0(x,c^*_i,\underline{a})\rightarrow \neg \varphi_0

(x,c_i,\underline{a}) \] 

and so the smallest counterexample $x_i$ to $c_i$ is smaller (or equal) 

than the smallest counterexample $x^*_i$ to $c^*_i$ and so 

$c^*_{i+1}=L^*(x^*_i,\underline{a}^*) \ge L(x_i,\underline{a})=c_{i+1}.$ 

Inductively, we get $c^*_i\ge c_i$ for all $i\le B(\underline{a}).$ 

\hfill $\Box$  





\begin{rmk} If the parameters $\underline{a}$ have all types of degree 

$\le 1$, then $\varphi$ in the above proposition is learnable in $B^*,L^*$ 

uniformly in $\underline{a}$ since $a^M \ maj \ a^1,$ where 

$a^M(n):=\max\{ a(i):i\le n.\}.$

\\ In this sense, we can extend the term effectively learnable as follows. A monotone $\Sigma^0_2$-formula $\phi(\tup a)$ is 

{\em effectively learnable with finitely many mind changes uniformly in 

majorants $\underline{a}^*$ of the parameters $\underline{a}$} if it is 

$(B^*,L^*)$-learnable (uniformly in majorants $\underline{a}^*$ of the 

parameters $\underline{a}$ by computable functionals $B^*,L^*$ and all elements of $\tup a^*$ are of type level at most one). 

Note that this means that in the System $\ha[X,\|\cdot\|]$, $\tup a$ could include parameters of types like $X,\NN\to X,X\to X$.

\end{rmk}







There are several ways to generalize our learnability definition. 

Of course one can drop the monotonicity condition, but we can also allow higher or abstract types for $n$ and $x$ 

(for $x$ we would need to consider all sequences $c_{(\cdot)}$ since we can provide only a counterexample $x$, 

not the smallest counterexample $x$ -- see also Remark~\ref{r:smallestCE}). 

The question here is, what kind of information we do allow the learning function(al) $L$ to access. At the moment, it seems 

that there is not such a nice and definitive answer as in the monotone case. However, we will stick with a (not necessarily unique) definition (see Definition~\ref{d:fmcNum}), which

\begin{enumerate} 

\item generalizes Definition~\ref{d:fmcMon}, i.e. if a monotone formula (assuming the bound variables to be of type $0$) is learnable according to our new definition, it is also learnable in the sense of 

Definition~\ref{d:fmcMon} and vice-versa,

\item while keeping the arguments of the learner $L$ simple, still is equivalent to the case where the learner has access to the full finitary information in the sense of Proposition~\ref{p:allx} (see Remark~\ref{r:allx})

\item fits very nicely into the hierarchy of different concepts for computational information (see Proposition~\ref{p:bg2meta}),

\item makes effective learnability guaranteed by very clear logical conditions on the provability of the learned formula (see Theorem~\ref{t:bdLem}).

\end{enumerate} 

The second point seems a very natural requirement and is the cause for 

the main difference to the monotone case, which is that in Definition~\ref{d:fmcNum} the learning process may depend on a whole tuple of all counterexamples used so far, rather than only on the 

last one (last in the sense of number of guesses/candidates, not the index of the counterexample). 

%As a consequence, it might be that in a general scenario a  $(B,L)$-learnable formula is not $(B^*,L^*)$-learnable and it gets very confusing to even formulate an analogous

%version of Remark~\ref{r:smallestCE}.

\\[1mm] 

Although we do not treat the case of learnability for higher type objects 

in this paper, the following definition easily applies to this case as well 

and, therefore, is written in this generality:

\begin{dfn}[(B,L)-learnability for general (not necessarily monotone) 

formulas]\label{d:fmcNum}

Consider an $\exists\forall$ formula $\phi$ with the only parameters $\tup a^{\tup \sigma}$, i.e.

\[\phi\ \equiv\ \exists n^\rho \forall x^0\ \phi_0(x,n,\tup a).\]

We call such a formula $\phi$ 

{\em $(B,L)$-learnable},

if there are function(al)s $B$ and $L$ such that the following holds:

\[ \exists i\leq B(\tup a)\ \forall x\ \phi_0(x,c_i,\tup a),\] where

\begin{align*}

c_0&:=0^{\rho},\\

c_{i+1}&:=

\begin{cases}

L(\langle x_0,\ldots,x_i\rangle, \tup a), &\text{for those $x_j$, $j\leq i$ with }\\

 &\neg\phi_0(x_j,c_j,\tup a)\ \wedge\ \forall y<x_j\ \phi_0(y,c_j,\tup a)\\

  &\text{ if each exists},\\

c_i, &\Telse.

\end{cases}

\end{align*}

We call such a $\phi$ {\em effectively learnable}, if it is $(B,L)$-learnable,

$\sigma_i$ and $\rho$ have type level at most one, and $B$ and $L$ are computable.

\end{dfn}

\newcommand{\seq}{_{(\cdot)}}

\rmk{\label{r:allx}Again, this definition already captures (so to say in a primitive recursive way) the case where the learner could access the previous values of $c\seq$ as well. Simply consider

\begin{align*}

&L(\overbrace{\langle x_0,\ldots,x_i}^{\tup x:=}\rangle, \tup a):=\\&\quad \quad

L'\big(\langle \tup x\rangle, \big\langle

 \underbrace{L'(\langle x_0\rangle, \langle 0\rangle,\tup a)}_{c'_1:=},

 \underbrace{L'(\langle x_0,x_1\rangle, \langle 0,c'_1\rangle,\tup a)}_{c'_2:=}, 

 \ldots,

 L'(\langle \tup x\rangle, \langle 0,c'_1,\ldots,c'_i\rangle,\tup a)

 \big\rangle, \tup a\big). 

\end{align*}

Of course, one could consider weaker concepts, like a learner which can access only the last counterexample (as in the monotone case). We considered also a learner of the kind $L'(x,c,\tup a)$ (i.e. a learner who is allowed to use in addition only the lastly learned solution candidate), which doesn't seem to be equivalent to any of the other two concepts.\\[2mm]

Let us make the properties of our learnability definition discussed above 

more transparent by proving the following results which we first briefly 

motivate: as mentioned already in the introduction (and proved further 

below in section 3), any classical proof (in a suitable formal system) 

of a Cauchy statement $\varphi(k):=\exists n\forall i,j\ge n 

 \,(d(x_i,x_j)<2^{-k})$ can be reformulated to use classical logic 

only up to $\forall l^0 (\Sigma^0_1$-LEM$(s(l,k)))$ (for some closed term 

$s$), i.e. 

\[ (a) \ 

\forall k^0 \ \big( \forall l^0\,(\Sigma^0_1\mbox{-LEM}(s(l,k)))\rightarrow 

\varphi(k)) \] 

follows intuitionistically. This, in particular, applies to the example 

from section \ref{section4.2} of a computable Cauchy sequence in $\RR$ which 

does not possess any computable learnability bounds $B,L.$ So in order to be 

able to extract such computable data $B,L,$ the Cauchy proof has to be further 

restricted, namely, to the situation where the proof implicitly contains a 

bound $t(k)$ on $\forall l^0,$ i.e. on the instances of 

$\Sigma^0_1$-LEM$(s(l,k))$ used. This is guaranteed (as we will see) when 

$(a)$ is strengthened to the (only noneffectively equivalent) form 



\[ (b) \ \forall k^0 \,\exists l^0\, 

\big( \forall m\le_0 l \,(\Sigma^0_1\mbox{-LEM}(s(m,k)))

\rightarrow 

\varphi(k)). \]  Then from a (semi-intuitionistic) proof of $(b)$ 

one can extract a term $B(k)$ computing $l$ and two further terms which 

allow one to build a learning procedure $L$ so that $\varphi$ is 

$(B,L)$-learnable. 

 

In the following IP$^{\omega}_{\forall}$ denotes 

the independence-of-premise principle for universal 

premises in all finite types:

\[\mbox{IP}^{\omega}_{\forall}: \ \ (\forall \underline{x}\,A_0(\underline{x})

\to\exists y \,

B(y))\rightarrow \exists y\,( \forall \underline{x}\,

A_0(\underline{x})\to B(y)), \]

where $A_0$ is a quantifier-free formula and $\underline{x},y$ are 

variables of 

arbitrary types.

\begin{thm}\label{t:bdLem}

Given that

\[ \ba{l} 

\ha[X,\|\cdot\|]+\, \mbox{\rm AC$+$M$^{\omega}+$IP}^{\omega}_{\forall} 

\vdash\\[1mm] \hspace*{2cm} \forall \underline{a}\,\exists l^0\ \big( 

\forall m\leq_0 l \exists u^0 \forall v^0\ 

\big(\psi_0(u,m,\tup a) \vee \neg\psi_0(v,m,\tup a)\big)\rightarrow \exists 

n^0 \forall x^0 \phi_0(x,n,\tup a)\big),\ea 

\]

where $\phi_0,\psi_0$ are quantifier-free formulas (containing at most the 

parameters $\underline{a}$ free), 

then $\exists n\forall x \phi_0(x,n,\tup a)$ 

is (valid in ${\cal S}^{\omega,X}$) $(B,L)$-learnable (in the sense 

of Definition \ref{d:fmcNum} and, for monotone formulas, in the 

sense of Definition \ref{d:fmcMon}) by functionals given by 

closed terms of $\ha[X,\|\cdot\|].$ \\

To $B,L$ one can construct majorants 

$B^*,L^*$ given by closed terms of $\ha$ such that 

if $\exists n\forall x \phi_0(x,n,\tup a)$ is monotone (as in 

Definition~\ref{d:fmcMon}), then it is even learnable in $B^*,L^*$ uniformly in majorants $\underline{a}^*$ of the parameters $\underline{a}.$

\end{thm}



\begin{proof}

Suppose that 

\[ \ba{l} 

\ha[X,\|\cdot\|]+\mbox{AC$+$M$^{\omega}+$IP}^{\omega}_{\forall} 

\vdash \\[1mm] \hspace*{2cm} 

\forall \underline{a}\exists l^0\ \big( 

\forall m\leq l \exists u \forall v\ \big(\psi_0(u,m,\tup a) \vee \neg\psi_0(v,m,\tup a)\big)\rightarrow \exists n \forall x \phi_0(x,n,\tup a)\big).

\ea \]

Then by the soundness of the G\"odel functional (`Dialectica') 

interpretation for $\ha[X,\|\cdot\|]+\mbox{AC$+$M$^{\omega}+$IP}^{\omega}_{\forall}$ (see \cite{Kohlenbach(book)}) we obtain that (note that since we do not need 

bar recursion to interpret $\ha[X,\|\cdot\|]$ we do not have to go through 

the model of strongly majorizable functionals and so do not need to assume 

any smallness condition on the types of $\underline{a}$ to pass to 

${\cal S}^{\omega,X}$)   

\[ {\cal S}^{\omega,X} \models \exists l,V,N \forall U,x\ \Big(\forall m\leq l 

\big (\psi_0(Um,m,\tup a) \vee \neg\psi_0(VxU,m,\tup a)\big)

\rightarrow \phi_0(x,N(U),\tup a)\Big).

\]

where `$\exists l,V,N$' is witnessed (uniformly in $\underline{a}$) 

by closed terms $t,s_V,s_N$ of 

$\ha[X,\|\cdot\|]$. 

The result when the terms $s_V,s_N$ are applied to 

$\tup a$, we conveniently name $V$ and $N$.\\

To show the learnability, let $U_{\tup v}$ (where $\tup v$ is a $t(\tup a)$-tuple) denote the function \[

U_{\tup v}(i):=\begin{cases}

v_i&\Tif\ i<t(\tup a),\\

0&\Telse,

\end{cases}\] set $B(\tup a):=t(\tup a)$ and define $L$ in $N$ and $V$ via a sequence of $t(\tup a)$-tuples $\tup v^{(\cdot)}$.

More precisely to compute $L(\langle \underbrace {x_0,x_1,\ldots,x_i}_{\tup x:=}\rangle,\tup a)$ for some $i$ we need to

define the tuples $\tup v^{(0)},\ldots,\tup v^{(i)}$ as follows.

\begin{enumerate}

\item[$\tup v^0$] 

Set $\tup v^0:=0,\ldots,0$ and $c_1:=L(\langle x_0\rangle,\tup a):=

N(U_{\tup v^0})$.

\item[$\tup v^1$] If $\forall x \phi_0(x,c_1,\tup a)$ holds, then there is nothing to be 

done\footnote{

Of course this is undecidable, however the conclusion discussed next is. In this sense if the conclusion is wrong for the $x_1$ given as input to $L$, 

we can simply set $\tup v^1=\tup v^0$ and $L(\langle x_0,x_1\rangle,\tup a):=c_1$ (or even $0$ for all that it matters).

}. 

Otherwise, we have in particular (provided that $x_1$ is the minimal 

counterexample) 

\[

\exists m\leq t(\tup a) \big (\neg\psi_0(U_{\tup v^0}m,m,\tup a) \wedge \psi_0(Vx_1(U_{\tup v^0}),m,\tup a)\big)

\]

and 

so we can denote the least such an $m$ by $m_0$ (put $m_0:=0$ in case such an 

$m$ does not exist) and define $\tup v^1$ as $\tup v^0$ except that we set $v^1_{m_0}:=Vx_1(U_{\tup v^0})$.

Furthermore, we set 

$c_2=L(\langle x_0,x_1\rangle,\tup a):=N(U_{\tup v^1})$. 

Note that we have \[ \neg\psi_0(U_{\tup v^0}m_0,m_0,\tup a) \wedge \psi_0(v^1_{m_0},m_0,\tup a). \tag{v0}\label{e:v0}\]

\item[$\tup v^2$] Now, if $\forall x \phi_0(x,c_2,\tup a)$ then we are finished. Otherwise, similarly as before we have 

\[

\exists m\leq t(\tup a) \big (\neg\psi_0(U_{\tup v^1}m,m,\tup a) \wedge \psi_0(Vx_2(U_{\tup v^1}),m,\tup a)\big)

\]

and we can denote the least such an $m$ by $m_1$ and define $\tup v^2$ as $\tup v^1$ except that we set $v^2_{m_1}:=Vx_2(U_{\tup v^1})$. As before this means that

\[ \neg\psi_0(U_{\tup v^1}m_1,m_1,\tup a) \wedge \psi_0(v^2_{m_1},m_1,\tup a), \tag{v1}\label{e:v1}\]

so in particular we obtain that $m_1\neq m_0$ by~\eqref{e:v0} as $U_{\tup v^1}m_1=v^1_{m_1}$. We set $c_3=L(\langle x_0,x_1,x_2\rangle,\tup a):=

N(U_{\tup v^2})$ and continue.

\item[$\tup v^3$] Again, if $\forall x \phi_0(x,c_3,\tup a)$ then we are finished. Otherwise, as before, we have that

\[

\exists m\leq t(\tup a) \big (\neg\psi_0(U_{\tup v^2}m,m,\tup a) \wedge \psi_0(Vx_3(U_{\tup v^2}),m,\tup a)\big)

\]

and we can denote the least such an $m$ by $m_2$ and define $\tup v^3$ as $\tup v^2$ except that we set $v^3_{m_2}:=Vx_3(U_{\tup v^2})$. As before this means that

\[ \neg\psi_0(U_{\tup v^2}m_2,m_2,\tup a) \wedge \psi_0(v^3_{m_2},m_2,\tup a), \tag{v2}\label{e:v2}\]

so in particular we obtain that $m_2\neq m_1$ by~\eqref{e:v1}. Moreover, by~\eqref{e:v0} and~\eqref{e:v2} we have $m_2\neq m_0$, since from $m_1\neq m_0$ follows that $v^2_{m_0}=v^1_{m_0}$.

 We set $c_4=L(\langle x_0,x_1,x_2,x_3\rangle,\tup a):=N(U_{\tup v^3})$ and continue.\\

\item[$\tup v^{n+1}$] Finally, in general assume that for some $n$ we have that $\forall i< n\forall j<i\ m_i\neq m_j$

 and $\forall i\leq n+1 \neg \forall x \phi_0(x,c_i,\tup a)$.

Then we have also that

\[ 

\forall i<n \big (\neg\psi_0(U_{\tup v^i}m_i,m_i,\tup a) \wedge \psi_0(Vx_{i+2}(U_{\tup v^{i+1}}),m_i,\tup a)\big). \tag{vi}\label{e:vi}

\]

As usual we have in particular that

\[ 

\exists m\leq t(\tup a) \big (\neg\psi_0(U_{\tup v^n}m,m,\tup a) \wedge \psi_0(Vx_{n+1}(U_{\tup v^n}),m,\tup a)\big)

\]

and we can denote the least such an $m$ by $m_n$ and define $\tup v^{n+1}$ as $\tup v^n$ except that we set $v^{n+1}_{m_n}:=Vx_{n+1}(U_{\tup v^n})$. As before this means that

\[ \neg\psi_0(U_{\tup v^n}m_n,m_n,\tup a) \wedge \psi_0(v^{n+1}_{m_n},m_n,\tup a). \tag{vn}\label{e:vn}\]

From $\forall 0<i<n\ ( m_0\neq m_i )$ it follows that $\forall0<i<n\ ( v^n_{m_0}=v^i_{m_0})$. Assume that $m_n = m_0$, then

\[U_{\tup v^n}m_n=v^n_{m_n}=v^n_{m_0}=v^1_{m_0}\]

and we obtain a contradiction as $\neg\psi_0(v^1_{m_0},m_0,\tup a)$ follows from~\eqref{e:vi} and $\psi_0(v^1_{m_0},m_0,\tup a)$ follows from~\eqref{e:vn}. This shows that 

$m_n \neq m_0$, similarly one shows that \[ \forall i<n\ (m_n \neq m_i).\]

As usual, we set $c_{n+2}=L(\langle x_0,\ldots,x_{n+1}\rangle,\tup a):=

N(U_{\tup v^{n+1}})$.

\end{enumerate}

This leads to the following definition of $L$:

\begin{align*}

 L(x,\tup a):= 

 L(\langle \underbrace {x_0,x_1,\ldots,x_i}_{\tup x:=}\rangle,\tup a):=

N(U_{\tup v^{i}}). 

\end{align*}

Note that since $N$ and $U$ are total, so is $L$. 

Moreover, if the values of $i$, $\tup x$, and $c_i$ satisfy the conditions from Proposition~\ref{p:allx}, then $L$ 

behaves as described above.\\

Finally, since there can be only $t(\tup a)$ many different $m_i$'s, it can happen at most $t(\tup a)$ many 

times that $\forall x \phi_0(x,c_i,\tup a)$ does not hold, where

$c_i$ is defined as in Definition \ref{d:fmcNum} with $L$ as above. Hence 

$\varphi$ is $(B,L)$-learnable in the sense of  Definition \ref{d:fmcNum} 

and hence - for monotone formulas - also $(B,\tilde{L})$-learnable 

for some $\tilde{L}$ primitive recursive in $B,L$ (and $\varphi_0$) 

by Proposition~\ref{p:allx}. 

\\ The 

second claim follows from the fact that $t,s_V,s_v$ have majorants $t^*,

s^*_V,s^*_N$ 

given by closed terms of $\ha$ (see \cite{Kohlenbach(book)}) which then 

yield majorants $B^*,L^*$ of $B,L.$ Now apply  

Proposition \ref{p:majBL}.

\end{proof} 

\begin{rmk} \label{simple-L}

Assume that $\varphi_0$  in the theorem comes from a Cauchy statement 

\[ j_1(x),j_2(x)\ge n\to \widehat{\| 

a_{j_1(x)}-a_{j_2(x)}\|}(k+1)\le_{\QQ} 

2^{-k}, \] where -- referring to the representation of real numbers 

by number theoretic functions $f$ representing fast Cauchy sequences of 

rationals -- $\widehat{f}(k+1)$ is a $2^{-k-1}$-rational approximation to $f$ 

(see \cite{Kohlenbach(book)} for details).  Then 

$\varphi$ is monotone and a counterexample $x$ to $n$ satisfies 

$x\ge n$ (using that for the Cantor pairing function $x\ge j_i(x)$). 

Assume also that $\psi_0$ is 

monotone in $u$ (which always can be arranged by taking 

$\psi'_0(u,m,\underline{a}):\equiv \exists \tilde{u}\le u\,\psi_0

(u,m,\underline{a})$). \\ 

Then the complicated iteration used in defining 

$L$ can be avoided by taking simply 

\[ L^*(\langle x_0,\ldots,x_i\rangle,\underline{a}^*):= 

N^*_{\underline{a}^*}(\lambda k.V^*_{\underline{a}^*}(x_i,

\lambda n.x_i)), \] 

where $\lambda \underline{a}^*.N^*_{\underline{a}^*},\lambda \underline{a}^*.

V^*_{\underline{a}^*}$ are majorants of 

\[ \tilde{N}_{\underline{a}}(f):=\max\left\{ \max\left\{ 

N(v^0*0):lh(v)=t\underline{a}\wedge \forall l\le t\underline{a}(v_l\le f(l))

\right\},f(l):l\le t\underline{a}\right\}\] and 

\[ \tilde{V}_{\underline{a}}(x,f):=\max\left\{ \max\left\{ 

V(x,v^0*0):lh(v)=t\underline{a}\wedge \forall l\le t\underline{a}(v_l\le f(l))

\right\},f(l):l\le t\underline{a}\right\}\] 

with $N,V,t$ as in Theorem \ref{t:bdLem}. Note that with $N,V$ also 

$\tilde{N},\tilde{V}$ satisfy the claim in the proof and that $L^*$ 

(for counterexamples $x_0,\ldots,x_i$) is 

an upper bound for the $L$ defined in terms of $\tilde{N},\tilde{V}$ as 

an elementary calculation shows (using that -- by the form of 

$\varphi_0$ -- a counterexample $x$ to $n$ has to satisfy $x\ge n$). 

By monotonicity, $\varphi$ is then also $(L^*,B^*)$-learnable (uniformly 

in majorants $\underline{a}^*$ of $\underline{a}$) where $B^*$ is some 

majorant of $B.$ 

\end{rmk}  

\begin{rmk} The theorem remains valid if arbitrary ${\cal S}^{\omega,X }$-true 

purely universal sentences are added as axioms to 

{\rm HA$^{\omega}[X,\|\cdot\|]$}. The part about the majorizing terms $B^*,L^*$ 

even remains valid -- using monotone functional interpretation -- 

if one adds sentences of the form $\Delta:\equiv 

\forall a^{\delta}\exists b\le_{\rho} sa\forall c^{\tau} 

F_0(a,b,c)$ with quantifier-free 

$F_0$ and closed $s$ as axioms which covers the case of the binary (`weak') K\"onig's lemma 

WKL (which together with AC even implies K\"onig's lemma KL); see 

\cite{Kohlenbach(book)} for extensive details on all this.

\end{rmk}

Using the representation of real numbers from 

\cite{Kohlenbach(book)}, each sequence of type $0\to 1$ can be viewed 

as a name of a sequence $(a_n)$ of reals. Now define 

$\tilde{a}_n:=\max_{\RR}\big(0,\min\nolimits_{i\le n}a_i\big).$ 

Let PCM$_{ar}(a_n)$ denote the statement that the monotone decreasing 

sequence $(\tilde{a}_n)$ in $[0,1]$ is Cauchy (see \cite{Kohlenbach(book)} 

for details) 

\[\mbox{PCM}_{ar}(a_n): \ \forall k\in\NN\,\underbrace{\exists n\in\NN\, 

\forall m\ge n\ (|\tilde{a}_m- \tilde{a}_n|\le 

2^{-k})}_{\mbox{PCM}_{ar}((a_n),k)

:\equiv} \]

(if $(a_n)$ is already 

a decreasing sequence in $[0,1]$, then $(\tilde{a}_n)=(a_n)).$ 

The usual classical proof of PCM$(a_n)$ uses $\Sigma^0_2$-DNE, but it can be 

converted into a proof that only needs the weaker $\Sigma^0_1$-LEM (see 

Proposition \ref{p:limComp} below). 

In \cite{Toftdal}, an explicit such proof is constructed 

exhibiting a concrete sequence of instances of $\Sigma^0_1$-LEM sufficient 

for this. From this proof one can read off the following even more detailed 

fact:

\begin{prop} There is a primitive recursive functional (in the 

ordinary sense) $\Phi$ such that (using the Cantor pairing function)

\[ \mbox{\rm HA}^{\omega}\vdash \forall a^{0\to 1}_{(\cdot)}, k^0\ 

\big( \forall m \le j(2^k-1,2^k) \,\Sigma^0_1\mbox{\rm -LEM}(\Phi(a_{(\cdot)},m)

\rightarrow 

\ \mbox{\rm PCM}_{ar}(a_{(\cdot)},k)\big). \] 

\end{prop} 

{\bf Proof.} The crucial step in Toftdal's proof in \cite{Toftdal} is 

to show by induction on $k$ that 

\[ \forall k\exists i\in \{ 1,\ldots,2^k\} \exists n \forall m \, 

\left( \frac{i-1}{2^k}\le \tilde{a}_{n+m}\le \frac{i}{2^k}\right), \]

where in the induction step $\Sigma^0_1$-LEM is used in the form 

(note that, based on our representation of real numbers, $<_{\RR}\in\Sigma^0_1$) 

\[\exists n\ \left( \tilde{a}_n <\frac{2i-1}{2^{k+1}}\right) \vee 

\neg \exists n\ \left( \tilde{a}_n <\frac{2i-1}{2^{k+1}}\right). \]

So to establish PCM$_{ar}(a_{(\cdot)},k)$ one needs only the instances 

\[\exists n\ \left( \tilde{a}_n <\frac{i}{2^{l}}\right) \vee 

\neg \exists n\ \left( \tilde{a}_n <\frac{i}{2^{l}}\right) \] 

for $i\le l-1$ and $l\le 2^k,$ i.e. the codes $j(i,l)$ of the instances 

used can be bounded by $t(k):=j(2^k-1,2^k).$  The construction of $\Phi$ is 

clear. 

 \hfill $\Box$ 



\mbox{ } 



While the usual classical proof of PCM$_{ar}$ only needs $\Sigma^0_1$-induction 

(but $\Sigma^0_2$-DNE), the above proof due to Toftdal needs an instance of 

the $\Sigma^0_2$-induction rule ($\Sigma^0_2$-IR) which, apparently, is the 

price to be paid for using only $\Sigma^0_1$-LEM (instead of 

$\Sigma^0_2$-DNE). Classically, $\Sigma^0_2$-IR is quite strong and proves 

(relative to PRA) the same $\Pi^0_3$-sentences as $\Pi^0_2$-IA 

(see e.g. \cite{Sieg}[Theorem 3.11]) and so, 

in particular, the totality of the Ackermann function. In our intuitionistic 

context, however, it is weak and the functional interpretation used 

(without negative translation!) in the 

proof of Theorem~\ref{t:bdLem} (and the corollary below) to extract $B,L$ solves 

$\Sigma^0_2$-IR using only ordinary primitive recursion in the form of 

$R_0.$ \\ One can also modify Toftdal's proof so that only the $\Pi^0_1$-FAC 

principle (`$\Pi^0_1$-finite-axiom-of-choice') is used (which classical 

is equivalent to $\Pi^0_1$-CP which is relative to PRA $\Pi^0_3$-conservative 

of $\Sigma^0_1$-IA): w.l.o.g. we may assume that $(a_n)$ is a sequence 

of rational numbers (for, otherwise, replace it by $r_n:=\min_{i\le n} 

(\widehat{a}_i+2^{-i})$ using the representation of reals from 

\cite{Kohlenbach(book)}): 

by $\Sigma^0_1$-LEM we have 

\[ \forall i\le 2^{k}\exists n\forall m \,\big( a_n<\frac{i}{2^k} \vee 

a_m\ge \frac{i}{2^k}\big). \] 

Hence by $\Pi^0_1$-FAC 

\[ \exists n\forall i\le 2^k \,\big(a_{(n)_i} <\frac{i}{2^k}\vee 

\forall m (a_m\ge \frac{i}{2^k})\big). \] 

Now let $i_0\le 2^k$ be least s.t. $ a_{(n)_{i_0}} <\frac{i_0}{2^k}$ 

(if existent; otherwise $\forall m (a_m=1)$ and we are done). Then for 

$l:=(n)_{i_0}$ 

\[ \forall m\ge l \, (|a_m-a_l|<2^{-k}). \]   

\\[2mm]

As a corollary we obtain that Theorem~\ref{t:bdLem} also holds with 

the original assumption being replaced by 

PCM$_{ar}(s(\underline{a},l),t(\underline{a})),$ where 

$s(\underline{a},l)^{0\to 1}$ represents for each $l\in\NN$ 

some sequence of reals defined by 

a closed term $s$ in $\underline{a}:$

\begin{cor} \label{cor.2.11}

Given that 

\[\ba{l} \mbox{\rm HA$^{\omega}[X,\|\cdot\|]+$AC$+$M$^{\omega}+

$IP}^{\omega}_{\forall}\vdash \\[1mm] \hspace*{2cm}  

\forall \underline{a}\,\exists k^0 ,l^0\ \big( 

\ \forall m \le l\ \mbox{\rm PCM}_{ar}(s(\underline{a},m),k)

 \rightarrow \exists n^0\forall x^0\,\varphi_0(x,n,\underline{a})\big), 

\ea \] 

where $s$ is a closed term and $\varphi_0$ as in Theorem \ref{t:bdLem}, then 

$\exists n^0\forall x\,\varphi_0(x,n,\underline{a})$ 

is (valid in ${\cal S}^{\omega,X}$) 

$(B,L)$-learnable (uniformly in $\underline{a}$) by functionals given by 

closed terms of the system $\ha[X,\|\cdot\|]$.\\

To $B,L$ one can construct majorants 

$B^*,L^*$ given by closed terms of $\ha$ such that 

if $\exists n\forall x \phi_0(x,n,\tup a)$ is monotone (see Definition~\ref{d:fmcMon}) then it is even learnable in $B^*,L^*$ uniformly in

majorants $\underline{a}^*$ of the parameters $\underline{a}$.

\end{cor}   

\begin{remark}

Of course, instead of sequences in $[0,1]$ one can also consider sequences 

in any compact interval $[-C,C],$ where then the functionals $B,L$ will 

additionally depend on $C.$ \\[1mm] Likewise, instead of decreasing 

sequences we may also have increasing ones or, if the Cauchy property 

is changed into the existence of an approximate infimum 

\[ \exists n\,\forall m (a_n\le a_m+2^{-k}), \] 

also arbitrary sequences in $[-C,C].$

\end{remark}

The next proposition shows how to convert any majorants $(B^*,L^*)$ for 

a $(B,L)$-learnable formula into a rate of metastability. This not only 

guarantees a highly uniform (and for computable $(B^*,L^*)$) computable 

rate of metastability but, moreover, such a rate which has a particularly 

simple form (see the remark and discussion after the proposition):

\footnote{Note that in our examples, $\underline{a}$ will be data related to 

an abstract normed of Hilbert space for which (in contrast to 

$\underline{a}^*$) computability is not even defined.} 

\begin{prop}\label{p:bg2meta}

Let $\exists m^0\forall k^0\,\varphi_0(n,m,k,\underline{a})$ be a formula 

in the language of $\ha$ (or $\ha[X,\|\cdot\|]$) with $\varphi_0$ being 

quantifier-free that 

is $(B,L)$-learnable in the sense of Definition \ref{d:fmcNum}  

(uniformly in $n$ and  $\underline{a}$) and let $B^*,L^*$ be majorants of  

$B,L$. \\ 

Then a rate of metastability $\Omega$ 

(valid in ${\cal S}^{\omega,X}$) 

\[ \forall n^0\,\forall g^1\, \exists m\le_0\Omega(g,n) \ 

\phi_0(n,m,g(m),\underline{a})\tag{metastable} \label{(metastable)}\]

for\footnote{Note that in order to talk about metastability, we need  one of the parameters to have type 0 and we treat it separately.}  

\[

\forall n^0 \exists m^0 \forall k^0\ \phi_0(n,m,k,\underline{a})\tag{$\phi$}\label{e:phi}

\]

is given by $\Omega(B^*,L^*,\underline{a}^*)$ 

(uniformly in majorants $\underline{a}^*$ of 

the parameters $\underline{a}$), where 

$\tilde g(c):=\max\big(c,\max_{c'\leq c}(g(c'))\big)$ and

\begin{align*} 

  \Omega&:=\lambda B^*,L^*,\underline{a}^*,g,n\ .\ C(L^*,g,n,

B^*(n,\underline{a}^*),\underline{a}^*),\\

C(i):=C(L^*,g,n,i,\underline{a}^*)&:=

\begin{cases}0,&\Tif\ i=0,\\ L^+(\langle\overbrace{ \tilde g(C(i-1)+1),\ldots,\tilde g(C(i-1)+1) }^{i\times}\rangle,n,\underline{a}^*),&\Telse,\end{cases}

\end{align*}

with $L^+(x):=\max\{ L^*(x),x\}.$ \\ 

Note that $\Omega$ is defined using only recursion $R_0$ of type $0$ and 

hence is primitive recursive in the usual sense of Kleene.

\end{prop}

\begin{proof} We reason in ${\cal S}^{\omega,X}.$

Since $\phi_0$ is quantifier-free, there is a closed term $f$ with 

$f(n,m,k,\underline{a})=0\leftrightarrow \phi_0(n,m,k,\underline{a})$. 

Hence, we have that (for $\underline{a}^*$ majorizing $\underline{a}$) and 

for the succession $c_i$ from Definition \ref{d:fmcNum} 

\[ \forall n\ \exists i\leq B^*(n,\underline{a}^*)\ \forall k\ \ \big( 

f(n,c_i,k,\underline{a})=0 \big), \]

by the assumptions of the proposition, and we need to show that

\[ \forall g,n\ \exists m\leq \Omega(B^*,L^*,\underline{a}^*,g,n) 

\ \ \big( f(n,m,g(m),\underline{a})=0 \big). \]

Now, fix any $g$, $n$ and assume towards contradiction that 

$\Omega(B^*,L^*,\underline{a}^*,g,n)$ is not a rate of metastability, i.e. that

\be[e:NE] 

\forall m\leq \Omega(B^*,L^*,\underline{a}^*,g,n)  \ \ \big( f(n,m,g(m),\underline{a})\neq0 \big).

\ee

By induction on $i$ we obtain that

\be[e:IC] 

\forall i\leq B^*(n,\underline{a}^*) \,\big(c_i\leq 

C(L^*,g,n,i,\underline{a}^*)

\big).

\ee

The case $i=0$ is trivial as $c_0=C(L^*,g,n,0,\underline{a}^*)=0$. 

Next, suppose that for some 

$1\leq i\leq B^*(n,\underline{a}^*)$ the following holds

\be[e:IH] 

\forall j<i\ \big(c_j\leq C(L^*,g,n,j,\underline{a}^*)\big). 

\ee

Denoting the smallest $k$ s.t. $f(n,m,k,\underline{a})\neq0$ by $x_m$ 

(if this does not exist, we have $c_i=c_{i-1}$ and are done), 

we obtain by~\eqref{e:IH} that\footnote{Here

we simply assume that our encoding is monotone in its components. If for some reason it was not, we could use a $L'$ which returns

the maximal value among all codes coordinatewise bounded by the elements of the encoded input of $L^*$.}

\[

   c_i\leq L^*(\langle x_{c_0},\ldots,x_{c_{i-1}}\rangle, n,\underline{a}^*)

   \leq L^*(\langle \tilde g({c_0}),\ldots,\tilde g({c_{i-1}})\rangle, n,\underline{a}^*)\le C(L^*,g,n,i,\underline{a}^*),

\]

since by~\eqref{e:NE} we have (using that $C(i)$ is nondecreasing in $i$) 

that \[ \forall i\leq B^*(n,\underline{a}^*)\ 

\big(m\leq C(L^*,g,n,i,\underline{a}^*)\rightarrow f(n,m,g(m),\underline{a})\neq0 \big)\] and 

so, in particular, that

\[ \forall i\leq B^*(n,\underline{a}^*)\ \big(m\leq C(L^*,g,n,i,\underline{a}^*)

\rightarrow x_m\leq g(m)\leq \tilde g(m) \big). \]

Finally, we can infer from~\eqref{e:IC} that

\[ \forall i\leq B^*(n,\underline{a}^*)\ 

\big(c_i\leq \Omega(B^*,L^*,\underline{a}^*,g,n)\big),\]

and therefore and by~\eqref{e:NE} also

\[ \forall i\leq B^*(n,\underline{a}^*)\ \neg\forall k^0\big( 

f(n,c_i,k,\underline{a})=0\big) \]

which is a contradiction.



\end{proof}



\begin{rmk}\label{r:metastr}

Note that $\Omega$ has essentially the following form\footnote{Note that 

the additional dependency on the number $i$ of iterates in 

Proposition \ref{p:bg2meta} via the length of the sequence 

$\langle \ldots\rangle$ can also 

be covered by 

this normal form, since -- by $\tilde{g}(C(i-1)+1)\ge\tilde{g}(i)\ge i$ -- 

the length of the sequence $\langle\ldots\rangle$  can be majorized by 

$\tilde{g}(C(i-1)+1)$ itself.}

\begin{align*}

&(L_{n,\tup a^*}\circ \tilde{g})^{B^*(n,\tup a^*)}(0),\\

&L_{n,\tup a^*}:=\lambda x\ .\ L^*(x,n,\tup a^*).

\end{align*}

Moreover, if we have such a rate of metastability for some Cauchy statement 

$\varphi$ as considered in Remark \ref{simple-L} so that 

$\varphi$ is monotone and a counterexample $x$ is always greater than the 

witness candidate, and given any $n,\tup a^*$ we have an 

$f^1$ and a $b^0$ such that for all $\underline{a}$ 

that are majorized by $\underline{a}^*$ 

\be[e:nfbA]

\forall g \exists m \leq (f\circ \tilde{g})^b(0) \phi_0(n,m,g(m),\tup a),

\ee

then $\phi$ is $B,L$-learnable (uniformly in $n$ and majorants 

$\underline{a}^*$ for $\underline{a}$) with

\[

B(n,\tup a^*):=b,\quad L(x,n,\tup a^*):=f(x).

\]

To prove this fact, we argue as follows. Fix arbitrary $n,\tup a^*$ and

consider corresponding $f$ and $b$. Let

\[

g(m)=\min\{x\ : \neg\phi_0(n,m,x,\tup a) \},

\]

if such an $x$ exists and $0$ otherwise. Note that due to the monotonicity

of $\phi$ we have $g(m)\neq0\rightarrow \tilde g(m)=g(m)$. This implies that

as long as there is a (the smallest) counterexample $x_i$ to $c_i$, it holds that

\be[e:nfb0]

c_{i+1}=L(x_i,n,\tup a^*)=f(x_i)

=(f\circ \tilde g)c_i=(f\circ \tilde g)^{i+1}(0).

\ee

Given all this, assume towards contradiction that

\be[e:nfb1]

\forall i\leq B(n,\tup a^*) \exists x \neg\phi_0(n,c_i,x,\tup a),

\ee

and consider any $m\leq (f\circ \tilde{g})^b(0)$. Due to~\eqref{e:nfb0} and~\eqref{e:nfb1},

we get that $m\leq c_b$ and due to the monotonicity of $\phi$ this means that

there is a counterexample to $m$ (since there is one for $c_b$ by~\eqref{e:nfb1}, as $B(n,\tup a^*)=b$), which 

means that $\neg\phi_0(n,m,g(m),\tup a)$ by definition of $g$. This is a contradiction

to~\eqref{e:nfbA}.

\end{rmk}



{\bf Discussion:} What the main results in this section (Theorem 

\ref{t:bdLem} and Proposition \ref{p:bg2meta}), taken together, show 

is that if the proof of a (monotone) $\Pi^0_3$-statement (e.g. a 

Cauchy statement) uses only a bounded (in the 

parameters) number of unnested 

$\Sigma^0_1$-LEM$^-$-instances but may use induction 

of unrestricted complexity, then we get a rate of metastability which 

-- as a functional in the counterfunction $g$ -- has a very simple structure 

(namely only a single use of iteration of g). This is remarkable as e.g. 

HA$^{\omega}$ does, of course, 

prove $\forall g \exists x\,\psi_0(g,x)$-sentences 

which need 

much more complicated functionals in $g$ (namely every type-2 functional

definable in G\"odel's calculus $T$ arrives in this way). What we have shown, 

however, is that this cannot happen if $\forall g \exists x\,\psi_0(g,x)$

results as the Herbrand normal form of a $\Pi^0_3$-statement 

$\varphi\equiv \forall n^0\exists m^0\forall k^0\,\varphi_0(n,m,k)$ 

(with $g$ being the function variable playing the role of the Herbrand 

index function) that is provable 

in HA$^{\omega}$ from the aforementioned restricted uses of 

$\Sigma^0_1$-LEM$^-.$ To see the difference, 

let us consider the simple but already illuminating 

case where the proof of $\varphi$ does not use classical logic 

at all. Then one can use modified realizability or functional interpretation 

to extract a (definable in $T$) witnessing term $t$ for $\varphi$ 

(and hence a bound $t^*$ which is uniform in majorants of the parameters) 

which then a fortiori is also a rate of metastability for $\varphi$ 

which does not use the argument $g$ at all.

The `$g$-involvement' displayed by a rate of metastability reflects the 

amount of $\Sigma^0_1$-LEM$^-$ used in the proof and the former 

is simple if the latter is low. Moreover, the extraction of the rate of 

metastability via the extraction of the learning procedure $L^*$ proceeds 

without any use of negative translation but with direct functional 

interpretation (while modified realizability would not be sufficient as 

we need the functional witnessing $V$ in the proof of Theorem 

\ref{t:bdLem}).

\\[2mm] 

A sort of complementary scenario would be to allow full classical logic 

in a proof but to restrict the use of induction to a bounded number of 

unnested instances of $\Sigma^0_1$-IA, the latter being used e.g. in 

the form of PCM$_{ar}.$ Let {\rm G$_3$A$^{\omega}$} be 

the finite type extension of Kalmar-elementary 

arithmetic (based on quantifier-free induction only but with classical logic) 

from \cite{Kohlenbach(lowrate)} (see also \cite{Kohlenbach(book)}).

Now consider a proof 

\[\mbox{\rm G$_3$A}^{\omega}[X,\|\cdot\|]\,

\vdash   

\forall \underline{a}\, \big( 

\ \mbox{\rm PCM}_{ar}(t_1(\underline{a}),

t_2(\underline{a}))

 \rightarrow \exists n^0\forall x^0\,\varphi_0(x,n,\underline{a})\big).  \] 

Then by negative translation and functional interpretation one can 

extract a rate of metastability for the conclusion 

making a single use of the rate of 

metastability of PCM$_{ar}$ given by a single application of the iteration 

$\tilde{g}^{k}(0)$ (see \cite{Kohlenbach(book)} prop.2.26) 

and terms $t[g]$ of G$_3$A$^{\omega}$ which can be 

majorized by terms using only a fixed number of $g$-nestings (reasoning 

as in the proof of Proposition \ref{p:nonLearnablePhi} below). 

This again leads to a rate of 

metastability which can be put into the form in Remark \ref{r:metastr}.\\[1mm] 

So to get a more complicated rate of metastability (e.g. of the form 

$\tilde{g}^{(\tilde{g}^x(0))}(0)$) requires a nested use of a {\bf combination} of 

$\Sigma^0_1$-LEM$^-$ and (at least) 

$\Sigma^0_1$-IA$^-$ as provided in our example 

of a sentence $\varphi$ in Proposition 

\ref{p:nonLearnablePhi} that is not effectively learnable (where 

$\Sigma^0_1$-LEM$^-$ together with $\Pi^0_1$-CP$^-$ is used).

 



\section{Cauchy statements and unrestricted use of $\Sigma^0_1$-LEM}

In the following, we refer to Friedman's so-called $A$-translation from 

\cite{Friedman(78)} (see e.g. \cite{Kohlenbach(book)}). Since we 

work in the context of weakly extensional systems and the quantifier-free 

rule of extensionality QF-ER is not sound under the $A$-translation we simply 

add for the reminder of this section 

all ${\cal S}^{\omega}$-true (resp. ${\cal S}^{\omega,X}$-true) purely 

universal sentences ${\cal P}$ in the language of the respective system 

as axioms (making the use of QF-ER in proofs superfluous as it only proves 

universal consequences). This, anyhow, is a common device in proof mining 

as universal axioms do not contribute to the computational content of a 

proof (this has been stressed by G. Kreisel since the 50's). We denote 

the extension of the theory ${\cal T}$ by the axioms ${\cal P}$ by 

${\cal T}_*.$ 



\begin{lemma}\label{l:Atrans+}

Friedman's $A$-translation is sound also for $\ha_* + \LEM$.\\ Similarly for 

{\rm HA$_*^{\omega}[X,\|\cdot\|]$} (and related extensions) instead of 

$\ha_*.$

\end{lemma}

\begin{proof}

Consider the following instance of $\LEM$

\[\forall y \phi_0(\tup a,y)  \vee \exists y \neg\phi_0(\tup a,y).\] 

W.l.o.g assume $\phi_0$ is atomic. 

It suffices to extend Friedman's proof by showing that

\[\ha_* + \LEM\vdash (\LEM)^A.\]

This means we need to prove

\[ \forall y \big( \phi_0(\tup a,y) \vee A\big) \vee \exists y \big(  (\phi_0(\tup a,y) \vee A) \rightarrow A\big),\tag{1}\]

in $\ha_* + \LEM$.



Suppose that

\begin{enumerate}

\item $\forall y \phi_0(\tup a,y)$ holds. Then also $\forall y \big( \phi_0(\tup a,y) \vee A\big)$ holds and therefore also (1).

\item $\exists y \neg\phi_0(\tup a,y)$ holds. Then fix such a $y$. For this $y$ we get 

\[(\phi_0(\tup a,y) \vee A) \rightarrow A\]

and so $ \exists y \big(  (\phi_0(\tup a,y) \vee A) \rightarrow A\big)$ holds and therefore also (1). 

\end{enumerate}

For HA$_*^{\omega}[X,\|\cdot\|]$ one just has to observe that still every 

quantifier-free formula can be written as an atomic formula of the form 

$t\underline{a}=_00$ and that the additional axioms are all purely universal 

and so easily imply their own $A$-translation.  

\end{proof} 



For HA instead of 

$\ha_*$ and HA$_*^{\omega}[X,\|\cdot\|]$ 

(also for $\Sigma^0_{n+1}$-LEM and $\Sigma^0_{n+2}$-DNE), 

the next proposition is stated (without proof) in \cite{Hayashi/Nakata}. 

\begin{prop}\label{p:Atrans}

The theory $\ha_* + \LEM$ is closed under the $\Sigma^0_2\m\DNE$ rule. \\ 

Similarly for {\rm HA$_*^{\omega}[X,\|\cdot\|].$}

\end{prop}

\begin{proof} 

Suppose \[\ha_* + \LEM \vdash \neg\neg \exists x\forall y\ \phi_0(\tup a,x,y),\]

where $\phi_0$ is quantifier free and contains only $\tup a$ as free variables 

(in addition to $x,y$). 

Moreover, w.l.o.g we assume that $\phi_0$ is atomic.  \\

Rewriting the negations in terms of ``$\rightarrow$'' and ``$\perp$'' we obtain that

\[\ha_* + \LEM \vdash \big(\ \exists x\forall y\ \phi_0(\tup a,x,y)\rightarrow\perp\big)\rightarrow \perp,\]

and using Friedman's A-translation (with Lemma~\ref{l:Atrans+}) that

\[\ha_* + \LEM \vdash \Big(\ \exists x\forall y\ \big(\phi_0(\tup a,x,y) \vee A\big)\rightarrow A\Big)\rightarrow A,\]

for any formula $A$ (not containing $x,y$ free). By setting \[A:\equiv\ \exists x'\forall y' \phi_0(\tup a,x',y'),\]

(we consider only this $A$ throughout the remainder of the proof) we obtain that $\ha_* + \LEM$ proves

\[\big(\ \exists x\forall y\ (\phi_0(\tup a,x,y) \vee \exists x'\forall y' \phi_0(\tup a,x',y'))\rightarrow \exists x'\forall y' \phi_0(\tup a,x',y')\big)\rightarrow \exists x'\forall y' \phi_0(\tup a,x',y').\tag{1}\]

Now the claim follows from

\[

\LEM\vdash \forall y\big(\phi_0(\tup a,x,y) \vee \exists x'\forall y' \phi_0(\tup a,x',y') \big) \rightarrow \big(\forall y\ \phi_0(\tup a,x,y) \vee \exists x'\forall y' \phi_0(\tup a,x',y')\big),

\tag{2}

\]

since using (2) the statement (1) is equivalent to

\[  \big((\exists x \forall y \phi_0(\tup a,x,y) \vee \exists x'\forall y' \phi_0(\tup a,x',y'))\rightarrow \exists x'\forall y' \phi_0(\tup a,x',y')\big)\rightarrow \exists x'\forall y' \phi_0(\tup a,x',y'),\]

which is equivalent to $\exists x\forall y \phi_0(\tup a,x,y)$.\\

To show (2) consider the following instance of $\LEM$

\[

\forall y\ \phi_0(\tup a,x,y) \vee \exists y\ \neg\phi_0(\tup a,x,y).

\]

Now suppose that

\begin{enumerate}

\item $\forall y\ \phi_0(\tup a,x,y)$ holds, then (2) is trivially true.

\item $\exists y\ \neg \phi_0(\tup a,x,y)$ holds, then for such a $y$ we have

\[

\big(\phi_0(\tup a,x,y) \vee \exists x'\forall y' \phi_0(\tup a,x',y') \big) \rightarrow \exists x'\forall y' \phi_0(\tup a,x',y')

\]

and so certainly we have also that

\[

\forall y\big(\phi_0(\tup a,x,y) \vee \exists x'\forall y' \phi_0(\tup a,x',y') \big) \rightarrow \exists x'\forall y' \phi_0(\tup a,x',y').

\]

Finally $\big(\forall y\ \phi_0(\tup a,x,y) \vee \exists x'\forall y' \phi_0(\tup a,x',y')\big)$ follows from $\exists x'\forall y' \phi_0(\tup a,x',y')$ so (2) holds as well.

\end{enumerate}

\end{proof}



It is known, that a Cauchy rate is limit computable (which corresponds 

to $\Sigma^0_2$-DNE which -- as mentioned in the introduction -- is 

strictly stronger than $\Sigma^0_1$-LEM). However, for every provable   

Cauchy sequence we have that $\Sigma^0_1$-LEM is sufficient:

\begin{prop}\label{p:limComp}

If a sequence of real numbers $(a_n)$ (or in some $\pa_*$-definable Polish 

space) defined by a term of 

$\pa_*$, can be proved to be Cauchy in $\pa$, then the proof can be carried 

out already in $\ha_*+\LEM$. Similarly for $\pa_*[X,\|\cdot\|]$ and 

sequences in 

$X.$

\end{prop}

\begin{proof}

Consider a sequence $x_{(\cdot)}$ and suppose

\[ \pa_*\vdash \forall k\exists n\forall i,j>n\ 

\big(| x_i - x_j| \leq 2^{-k}\big). \]

Then by the Kuroda negative translation (see e.g. \cite{Kohlenbach(book)}) 

we obtain that

\[ \ha_*\vdash \forall k\neg\neg\exists n\forall i,j>n\ \big(| 

x_i - x_j| \leq 2^{-k}\big). \]

By Proposition~\ref{p:Atrans} this implies that

\[ \ha_*+\LEM\vdash \forall k\exists n\forall i,j>n\ \big(| x_i - x_j| 

\leq 2^{-k}\big) \] (recall that $\le_{\RR}\in\Pi^0_1$).

\end{proof}



\section{Which Cauchy statements are effectively learnable and which are not}



\begin{prop}[Implications between different bounding information 

for Cauchy statements] \label{prop.hierarchy}

Let $(x_n)$ be a Cauchy sequence in a metric space $(X,d).$ 

\begin{enumerate}

\item A rate of convergence is a bound on the number of fluctuations.

\item A bound for the number of fluctuations is a bound $B$ 

on the number of mind changes to learn a rate of convergence (with a  

simple projection function as learning procedure $L$).

\item 

Primitive recursively (in the ordinary sense of Kleene) in majorants 

$B^*,L^*$ of functionals $B,L$ such that the Cauchy rate is $(B,L)$-learnable 

one can obtain a rate of metastability. 

\end{enumerate}

\end{prop}

\begin{proof}

Consider a Cauchy sequence $x_{(\cdot)}$.

\begin{enumerate}

\item Let $b$ be a rate of convergence, i.e.

\[ \forall k \forall n,m\geq b(k)\ \big( d(x_n,x_m)\leq 2^{-k}\big). \]

Then $b(k)$ is also a bound on the number of $2^{-k}$ fluctuations, since any fluctuation has to occur before $b(k)$ (i.e. that one of the indexes of the 

fluctuation has to be smaller than $b(k)$) and there can be at most $b(k)$ many fluctuations indexed within $[0;b(k)]$.

\item Let $b$ be a bound on the number of $2^{-k}$ fluctuations, i.e.

\[ \forall k \forall n>b(k) \forall i,j \neg\Fluc_{2^{-k}}(n,i,j). \]

Then $b(k)$ is also a bound on the number of mind changes to learn a rate of convergence, since for 

$L(n):=n$ we have that

\[ \forall k\ \exists l\leq b(k)\ \forall n,m>c_l\ \ \big(  d(x_n,x_m)\leq 2^{-k}\big). \tag{BE}\label{e:be}\]

%



Formally, $L(i,x_{(\cdot)},k):=i$, and (where -- again -- to have 

$\varphi_0$ quantifier-free we officially have to use the $2^{-k-1}$-rational 

approximation $\widehat{d(x_n,x_m)}(k+1)$ of $d(x_n,x_m)$) 

\[

\phi_0(j(n,m),c_i,x_{(\cdot)},k):\equiv\ \big((n>c_i \wedge m>c_i) \rightarrow d(x_n,x_m)\leq 2^{-k}\big),

\]

where $j(n,m)$ is the Cantor pairing function.

%

The statement~\eqref{e:be} can be inferred from the fact that each mind change ($c_i$) corresponds to a (different) fluctuation (as it is based on a counterexample for $d(x_n,x_m)\leq 2^{-k}$, whose both indexes are greater than the last $c_i$).

\item Follows directly from Proposition~\ref{p:bg2meta}.

\end{enumerate}

\end{proof}







In the rest of this section we show that the hierarchy in Proposition \ref{prop.hierarchy} 

between the four different 

quantitative notations for Cauchy sequences discussed in the introduction 

is strict. That an effective bound on the number of fluctuations does not imply an 

effective rate of convergence, follows already from the existence of Specker 

sequences \cite{Specker(49)}. We can also use the following

very simple example with a $2^{-k}$-fluctuation bound $k$ and no effective rate of convergence, since such a rate would decide the halting problem:\\ 

\begin{prop}[$\alpha_{(\cdot)}$] 

We take the Cantor pairing function $j$ and set

\[

\alpha_{j(k,n)}:=

\begin{cases}

2^{-k}, &\Tif T(k,0,n),\\

0, &\Telse,

\end{cases}

\]

where $T$ is the primitive recursive Kleene $T$-predicate. Then $(\alpha_n)$ 

is a convergent (towards $0$)

primitive recursive sequence of rationals in $[0,1]$ with $2^{-k}$-fluctuation 

bound $k$ which has no computable 

rate of convergence.

\end{prop}

We next construct primitive recursive sequence $\beta_{(\cdot)}$ of rational 

numbers in $[0,1]$ with an effectively (even primitive recursively) learnable Cauchy rate (so in particular with a primitive recursive rate of metastability), which has no computable bound on fluctuations (this example is not captured by the rough sketch of Avigad and Rute as here the number of the oscillations is determined by the length of the computation, not by the index of the machine as suggested in \cite{Avigad/Rute}). 

Furthermore, we also give an example of a primitive recursive (in the ordinary 

sense) sequence $\gamma_{(\cdot)}$ of rational numbers in $[0,1]$ 

which (provably in the fragment 

of PA based on $\Sigma^0_1$-IA only) converges to $0$ (and so 

has a primitive recursive in the sense of Kleene rate of metastability 

for the convergence towards $0$) 

which does not have an effectively learnable Cauchy rate. 



\subsection{A primitive recursive sequence of rationals 

with a primitive recursively learnable Cauchy rate but with no computable 

bound on fluctuations}



\begin{dfn}[$\beta_{(\cdot)}$]\label{d:beta}

We fix a primitive recursive 

surjective encoding of triples which is monotone in the third component 

satisfying $\langle k,n,m\rangle\ge k,n,m$ and set

\[

\beta_{\langle k,n,l\rangle}:=

\begin{cases}

2^{-k}, &\Tif T(k,0,n) \wedge l\leq n \wedge l\text{ is even},\\

0, &\Telse.

\end{cases}

\]

\end{dfn}



In the next propositions we will show that the sequence $\beta_{(\cdot)}$

\begin{itemize}

\item is Cauchy (in fact, it converges to zero) -- 

Proposition~\ref{p:alphaIsCauchy},

\item its Cauchy rate is effectively learnable -- Proposition~\ref{p:alphaIsLearnable},

\item there is no computable (in $\epsilon$ and $\beta$) bound on the number of $\epsilon$-fluctuations -- Proposition~\ref{p:alphaHasNoFlucBd}.

\end{itemize}



\begin{prop}\label{p:alphaIsCauchy}

The sequence $\beta_{(\cdot)}$ is convergent towards $0,$ provably in 

{\rm $\ha+\Sigma^0_1$-LEM$^-$}. More precisely we show that

\[ 

\ha

\vdash 

\forall k \Big( 

\forall m\leq k \,\big(\exists u\,T(m,0,u) \vee \forall v\,\neg T(m,0,v)\big)\rightarrow \exists n \forall x\geq n\big(\beta_x\leq2^{-k}\big)\Big).

 \]

\end{prop}

\begin{proof}

Consider the terminating computations on input $0$ of the Turing machines encoded by $0,\ldots,k$. Then for every $k$ there is an $n$ corresponding to the

code of the longest such computation. W.l.o.g. we can assume that $n\geq k$ (otherwise set $n:=k$). This means we have that

\begin{align}

n\geq k\ \wedge\ \forall n'\forall k'\leq k\ \big( T(k',0,n')\rightarrow n'\leq n\big).\label{e:ac-n}

\end{align}

Now, set \[ c(k):=\max\{ \langle k',n',l' \rangle\ :\ n'\leq n,\ k'\leq k,\ l'\leq n'\}.\]

Then $c$ is even a rate of convergence, since

\[\ba{l} 

\langle k',n',l'\rangle\ >\ c(k)\ \rightarrow\ k'>k \vee (k'\le k \wedge n'>n)

\vee (k'\le k\wedge n'\le n\wedge l'>n')\\  

\rightarrow\ k'>k \vee \beta_{\langle k',n',l'\rangle}=0 \ \rightarrow\ \beta_{\langle k',n',l'\rangle} < 2^{-k}. \ea 

\]

These arguments are constructive, except for the existence of the longest computation $n$.

This existence is a consequence of $\LEM^-$ and $\Pi^0_1\m\CP^-,$ where 

$\Pi^0_1$-CP is the bounded collection principle for $\Pi^0_1$-formulas 

(also called $B\Sigma^0_2$ in the literature) which is easily provable by 

induction in HA$^{\omega}$.

Consider the following $k+1$ instances of $\LEM^-$:

\[ \forall j\leq k\ \big( \exists n T(j,0,n) \vee \forall m \neg T(j,0,m)\big) \] which over HA$^{\omega}$ implies 

\[ \forall j\leq k \exists n_j \,\forall m\,\big( T(j,0,n_j) \vee \neg T(j,0,m)\big). \] 

By an application of $\Pi^0_1$-CP$^-$, this in turn

implies

\begin{align*}

\exists n\forall j\leq k ( \exists n'\leq n T(j,0,n') \vee \forall m \neg T(j,0,m) ),

\end{align*}

 (consider $n=\max \{n_j : j\leq k\}$).\\

This shows that the convergence is provable in $\ha$+$\LEM^-$ and the convergence up to an error $2^{-k}$ in $\ha$ uses

only $k+1$ instances of $\LEM^-$.

\end{proof}







\begin{lemma}\label{l:G3ACP} 

For a quantifier-free formula $\phi_0$ with parameters only of type $0$, we have that

\[

\mbox{\rm G$_3$A}^\omega + 

\SiLm\IA^- \vdash 

\forall x^0\ \exists u^0 

\forall\tilde x\leq x \big(  

\forall y^0 \phi_0(\tilde x,y)

\vee\exists\tilde u\leq u\neg\phi_0(\tilde x,\tilde u) \big).

\] 

\end{lemma}

\begin{proof}

See \cite{Kohlenbach(book)} Lemma 3.18.

\end{proof}



\begin{rmk}

$\Sigma^0_1$-IA$^-$ is (over G$_3$A$^{\omega}$) strictly weaker than 

$\Pi^0_1$-CP$^-$ but the proof in Lemma \ref{l:G3ACP} needs $\Sigma^0_2$-DNE 

and so more of classical logic than necessary in the proof based on 

$\Pi^0_1$-CP$^-$.

In general, it seems that considering the extraction of computational content 

from proofs, often some amount of classical logic can be reduced on the cost of more recursion.\\

If one is interested (only) in a classical proof, we obtain due to Lemma~\ref{l:G3ACP}

(simply consider $\varphi_0(x,y):\equiv \neg T(x,0,y)$) a proof of the 

convergence of $\beta_{(\cdot)}$ 

without the use of $\PiLm\CP$, which can be formalized in G$_3$A$^{\omega}+\SiLm\IA^-$.

\end{rmk}





\begin{prop}\label{p:alphaIsLearnable}

The rate of convergence is effectively learnable in $k$, i.e.

there are total (elementary) 

recursive functions $B$ and $L$, s.t. for any $k$ we have that

\[ \forall k\ \exists n\leq B(k)\ \forall m > c_n\quad \big(\beta_m\leq 2^{-k}\big),\]

where $c_{(\cdot)}$ is defined as in Definition~\ref{d:fmcMon} with

\[

\phi_0(x,n,k):\equiv\ x> n\rightarrow \beta_x\le 2^{-k}.

\]

\end{prop}

\begin{proof}

Obviously, this follows already from Proposition~\ref{p:alphaIsCauchy}. Also, it is easy to see that the 

rate is $(B,L)$-learnable with the following $B$ and $L$:

\begin{align*}

B(k)&:=k+1  \\

L(n,k)&:= \langle k,n,n \rangle+1.

\end{align*}

Let $x\ge L(n,k)>\langle k,n,n\rangle$ be a counterexample. Then (using 

the definition of $\beta_{(\cdot)}$)  

\[ j_1(x)\le k\,\wedge\,(j_2(x)>n\vee j_3(x)>n)\,\wedge \,j_3(x)\le j_2(x) \] 

and so 

\[ j_1(x)\le k\,\wedge\,j_2(x)>n\,\wedge\,j_3(x)\le j_2(x). \]

The 2nd conjunct implies $j_2(x)>j_2(n)$ and hence $j_1(x)\not= j_1(n)$ 

if $n$ is a preceding counterexample. However, for numbers $\le k$ this can 

happen at most $k$-many times. Hence $B(k):=k+1$ and $L$ do the job.

\end{proof}





\begin{cor} \label{metastability-alpha}

$\beta_{(\cdot)}$ has a primitive recursive (in the ordinary sense 

of Kleene) rate of metastability for the convergence towards $0.$

\end{cor}

\begin{proof} One can apply Proposition \ref{p:bg2meta} to convert 

the bounds $(B,L)$ from Proposition \ref{p:alphaIsLearnable} (which are 

trivially self-majorizing using standard monotonicity properties of the 

triple coding) into a primitive 

recursive rate of metastability. Alternatively, one can use that by 

Lemma \ref{l:G3ACP} the convergence of $\beta_{(\cdot)}$ towards $0$ 

is provably in 

G$_3$A$^{\omega}+\Sigma^0_1$-IA$^-$ and so a fortiori  in 

$\widehat{\rm PA}^{\omega}\res+$QF-AC (see \cite{Kohlenbach(book)}, Prop.3.31).

Then proposition 10.54 in \cite{Kohlenbach(book)} implies the extractability 

of a primitive recursive rate of metastability for the convergence towards 

$0$ (the latter being essentially 

the rate of metastability for the statement in Lemma \ref{l:G3ACP} 

which is computed in \cite{Kohlenbach(book)}[Prop.3.19]).

\end{proof} 



\begin{rmk}\label{r:metaB}

One can obtain such a rate of metastability for $\beta_{(\cdot)}$ directly, using previous results

of the first author.\\

By~\cite{Kohlenbach(book)} (Prop.13.19) we have that

\be[e:maU1]

\forall x,f \forall \tilde x\leq x\big(\exists y\leq\Phi xf\ T(\tilde x,0,y)\vee \forall z\leq f(\Phi xf)\neg T(\tilde x,0,z)\big),

\ee

where $\Phi xf\leq \max\{f^{i}(0)\ 

:\ i\leq x+1\}=:\Phi^*xf$ (here $f^i(0)$ again denotes the 

$i$-times iteration of $f$). \eqref{e:maU1} implies

\be[e:maU2]

\forall z \big(\Phi xf< z\leq f(\Phi xf)\rightarrow \forall \tilde x \leq x \neg T(\tilde x,0,z)\big).

\ee

Define $\tilde f(n):=\max\{f(n),n\}$ and $f_k(n):=\tilde f(\langle k,n,n\rangle+1)$ and let

\[

\Psi(k,f):=\langle k,\Phi^*kf_k,\Phi^*kf_k\rangle + 1.

\]

Then $\Psi$ is a rate of metastability for the convergence of 

$\beta_{(\cdot)}$ towards $0,$ i.e.:

\be[e:rmB]

\forall k,f\ \exists n\leq \Psi(k,f)\ \forall z\in[n,\tilde f(n)]\ \ 

(|\beta_z|< 2^{-k}).

\ee

To prove~\eqref{e:rmB}, define $n:=\langle k, \Phi kf_k, \Phi kf_k\rangle+1\leq\Psi(k,f)$ and let $z\in[n,\tilde f(n)]$. Then 

$z\geq n > \langle k, \Phi kf_k, \Phi kf_k\rangle$ and so one of the following cases holds:

\hspace{35mm}\begin{enumerate}

\item $j_1(z)>k$. Then $|\beta_z|\leq2^{-j_1(z)}<2^{-k}$.

\item $j_2(z)>\Phi kf_k\wedge j_1(z)\leq k$. Then 

$

\Phi kf_k < j_2(z)\leq z \leq \tilde f(n)=f_k(\Phi kf_k).

$

Hence, by~\eqref{e:maU2} (applied to $k,j_1(z),f_k,j_2(z)$ 

for $x, \tilde{x}, f, z$), we get 

$\neg T(j_1(z),0,j_2(z))$ and so $\beta_z=0$.

\item $j_3(z)>\Phi kf_k\wedge j_2(z)\leq\Phi kf_k\wedge j_1(z)\leq k$. Then $j_3(z)>j_2(z)$ and so $\beta_z=0$.

\end{enumerate}

\end{rmk}





\begin{lemma}[Termination causes at least $n$ fluctuations]\label{l:2n}

Suppose the $k^\text{th}$-machine terminates on $0$ with computation encoded by $n$ (i.e. $T(k,0,n)$ holds). Then the sequence $\beta_{(\cdot)}$ contains at 

least $n$ many $2^{-k}$\nbd fluctuations.

\end{lemma}

\begin{proof}

Consider the tuples of indexes $\tup i$, $\tup j$, 

s.t. $i_l:=\langle k,n,l\rangle$, $j_l:=\langle k,n,l+1\rangle$ and $l+1\leq n$. Then we have by definition of $\beta_{(\cdot)}$ (using the monotonicity of the encoding in $l$) that

$

\Fluc_{\beta_{(\cdot)},2^{-k}}(n,\langle \tup i \rangle, \langle \tup j\rangle).

$

\end{proof}



\begin{prop}\label{p:alphaHasNoFlucBd}

There is no computable bound on the fluctuations of $\beta_{(\cdot)}$.

\end{prop}

\begin{proof}

Suppose $b_k$ is a bound on the number of fluctuations by $2^{-k}$, 

then $b_k$ can be used to effectively compute whether the $k^{\text{th}}$ Turing machine terminates on input $0$ as follows.\\

Let the machine run until the code of the computation reaches $b_k$ (or until it stops). If it terminated, we are done.\\

Now suppose it terminates with a computation encoded by some $n>b_k$. Then $\beta_{(\cdot)}$ would have at least $n$ many $2^{-k}$\nbd fluctuations by Lemma~\ref{l:2n}, which is a contradiction.\\

Therefore if the machine does not terminate with a code of computation at most $b_k$ it does not terminate at all.

\end{proof}







\subsection{Metastability of Cauchy sequences does not imply effective learnability} \label{section4.2}



In the next propositions we define a primitive recursive sequence 

$\gamma_{(\cdot)}$ of rational numbers in $[0,1]$ 

(defined in Corollary~\ref{c:gammaee} using Definition~\ref{d:gammaf}) that

\begin{itemize}

\item is Cauchy (in fact, it converges to zero) -- by 

Proposition~\ref{p:gammaf},

\item has a primitive recursive rate of metastability of its 

convergence towards $0$ -- by Proposition~\ref{p:gammaf},

\item has no effectively learnable Cauchy rate -- by Corollary~\ref{c:gammaee}.

\end{itemize}



We use the upper index as a name extension (like $\tup k^K$, meaning a tuple $\tup k$ corresponding to a particular $K$) and as iteration of functions (like $f^n(x)$, meaning we iterate the function $f$ $n$-many times with the starting point $x$). When unclear, we use the notation $(f)^n$ to make explicit, that we mean the iteration.





\begin{dfn}\

\begin{enumerate}

\item

For any function $f:\NN^2\to \NN$ we define \[ 

\mbox{$\widehat{f}(k,n):=0$, 

if $f(k,n)=0\wedge\forall m<n\ (f(k,m)\neq 0)$ and $\widehat{f}(k,n):=1,$ 

otherwise.}\] Note that $\widehat{f}(k,\cdot)$ has a root iff $f(k,\cdot)$ has 

one but also that it has at most one root.

\item 

For any $f$ as above and $x,y\in\NN$ define

\[ \hspace*{-5mm}

y_{f,x}:=\hspace*{-1mm}\begin{cases}

\max\big\{y'\leq y\, :\, \exists x'\le x\,

(y'=\min\{y''\leq y\ :\ f(x',y'')=0

\ \})\big\},& \hspace*{-2mm} \text{if such $y'$ exists}\\

x,&\Telse.

\end{cases}

\]

Then for $p=j(y,u)$ define 

\[ \varphi_1(f,x,p,z): \equiv

\forall \tilde x \leq x \exists \tilde y\leq y \forall \tilde z\leq z 

\big(f(\tilde x,\tilde y)=0\vee f(\tilde x,\tilde z)\neq0\big) \] 

and 

\[ \varphi_2(f,x,p,z):\equiv 

\forall \tilde y \leq y_{f,x} \exists \tilde u\leq u \forall \tilde z\leq z 

\big(f(\tilde y,\tilde u)=0\vee f(\tilde y,\tilde z)\neq0\big) \] 

and, finally,\footnote{In connection with $f$ we write the type $1$ even 

though it officially is the type $0\to (0\to 0).$} 

\[ \varphi_0:\equiv \varphi_1\wedge\varphi_2, \ \ \varphi:\equiv

\forall f\leq_1  \forall x^0 \exists p  \forall z \phi_0 

(\widehat{f},x,p,z)

\]

\end{enumerate}

\end{dfn}

Note that the $\varphi_1$-part of $\varphi$ combines a sequence of instances 

of $\Sigma^0_1$-LEM with induction (in the form of $\Pi^0_1$-CP) and that the 

$\varphi_2$-part repeats this construction but taking (essentially) the result 

`$y$' from $\varphi_1$ as input thereby making it no longer possible to give 

a computable bound on the number of instances of $\Sigma^0_1$-LEM used in 

total. We 

will show in the next proposition that $\varphi$ is not $(B,L)$-learnable 

by showing that it implies over a system as weak as 

G$_3$A$^{\omega}$ (which does not allow for the iteration of a function 

variable $g$) 

\[ \forall g^1\exists y^0 \,(y=g^{g^x(0)}(0)) \] 

which grows too fast as a functional in $g$ to be derivable from a rate 

of metastability for $\varphi$ having the simple form from Remark 

\ref{r:metastr} whose existence would follow from the $(B,L)$-learnability 

of $\varphi.$  

Here it is crucially used that computable functionals $B,L$ in the function 

parameter $f$ which can be taken to be bounded by $1$ can be effectively 

majorized by bounds which no longer depend on $f.$



\begin{prop}\label{p:nonLearnablePhi} $\varphi$  

is provable using {\rm $\Sigma^0_1$-LEM$^-$} combined with 

{\rm $\Pi^0_1$-CP$^-$} (uniformly in $f$ treated as parameter) but  

is not effectively learnable.

\end{prop}

\begin{proof}

We first show that $\varphi$ is provable: by a suitable instance 

$\Sigma^0_1$-LEM$(t(f))$ of $\Sigma^0_1$-LEM$^-$ one obtains 

\[ \forall x\forall \tilde{x}\le x\exists y\forall z \big( 

\exists \tilde{y}\le y\,\widehat{f}(\tilde{x},\tilde{y})=0 \vee \forall 

\tilde{z}\le z 

\,\widehat{f}(\tilde{x},\tilde{z})\not=0\big) \] 

and so by a suitable instance $\Pi^0_1$-CP$(s(f))$ 

\[ \forall x\exists y \forall \tilde{x}\le x \forall z 

\big( 

\exists \tilde{y}\le y\,\widehat{f}(\tilde{x},\tilde{y})=0 \vee \forall 

\tilde{z}\le z 

\,\widehat{f}(\tilde{x},\tilde{z})\not=0\big) \] which implies  

\[ \forall x\exists y \forall z\forall \tilde{x}\le x \exists \tilde{y}\le y

\forall \tilde{z}\le z\, 

\big( \widehat{f}(\tilde{x},\tilde{y})=0 \vee 

\widehat{f}(\tilde{x},\tilde{z})\not=0\big). \]



Now we repeat the same argument with $y_{f,x}$ instead of $x.$

\\[1mm] 

To show that $\varphi$ is not learnable we proceed in three steps.

\paragraph{Step 1} We will show that (informally speaking, since 

formally we cannot express function iteration and the conclusion would have to use $\psi_0$ and $\psi^y_0$ defined below)

\[

\mbox{G$_3$A}^\omega\vdash \forall g \forall x\ \big(\exists p\forall z\phi_0(f_g,x,p,z)\rightarrow \exists y'(y'=g^{g^{x}(0)}(0))\big),

\tag{GA}\label{e:GA}

\]

for \[

f_g(b,d):=\begin{cases}

0,&\Tif \ \lh(d) = b+1 \wedge d_0=0\wedge \forall i< b\ (d_{i+1}=g(d_i),\\

1,&\Telse.

\end{cases}

\]

Formally, $\exists y' (y'=g^{g^x(0)}(0))$ is to be read as 

\[ \exists y,u \, (f_g(x,y)=0\wedge f_g(y_x,u)=0),

\tag{GA$^*$}\label{e:GA$^*$} \] 

where then $y':=u_{y_x}.$  

Note that we have $f_g=_1\widehat{f_g}$. To show~\eqref{e:GA} fix arbitrary $g^1$ and $x^0$.

Now assume $\exists p\forall z\phi_0(f_g,x,p,z)$ and let us fix such 

a $p=j(y,u)$ to obtain:

\begin{align}

&\forall z\forall\tilde x \leq x \exists \tilde y\leq y \forall \tilde z\leq z 

\big(f_g(\tilde x,\tilde y)=0\vee f_g(\tilde x,\tilde z)\neq0\big) \label{e:A}

\wedge \\

&\forall z\forall\tilde y \leq y_{f_g,x} \exists \tilde u\leq u \forall \tilde z\leq z 

\big(f_g(\tilde y,\tilde u)=0\vee f_g(\tilde y,\tilde z)\neq0\big). \label {e:B}

\end{align}

We now show by quantifier-free induction that 

\be[e:IC1]

\forall x'\leq x \exists y'\leq y\ f_g(x',y')=0.

\ee

Note that $\exists y'\,f_g(x',y')=0$ implies that $y'_{x'}=g^{x'}(0).$ 

The case $x'=0$ is trivially satisfied by $y'=\langle 0 \rangle$. Then 

$y'\le y$ by ~\eqref{e:A}. So suppose for some $x'<x$ we have

$\exists y'\leq y\,f_g(x',y')=0.$ Then we can set 

\[

z:=y'*\langle g(y'_{x'}) \rangle =\langle y'_0,y'_1,\ldots,g(y'_{x'})\rangle

\] to get $f_g(x'+1,z)=0$

which concludes the proof of~\eqref{e:IC1} since -- again by 

~\eqref{e:A} -- $z\le y.$ Note, furthermore,

that for $y'\leq y$ (and $x'\leq x$),

\[ f_g(x',y')=0\ \rightarrow\ y'\leq y_{f,x}.\] 

So we have even that \\

\be[e:IC1+]

\forall x'\leq x \exists y'\leq y_{f_g,x}\, f_g(x',y')=0.

\ee

Now, let $y^*$ denote the $y'$ which satisfies~\eqref{e:IC1+} for $x'=x$ and 

note that $y^*_x\le y^*\le y_{f_g,x}.$ 

By quantifier-free induction we show that 

\be[e:IC2]\forall x'\leq y_{f_g,x} \exists u'\leq u \, f_g(x',u')=0.\ee

The case $x'=0$ is again trivially satisfied by $u'=\langle 0 \rangle \le u$ 

(using ~\eqref{e:B}). 

So suppose for some $x'< y_{f_g,x}$ that

$\exists u'\leq u \,f_g(x',u')=0$. Then we can set

\[z:=u'*\langle g(u'_{x'})\rangle= \langle u'_0,u'_1,\ldots,g(u'_{x'})\] to get $f_g(x'+1,z)=0$,

which by~\eqref{e:B} implies 

\[\exists \tilde u\leq u \,f_g(x'+1, \tilde u)=0\]

and so concludes the proof of ~\eqref{e:IC2}. Applying ~\eqref{e:IC2} 

we obtain (for $x'=y^*_x$)

\[

\exists u'\leq u\, (f_g(x,y^*)=0\wedge f_g(y^*_x,u')=0),

\]

which concludes the proof of~\eqref{e:GA$^*$} and, therefore, 

also the proof of~\eqref{e:GA}.\\

\paragraph{Step 2} We investigate the terms witnessing the implication~\eqref{e:GA}. By prenexation we obtain

\[

\mbox{G$_3$A}^\omega\vdash \forall g,x,p\exists y',z\ \big(\phi_0(f_g,x,p,z)\rightarrow y'=g^{g^{x}(0)}(0)\big),

\]

and, therefore, by program extraction theorems (see Corollary 3.1.3 in~\cite{Kohlenbach(lowrate)}), we get closed terms $s$ and $t$ in G$_3$A$^\omega$, s.t.

\be[e:st]

\forall g,x,p\ \big(\phi_0(f_g,x,p,sgxp)\rightarrow tgxp=g^{g^{x}(0)}(0)\big).

\ee

\paragraph{Step 3} Finally, we show that with sufficiently large (in the sense of growth) $g$, this contradicts the effective learnability of $\phi$. Suppose namely that $\phi$ were learnable by computable functionals $B(f,x),L(y,f,x).$ 

Then also 

\[ \ba{l} B^*(x):=\sup \{ B(f,\tilde{x}): f\le_1 1, \tilde{x}\le x\} \ 

\mbox{and} \\ 

L^*_x(y):=\max\{ y,\sup \{ L(\tilde{y},f,\tilde{x}): f\le_1 1,\tilde{y}\le y,

\tilde{x}\le x\}\}\ea \] 

are computable (in $x$ resp. in $x,y$) and majorize $B,L.$ 

Now by Proposition~\ref{p:bg2meta} and the fact that $f_g$ is trivially 

majorized by $1$ we get, in particular, that 

\[

\exists p\leq \Omega(B^*,L^*_x,h_x,x)\ \phi_0(f_g,x,p, sgxp),

\]

for any $g$ and $x$, by setting \[h^1_x:=\lambda p\ .\ sgxp.\]

So, we obtain together with~\eqref{e:st} that

\[

\forall g,x \exists p_x\leq \Omega(B^*,L^*_x,h_x,x)\ 

\big( tgxp_x =g^{g^{x}(0)}(0)\big).

\]

Since $s$ and $t$ are closed terms of G$_3$A$^\omega$ and the variables 

$g,x,p_x$ have types $\le 1$ by normalization arguments

(see Corollary 2.2.24 and Remark 2.2.25 in~\cite{Kohlenbach(lowrate)}) we know that there is 

a constant $D$, s.t. (for any $g$ that majorizes $\lambda n.2^n$) 

\[ \forall x,v\, \big( \tilde{g}^D(x+v)\ge sgxv,tgxv\big).\] 

Since we may assume that $\tilde{g}(n)>n$ and, therefore, 

$\tilde{g}^x(v)\ge x+v,$ this yields 

\[ \forall x,v\,\big( \tilde{g}^{D+x}(v)\ge sgxv,tgxv\big). \]

As a consequence, we get (using the $\Omega$-definition 

and that that $B^*,L^*_x$ are selfmajorizing and that $L^*_x(y)\ge y$) 

that for all $x$  

\[ \tilde{g}^{D+x}\left(\Omega(B^*,L^*_x,\tilde{g}^{D+x},x)\right) \ge 

\tilde{g}^{D+x}\left(\Omega(B^*,L^*_x,h_x,x)\right) \ge 

tgxp_x=g^{g^x(0)}(0). \]

By the definition of $\Omega$ (see also Remark~\ref{r:metastr}) 

\[   \ba{l} \tilde{g}^{D+x}\big(\Omega(B^*,L^*_x,\tilde{g}^{D+x},x\big)\le 

\tilde{g}^{D+x}

\left((L^*_x\circ\tilde{g}^{D+x})^{B^*(x)}(0)\right) \le \\ 

\tilde{g}^{D+x}\left((L^*_x\circ\tilde{g})^{(D+x)B^*(x)}(0)\right)\le 

(L^*_x\circ \tilde{g})^{\widehat{B}^*(x)}(0)\ea \] 

and so (for all $x$)

\[  g^{g^x(0)}(0)\le (L^*_x\circ \tilde{g})^{\widehat{B^*}(x)}(0) \]

where $\lambda x,y.L^*_x(y)$ and $\widehat{B^*}(x):=(D+x)(B^*(x)+1)$ 

are fixed total recursive functions that do not depend 

on $g$ which is not possible for sufficiently fast growing $g$.

\end{proof}



\begin{cor}\label{c:anyStrongerPhi}

Let $\phi_0$ be as in the previous Proposition. If for a quantifier free formula $\psi_0$ (with no hidden parameters)

\[

\mbox{\rm G$_3$A}^\omega+\QF\m\AC 

\vdash \forall f\leq 1 \forall x^0\big( \exists y^0\forall z^0 \psi_0(\xi(f),\chi(x),y,z)

 \rightarrow \exists p^0\forall z^0 \phi_0(\widehat{f},x,p,z) \big),

\]

where $\xi$ and $\chi$ are closed terms of {\rm G$_3$A$^\omega$,}

then $\forall f\leq 1\forall x^0\exists y^0\forall z^0 \psi_0(f,x,y,z)$ is also not effectively learnable. Here 

\[ \mbox{\rm QF-AC}: \ \forall x\,\exists y\,F_0(x,y)\rightarrow 

\exists f\,\forall x\,F_0(x,f(x)), \] 

with quantifier-free $F_0$ and $x,y$ of arbitrary types.

\end{cor}

\begin{proof}

This follows analogously from~\cite{Kohlenbach(lowrate)} (Corollary 3.1.3.) 

and our Proposition~\ref{p:bg2meta} as in the proof

of Proposition~\ref{p:nonLearnablePhi}.

\end{proof}



\begin{rmk}

We can prove in {\rm G$_3$A$^\omega+\SiLm\CP$} (which is included in 

{\rm G$_3$A$^\omega+\QF\m\AC^{0,0}$}) that $\phi$ in Proposition

~\ref{p:nonLearnablePhi} is actually equivalent to its monotone version, 

\[

\tilde\phi\equiv \forall f^1\leq 1\forall x^0\exists q\forall z \exists 

p\leq q\forall \tilde z\leq z \phi_0(\widehat{f},x,p,z).

\]

Since this equivalence holds also pointwise (in $f,x$), 

we can use Corollary \ref{c:anyStrongerPhi} to 

infer that there is actually a monotone formula, which is not effectively 

learnable.

\end{rmk}



\begin{dfn}\label{d:gammaf} Define (using a surjective quadruple coding) a 

primitive recursive sequence of rational numbers in $[0,1]$ by

\[

\gamma(f)_{\langle k,n,i,m\rangle}:=\begin{cases}

2^{-k},&\Tif\ \widehat{f}(k,n)=0\wedge i\leq n\wedge \widehat{f}(i,m)=0,\\

0,&\Telse.\end{cases}

\]

\end{dfn}



\begin{prop}\label{p:gammaf} The sequence $(\gamma(f)_z)_{z\in\NN}$ converges 

to $0$ but

the formula stating the existence of a Cauchy point for any $f$ 

\[

\psi:=\forall f^1\leq_1 1,x^0\exists z \forall k\geq z\ \big( \gamma(f)_k<2^{-x} \big).

\] is not effectively learnable. However, there is

a primitive recursive (in the ordinary sense of Kleene) rate 

of metastability for the convergence of $\gamma(f)\seq$ towards $0,$ 

which does not 

depend on $f$.

\end{prop}

\begin{proof}

The existence of a metastability rate follows from the fact

that $\gamma(f)\seq$ converges to $0$ for any $f\leq_1 1$. Moreover, since the

proof can be formalized in G$_3$A$^\omega+\SiLm\IA$ there is a primitive recursive rate and since $f$ is trivially majorizable it is also

clear that there is even a primitive recursive rate which does not 

depend on $f$. (In Remark \ref{metastable} below, we actually give such a 

rate explicitly.)\\

To show the unlearnability, due to Corollary~\ref{c:anyStrongerPhi} it 

suffices to show that

\[ \mbox{G$_3$A}^\omega+\QF\m\AC^{0,0}\vdash 

\forall f\leq 1 \forall x^0\big( \exists z \forall k\geq z\ \big( \gamma(f)_k<2^{-x} \big)

 \rightarrow \exists p\forall z' \phi_0(\widehat{f},x,p,z') \big).

 \]

To prove $\phi,$ fix arbitrary $f^1$ and $x^0$ and suppose that

\[\exists z\forall k\geq z \big( \gamma(f)_k<2^{-x} \big).\]

Moreover, assume towards contradiction

\be[e:CD]

\exists \tilde x\leq x \exists a\geq\max(z,x) \widehat{f}(\tilde x,a)=0. 

\ee

Since $a\ge \tilde{x},z,$  

this implies that $k:=\langle \tilde x, a, \tilde x, a\rangle \ge z$ 

and $\gamma(f)_k=2^{-\tilde x}$, which is a contradiction.\\

Hence we can conclude that

\[

\forall \tilde x\leq x\big( \exists\tilde y<\max(x,z) \widehat{f}(\tilde x,\tilde y)=0\ \vee\ \forall a \widehat{f}(\tilde x,a)\neq 0\big),

\] 

which is equivalent to

\be[e:A1]

\forall \tilde x\leq x \forall a\exists\tilde y<\max(x,z) \ \big(  

\widehat{f}(\tilde x,\tilde y)=0\ \vee\   \forall \tilde{a}\le a 

\,\widehat{f}(\tilde x,\tilde{a})\neq 0\big).

\ee 

Next, set $y:=\max(x,z)$ and assume towards contradiction that

\be[e:CD2]

 \exists \tilde y\leq y_{\widehat{f}, x} \exists a\geq z \widehat{f}(\tilde y,a)=0. 

\ee

Recall that

\be

y_{\widehat{f}, x}:=\begin{cases}

\max\big\{y'\leq y\ :\ \exists x'\le x\,(

y'=\min\{y''\leq y\ :\ 

\widehat{f}(x',y'')=0\, \})\big\},&\text{if such $y'$ exists}\\

x,&\Telse.

\end{cases}

\ee

Note that if $y_{\widehat{f},x}=x$, then $\phi$ follows already from \eqref{e:A1}. Otherwise, denote the smallest $x'\leq x$ for which

$\widehat{f} (x',y_{\widehat{f},x})=0$ by $\tilde x$. Then $k:=\langle 

\tilde x, y_{\widehat{f}, x}, \tilde y, a\rangle \ge z$ and $\gamma(f)_k=2^{-\tilde x}$, which is a 

contradiction.\\

Finally, $\exists p\forall z' \phi_0(\widehat{f},x,p,z')$ follows from not-\eqref{e:CD2} and~\eqref{e:A1} (with $y:=\max(x,z)$, $p:=\langle y, z\rangle$).

\\

\end{proof}



\begin{rmk}\label{metastable} 

Similarly as before, we can give an explicit such rate of metastability for 

the convergence of $(\gamma(f)_z)$ toward $0.$ 

As in Remark~\ref{r:metaB}, there is a $\Phi_fxg\le\Phi^*xg:=

\max\{ g^i(0):i\le x+1\}$ such that 

\[ \forall x,f,g\forall\tilde{x}\le x\,\big(\exists y\le\Phi_fxg \,(

\widehat{f}(\tilde{x},y)=0)\vee \forall z\le g(\Phi_fxg)\,

(\widehat{f}(\tilde{x},z)\not= 0)\big). \]

Define 

\begin{align*}

\Phi_1 xg&:= \Phi_f\Big(x,\lambda y.g_y\big(\Phi_f(y,g_x)\big)\Big),\\

\Phi_2 xg&:= \Phi_f\big(\Phi_1 xg,g_{\Phi_1 xg}\big),

\end{align*}

where $g_y(n):=g(y,n)$. Then

\[ \ba{l}

\forall x,f,g\ \forall\tilde x\leq x\,\forall \tilde{y}\le \Phi_1xg \\[1mm] 

\hspace*{1cm} \left\{ \ba{l} \Big( 

(\exists y\leq \Phi_1 xg \big(\widehat{f}(\tilde x, y)=0\big) \vee 

\forall z\leq g(\Phi_1 xg,\Phi_2xg)\ \big(\widehat{f}(\tilde x,z)\neq0)) \ 

\wedge \\ \hspace*{2mm}

 (\exists u\le\Phi_2xg\,(\widehat{f}(\tilde{y},u)=0)\vee \forall z\le 

g(\Phi_1xg,\Phi_2xg)\,(\widehat{f}(\tilde{y},z)\not= 0)\big) \Big).

\ea \right. \ea \]

This implies

\be[e:rmGU2]

\forall z \big(\Phi_1 xg< z\leq g(\Phi_1 xg,\Phi_2 xg)\rightarrow \forall \tilde x \leq x\ (\widehat{f}(\tilde x,z)\neq0)\big)

\ee

and

\be[e:rmGU3]

\forall z \big(\Phi_2 xg< z\leq g(\Phi_1 xg,\Phi_2 xg)\rightarrow \forall \tilde x \leq \Phi_1 xg\ (\widehat{f}(\tilde x,z)\neq0)\big).

\ee

Define $\tilde g(n):=\max\{g(n),n\}$ and $g_k(n,m):=\tilde g(\langle k,n,n,m\rangle+1)$ and 

\[

\Psi(k,g):=\langle k,\Phi^*_1 kg_k,\Phi^*_1 kg_k,\Phi^*_2 kg_k\rangle+1,

\]

where $\Phi^*_i$ is defined as $\Phi_i$ but with $\Phi^*$ and $(g_k)^{M}$ instead of $\Phi_f$ and $g_k$.

To show

\[

\forall k,g,f\ \exists n\leq \Psi(k,g)\ \forall z\in[n,\tilde g(n)]\ \ (|\gamma(f)_z| < 2^{-k}), \] 

define $n:=\langle k,\Phi_1 kg_k,\Phi_1 kg_k,\Phi_2 kg_k\rangle+1\leq \Psi(k,g)$ and let $z\in[n,\tilde g(n)]$. Then 

$z\geq n>\langle k,\Phi_1 kg_k,\Phi_1 kg_k,\Phi_2 kg_k\rangle$. Hence one of the following holds:

\begin{enumerate}

\item $j_1(z)>k$. Then $|\gamma(f)_z|\leq2^{-j_1(z)}<2^{-k}$.

\item $j_2(z)>\Phi_1 kg_k\wedge j_1(z)\leq k$. Then 

$

\Phi_1 kg_k < j_2(z)\leq z \leq \tilde g(n)=g_k(\Phi_1 kg_k,\Phi_2 kg_k).

$

Hence, by~\eqref{e:rmGU2} (applied to $j_1(z),k,g_k,j_2(z)$ for 

$\tilde{x},x,g,z$), $\gamma(f)_z=0$.

\item $j_3(z)>\Phi_1 kg_k\wedge j_2(z)\leq\Phi_1 kg_k\wedge j_1(z)\leq k$. Then $j_3(z)>j_2(z)$ and so $\gamma(f)_z=0$.

\item $j_4(z)>\Phi_2 kg_k\wedge j_3(z)\leq\Phi_1 kg_k\wedge j_2(z)\leq\Phi_1 kg_k\wedge j_1(z)\leq k$. 

Then 

$

\Phi_2 kg_k < j_4(z)\leq z \leq \tilde g(n)=g_k(\Phi_1 kg_k,\Phi_2 kg_k).

$

Hence, by~\eqref{e:rmGU3} (applied to $j_3(z),k,g_k,j_4(z)$ for 

$\tilde{y},x,g,z$), $\gamma(f)_z=0$.

\end{enumerate}

Note that the 2-nested use of primitive recursive iteration hidden in the 

2-nested application of 

$\Phi_f$ in the definition of $\Phi_i$ very much resembles the basic 

structure of the rate of metastability extracted from a concrete proof 

in ergodic theory in \cite{Safarik(11)} (see the discussion in the 

introduction).

\end{rmk}



\begin{cor}\label{c:gammae}

Let $\gamma(e)_{(\cdot)}$ be defined as $\gamma(f)_{(\cdot)}$ but 

with $\widehat{f}(x,y)=0$ being replaced by 

\[ P(e,x,y):\equiv \mbox{\rm lh} (y)=x+1\wedge y_0=0\wedge\forall i<x(T(e,y_i,y_{i+1})).\]  

Then $(\gamma(e)_n)$ is a sequence of rational numbers in $[0,1]$ that 

converges to $0$ but 

the formula stating the existence of a Cauchy point of 

$\gamma(e)\seq$ for any number $e$  

\[

\psi:=\forall e^0,x^0\exists z \forall k\geq z\ \big( \gamma(e)_k<2^{-x} \big).

\]

is not effectively learnable.

\end{cor}



\begin{proof}

First note that the convergence (towards $0$) 

of $\gamma(e)_n$ follows from this property of $\gamma(f)_n$ 

since taking $f(x,y):=0,$ if $P(e,x,y)$, 

and $f(x,y):=1,$ otherwise, both sequences coincide (note that 

$f=\widehat{f}$).\\ 

Let $e$ by a code of a total recursive function and define $g(x):=\mu 

y\,.\,T(e,x,y).$

The arguments of both Proposition~\ref{p:nonLearnablePhi} and Proposition~\ref{p:gammaf} then remain valid,

except the fact that the set of 

G\"odel numbers $e$ -- in contrast to $f_g$  -- is not majorizable. 

We also need the additional assumption $\forall x\,(T(e,x,g(x)))$ in 

(\ref{e:GA}), which 

expresses that $g(x):=\mu y\,.\,T(e,x,y)]$ defines a total function. This 

assumption, however, does not contribute in the course of the functional 

interpretation argument applied to (\ref{e:GA}) (`Step 2' in the proof of 

Proposition~\ref{p:nonLearnablePhi}) as it is purely universal. 

So as before, the learnability would lead to a constant $D$ and 

total recursive functions $\lambda e,x,y.

L^*_{e,x}(y)$, $B^*$ such that

\[ \forall e\,\forall g\,\left(\forall x^0 \,T(e,x,g(x))\rightarrow 

\forall x^0\ (L^*_{e,x}\circ \tilde{g})^{\widehat{B^*}(e,x)}(0)

\geq g^{g^{x}(0)}(0))\right)\] where 

$\widehat{B^*}(e,x):=(D+x+e)(B^*(x)+1).$  \\ 

We can argue similarly as in the proof of Proposition~\ref{p:nonLearnablePhi}, 

since for any fixed $g$ given by some $e$ as above, eventually we have $x>e$, 

so we can simply choose

a total recursive $g$ which grows much faster than $L^*_{x,x}(x)$ 

and $B^*(x,x)$.

\end{proof}



\begin{cor}\label{c:gammaee}

Define the primitive recursive sequence of rational numbers in $[0,1]$  

(using the Cantor pairing function)

\[

\gamma_{j(e,n)}:= \gamma(e)_n\cdot 2^{-e}.

\]

This sequence converges to $0$ (provably in {\rm 

G$_3$A$^{\omega}+\Sigma^0_1$-IA$^-$} and hence with a primitive recursive -- 

in the sense of Kleene -- rate of metastability) but 

the formula stating the existence of a Cauchy point of $\gamma\seq$

\[

\psi:=\forall x^0\exists z \forall k\geq z\ \big( \gamma_k<2^{-x} \big).

\]

is not effectively learnable.

\end{cor}

\begin{proof} 

Let $\rho_e(x)$ be a rate of convergence for $(\gamma(e)_n)_n$ and define 

$\rho(x):=\max\{ \rho_{\tilde{x}}(x):\tilde{x}\le x\}.$ Then 

\[ \widehat{\rho}(x):=j(x,\rho(x)) \] is a rate of convergence for 

$(\gamma_n)$ towards $0.$

\\

Conversely, for any rate $\rho$ of convergence for $(\gamma_n)$ we have that 

$\rho_e(x):=\rho(e+x)$ is a rate of convergence of $(\gamma(e)_n)_n$. In 

particular, the $2^{-e-x}$-Cauchy property for $(\gamma_n)$ implies the 

$2^{-x}$-Cauchy property of $(\gamma(e)_n)_n.$ Hence by Corollary 

\ref{c:gammae}, 

$(\gamma_n)$ does not have an effectively learnable Cauchy rate.

\end{proof}





\section{When learnability implies fluctuation bounds}





In some cases, the effective $(B,L$)-learnability of a convergence rate 

(meaning that the convergence rate can be learned by $L$ with 

$B(\underline{a})$-many mind 

changes) gives a bound on the number of fluctuations. This, for instance, 

is the case for bounded monotone sequences. In general, we can say that

effective learnability implies the existence of an

effective bound of fluctuations, if the learner and the sequence satisfy

certain gap conditions. Informally, if

\begin{itemize}

\item any two exceptions $i,\tilde{\imath}$ to the Cauchy property for $2^{-k}$ have distance at least $\Delta_*(\max(i,\tilde{\imath}),k)$, and

\item the learning map jumps at most by $J^*(i,k)$ from $i$, and

\item (for any $k$) $J^*(\cdot, k)$ is asymptotically at most equivalent to $\Delta_*(\cdot, k)$, 

\end{itemize}

then the number of fluctuations is asymptotically bounded by $B^*$. Below, we discuss an example from ergodic theory, where these conditions are met.



\begin{prop}[{Gap conditions on the learner}]\label{p:gap}

Let $a\seq$ be some sequence in a metric space $(X,d)$ and let 

\[

\phi:\equiv \forall k\exists n\forall i\ \overbrace{\forall \tilde{\imath}<i \big( n\leq \tilde{\imath}\rightarrow d(a_{\tilde{\imath}},a_{i})\leq 2^{-k}\big)}^{\phi_0(i,n,k):\equiv}.

\] 

be a $(B,L)$-learnable formula (which states simply the Cauchy property 

of the sequence, we use $\tilde{\imath}$ simply because the natural choice $j$ already denotes the pairing function) and let $B^*,L^*$ be

majorants of $B,L.$ \\

Moreover, suppose that there are functions $\Delta_*>0$, $J^*$, s.t.

\[\forall n,i,i'\ \Big( \big( \neg\phi_0(i,n,k)\wedge\neg\phi_0(i',n,k))\rightarrow

|i'-i|\geq\Delta_*(i,k)\Big),\]

\[\forall n,i \big( \neg\phi_0(i,n,k) \rightarrow L^*(i,k)-i\leq J^*(i,k)\big),\]

and

\be[e:O]

\big( J^*\in\mathcal{O}\big(\Delta_*\big)\big)\equiv \forall k\exists N_k,K_k\ 

\forall x\geq N_k\ \big( 

K_k\Delta_*(x,k)\geq J^*(x,k)\big).

\ee

Then there is a bound on the number of $2^{-k}$-fluctuations,

which is primitive recursive 

in $B^*,\Delta_*,J^*$ and $N_k,K_k$ (which witness~\eqref{e:O}) given by

\[b(k):=B^*(k)\bigg(2+K_k+\max_{n<N_k}\Big(\frac{J^*(n,k)}{\Delta_*(n,k)}\Big)\bigg).\]

Note that in the case where $N_k=0,$ we get

\[b(k)=(2+K_k)B^*(k).\]

\end{prop}

\begin{proof}

For simplicity, assume $N_k=0$ (otherwise we could just replace every occurrence 

of $K_k$ by $(K_k+\max_{n<N_k}(\frac{J^*(n,k)}{\Delta_*(n,k)}))$).\\

Firstly, note that any $2^{-k}$-fluctuation between two indexes $\tilde{\imath}$ and $i$ corresponds to a counterexample $i$. So, by definition there 

is at most one fluctuation in the interval $[c_l,i_l]$, where 

$i_l$ is the smallest counterexample to the solution 

candidate $c_l$. Moreover, if there is such a fluctuation, its greater index is $i_l$.\\

Secondly, from our assumption on $\Delta_*$ we get that there are at most

\[

\left\lceil \frac{c_{l+1}-i_l}{\Delta_*(i_l,k)} \right\rceil \leq \left\lceil \frac{J^*(i_l,k)}{\Delta_*(i_l,k)} \right\rceil \leq K_k

\]

many fluctuations within an interval $[i_l,c_{l+1}]$.\\

There are at most $B^*(k)$ such pairs of intervals, before a $2^{-k}$-Cauchy point is reached, but there might be fluctuations, which arise only when we

unite two such intervals.\\

By incrementing $K_k$ by $1$, we already account for any additional fluctuation due to

combining the intervals $[c_l,i_l]$ and $[i_l,c_{l+1}]$. This is because if there was a fluctuation within $[c_l,i_l]$, then there cannot be an additional one which results from combining such a pair of intervals, as its greater index would be $i_l$. There can, however, be an additional fluctuation, when we combine the intervals $[i_l,c_{l+1}]$ and $[c_{l+1},i_{l+1}]$.\\

\end{proof}



We now consider the general form of the structure of Birkhoff's proof 

of the mean ergodic theorem as analyzed in \cite{Kohlenbach/Leustean4} and 

the argument used in \cite{Avigad/Rute} to convert the rate of metastability 

obtained in \cite{Kohlenbach/Leustean4} into a bound on the number of 

fluctuations: \\[1mm]

Let $x_{(\cdot)}$ be a sequence in some normed space $X$ (in the case at hand 

$X$ is a uniformly convex Banach space) and $y_{(\cdot)}$ be a sequence in $\RR_+$ definable by terms in $\ha[X,\|\cdot\|,\ldots].$ 

Suppose the Cauchyness of $x_{(\cdot)}$ 

is proved using that $y_{(\cdot)}$ has arbitrarily good approximate 

infima, i.e.

\begin{align}

\forall \delta>0\exists n &\forall k \forall \tilde k\le k\ (y_{\tilde k}\geq y_n-\delta)\\

&\rightarrow \forall \epsilon>0\exists m \forall u \forall i,j\in[m,u]\ (\|x_i-x_j\|\leq \epsilon).

\end{align} 

This implication is classically equivalent to 

\begin{align*}

\forall \epsilon>0 \exists \delta>0\ \Big( \exists n &\forall k \forall \tilde 

k\le k\ (y_{\tilde k}\geq y_n-\delta)\rightarrow \\ \tag{+}\label{e:U2}

&\exists m \forall u \forall i,j\in[m,u]\ (\|x_i-x_j\|\leq \epsilon) \Big).

\end{align*}

Suppose now that we are in the situation of Corollary \ref{cor.2.11}, i.e.  

\[

\ha[X,\|\cdot\|,\ldots]+\mbox{AC$+$M$^{\omega}+$IP}^{\omega}_{\forall} 

\vdash\text{\eqref{e:U2}}

\]

then

\begin{align*}

\ha&[X,\|\cdot\|,\ldots]+{\rm AC}+{\rm M}^{\omega} +{\rm IP}^\omega_\forall \vdash\\

&\forall\epsilon>0\exists\delta>0\forall n\exists m\geq n\forall u\exists k(\forall \tilde k\le k(y_{\tilde k}\geq_\RR y_n-\delta)

\rightarrow \forall i,j\in[m,u](\|x_i-x_j\|<_\RR\epsilon).

\end{align*}

Hence by monotone functional interpretation one extracts terms 

 $\delta_\epsilon>0$, 

$m_\epsilon$ and $k_\epsilon$ (depending additionally only 

on majorants of the parameters $\underline{a}$ used in the definition of 

our sequences) s.t. (valid in 

${\cal S}^{\omega,X}$) for all majorants $\underline{a}^*$ of $\underline{a}$ 

\[ \hspace*{-5mm} \ba{l} 

\forall \epsilon>0\ \forall n,u\\[1mm] 

\bigg( m_\epsilon(n)\ge n \wedge \Big(\forall \tilde k\le k_\epsilon(n,u)\ (y_{\tilde k}\geq y_n-\delta_\epsilon)\rightarrow 

 \forall i,j\in[m_\epsilon(n),u]\ (\|x_i-x_j\|\leq \epsilon) \Big)

\hspace*{-1mm} \bigg).

\ea  \hspace*{-1mm}\tag{$*$}\label{e:U4-me}\]

Now define $k^*_\epsilon(u):=\max\{k_\epsilon(i,u)\ :\ i\leq u\}$ and consider

\[

\forall \epsilon>0\ \forall n,u\ \Big( \forall \tilde k\le k^*_\epsilon(u)\ (y_{\tilde k}\geq y_n-\delta_\epsilon)\rightarrow 

 \forall i,j\in[m_\epsilon(n),u]\ (\|x_i-x_j\|\leq \epsilon) \Big).

\tag{$**$}\label{e:U4}\]

We can infer \eqref{e:U4} from \eqref{e:U4-me} by the following case 

distinction:\footnote{We are grateful to P. Oliva for pointing this out to us.}

Fix $\varepsilon >0$ and $n.$ 

\begin{itemize}

\item[Case]  1: $u<m_\epsilon(n)$. Then the conclusion and hence the whole implication is trivially true.

\item[Case]  2: $u\geq m_\epsilon(n) \geq n$.Then $k^*_\epsilon(u)\geq 

k_\epsilon(n,u)$ and so $\forall \tilde k\le k^*_\epsilon(u)\ 

(y_{\tilde k}\geq y_n-\delta_\epsilon)$ implies $\forall \tilde k\le 

k_\epsilon(n,u)\ (y_{\tilde k}\geq y_n-\delta_\epsilon)$ and so the claim 

follows as well.

\end{itemize}



Now suppose w.l.o.g. that $k^*_\epsilon:\NN\to\NN$ is injective and for any given $u$ define

\[

l_u\ :=\ (k^*_\epsilon)^{-1}(u).

\]

Then \eqref{e:U4} applied to $u:=l_u$ yields

\[

\forall \epsilon>0\ \forall n,u\ \Big( \forall \tilde k\le u\ 

(y_{\tilde k}\geq y_n-\delta_\epsilon)\rightarrow 

 \forall i,j\in[m_\epsilon(n),(k^*_\epsilon)^{-1}(u)]\ 

(\|x_i-x_j\|\leq \epsilon) \Big).

\tag{-}\label{e:U5}\]

Now let $N_0$, $N_1$, ..., $N_{S_\epsilon}$ be integers s.t. $N_0=0$ and $N_{i+1}$ is the least $m>N_i$ s.t. $y_m<y_{N_i}-\delta_\epsilon$ as long as such an $m$ exists.

Assume that $b\geq y_0$ (for some $b$) and so $S_\epsilon\leq\frac{b}{\delta_\epsilon}$.\\

By \eqref{e:U5} there are no $\epsilon$-fluctuations of $x_{(\cdot)}$ on the 

$S_\epsilon$ many intervals $[m_\epsilon(N_i),(k^*_\epsilon)^{-1}

(N_{i+1})]$ for $i=0,\ldots,S_{\varepsilon}-1$.\\

In the intervals $[(k^*_\epsilon)^{-1}(N_{i}),m_\epsilon(N_i)]$ 

for $i=1,\ldots,

S_\epsilon$ and $[0,m_\epsilon(N_0)]$ we have to show that if we have for 

any $N\in\NN$ $s$ many

fluctuations indexed within $[(k^*_\epsilon)^{-1}(N),m_\epsilon(N)]$ 

(or in $[0,m_\epsilon(N_0)]$) each indexed by a pair of indexes $(i,j)$ 

then the highest index of such fluctuation ($j_s$) has to be greater than 

(or equal to) some $\phi_\epsilon(s,N)$, where $\phi_\epsilon$ is such that

\[

\exists \tilde s\forall n\ \big(\phi_\epsilon(\tilde s,n)>m_\epsilon(n)\big).

\]

Then, given such an $\tilde s$, we have at most

\[

\frac{b}{\delta_\epsilon}+\tilde s\left(\frac{b}{\delta_\epsilon}+1\right)

\]

many fluctuations.

\\[2mm]

In the case of Birkhoff's proof, the analysis in \cite{Kohlenbach/Leustean4} 

and the discussion in \cite{Avigad/Rute} 

gives the following data used in \cite{Avigad/Rute}: 

\begin{align*}

\delta_\epsilon &:= \frac{\epsilon^2}{512b},& m_\epsilon(n)&:=\left\lceil 

\frac{16b}{\epsilon}\right\rceil n,\\

(k^*_\epsilon)^{-1}(n)&:=\left\lfloor\frac{n}{2}\right\rfloor,& 

\phi_\epsilon(s,n)&:=\big(1+ \frac{\epsilon}{2b}\big)^s n,

\end{align*}

and so (for $\varepsilon <2b$) \[ \tilde s \leq \frac{ 4 b\log \big\lceil 

\frac{16b}{\epsilon}\big\rceil }{\epsilon}.\]

The function $\varphi_{\varepsilon}$ results (see \cite{Avigad/Rute} for 

the calculation) from the fact that 

\[ \| x_ {n+k}-x_k\|\le 2n\| x\|/(n+k) \]

which is established already in Birkhoff's proof and which -- for $n=1$ -- 

shows that $(x_k)$ has a linear rate of asymptotic regularity.

\begin{rmk}

Naturally, we could use the data, which led to the bound 

of $\tilde s$ above, also simply with Proposition~\ref{p:gap} to

obtain a similar fluctuation bound (which has the same structure in $\epsilon$).

\end{rmk}



For the case of Halpern iterations (with scalar $1/(n+1)$) mentioned in 

the introduction, the analysis given in \cite{Kohlenbach/Leustean6}   

yields (roughly) the following data for Hilbert spaces $X$ 

(see \cite{Kohlenbach/Leustean6,Addendum} for the 

detailed definition of $\Theta_n$):

\begin{align*}

\delta_\epsilon &:= \frac{\epsilon^4}{576(b+1)^4},& m_\epsilon(n)&\approx \Theta_n(\frac{\epsilon^2}{4})\approx n^2,\\

(k^*_\epsilon)^{-1}(n)&:=\left\lfloor\frac{n\cdot\epsilon}{3b^2}

\right\rfloor.& 

\end{align*}

Similar data are also obtained  

in the recent \cite{Koernlein} which is based on the analysis of a different 

proof for the strong convergence of the Halpern iteration from 

\cite{Xu}.\\

However, now the rate of asymptotic regularity roughly is of order (see 

corollary 6.3 in \cite{Kohlenbach/Leustean6}) 

\[ \| x_{k+1}-x_k\| \le \frac{b}{\sqrt{k}}, \] 

which does not lead to a linear (in $n$) $\varphi_{\varepsilon}(s,n)$ and 

even if it would, this would not suffice to dominate $m_{\varepsilon}(n).$ 

So as it stands, the analysis does not seem to yield any fluctuation 

bound for the Halpern iteration $(x_k).$



\begin{prop}

Given a bound $B_{\varepsilon}$ on the number of fluctuations, there is an in $B_{\varepsilon}$ (and the given data 

$(k^*_{\varepsilon})^{-1}$ and $m_{\varepsilon}$ and the majorants $\underline{a}^*$ of their parameters including $\varepsilon)$) 

primitive recursive $\phi_\epsilon$

satisfying the conditions in the proof:

\begin{align}

\forall \varepsilon >0 \forall s,N,i,j \big( i,j\in[(k^*_\epsilon)^{-1}(N),

m_\epsilon(N)]\ \wedge\ \Fluc_{\varepsilon}(s,i,j)\ 

\rightarrow j_s\geq \phi_\epsilon(s,N) \big),\label{e:FU1}\\

\exists \tilde s\forall n\ \big(\phi_\epsilon(\tilde s,n)>m_\epsilon(n)\big).\label{e:FU2}

\end{align}

\end{prop}

\begin{proof}

Set

\[

\phi_\epsilon(s,n):=\begin{cases}

(k^*_\epsilon)^{-1}(n)&\Tif\ s\leq B_{\varepsilon},\\

m_\epsilon(n)+1&\Telse.

\end{cases}

\]

\end{proof}

\vspace*{-3mm}

\begin{rmk}

In particular, this means that if we know there is for computable 

(in the majorants $\underline{a}^*$ of the parameters including $\varepsilon$) 

$(k^*_{\varepsilon})^{-1}$ and $m_{\varepsilon}$ 

no computable $\phi_{\varepsilon}$ (in $\tup a^*$) satisfying these conditions,

then there cannot be a bound on the number of fluctuations, which is computable (in $\tup a^*$).

\end{rmk}





{\bf Acknowledgements:} 

This research was supported by the German Science Foundation (DFG Project KO 1737/5-1). 



\vspace*{-3mm}

\begin{thebibliography}{00}

\bibitem{Akama} Akama, Y., Berardi, S., Hayashi, S., Kohlenbach, U., 

An arithmetical hierarchy of the law of excluded middle and related 

principles. Proc. of the 19th Annual IEEE Symposium on Logic in 

Computer Science (LICS'04), pp. 192-201, IEEE Press (2004).

\bibitem{Artemov} Artemov, S., Explicit provability and

constructive semantics. Bull. Symbolic Logic {\bf 7}, pp. 1-36 (2001).

\bibitem{Artemov11} Artemov, S., Yavorskaya, First-order logic of proofs. 

Technical Report TR-2011005, Cuny PhD Program in Computer Science 2011.

\bibitem{Aschieri1} Aschieri, F., A contructive analysis of learning 

in Peano Arithmetic. To appear in: Ann. Pure Appl. Logic.

\bibitem{Aschieri2} Aschieri, F., Learning based on realizability 

for HA+EM1 and 1-Backtracking games: Soundness and completeness. 

To appear in: Ann. Pure Appl. Logic.

\bibitem{Aschieri/Berardi} Aschieri, F., Berardi, S., 

A new use of Friedman's translation: interactive realizability.  

in: Logic, Construction, Computation, Ontos-Verlag Series in 

Mathematical 

Logic, Berger et al. editors, 2012

\bibitem{Avigad/Gerhardy/Towsner} Avigad, J., Gerhardy, P., Towsner, H., 

Local stability of ergodic averages. Trans. Amer. 

Math. Soc. {\bf 362}, pp. 261-288 (2010). 

\bibitem{Avigad/Iovino} Avigad, J., Iovino, J., Ultraproducts and 

metastability. arXiv:1301.3063, Preprint 10pp., 2013.

\bibitem{Avigad/Rute} Avigad, J., Rute, J., Oscillation and the 

mean ergodic theorem. arXiv:1203.1743, Preprint 7pp., 2012.

\bibitem{Baillon(75)} Baillon, J.B., Un th\'eor\`eme de type ergodique 

pour les contractions non lin\'eaires dans un espace de Hilbert. 

C.R. Acad. 

Sci. Paris S\`er. A-B {\bf 280}, pp. 1511-1514 (1975).

\bibitem{Baillon(76)} Baillon, J.B., Quelques propri\'et\'es de 

convergence asymptotique pour les contractions impaires. 

C.R. Acad. Sci. Paris S\`er. A-B {\bf 283}, pp. 587-590 (1976). 

\bibitem{Berardi/Coquand/Hayashi} Berardi, S., Coquand, T, Hayashi, S., 

Games with 1-Backtracking. Ann. Pure Appl. Logic {\bf 161}, pp. 

1254-1264 (2010).

\bibitem{Bir39} Birkhoff, G., 

The mean ergodic theorem. Duke Math. J. {\bf 5}, pp. 19-20 (1939). 

\bibitem{Coquand} Coquand, T., A semantics of evidence for classical

arithmetic. J. Symbolic Logic {\bf 60}, pp. 325-337 (1995).

\bibitem{Friedman(78)} Friedman, H., Classical and intuitionistically

provably recursive functions. In: M\"uller,G.H., Scott, D.S. (eds.),

Higher Set Theory, pp. 21-27. Springer LNM {\bf 669} (1978).

\bibitem{GerKoh06}

Gerhardy, P., Kohlenbach, U.,  

Strongly uniform bounds from semi-constructive proofs, Ann. Pure Appl. 

Logic {\bf 141}, pp. 89-107 (2006).

\bibitem{Gold(67)} Gold, E. M., Language identification in the limit.

Information and Control {\bf 10}, pp. 447-474 (1967).

\bibitem{Hayashi02} Hayashi, S., Mathematics based on learning.

Proc. 13th Internat. Conf. ALT 2002, Springer Lecture Notes in 

Artificial Intelligence Vol. 272, pp. 7-21 (2002).

\bibitem{Hayashi06} Hayashi, S., Mathematics based on incremental learning -- 

Excluded middle and inductive inference Original Research Article

Theoretical Computer Science {\bf 350}, pp. 125-139 (2006) (Journal version 

of \cite{Hayashi02}).

\bibitem{Hayashi/Nakata} Hayashi, S., Nakata, M., Towards limit computable 

mathematics. In: P. Callaghan et al. (eds.), TYPES 2000, Springer 

LNCS {\bf 2277}, pp. 125-144 (2002).

\bibitem{Higuchi/Kihara} Higuchi, K., Kihara, T., Inside the Muchnik 

degrees: discontinuity, learnability, and constructivism. Preprint 2012. 

\bibitem{Jones} Jones, R.L., Ostrovskii, I.V., Rosenblatt, J.M., 

Square functions in ergodic theory. Ergodic Theory and Dynamical 

Systems {\bf 16}, pp. 267-305 (1996).

\bibitem{Kohlenbach(lowrate)} Kohlenbach, U., Mathematically strong subsystems of analysis with low rate of growth of provably recursive functionals. Arch. Math. Logic {\bf 36}, pp. 31-71 (1996). 

\bibitem{Kohlenbach(relative)} Kohlenbach, U., Relative constructivity.

J. Symbolic Logic {\bf 63}, pp. 1218-1238 (1998). 

\bibitem{Kohlenbach(metapaper)} Kohlenbach, U., Some logical

metatheorems with applications in functional analysis. 

Trans. Amer. Math. Soc. {\bf 357}, no. 1, pp. 89-128 (2005).

\bibitem{Kohlenbach(book)} Kohlenbach, U., Applied Proof Theory: 

Proof Interpretations and their Use in Mathematics. 

Springer Monographs in Mathematics. xx+536pp., Springer 

Heidelberg-Berlin, 2008.

\bibitem{Kohlenbach(Browder)} Kohlenbach, U.,On quantitative 

versions of  theorems due to F.E. Browder and R. Wittmann. Advances in 

Mathematics {\bf 226}, pp. 2764-2795 (2011).   

\bibitem{Kohlenbach(Baillon)} Kohlenbach, U., A uniform quantitative form of 

sequential weak compactness and Baillon's nonlinear ergodic theorem. 

Communications in Contemporary Mathematics {\bf 14}, 20pp. (2012). 

\bibitem{Kohlenbach/Leustean4} Kohlenbach, U., Leu\c{s}tean, L., A 

quantitative mean ergodic theorem for uniformly convex Banach spaces. 

Ergodic Theory and Dynamical Systems {\bf 29}, pp. 1907-1915 (2009). 

\bibitem{Kohlenbach/Leustean3} Kohlenbach, U., Leu\c{}stean, L., 

Asymptotically nonexpansive mappings in uniformly convex hyperbolic spaces. 

Journal of the European Mathematical Society {\bf 12}, pp. 71-92 (2010). 

\bibitem{Kohlenbach/Leustean6} Kohlenbach, U., Leu\c{s}tean, L., 

Effective metastability of Halpern iterates in CAT(0) spaces.

Adv. Math. {\bf 231}, pp. 2526-2556 (2012).

\bibitem{Kohlenbach/Leustean7} Kohlenbach, U., Leu\c{s}tean, L., 

On the computational content of convergence proofs via Banach limits. 

Philosophical Transactions of the Royal Society A  {\bf 370}, 

pp. 3449-3463 (2012). 

\bibitem{Addendum} Kohlenbach, U., Leu\c{s}tean, L., 

Addendum to \cite{Kohlenbach/Leustean6}. 2pp. (2013).

\bibitem{Koernlein} K\"ornlein, D., Analysis of a proof due to Xu. 

Preprint 2013.

\bibitem{Luckhardt(89)} Luckhardt, H., Herbrand-Analysen zweier Beweise

des Satzes von Roth: Polynomiale Anzahlschranken. J. Symbolic Logic

{\bf 54}, pp. 234-263 (1989).

\bibitem{Saejung} Saejung, S., Halpern's iteration in CAT(0) spaces. 

Fixed Point Theory and Applications {\bf 2010} (2010).

\bibitem{Safarik(11)} Safarik, P., A quantitative nonlinear strong 

ergodic theorem for Hilbert spaces. J. Math. Anal. Appl. {\bf 391}, 

pp. 26-37 (2012).

\bibitem{Sieg} Sieg, W., Fragments of arithmetic. Ann. Pure Appl. Logic. 

{\bf 28}, pp. 33-71 (1985).

\bibitem{Specker(49)} Specker, E., Nicht konstruktiv beweisbare S\"atze der 

Analysis. J. Symb. Logic {\bf 14}, pp. 145-158 (1949).

\bibitem{Toftdal} Toftdal, M., Calibration of Ineffective Theorems of 

Analysis in a Constructive Context. Master Thesis, Aarhus Universitet, 

2004.

\bibitem{Wittmann(90)} Wittmann, R., Mean ergodic theorems for nonlinear 

operators. Proc. Amer. Math. Soc. {\bf 108}, pp. 781-788 (1990).

\bibitem{Wittmann(92)} Wittmann, R., Approximation of fixed points of 

nonexpansive mappings. Arch. Math. {\bf 58}, pp. 486-491 (1992).

\bibitem{Xu} Xu, H.-K., Iterative algorithms for nonlinear operators.

J. London Math. Soc. {\bf 66}, pp. 240-256 (2002).

\bibitem{Ziegler(07)} Ziegler, M., Real hypercomputation and continuity. 

Theory of Computing Systems vol.41, pp. 177-206 (2007).

\end{thebibliography}



\end{document}