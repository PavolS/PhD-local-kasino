\section*{Introduction}
\label{s:intro}
\markboth{\small\sffamily\fontseries{c}\selectfont {Introduction}}
         {\small\sffamily\fontseries{c}\selectfont {Introduction}}
\addcontentsline{toc}{section}{Introduction}
%
%
\mySubsection{Foreword}
In \cite{Tao07}, a recent Fields medallist, Terence Tao, 
published some thoughts on the relation between
 ``hard (quantitative, finitary)'' and ``soft (qualitative, 
infinitary)'' analysis. In this essay, he also emphasized
the importance of the so called ``hard'' analysis:
\theQuote{ ... I therefore feel that it is often profitable for a practitioner 
of one type of analysis to learn about the other, as they both offer their own 
strengths, weaknesses, and intuition, and knowledge of one gives 
more insight into the workings of the other. ... }
His point is illustrated by the proof of Theorem~1.6 in~\cite{tao-2007},
which is carried out in the context of ``finitary ergodic theory'': 
\theQuote{ ... The main advantage of working in a finitary setting ... is that
the underlying dynamical system becomes extremely explicit. ... }
He goes on to connect the finitisation to the methods we will employ
in this thesis:
\theQuote{ ... In proof theory, this finitisation is known as G\"odel functional
interpretation ... }
In the case of convergence theorems Tao calls the finitary formulation
metastability and the corresponding explicit content rate(s) of metastability.
It was recently observed by Avigad and Rute \cite{Avigad/Rute} that in the 
case of the von Neumann mean ergodic theorem a particular such rate of metastability 
(extracted in \cite{kohlenbachleustean09}) can be used to obtain even a simple effective (and 
also highly uniform) bound on the number of fluctuations (for the case of Hilbert spaces this 
was already obtained with an even better bound in \cite{Jones}).\\
This leads us to a very natural question: What kind of effective bounds (explicit information) is extractable
under which general logical conditions on convergence proofs.  \\
In this thesis we discuss four natural kinds of such finitary information and analyze the corresponding
conditions.\\
While the results we obtain in the process explain a very common form of realizers for strong ergodic theorems
in seemingly unrelated logical circumstances, there is a notable exception to this pattern published in~\cite{Safarik(11)}. We will see how the nested realizer from~\cite{Safarik(11)} relates to 
a separating example for two of our kinds of finitary information mentioned above.\\
Once we established this basis for computational content extraction, we also investigate, so to say, the opposite approach. 
Namely, given the fact that one is interested in such extraction, the question arises in which systems this is still possible and how.
We answer this question for a very prominent case in the last chapter, where we introduce an intrinsic formalisation
of non-standard analysis and discuss the extractability of effective bounds from proofs in that system.

%As follows from the soundness of the N(M)D-interpretation (see theorems \ref{t:sFI} and \ref{t:mfi})
%the precise realizing terms of the Bolzano-Weierstra{\ss} theorem, which we derive in this thesis,  
%constitute the computational contribution of this theorem in proofs.\\
%As it turns out, in contrast to Tao's concern indicated by his reference
%to the Paris-Harrington theorem (see~\cite{PH77}), this contribution is comfortably low. 
%In many contexts, which include systems formalizing large parts of Analysis, 
%%this contribution is covered by the $<\nolinebreak[4]\!\!\epsilon_0$\nbd recursion and, in some cases,
%this contribution is covered by \lOrdm{\epsilon_0}re\-cur\-sion or, in some cases,
%%even by $<\!\!\omega^{\omega^{\omega}}$-recursion.
%even by \lOrdm{\omega^{\omega^{\omega}}}re\-cur\-sion (see theorems \ref{t:PEfBW} and \ref{t:PEBW}). 
%Furthermore, it follows
%from Kohlenbach's results in \cite{Kohlenbach98} that for very weak systems 
%the contribution can in fact be covered by \lOrdm{\omega^{\omega}}re\-cur\-sion.
%
\mySubsection{Scientific Context}
At least since the presentation of G\"odel's functional 
interpretation (also called Dialectica or simply ``D-'' interpretation)
by K. G\"odel in 1958, proof theoretic methods can be used not only simply 
to analyze the syntactic structure of proofs, 
but also to deliver new information about the mathematical theorems being proved. The
first to formulate this idea of ``unwinding proofs'' was G.~Kreisel: 
\theQuote{ What more do we know if we have proved a theorem by 
  restricted means than if we merely know that it is true. }
%In our case, as mentioned above, it will be the theorem of Bolzano-Weierstra{\ss}.
%The first work in this new direction is also due to G. Kreisel in early 50's and
%his no-counterexample interpretation of Peano arithmetic (further $\PA$) experiencing
%a rather great interest and development thereafter. He used
%W. Ackermann's $\epsilon$-substitution method to prove
%that for any theorem of the first-order $\PA$ one can find ordinal recursive functionals,
%of order type \lOrd{\epsilon_0}, which realize the theorem's Herbrand
%normal form (see \cite{Kreisel51}, \cite{Kreisel52}).\\
Kreisel's ``Unwinding of Proofs'' 
developed into a new field of Mathematical Logic providing many new 
results in Algebra, Number Theory, and Numerical and Functional Analysis. 
To describe better the character and aim of this field, D. Scott suggested to call
this discipline \defkey{Proof Mining} and the new name was quickly adopted.\\
In this thesis, we make mainly use of the above mentioned functional interpretation (in various forms, 
we will even give a new variant suitable for our formalization of non-standard analysis in the last chapter).
G\"odel's functional interpretation requires an \em{intuitionistic} proof, 
(also called a \em{constructive} proof), which is -- roughly speaking -- a proof which doesn't make any use
of the law of excluded middle:
\[ \usftext{LEM}\quad:\quad \phi \vee \neg \phi\text{.} \]
Of course, one often works with semi-intuitionistic (semi-constructive) systems, 
where $\usftext{LEM}$ is only partially added, e.g. only for negated formulas (i.e. formulas in the form $\neg\phi$) -- denoted by $\usftext{LEM}_\neg$ -- or for formulas with only one, existential quantifier (i.e. formulas in the form $\exists\phi_\QF$) -- denoted by $\LEM$.
An intuitionistic proof can be obtained from a classical one by applying the negative translation
as a pre-processing step. The negative translation also goes back to 
G\"odel who defined a transformation of 
classically provable arithmetical formulas into classically equivalent and 
intuitionistically provable formulas in early 30's. We will use a 
modified version of this interpretation as 
presented by Kuroda in 1951 (see definition \ref{d:NT}).
The application of the combination of these interpretations, 
i.e. first the negative translation and then
the functional interpretation, is often referred to as the 
ND-interpretation (ND for negative D-interpretation). We will discuss all of this
in great detail in the first chapter and partially in the next paragraph.
For even more thorough and self-contained presentation of all these topics,
see U. Kohlenbach's book \cite{Kohlenbach08}. \\
% At this point, it is important to mention a result of C. Spector \cite{Spector62}.
In 1962, Spector \cite{Spector62} defined a particularly simple form of bar recursion (recursion on well-founded trees), which is sufficient
to solve the functional interpretation of (negative translation of) 
the schema of full comprehension over numbers:
\[
\CA^0\ :\quad \exists f:\NN\to\NN\forall x\in\NN\ \big(f(x)=_\NN 0 \leftrightarrow \phi(x)\big)\text{.}
\]
One could call this result a major break-through for proof mining, since it suffices to
prove most of classical analysis within a simple logical system based on Peano arithmetic (again see also \cite{Kohlenbach08}).\\
In particular, with Spector's bar recursion one can extract realizers (using ND-interpretation) for almost all
theorems of classical analysis (given their formal proof).\\
In fact, for many theorems the existence of a uniform bound on the realizers is guaranteed by Kohlenbach's metatheorems
introduced in~\cite{Kohlenbach05meta} and refined in~\cite{GK08}. Additionally, 
proof theoretic methods such as Kohlenbach's monotone functional interpretation (see~\cite{Kohlenbach96mfi}) can be used to
systematically obtain such effective bounds from their proofs, which tend to be a lot simpler than the actual realizers, though they remain
equally useful (again \cite{Kohlenbach08} is a very good reference for a comprehensive discussion on this topic). \\
We use Spector's solution together with Kohlenbach's monotone interpretation of Weak K�nig's Lemma (WKL, see~\ref{l:WKL-Feferman} in first chapter or for more details and background
e.g. \cite{Kohlenbach92, Kohlenbach08, Howard81, Troelstra74}) to give the ND-interpretation
of the Bolzano-Weierstrass theorem:
\[
 \BW_\RR\quad:\quad
  \forall f^{1(0)}
      \underbrace{ \exists a^1\forall k^0\exists l^0\geq_0 k
           \ |\hat a-_\RR\widetilde {fl\ }|\leq_\RR (\lambda n.\langle 2^{-k}\rangle) }_{\equiv:\BW_\RR(f^{1(0)})}
\text{.}
\] \\
The computational contribution of (instances of) BW and other principles 
describing sequential compactness
has been analyzed by Kohlenbach in \cite{Kohlenbach98} for proofs in somewhat 
weaker base systems: the Grzegorczyk arithmetic of level $n$, \defkeyn{$\GA$}. 
Kohlenbach and Oliva gave bounds for the bar-recursive realizers of the 
Principle of monotone convergence, $\PCM$, in the last section of \cite{KO02} and
discuss various general scenarios, where $\PCM$ has provably significantly smaller 
computational contribution.\\ 
Still, it is a bit surprising that the computational contribution of 
this basic theorem was first fully (in the above sense) analyzed in \cite{Safarik08}.
%? cite KohlenbachSafarik10
Thereafter the interest in BW's computational content continued. It was analyzed as part of
two different larger projects. Encouragingly, as far as the results can be compared, in both cases
the computational content corresponds very closely to the one obtained in \cite{Safarik08}. \\
The more recent of these results were obtained by P. Oliva and T. Powell in \cite{OP12}, where the authors
take the same approach as in \cite{Safarik08}. The main difference is, that
Spector's bar recursion is replaced by the use of selection functions. The authors
show the equivalence of unbounded products of selection functions and bar-recursion in \cite{OP12br}
and demonstrate how this can be used to give a different formalization
of the realizers in \cite{Safarik08}. Moreover, they give a very nice game-theoretic
interpretation of the realizers which allows for better understanding and intuition
for he original bar-recursive solutions.\\
On the other hand, V. Brattka analyzed the BW theorem as part of the Weihrauch-Lattice (see e.g. \cite{BG11, HP11})
in \cite{BGM12}. Though using completely different techniques, also he arrives at the conclusion
that while using BW in its full generality in a proof implies very strong forms of recursion, when used
in a more natural way, i.e. applied as a single instance of the principle, the
complexity is rather low.\\
In general, the proof mining experience shows that very often proofs of well known and influential theorems
do not need this kind of strong principles, i.e. principles like $\CA^0$ requiring more than normal recursion 
(though in most cases the proofs themselves are formulated in ways, which would suggest otherwise).\\
In \cite{AGT10}, J. Avigad, P. Gerhardy and H. Towsner analyzed the mean ergodic theorem,
implicitly showing that, when proving that the sequence of averages is Cauchy (rather than its
full convergence), it can be proved with strictly arithmetical
principles\footnote{more precisely, with arithmetic principles and arithmetic versions of analytical and higher principles},
i.e. statements about numbers without the need of a specific true higher type object like an actual 
cluster point for a sequence as in the BW theorem (please refer to section~\ref{s:ArProof} for more details on arithmetization in this context and
to \cite{Kohlenbach98} in general). \\
%Wittmann intro
Shortly after \cite{AGT10}, Kohlenbach and Leu{\c s}tean gave a more efficient analysis of MET in \cite{kohlenbachleustean09},
which was further investigated by 
Avigad and Rute in \cite{Avigad/Rute}, where they derive the above discussed
bound on the number of fluctuations (roughly speaking a bound on how many times does the distance 
between the values of the sequence exceed a given $\epsilon$, see Definition~\ref{d:nfluc}).
Such a bound gives us, of course, much more than simply a rate of metastability. The question under which
circumstances can a fluctuation bound be (systematically) obtained -- hoping for a similarly clear-cut answer
as in the case of metastability -- was answered in~\cite{KS13} and led to a new but very natural and straight-forward
definition of (effective) learnability. As expected, the kind of computational information we get, strongly depends on the ``amount of classicallity'' we use. Or in other words, how much of $\usftext{LEM}$ was truly needed to prove a given theorem.  \\
Interestingly, to some extent, we get a bit of ``classicallity''
in constructive formalization(s) of non-standard analysis as well (see e.g.~\cite{avigadhelzner02}).
Of course, this makes logical formalization of non-standard analysis and possible related conservation results interesting on its own account. Moreover, it would be hugely beneficial, to have an extraction procedure
and some meta-theorems to guarantee specific uniform bounds in the non-standard context as well.
Inspired by the work of Nelson~\cite{nelson77, nelson88}, Berg et al. give a formalization
of non-standard analysis together with an adapted form of Shoenfield interpretation (while the similarity between Shienfield's interpretation and Nelson's translation procedure between classical and non-standard proofs was actually the original point of departure), which are both intended for extraction of computation content from proofs based on non-standard methods, in~\cite{BBS12}. A major
step towards that goal would be solving the interpretation of the saturation principles, which can be used to formalize the concept of Loeb measures -- one of the most prominent non-standard techniques.  
For certain systems, it has turned out that extending them with saturation principles has resulted in an increase in proof-theoretic strength (see \cite{hensonkeisler86, keisler07}). So far, it seems that in the
context of~\cite{BBS12} it is also the case and we need to add a version of bar recursion
which corresponds precisely to Spector's, but adopted to our special form of sequence application as defined in~\cite{BBS12}.




% Finally talk a little about Intiotionistic vs. classic and how this is partially included in NSA
% and that we deal with it.


%%
%
% Here comes the part about complexity
%
%%%%%%%%%%%%%
%Of course, we are interested in the computational complexity of these functionals. 
%While investigating this kind of questions definitely should
%be considered being proof mining, it hardly fits into the proof-theoretic bounds. 
%It is not easy, if not impossible, to assign every
%single problem or rather approach to solve a problem of mathematical logic 
%to a specific branch of symbolic logic, nevertheless one would probably consider this
% sort of analysis more a part of recursion theory than proof theory. 
%
%

\mySubsection{Goals and Organization of the Material}
While in the context description above, we follow more or less the chronological order of
development, this thesis is structured according to the underlying proof-theoretic strength
or, correspondingly, the complexity of the computational content in question. We start
with low complexity realizers for simple convergence and work towards bar-recursive solutions
in the last two chapters.
Large parts of this thesis are based on previous publications. We precede
each section with a short quote of some of the journal's referee comments on the corresponding
article. These quotes are intended to offer a short abstract on the topic at hand from a more detached perspective.
In the following
we give an overview of the contents of the chapters, their relation and which
publications are they based on.\\
%
The {\em Preliminaries chapter} is somewhat different from the others. It is based on parts from
both a joint paper with U. Kohlenbach~\cite{KS10} and~\cite{Safarik08}.
It gives a quick introduction to the basic 
methods and results in the area of proof mining, which might be considered folklore or common knowledge. We do give some new results (and fill tiny gaps in known proofs), but mostly it is a collection of
known or not surprising material and can be found in
standard literature (mainly~\cite{Troelstra73,Howard81,Kohlenbach08}). In fact, it is meant more
as a quick reference for results we need in later chapters. We capture the most relevant results in short summaries at the end of the corresponding sections. For an interested reader, 
it is probably best to refer to \cite{Kohlenbach08} directly instead.\\
%
The {\em second chapter} is about effective learnability and discusses in great detail what kind of computational content can be extracted, and under which circumstances, from convergence proofs. 
It is based on the most recent joint work with U. Kohlenbach~\cite{KS13}, where both authors contributed
equally~\footnote{though the proof of Proposition~\ref{p:nonLearnablePhi} is solely due to Kohlenbach} and I wrote the parts presented in this thesis. We start with this topic, because the analysis
deals with the lowest level of computational complexity (compared to the remainder of this thesis).
The goal is to precisely characterize when can we obtain more than a (computable) rate of metastability 
from a proof of a convergence statement in analysis. Of course, it is known when 
(computable) full rate of convergence is extractable. Here, we are concerned
with computational content strictly in between. Namely a (computable) bound on fluctuations or at least an (effective) learning procedure for the limit point.\\
%
We take this, so to say, a step further in the {\em third chapter}, where we dive into a
case study on a proof of a strong non-linear ergodic theorem due to R. Wittmann. 
The situation here, closely resembles our abstract separating example for a convergence statement which is non-learnable but has an effective rate of metastability from second chapter. This is due to a specific nested structure of the extracted realizer, which is unprecedented by similar results in proof mining so far.
It is a rare example, where a proof of strong convergence doesn't lead to a realizer of the specific
form which is typically guaranteed for effectively learnable statements. This chapter is based on~\cite{Safarik(11)} extended by an in depth analysis of an arithmetized version of Wittmann's proof.\\
%
In the {\em fourth chapter} we go even further and investigate how a specific compactness principle, the theorem of Bolzano-Weierstrass, contributes to the complexity of realizers when used
as the main principle in proofs of logical statements in a particular, but rather general, logical form.
To achieve this, we need to use advanced techniques like bar-recursion, but we demonstrate that even in systems formalizing large parts of full classical analysis, powerful principles like $\BW$
contribute by the minimal amount possible, when they are used naturally and are treated correctly.
The results in this chapter improve, generalize and extend preliminary results going back to my Diplom-Thesis \cite{Safarik08} and
have been published in~\cite{KS10}. In particular, we now treat the Bolzano-Weierstrass principle for general
Polish spaces of the form $\PP^b$ (see Definition~\ref{d:PPs}) which subsequently has been crucially used
by U. Kohlenbach in his proof-theoretic analysis of weak compactness in~\cite{Kohlenbach2012(weak)}.\\ 
%
The {\em last chapter} surveys my joint work with B. van den Berg and E. Briseid in~\cite{BBS12} presenting in 
detail my contributions (sections~\ref{ss:dst:negative} and~\ref{s:Shoenfield}), here actually extended by an example (interpretation of $\DNS{\st}$). While in the chapters three and four we use established techniques already fine-tuned for proof-mining, here we want to contribute to
the field in general and provide extraction procedures for non-standard analysis. Such a challenging task certainly needs more than a single publication on a system formalizing the non-standard methods and a corresponding (negative) functional interpretation, but we believe we made a very solid first step. In this thesis we go actually further and discuss a yet unpublished successful part of a larger unfinished attempt on interpreting the principle of countable saturation, which is the basic step needed for interpreting proofs based on Loeb measures (see section~\ref{ss:dst:csat}). This part is of particular interest in the context of this thesis, since it was achieved
using much of the ideas and techniques needed for our previous results.

\mySubsection{Notation and common Expressions}
%
By "$\equiv$" we mean the syntactical identity.
We will write \defkeyn{$\PiL$} and \defkeyn{$\SiL$} for the
purely universal arithmetic formulas, i.e. $\forall n^0\ \phi_{_\QF}(n)$, and
the purely existential arithmetic formulas, i.e. $\exists n^0\ \phi_{_\QF}(n)$, where
in both cases $\phi_{_\QF}$ denotes a quantifier-free formula, 
which may contain parameters of arbitrary type. 
However, in general, we allow quantification over variables of any finite type.\\
For the encoding of a given finite sequence $s$ of natural numbers we 
write \defkey{$\lh(s)$} for the length of $s$ and
denote by \defkey{$[s]$} the type one function:
\[
  [s](i^0)\ :=_0\ \begin{cases}s(i)&\text{if}\ \ i<_0\lh(s)\\0&\text{else}\end{cases}
\text{.}\]
For a type one function $f$ and a natural number $n$ we define the corresponding encoding of
the finite sequence \defkey{$\overline{f}n$} of length $n$
as follows:
\[
  \overline{f}n\ :=\ \big\langle f(0), f(1), \cdots, f(n-1) \big\rangle
\text{.} \]
Given two finite sequences $s$ and $t$ we write $s*t$ for the concatenation of $s$ and $t$. 
We write shortly $s*\langle 0\rangle$ and $s*\langle 1\rangle$ as
$s*0$ and $s*1$. Following the notation of Avigad and Feferman \cite{AF98} by
$s\subseteq t$ we mean $t$ is an extension of $s$ (i.e. the sequence $t$ starts
with the sequence $s$, or is $s$). We denote the empty sequence by \defkeyn{$\emptyset$}.\\
For finite tuples of variables (not necessarily of the same type) $x_1,x_2,\ldots,x_k$ we
write $\tup x$. By $\tup x^{\tup \rho}$ we 
mean $x_1^{\rho_1},x_2^{\rho_2},\ldots,x_k^{\rho_k}$.\\
Generally we will use the Greek letters $\phi$,$\psi$, $\chi$ to 
denote formulas (as an exception
a predicate in section~\ref{s:bw} is named $I$), the lower case Latin letters $f$,$g$,$h$
for functions, the letters $a$,$b$,$i$,$j$,$k$,$\ldots$ for 
natural numbers and encodings, and the capitals like $A$,$B$,$\ldots$ for functionals.\\
We denote $\lambda n^0.1^0$, $\lambda f^1.1^0$ and so on by $\one\equiv\one^1$, 
$\one^2$ and so on.
We use bold numbers to indicate the type level of a term, e.g. we would write 
$t\tp 1$ for $t^{1(0)}$. In general, we use this superscript as a shortcut for a specific type having
the given type level. So, by $\forall X\tp2$ we mean for all $X$ of an appropriate type,
e.g. $2(1(0))$, not for all $X$ which are of any type with level $2$.\\
We will write ``holds classically/intuition\-istic\-ally'' or 
"is equivalent classically/intuition\-istic\-ally"
and so on meaning in fact "is provable" or "the equivalence is provable" and so on
in the classical, i.e. in $\wepa + \QFm\AC$, or intuitionistic, i.e. in
$\weha + \M + \AC$, system respectively.\\
We will treat tuples and pairs somewhat special in the second chapter, but we will clarify this in its first section.\\
Finally, in our last chapter on non-standard systems, we will need some more elaborate conventions, however we define those
there at the beginning, since we don't need them anywhere else.

%% Sysint (1)
%The basic notions of Proof Mining, some of which we mentioned already in this introduction, are given in section~\ref{s:si}.
%% Basics (2)
%We present the functional interpretations of some principles we will need, mainly the schema 
%of arithmetical comprehension, in section~\ref{s:basicInterpretations}.\\
%% CompPM (3)
%In section \ref{s:compPM}, we give a basic introduction to some computability theory related to proof mining.
%% Schema/rule (4)
%Section \ref{s:sr} is a brief discussion on the different effects on
%the ND-interpretation of proofs based on full schemas and proofs based only 
%on their concrete instances.\\
%% WKL (5)
%In section \ref{s:wkl}, we present and extend, by
%giving a few more details, Howard's results on the
%functional interpretation of Weak K\"onigs lemma.\\
%% BW (6)
%We use this lemma  together with the results from section \ref{s:basicInterpretations} 
%in section \ref{s:bw} to give
%a proof of $\BW$ and its bar-recursive
%functional interpretation. Also, we tie up to sections \ref{s:compPM} and \ref{s:sr} by analyzing
%the complexity of the realizers in both cases: the general schema and its
%concrete instance.\\

\mySubsection{Acknowledgments}
First of all, I would like to thank my supervisor Prof. Dr. Ulrich
Kohlenbach. Professor Kohlenbach not only proposed the topic for this thesis and supported
my research in a very flexible way depending on the most recent
results, but also went out of his way and accorded me his time whenever I
encountered a problem providing hints, explanations and encouragement. 
Moreover, he led me to a large field
of applied logic far out of the bounds of the mere subject at hand.\\
I am also grateful to his assistants Dr. Eyvind Briseid, Dr. Alexander Kreuzer and Dr. Laurentiu
Leustean for their direct advice as well as for pointing me to some additional 
literature. I thank Prof. Dr. Benno van den Berg and Prof. Dr. Thomas Streicher
for several fruitful discussions and tips.\\
I also appreciate the continuing support of my family and friends. %A lot.

